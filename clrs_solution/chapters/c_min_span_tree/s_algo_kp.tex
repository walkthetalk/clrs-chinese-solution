\startsection[
  title={The algorithms of Kruskal and Prim},
]

%e23.2-1
\startEXERCISE
對於同一個輸入圖, Kruskal 算法返回的最小生成樹可以不同。
這種不同來源於對邊進行排序時,對權重相同的邊進行的不同處理。
證明:對於圖 \m{G} 的每棵最小生成樹 \m{T},
都存在一種辦法來對 \m{G} 的邊進行排序,
使得 Kruskal 算法所返回的最小生成樹就是 \m{T}。
\stopEXERCISE

\startANSWER
排序時,對於同樣權重的邊,讓出現在 \m{T} 中的排在前面。
\stopANSWER

%e23.2-2
\startEXERCISE
假定我們用鄰接矩陣來表示圖 \m{G=(V,E)}。
請給出 Prim 算法的一種簡單實現,
使其運行時間爲 \m{O(V^2)}。
\stopEXERCISE

\startANSWER
\CLRSH{MST-PRIME2(G, r)}
\startCLRS
for each u in V[G]
	key[u] = infinite
	pi[u] = NIL
key[r] = 0
Q = V[G]
while Q is not empty
	u = EXTRACT-MIN(Q)
	for each v in V[G]
		if A[u,v] != 0 and v in Q and A[u,v] < key[v]
			pi[v] = u
			key[v] = A[u,v]
\stopCLRS

兩層循環次數均爲 \m{|V|},因此算法的時間複雜度爲 \m{O(V^2)}。
\stopANSWER

%e23.2-3
\startEXERCISE
對於稀疏圖 \m{G=(V,E)},這裏 \m{|E|=\Theta(V)},
使用 Fibonacci 堆實現的 Prim 算法是否比使用二叉堆實現的算法更快?
對於稠密圖又如何呢?
 \m{|E|} 和 \m{|V|} 必須具備何種關係才能使 Fibonacci 堆的實現在漸進級別上比二叉堆的實現更快?
\stopEXERCISE

\startANSWER
令 \ALGO{EXTRACT-MIN} 時間爲 \m{T_1}, \ALGO{DECREASE-KEY} 時間爲 \m{T_2}。
時間複雜度爲 \m{\Theta(VT_1 + ET_2)}。
 Fibonacci 堆 \m{T_1=\lg V}, \m{T_2=1};
二叉堆 \m{T_1=\lg V}, \m{T_2=\lg V}。

若 \m{E=\Theta(V)},則兩種複雜度均爲 \m{\Theta(V\lg V)}。
若 \m{E=\Theta(V^2)}, Fibonacci 堆爲 \m{\Theta(V^2)},二叉堆爲 \m{\Theta(V^2\lg V)}。

當 \m{E=\omega(V)} 時, Fibonacci 堆漸進更快。
\stopANSWER

%e23.2-4
\startEXERCISE
假定圖中的邊權重全部爲整數,且在範圍 \m{1}~\m{|V|} 內。
 Kruskal 算法最快能多快?
如果邊的權重取值範圍在 1 到某個常數 \m{W} 之間呢?
\stopEXERCISE

\startANSWER
無論在什麼範圍內,排序用的時間都是 \m{O(E)}。
總時間爲 \m{O(E)+O((E+V)\alpha(V))=O(E\alpha(V))}。
\stopANSWER

%e23.2-5
\startEXERCISE
假定圖中的邊權重全部爲整數,且在範圍 \m{1}~\m{|V|} 內。
 Prim 算法最快能多快?
如果邊的權重取值範圍在 1 到某個常數 \m{W} 之間呢?
\stopEXERCISE

\startANSWER
用數列加雙向鏈表的形式實現, \m{T_1=1}, \m{T_2=1},總時間複雜度爲 \m{\Theta(E)}。
\stopANSWER

%e23.2-6
\startEXERCISE\DIFFICULT
假定一個圖中所有的邊權重均勻分佈在半開區間 \m{[0,1)} 內。
 Prim 算法和 Kruskal 算法哪一個可以運行得更快?
\stopEXERCISE

\startANSWER
Kruskal 算法可以運行得更快,可以使用桶排序,期望運行時間爲 \m{O(E)}。
總時間爲 \m{O(E)+O((E+V)\alpha(V))=O(E\alpha(V))}。
\stopANSWER

%e23.2-7
\startEXERCISE\DIFFICULT
假定圖 \m{G} 的一棵最小生成樹已經被計算出來。
如果在圖中加入一個新節點及其相關的新邊,
我們需要多少時間來對最小生成樹進行更新?
\stopEXERCISE

\startANSWER
如果只新增一條邊,則將這條邊加入原最小生成樹即可。
如果新增了 \m{k(k>1)} 條邊,那麼要刪除 \m{k-1} 條邊。

假設新節點是 \m{v},那麼必然存在包含 \m{v} 的環路。
遍歷 \m{k-1} 次,每次都能找到一個環路,從該環路中刪除一條權值最大的邊。
\stopANSWER

%e23.2-8
\startEXERCISE
Borden 教授提出了一個新的分治算法來計算最小生成樹。
該算法的原理如下:
給定圖 \m{G=(V,E)},將 \m{V} 劃分成兩個集合 \m{V_1} 和 \m{V_2},
使得 \m{|V_1|} 和 \m{|V_2|} 的差最多爲 1。
設 \m{E_1} 爲端點全部在 \m{V_1} 中的邊的集合,
 \m{E_2} 爲端點全部在 \m{V_2} 中的邊的集合。
我們遞迴地解決兩個子圖 \m{G_1=(V_1,E_1)} 和 \m{G_2=(V_2,E_2)} 的最小生成樹問題。
最後,在邊集合 \m{E} 中選擇橫跨切割 \m{V_1} 和 \m{V_2} 的最小權重的邊來
將求出的兩棵最小生成樹連接起來,
從而形成一棵最後的最小生成樹。
請證明該算法能正確計算除一棵最小生成樹,
或者舉出反例來說明該算法不正確。
\stopEXERCISE

\startANSWER
不正確,無法保證第一條邊屬於某個最小生成樹。如下圖:

\externalfigure[output/e23_2_8-1]
\stopANSWER

\stopsection
