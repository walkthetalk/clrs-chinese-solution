\startsubject[
  title={Problems},
]

%p14-1
\startPROBLEM
(Point of maximum overlap)
假設我們希望記錄一個區間集合的{\EMP 最大重疊點},即覆蓋此點的區間數目最多。
\startigBase[a]
\startitem
證明:總有一個最大重疊點是某個區間的端點。
\stopitem
\startANSWER
\TODO{略。}
\stopANSWER

\startitem
設計一個數據結構,使其能有效地支持 \ALGO{INTERVAL-INSERT}、 \ALGO{INTERVAL-DELETE},
以及返回最大重疊點的 \ALGO{FIND-POM} 操作。(\hint 使紅黑樹記錄所有的端點。
左端點關聯 +1,右端點關聯 -1,並且給樹中的每個節點擴張一個額外信息來維護最大重疊點。)
\stopitem
\startANSWER
令 \m{e_1,e_2,\cdots,e_n} 爲已排序的所有端點,滿足 \m{e_1\le e_2 \le \cdots \le e_n}。
從左至右掃描,若爲左端點,則 p 的值爲 +1,若爲右端點,則 p 的值爲 -1。
令 \m{s(i,j)=p_i + p_{i+1} + \cdots + p_j},其中 \m{1\le i\le j\le n}。
目標是要找到使 \m{s(1,j)} 最大的 j。

每個節點記錄三個值: v,子樹所有節點的 p 之和; m,子樹上 \m{s(i,j)} 的最大值;
 j,即 m 所對應的 j。
定義 NIl 的 m 和 v 均爲 0。
用自底向上的方式計算這些值。
\startformula
x.m = \startcases
\NC x.left.m	\NC (x.j = x.left.j) \NR
\NC x.left.m + x.v \NC (x.j = x) \NR
\NC x.left.m + x.v + x.right.m \NC (x.j = x.right.j) \NR
\stopcases
\stopformula

\ALGO{FIND-POM} 直接返回 T.root.j 即可。
\stopANSWER
\stopigBase
\stopPROBLEM

%p14-2
\startPROBLEM
(Josephus permutation)
此問題定義如下:假設有 n 個人,編號分別爲 \m{1,2,3,\ldots,n},按順序圍成一圈。
給定一個正整數 m 滿足 \m{m\le n}。
從某個指定的人開始報數,數到 m 的那個人出列;下一個人又從 1 開始報數,
以此類推,直到所有人全部出列。
按出列的次序將所有人的編號進行排列,稱其爲 (n,m)-Josephus 排列,
例如, (7,3)-Josephus 排列爲 \m{\langle 3,6,2,7,5,1,4\rangle}。
\startigBase[a]
%a
\startitem
假設 m 是常數,描述一個 \m{O(n)} 時間的算法,
使得對於給定的 n,能夠輸出 (n,m)-Josephus 排列。
\stopitem

\startANSWER
用循環鏈表即可,總時間爲 \m{O(mn)},由於 m 是常數,有 \m{O(mn)=O(n)}。
\stopANSWER

%b
\startitem
假設 m 不是常數,描述一個 \m{O(n\lg n)} 時間的算法,
使得對於給定的 n,能夠輸出 (n,m)-Josephus 排列。
\stopitem

\startANSWER
使用順序統計樹。順序統計樹相關操作的時間爲 \m{O(\lg n)},總時間爲 \m{O(n\lg n)}。

\CLRSH{JOSEPHUS(n,m)}
\startCLRS
T = NIL
for j = 1 to n
	create a node x with key[x] = j
	OS-INSERT(T, x)
j = 1
for k = n downto 1
	j = (j + m - 2) mod k + 1
	x = OS-SELECT(root[T], j)
	print key[x]
	OS-DELETE(T, x)
\stopCLRS
\stopANSWER
\stopigBase
\stopPROBLEM

\stopsubject%Problems
