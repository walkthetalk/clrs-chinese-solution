\startsubject[
  title={Problems},
]

\startPROBLEM
(Persistent dynamic sets)
在算法的執行過程中,我們有時會發現更新動態集合時需要維護其歷史版本。
我們稱這樣的集合爲{\EMP 持久的(persistent)}。
實現持久集合的一種方法是每次修改都將其完整的復制一份,
但這種方法會降低一個程序的執行速度,且佔用空間過多。
有時我們可以做得更好。

考慮結合 \m{S},支持 \ALGO{INSERT}、 \ALGO{DELETE} 和 \ALGO{SEARCH},
我們使用圖 13-8(a)所示的二叉搜索樹來實現。
對集合的每一個版本都維護一個不同的根。
爲了將關鍵字 5 插入其中,創建一個具有關鍵字 5 的新節點。
該節點會成爲關鍵字 7 的左孩子,由於我們不能修改關鍵字 7 所在節點,
因此我們需要一個關鍵字爲 7 的新節點。
以此類推,關鍵字 8 和 4 都需要一個新節點。
新節點 4 是新的根結點。
從而我們只需要復制一部分,就能構造一顆新樹,與原樹共享部分節點,
如圖 13-8(b)所示。

假設樹中每個節點都有屬性 key、 left 和 right,但沒有屬性 parent。
(參見\refexercise{13.3-6})
\startigBase[a]
\startitem%a
對於一顆一般的持久二叉搜索樹,爲插入一個關鍵字 k 或刪除一個節點 y,
需要改變哪些節點。
\stopitem

\startANSWER
需要改變從根結點開始,到要插入或刪除節點路徑上的所有節點。

如果是刪除,且要刪除的 z 有兩個孩子,則 z 的後繼 y 會代替 z 的位置。
 z 的所有祖先以及 y 都會發生變化,即 y 及其祖先都會改變。
\stopANSWER

\startitem%b
請寫出一個過程 \ALGO{PERSISTENT-TREE-INSERT},
使得在給定一棵持久樹 T 和一個要插入的關鍵字 k 時,
他返回將 k 插入 T 後得到的新的持久樹 T'。
\stopitem

\startANSWER
\CLRSH{PERSISTENT-TREE-INSERT(T, k)}
\startCLRS
y = T.root
if y == nil
	return new_by_key(k)
else
	new_root = new_copy(T.root)
	x = new_root
	while y != nil
		z = x
		if k < y.key
			y = y.left
			x = new_copy(y)
			z.left = x
		else
			y = y.right
			x = new_copy(y)
			z.right = x
			z = new_by_key(k)
			if k < x.key
				x.left = z
			else
				x.right = z
	return new_root
\stopCLRS
\stopANSWER

\startitem%c
如果持久二叉搜索樹 T 的高度爲 h,
實現 \ALGO{PERSISTENT-TREE-INSERT} 過程的時間和空間需求分別是多少?
(空間需求與新分配的節點數成正比)
\stopitem

\startANSWER
時間和空間都是 \m{O(h)}。
\stopANSWER

\startitem%d
假設在每個節點中增加一個父節點屬性。
在這種情況下, \ALGO{PERSISTENT-TREE-INSERT} 需要做一些額外的復制工作。
證明: \ALGO{PERSISTENT-TREE-INSERT} 的時間需求和空間需求爲 \m{\Omega(n)},
其中 n 爲樹中的節點個數。
\stopitem

\startANSWER
如果有屬性 parent,由於新根節點的存在,根節點孩子的 parent 會指向新根節點,
然後他們的孩子也會改成指向他們,以此類推。
因此每個節點都需要有一份拷貝,
插入操作會創建 \m{\Omega(n)} 個新節點,所需時間爲 \m{\Omega(n)}。
\stopANSWER

\startitem%e
說明如何利用紅黑樹來保證每次插入或刪除的最壞情況運行時間和空間爲 \m{O(\lg{n})}。
\stopitem

\startANSWER
由(a)和(c)可知,高爲 h 的持久二叉搜索樹與普通二叉搜索樹相同,
插入操作的最壞情況時間都是 \m{O(h)}。
紅黑樹的高度 \m{h=O(\lg n)},
因此普通紅黑樹的插入操作需要的時間爲 \m{O(\lg n)}。
如果紅黑樹是持久的,
則插入操作仍然可以在時間 \m{O(\lg n)} 內完成。
\stopANSWER

\stopigBase
\stopPROBLEM

%p13-2
\startPROBLEM
(Join operation on red-black trees)
{\EMP 連接(join)}操作指的是取兩個動態集合 \m{S_1}、 \m{S_2} 和一個元素 \m{x},
使得對於任何 \m{x_1\in S_1} 和 \m{x_2\in S_2},均滿足 \m{x_{1}.key \le x.key \le x_{2}.key}。
該操作返回一個集合 \m{S=S_1 \cup \{x\} \cup S_2}。
在這個問題中,討論如何在紅黑樹上實現連接操作。
\startigBase[a]
\startitem%a
給定一棵紅黑樹 \m{T},其黑高存儲在新屬性 \m{T.bh} 中。證明:
在不需要樹中節點的額外存儲空間和不增加漸進運行事件的前提下,
 \ALGO{RB-INSERT} 和 \ALGO{RB-DELETE} 可以維護這個屬性。
並證明:當沿 \m{T} 下降時,可以對每個被訪問的節點在 \m{O(1)} 時間內確定其黑高。
要求實現操作 \ALGO{RB-JOIN(T1, x, T2)},
他會銷燬 \m{T_1} 和 \m{T_2} 並返回一棵紅黑樹 \m{T=T_1\cup \{x\} \cup T_2}。
令 \m{n} 爲 \m{T_1} 和 \m{T_2} 中的節點總數。
\stopitem
\startANSWER
對於空的紅黑樹,將其 bh 初始化爲 0。
插入只會增加黑高,刪除只會減小黑高。
插入過程中,如果 \ALGO{BALANCING} 到達了根節點,
將根節點由紅色改成了黑色,則 bh 會增 1。
這是往樹中添加黑色節點唯一的地方。
刪除過程中,如果額外的黑節點到達了根節點,則將 bh 減 1;
這是往樹中刪除黑色節點唯一的地方。
因此每次操作過程中, bh 的開銷爲 \m{O(1)}。
樹中每個節點的黑高,都可以由根節點的 bh 開始,每過一個黑色節點就減 1;
其開銷也爲 \m{O(1)}。
\stopANSWER

\startitem%b
假設 \m{T_{1}.bh\ge T_{2}.bh}。
試給出一個 \m{O(\lg n)} 時間的算法,
使之能從黑高爲 \m{T_{2}.bh} 的節點中選出具有最大關鍵字的 \m{T_1} 中的黑色節點 \m{y}。
\stopitem
\startANSWER
我們只需要遍歷樹的最右路徑,即,
如果有右孩子,我們就可以一直往右走,否則就往左走。
在此過程中,每過一個黑色節點,就將黑高減 1。
直到所到達黑色節點的黑高等於 \m{bh[T_2]},
我們就找到了黑高爲 \m{bh[T_2]} 的最大關鍵字。
由於紅黑樹的高度爲 \m{O(\lg n)},此操作需要時間爲 \m{O(\lg n)}。
\stopANSWER

\startitem%c
令 \m{T_y} 爲以 \m{y} 爲根節點的子樹。
試說明如何在不破壞二叉搜索樹性質的前提下,
在 \m{O(1)} 時間內用 \m{T_y\cup \{x\} \cup T_2} 來取代 \m{T_y}。
\stopitem
\startANSWER
在 \m{T_1} 中 y 所在位置插入 x。
令 y 爲左孩子, x 的右孩子。
假定連接操作滿足 \m{x_1.key\le x.key\le x_2.key},
則會保持二叉搜索樹的性質,所需時間爲 \m{O(1)}。
\stopANSWER

\startitem%d
要保持紅黑性質 1、 3 和 5,應將 \m{x} 着成什麼顏色?
試說明如何在 \m{O(\lg n)} 時間內維護性質 2 和 4。
\stopitem
\startANSWER
將 x 着爲紅色。
我們可以執行 \ALGO{RB-INSERT-FIXUP(T_1, x)} 來維持性質 2、 4。
 \m{T_2} 的黑高等於 \m{T_y},
因此 \ALGO{RB-INSERT-FIXUP} 可以工作。
運行時間爲 \m{O(\lg n)}。
\stopANSWER

\startitem%e
論證使用(b)部分的假設是不失一般性的,
並描述當 \m{T_1.bh\le T_2.bh} 時所出現的對稱情況。
\stopitem
\startANSWER
如果 \m{T_1.bh < T_2.bh},
我們會掃描 \m{T_2} 最左側的路徑,找到黑高爲 \m{T_1.bh} 的最小的黑色節點。
\stopANSWER

\startitem%f
證明: \ALGO{RB-JOIN} 的運行時間是 \m{O(\lg n)}。
\stopitem
\startANSWER
由之前幾部分來實現 \ALGO{RB-JOIN}:
可以在 \m{O(1)} 時間內計算維護黑高。
可以在 \m{O(\lg n)} 時間內找到所需黑色節點 y。
在 \m{O(1)} 時間內完成連接,
最後花 \m{O(\lg n)} 時間來保持紅黑樹的性質。
總共需要時間爲 \m{O(\lg n)}。
\stopANSWER

\stopigBase
\stopPROBLEM

%p13-3
\startPROBLEM
(AVL 樹)
 AVL 樹是一種{\EMP 高度平衡的(height balanced)}二叉搜索樹:
對每一個節點 x,其左子樹和右子樹的高度最多差 1。
要實現一棵 AVL 樹,
需要在每個節點內維護一個額外的屬性: \m{x.h} 爲節點 x 的高度。
與任何其他二叉搜索樹 T 一樣,
假設 \m{T.root} 指向根節點。
\startigBase[a]
\startitem%a
證明:一棵有 n 個節點的 AVL 樹高度爲 \m{O(\lg n)}。
(\hint 證明高度爲 h 的 AVL 樹至少有 \m{F_h} 個節點,
其中 \m{F_h} 是 Fibonacci 數列的第 h 個數)
\stopitem
\startANSWER
對於一棵高度爲 \m{h} 的高度平衡樹,令 \m{T(h)} 時節點數目的最小值。
現在來歸納。
對於基準情況, \m{T(1)\ge T(0)\ge 1},因此 \m{T(1)\ge F_1} 且 \m{T(0)\ge F_0}。
現在假設對於所有 \m{h'<h},都有 \m{T(h')\ge F_{h'}}。
高度爲 h 的 AVL 樹中的根節點有兩個孩子:
一個孩子高度爲 \m{h-1},二另一個孩子高度至少爲 \m{h-2}。
高度爲 h 的 AVL 樹中節點數用 \m{T(h-1)} 和 \m{T(h-2)} 表示爲 \m{T(h)\ge T(h-1)+T(h-2)}。
由歸納假設,這意味着 \m{T(h)\ge F_{h-1} + F_{h-2} = F_h}。
所以 \m{F_h\ge 1.6^h},因此 \m{n\ge 1.6^h}, \m{h=O(\lg n)}。
\stopANSWER

\startitem%b
要在一棵 AVL 樹中插入一個節點,
首先以二叉搜索樹的順序把該節點放在適當的位置上。
此時,這棵樹可能不再是高度平衡的。
也就是說,某些節點的左子樹和右子樹高度差可能會是 2。
請描述一個過程 \ALGO{BALANCE(x)},
輸入一棵以 x 爲根的子樹,
其左子樹與右子樹是高度平衡的,且高度差最多是 2,
即 \m{|x.right.h - x.left.h|\le 2},
並將這棵以 x 爲根的子樹轉換成高度平衡的。
(\hint 使用旋轉)
\stopitem
\startANSWER
\CLRSH{BALANCE(x)}
\startCLRS
if height(x.left) - height(x.right) \le 1
	return x
else if height(x.left) < height(x.right)
	y = x.right
	if y.left < y.right
		return left-rotate(x)
	else
		right-rotate(y)
		return left-rotate(x)
else
	y = x.right
	if y.right < y.left
		return right-rotate(x)
	else
		left-rotate(y)
		return right-rotate(x)
\stopCLRS
\stopANSWER

\startitem%c
利用(b)來描述一個過程 \ALGO{AVL-INSERT(x, z)},
該操作輸入一個 AVL 樹中的節點 x 以及一個新節點 z (其關鍵字已經填好),
然後將 z 添加到以 x 爲根的子樹中,
並保持 x 是一棵 AVL 樹的根節點。
和 12.3 節中的 \ALGO{TREE-INSERT} 一樣,
假設 \m{z.key} 已經填好,且 \m{z.left=NIL}, \m{z.right=NIL};
再假設 \m{z.h=0}。
因此要把節點 z 插入到 AVL 樹 T 中,
需要調用 \ALGO{AVL-INSERT(T.root, z)}。
\stopitem
\startANSWER
\CLRSH{AVL-INSERT(x, z)}
\startCLRS
if x == nil
	height[z] = 0
	return z
if key[z] \le key[x]
	y = AVL-INSERT(left[x], x)
	left[x] = y
else
	y = AVL-INSERT(right[x], x)
	right[x] = y
	parent[y] = x
	height[x] = height[y] + 1
	x = BALANCE(x)

return x
\stopCLRS
\stopANSWER

\startitem%d
證明:在一棵 n 個節點的 AVL 樹上 \ALGO{AVL-INSERT} 操作需花費 \m{O(\lg n)} 時間,
且執行 \m{O(1)} 次旋轉。
\stopitem
\startANSWER
\stopANSWER

\stopigBase
\stopPROBLEM

%p13-4
\startPROBLEM
(treap 樹)
如果將一個含 n 個元素的集合插入到一棵二叉搜索樹中,
所得到的樹可能會相當不平衡,
從而導致查找事件很長。
然而從 12.4 節中可知,
隨機構造二叉搜索樹是趨向於平衡的。
因此,一般來說,要爲一組固定的元素建立一棵平衡樹,
可以採用的一種策略就是先隨機排列這些元素,然後按照排列的順序將他們插入到樹中。

如果沒法同時得到所有的元素,應該怎樣處理呢?
如果一次收到一個元素,
是否仍然能用他們來隨機建立一棵二叉搜索樹?

我們將通過考察一個數據結構來正面回答這個問題。
一棵 treap 樹是一棵更改了節點排序方式的二叉搜索樹。
圖 13-9 顯示了一個例子。
通常,樹內的每個節點 x 都有一個關鍵字 x.key。
另外,還要爲每個節點指定 x.priority,
他是一個獨立選取的隨機數。
假設所有的優先級都是不同的,
而且所有的關鍵字也是不同的。
在 treap 樹中,所有節點的關鍵字排列順序遵循二叉搜索樹的性質,
且優先級遵循最小堆順序性質:
\startigBase[2]
\item 如果 v 是 u 的左孩子,則 v.key < u.key。
\item 如果 v 是 u 的右孩子,則 v.key > u.key。
\item 如果 v 是 u 的孩子,則 v.priority > u.priority。
\stopigBase
(這兩個性質的結合就是這種樹被稱爲“treap”樹的原因:
他同時具有二叉搜索樹和堆的特徵)

用以下方式考慮 treap 樹是會有幫助的。
假設將已有相應關鍵字的節點 \m{x_1}、 \m{x_2}、 \m{\ldots}、 \m{x_n} 插入到一棵 treap 樹內。
通過將這些節點以他們的優先級(隨機選取的)順序插入一棵正常的二叉搜索樹形成一棵 treap 樹,
即 \m{x_i.priority < x_j.priority} 表示 \m{x_i} 在 \m{x_j} 之前被插入。

\startigBase[a]
\startitem%a
證明:給定一個已有相應關鍵字和優先級(互異)的節點  \m{x_1}、 \m{x_2}、 \m{\ldots}、 \m{x_n} 組成的集合,
存在唯一的一棵 treap 樹與這些節點相關聯。
\stopitem
\startANSWER
首先,根節點是唯一的,是優先級最小的那個節點。
剩下的節點分成了左右兩棵子樹。
子樹的根節點時子樹中優先級最小的那個節點。
以此類推,每個節點的位置都是固定的。因此所對應的 treap 樹是唯一的。
\stopANSWER

\startitem%b
證明: treap 樹的期望高度是 \m{\Theta(\lg n)},
因此在 treap 樹內查找一個值所花的時間爲 \m{\Theta(\lg n)}。
讓我們看看如何將一個新的節點插入到一個已存在的 treap 樹中。
要做的第一件事就是將一個隨機的優先級賦予這個新節點。
然後調用稱爲 \ALGO{TREAT-INSERT} 的插入算法,
其操作如圖 13-10 所示。
\stopitem
\startANSWER
我們直到隨機構建的二叉樹,其高度期望值爲 \m{O(\lg n)}。
 treap 樹與通過按優先級升序插入 n 個節點所構成的 BST 是一樣的。
如果優先級時隨機指定的,則優先級所有排列的可能性相同。
因此 treap 樹高度的期望值與隨機構建的 BST 高度的期望值相同,
均是 \m{O(\lg n)}。
\stopANSWER

\startitem%c
解釋 \ALGO{TREAP-INSERT} 是如何工作的。
說明其思想並給出僞代碼。
(\hint 執行通常的二叉搜索樹插入過程,然後做旋轉來恢復最小堆順序的性質)
\stopitem
\startANSWER
我們調用 \ALGO{BST-INSERT} 將新節點 x 插入 treap。
接着我們來看 x 的父節點。
如果 x 的優先級比其父節點的優先級高,
我們根據 x 是左孩子還是右孩子來進行旋轉。
一級一級往上,直到到達根節點。

\CLRSH{TREAP-INSERT(T, x, priority)}
\startCLRS
BST-INSERT(T, x)
while x.parent != NIL && x.priority < x.parent.priority
	if x.parent.left == x
		RIGHT-ROTATE(T, x.parent)
	else
		LEFT-ROTATE(T, x.parent)
\stopCLRS
\stopANSWER

\startitem%d
證明: \ALGO{TREAP-INSERT} 的期望運行時間是 \m{\Theta(\lg n)}。
\stopitem
\startANSWER
\ALGO{TREAP-INSERT} 首先執行 \ALGO{BST-INSERT},所需時間爲 \m{\Theta(h)}。
然後沿着路徑往根節點前進,並保持最小堆的性質,
所花時間爲 \m{\Theta(h)},因爲我們所到達的每個節點最多秩序一次旋轉。
所以 \ALGO{TREAP-INSERT} 的期望運行時間爲 \m{\Theta(h) = \Theta(\lg n)}。
\stopANSWER

\ALGO{TREAP-INSERT} 先執行一次查找,然後做一系列旋轉。
雖然這兩種操作的期望運行時間相同,但其實際代價不同。
查找操作會從 treap 樹中讀取信息而不做修改。
相反,旋轉操作會改變 treap 樹內的父節點和子節點的指針。
在大部分的計算機上,讀取操作要比寫入操作快很多。
所以我們希望 \ALGO{TREAP-INSERT} 執行少量的旋轉。
後面將說明所執行旋轉的期望次數有一個常數界。

爲此,需要做一些定義,如果 13-11 所示。
一棵二叉搜索樹 T 的{\EMP 左脊柱(left spine)}是從根節點到最小關鍵字節點的簡單路徑。
換句話說,左脊柱是從根節點開始只包含左邊緣的簡單路徑。
對稱地, T 的{\EMP 右脊柱(left spine)}是從根節點開始只包含右邊緣的簡單路徑。
一條脊柱的長度是他包含的節點數目。

\startitem%e
考慮利用 \ALGO{TREAP-INSERT} 插入節點 x 後的 treap T。
設 C 爲 x 左子樹的右脊柱的長度, D 爲 x 右子樹的左脊柱的長度。
證明:在插入 x 期間所執行的旋轉總次數等於 \m{C+D}。
現在來計算 C 和 D 的期望值。
不失一般性,假設關鍵字爲 1、 2、 …、 n,因爲只是將他們兩兩比較。
\stopitem
\startANSWER
剛調用完 \ALGO{TREAP-INSERT}, x 沒有孩子, \m{C+D=0}。
每次在 x 的父節點 y 上執行左旋, x 的右孩子保持不變。
 x 的左孩子變成 y, x 之前的左子樹變成 y 的右子樹。
換句話說,我們給集合 \m{C+D} 添加新節點, x 會向根節點上升,
 y 永遠在 x 左子樹的右脊柱上。
右旋類似。
所以旋轉的總次數爲 \m{C+D}。
\stopANSWER

對 treap T 中的節點 x 和 y,其中 \m{y\ne x},
設 \m{k=x.key} 以及 \m{i=y.key}。
定義指示器隨機變量:
\startformula
X_{ik} = I\{\text{ \m{y} 在 \m{x} 的左子樹的右脊柱中}\}
\stopformula

\startitem%f
證明:當且僅當滿足下列條件時, \m{X_{ik}=1}:
\m{y.priority > x.priority}、 \m{y.key < x.key},
且對於每個滿足 \m{y.key < z.key < x.key} 的 \m{z},有 \m{y.priority < z.priority}。
\stopitem
\startANSWER
首先,假設 \m{X_{ik} = 1}。
由最小堆性質可知, y 是 x 的後裔,才有 \m{y.priority < x.priority}。
由 BST 屬性可知, y 在 x 的左子樹中,才有 \m{y.key < x.key}。
最後,考慮任一節點 z 滿足 \m{y.key < z.key < x.key},
以中序遍歷 treap。
打印出 y 後,我們會打印 y 的右子樹,而 x 的左子樹已經打印完了,
因爲 y 位於 x 左子樹的右脊柱上。
因此,在 y 之後, x 之前打印的節點均位於 y 的右子樹中;
 z 一定時在 y 的右子樹中。

然後,我們假設三個性質保持不變,並證明 \m{X_{ik} = 1}。
首先考慮 y 在 x 的左子樹中,但不在此子樹的右脊柱上的可能性。
在此脊柱上存在節點 z,y 在 z 的左脊柱上。
 z 滿足 \m{y.key < z.key < x.key},
但是 \m{z.priority < y.priority} 不滿足第三個性質。
顯然 y 不能在 x 的右子樹中,否則違反了第二個性質。
且 x 不能時 y 的後裔,否則會違反第一個性質。
假定 x 和 y 不是彼此的後裔, 而 z 是他們共同的祖先。
則 \m{y.key < z.key < x.key},但是 \m{z.priority < y.priority},
違反了第三個性質。
剩下的唯一可能就是 y 在 x 左子樹的右脊柱上,即 \m{X_{ik} = 1}。
\stopANSWER

\startitem%g
證明:
\startformula
\Pr\{X_{ik}=1\} = \frac{(k-i-1)!}{(k-i+1)!} = \frac{1}{(k-i+1)(k-i)}
\stopformula
\stopitem
\startANSWER
假定 \m{k>1}, \m{\Pr\{X_{ik} = 1\}} 是滿足(f)中所有條件的概率。
考慮關鍵字爲 \m{i}、 \m{i+1}、 \m{\ldots}、 \m{k} 的所有元素。
共有 \m{k-i+1} 個元素。其優先級的順序隨機。
共有 \m{(k-i+1)} 種可能的排列(且概率相同)。
爲滿足條件,需要對於所有 \m{z\in i+1, i+2, \ldots, k-1},
都有 \m{z.priority > y.priority > x.priority}。
 x 和 y 的優先級是剩餘元素中最小的兩個,剩餘元素排列方式有 \m{(k-i-1)!} 種。
\stopANSWER

\startitem%h
證明:
\startformula
E[C] = \sum_{j=1}^{k-1}\frac{1}{j(j+1)} = 1 - \frac{1}{k}
\stopformula
\stopitem
\startANSWER
\startformula\startmathalignment
\NC E[C] \NC = \sum_{j=1}^{k-1}\frac{1}{j(j+1)} \NR
\NC      \NC = \sum_{j=1}^{k-1}(\frac{1}{j} - \frac{1}{j+1}) \NR
\NC      \NC = 1 - \frac{1}{k} \NR
\stopmathalignment\stopformula
\stopANSWER

\startitem%i
利用對稱性證明:
\startformula
E[D] = 1 - \frac{1}{n-k+1}
\stopformula
\stopitem
\startANSWER
x 左子樹的右脊柱,期望長度取決於元素 x 的秩 k。
右子樹的左脊柱,期望長度取決於 x 的反向秩 \m{n-k+1}。
因此, \m{E[D] = 1 - 1/(n-k+1)}。
\stopANSWER

\startitem%j
得出如下結論:
當在一棵 treap 樹中插入一個節點時,
執行旋轉的期望次數小於 2。
\stopitem
\startANSWER
\startformula
E[C+D] = E[C] + E[D] = 2 - \frac{1}{k} - \frac{1}{n-k+1} \le 2
\stopformula
\stopANSWER

\stopigBase
\stopPROBLEM

\stopsubject%Problems
