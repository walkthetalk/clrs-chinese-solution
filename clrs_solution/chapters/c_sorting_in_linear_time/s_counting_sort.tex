\startsection[
  title={Counting sort},
]

\startEXERCISE
參照圖 8-2 的方法,說明 \ALGO{COUNTING-SORT} 在數列 \m{A=\langle 6,0,2,0,1,3,4,6,1,3,2\rangle} 上的操作過程。
\stopEXERCISE

\startANSWER
\externalfigure[output/e8_2_1-1]
\externalfigure[output/e8_2_1-2]
\externalfigure[output/e8_2_1-3]
\externalfigure[output/e8_2_1-4]
\externalfigure[output/e8_2_1-5]
\externalfigure[output/e8_2_1-6]
\externalfigure[output/e8_2_1-7]
\externalfigure[output/e8_2_1-8]
\externalfigure[output/e8_2_1-9]
\externalfigure[output/e8_2_1-10]
\externalfigure[output/e8_2_1-11]
\externalfigure[output/e8_2_1-12]
\externalfigure[output/e8_2_1-13]
\stopANSWER

\startEXERCISE
試證明 \ALGO{COUNTING-SORT} 是穩定的。
\stopEXERCISE

\startANSWER
假定兩個下標爲 \m{i_1} 和 \m{i_2},且 \m{i_1 < i_2} 的元素相等。
排序後位置爲 \m{j_1} 和 \m{j_2}。
由於 \ALGO{COUNTING-SORT} 是從右向左處理數列元素的,
會先處理 \m{A[i_2]},將其放入 \m{B[j_2]} 中,
然後再處理 \m{A[i_1]},將其放入 \m{B[j_1]} 中。
由算法可知 \m{j_2 > j_1},因此算法是穩定的。
\stopANSWER

\startEXERCISE
假設我們在 \ALGO{COUNTING-SORT} 的第 10 行循環的開始部分,將代碼改爲:
\startCLRS[start=10]
for j = 1 to A.length
\stopCLRS
試證明該算法仍然是正確的。他還穩定嗎?
\stopEXERCISE

\startANSWER
算法仍然正確,但不穩定了。相等的元素會以相反的順序出現在排好序的數列中。
\stopANSWER

\startEXERCISE
設計一個算法,他能夠對於任何給定的介於 \m{0} 到 \m{k} 之間的 \m{n} 個整數先進行預處理,
然後在 \m{O(1)} 時間內回答輸入的 \m{n} 個整數中有多少個落在區間 \m{[a..b]} 內。
你設計的算法的預處理時間應爲 \m{\Theta(n+k)}。
\stopEXERCISE

\startANSWER
先利用 \ALGO{COUNTING-SORT} 中的部分代碼構建數列 \m{C},即預處理,時間爲 \m{\Theta(n) + \Theta(k) = \Theta(n+k)}。
然後直接由 \m{C[b] - C[a-1]} 即可得到想要的結果,時間爲 \m{O(1)}。
\stopANSWER

\stopsection
