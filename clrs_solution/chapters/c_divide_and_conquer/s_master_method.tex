\startsection[
  title={The master method for solving recurrences},
]

\startEXERCISE
用主定理求下列遞迴式的漸進緊確界。
\startigBase[a]
\item \m{T(n) = 2T(n/4) + 1}
\item \m{T(n) = 2T(n/4) + \sqrt{n}}
\item \m{T(n) = 2T(n/4) + n}
\item \m{T(n) = 2T(n/4) + n^2}
\stopigBase
\stopEXERCISE
\startANSWER
\startigBase[a]
\item \m{\Theta(n^{\log_4{2}}) = \Theta(\sqrt{n})}
\item \m{\Theta(n^{\log_4{2}}\lg{n}) = \Theta(\sqrt{n}\lg{n})}
\item \m{\Theta(n)}
\item \m{\Theta(n^2)}
\stopigBase
\stopANSWER

\startEXERCISE
Caesar 教授想設計一個漸進優於 Strassen 算法的矩陣相乘算法。
他的算法使用分治策略,將每個矩陣分解爲 \m{n/4 \times n/4} 的子矩陣,
分解和合並步驟共花費時間爲 \m{\Theta(n^2)}。
他需要確定,這個算法需要創建多少個子問題,才能擊敗 Strassen 算法。
如果他的算法創建 \m{a} 個子問題,則描述運行時間 \m{T(n)} 的遞迴式爲
\m{T(n) = a T(n/4) + \Theta(n^2)}。
Caesar 教授的算法如果要漸進優於 Strassen 算法, \m{a} 的最大整數值應爲多少?
\stopEXERCISE
\startANSWER
當 \m{a < 16} 時適用於主定理的第一種情況。
此時,此算法 \m{T(n) = \Theta(n^{\log_4{a}})}。
要想更快,需 \m{\log_4{a} < \log_{2}7},其中 \m{a < 14}。
因此,最大整數值爲 \m{13}。
\stopANSWER

\startEXERCISE
用主定理證明:而分查找遞迴式 \m{T(n) = T(n/2) + \Theta(1)} 的解爲 \m{T(n) = \Theta(\lg n)}。
(二分查找的描述參見\refexercise{bin_search})
\stopEXERCISE
\startANSWER
\startformula\startmathalignment
\NC a =\NC 1\NR
\NC b =\NC 2\NR
\NC f(n) =\NC \Theta(n^{\log_2^1}) = \Theta(1) \NR
\NC T(n) =\NC \Theta(\lg n) \NR
\stopmathalignment\stopformula
\stopANSWER

\startEXERCISE
主定理是否可用於遞迴式 \m{T(n) = 4 T(n/2) + n^2 \lg n}?
請說明爲什麼。
請給出此遞迴式的漸進上界。
\stopEXERCISE
\startANSWER
根據主定理, 有 \m{a = 4}、 \m{b = 2},
而 \m{f(n) = n^2\lg{n} \ne O(n^{2-\epsilon}) \ne \Omega(n^{2-\epsilon})},
因此無法使用主定理。

猜測 \m{\Theta(n^2\lg^2{n})}:
\startformula\startmathalignment
\NC T(n) \NC \le 4T(n/2) + n^2\lg{n} \NR
\NC      \NC \le 4c(n/2)^2\lg^2(n/2) + n^2\lg{n} \NR
\NC      \NC \le cn^2\lg(n/2)\lg{n} - cn^2\lg(n/2)\lg{2} + n^2\lg{n} \NR
\NC      \NC \le cn^2\lg^2{n} - cn^2\lg{n}\lg{2} - cn^2\lg(n/2) + n^2\lg{n} \NR
\NC      \NC \le cn^2\lg^2{n} + (1 - c)n^2\lg{n} - cn^2\lg(n/2) \qquad (c > 1) \NR
\NC      \NC \le cn^2\lg^2{n} - cn^2\lg(n/2) \NR
\NC      \NC \le cn^2\lg^2{n} \NR
\stopmathalignment\stopformula

推廣結論參見 \refexercise{master_method_generalized}。
\stopANSWER

\startEXERCISE \DIFFICULT
考慮主定理第三種情況中,對於某個常數 \m{c < 1},考慮其正則條件 \m{a f(n/b) \le c f(n)} 是否成立。
給出例子,使常數 \m{a \ge 1}、 \m{b > 1},且函數 \m{f(n)} 滿足主定理第三種情況中除正則條件外的所有條件。
\stopEXERCISE
\startANSWER
\startformula\startmathalignment
\NC a \NC = 1 \NR
\NC b \NC = 2 \NR
\NC f(n) \NC = n (2 - \cos n) \NR
\stopmathalignment\stopformula

如果想要證明:
\startformula\startmathalignment[n=1]
\NC \frac{n}{2}(2 - \cos \frac{n}{2}) < c n \NR
\NC \frac{1 - \cos(n/2)}{2} < c \NR
\NC c \ge 1 - \frac{\cos(n/2)}{2} \NR
\stopmathalignment\stopformula
由於 \m{\min \cos(n/2) = -1},因此需要 \m{c \ge \frac{3}{2}},但是 \m{c < 1}。
\stopANSWER

\stopsection
