\startsection[
  title={Priority queues},
]

%e6.5-1
\startEXERCISE
試說明 \ALGO{HEAP-EXTRACT-MAX} 在堆 \m{A = \langle
15, 13, 9, 5, 12, 8, 7, 4, 0, 6, 2, 1 \rangle} 上的操作過程。
\stopEXERCISE

\startANSWER
\startcombination[2*2]
{\externalfigure[output/e6_5_1-1]}{}
{\externalfigure[output/e6_5_1-2]}{}
{\externalfigure[output/e6_5_1-3]}{}
\stopcombination
\stopANSWER

%e6.5-2
\startEXERCISE
試說明 \ALGO{MAX-HEAP-INSERT(A, 10)} 在堆 \m{A = \langle
15, 13, 9, 5, 12, 8, 7, 4, 0, 6, 2, 1 \rangle} 上的操作過程。
\stopEXERCISE

\startANSWER
\startcombination[2*2]
{\externalfigure[output/e6_5_2-1]}{}
{\externalfigure[output/e6_5_2-2]}{}
{\externalfigure[output/e6_5_2-3]}{}
\stopcombination
\stopANSWER

%e6.5-3
\startEXERCISE
要求用最小堆實現最小優先隊列,請寫出 \ALGO{HEAP-MINIMUM}、 \ALGO{HEAP-EXTRACT-MIN}、
 \ALGO{HEAP-DECREASE-KEY} 和 \ALGO{MIN-HEAP-INSERT} 的僞碼。
\stopEXERCISE

\startANSWER
\CLRSH{HEAP-MINIMUM(A)}
\startCLRS
return A[1]
\stopCLRS

\CLRSH{HEAP-EXTRAC-MIN(A)}
\startCLRS
if A.heap-size < 1
	error "heap underflow"
min = A[1]
A[1] = A[A.heap-size]
A.heap-size = A.heap-size - 1
MIN-HEAPIFY(A, 1)
\stopCLRS

\CLRSH{HEAP-DECREASE-KEY(A, i, key)}
\startCLRS
if key > A[i]
	error "new key is bigger than current key"
A[i] = key
while i > 1 and A[PARENT(i)] > A[i]
	exchange A[i] with A[PARENT(i)]
	i = PARENT(i)
\stopCLRS

\CLRSH{MIN-HEAP-INSERT(A, key)}
\startCLRS
A.heap-size = A.heap-size + 1
A[A.heap-size] = +∞
HEAP-DECREASE-KEY(A, A.heap-size, key)
\stopCLRS
\stopANSWER

%e6.5-4
\startEXERCISE
在 \ALGO{MAX-HEAP-INSERT} 的第 2 行,爲什麼要先把關鍵字設爲 \m{-\infty},
然後又將其增加到所需的值?
\stopEXERCISE

\startANSWER
是爲了能通過 \ALGO{HEAP-INCREASE-KEY} 中的校驗 \m{key < A[i]}。
\stopANSWER

%e6.5-5
\startEXERCISE
試分析在使用下列循環不變式時, \ALGO{HEAP-INCREASE-KEY} 的正確性:

在算法的第 4~6 行 {\EMP while} 循環每次迭代開始時,
子數列 \m{A[1..A.heap-size]} 要滿足最大堆的性質。
如果由違背,只有一個可能: \m{A[i]} 大於 \m{A[PARENT(i)]}。

這裏,可以假定調用 \ALGO{HEAP-INCREASE-KEY} 時, \m{A[1..A.heap-size]} 是滿足最大堆性質的。
\stopEXERCISE

\startANSWER
{\EMP 初始化:} \m{A} 是最大堆,除非 \m{A[i]} 比他的父節點大,因此只有 \m{A[i]} 被修改過。
 \m{A[i]} 比其子節點大,否則無法通過參數校驗,即不會進入循環(新值大於舊值,
 且舊值大於其父節點);

{\EMP 保持:} 調換 \m{A[i]} 和其父節點時,仍然滿足最大堆的性質,
只有 \m{A[PARENT(i)]} 可能比其父節點大。 \m{i} 變爲 \m{PARENT(i)} 後此不變式仍然成立;

{\EMP 終止:}當到達根節點,或者 \m{A[i]} 和其父節點關系滿足最大堆的性質時,循環終止。
循環終止後, \m{A} 就是一個最大堆了。
\stopANSWER

%e6.5-6
\startEXERCISE
在 \ALGO{HEAP-INCREASE-KEY} 的第 5 行,一般要通過三次賦值才能完成交換操作。
想一想如何利用 \ALGO{INSERTION-SORT} 內循環部分的思想,
只用一次賦值就完成這一交換操作?
\stopEXERCISE

\startANSWER
\CLRSH{HEAP-INCREASE-KEY(A, i, key)}
\startCLRS
if key < A[i]
	error "new key is smaller than current key"
while i > 1 and A[PARENT(i)] < key
	A[i] = A[PARENT(i)]
	i = PARENT(i)
A[i] = key
\stopCLRS
\stopANSWER

%e6.5-7
\startEXERCISE
試說明如何使用優先隊列來實現一個先進先出隊列,以及如何使用優先隊列實現棧
(隊列和棧的定義見\refsection{stack_and_queue})。
\stopEXERCISE

\startANSWER
對於棧,新元素具有最高優先級,而對於隊列,新元素具有最小優先級。
對於棧,新元素優先級爲 \m{HEAP-MAXIMUM(A) + 1}。
而對於隊列,則需要跟蹤新元素,每次增加新元素時,都用更小的優先級。
但是兩者性能均不高。
而且如果優先級會上溢或下溢,我們需要重新指定優先級。
\stopANSWER

%e6.5-8
\startEXERCISE
\ALGO{HEAP-DELETE(A, i)} 能將節點 \m{i} 從堆 \m{A} 中刪除。
對於一個包含 \m{n} 個元素的堆,
請設計一個能在 \m{O(\lg{n})} 時間內完成的 \ALGO{HEAP-DELETE} 算法。
\stopEXERCISE

\startANSWER
\CLRSH{HEAP-DELETE(A, i)}
\startCLRS
A[i] = A[A.heap-size]
A.heap-size = A.heap-size - 1
MAX-HEAPIFY(A, i)
\stopCLRS
\stopANSWER

%e6.5-9
\startEXERCISE[exercise:k_merge]
請設計一個時間復雜度爲 \m{O(n\lg{k})} 的算法,
可以將 \m{k} 個有序鏈表合並爲一個有序鏈表,
這裏 \m{n} 是所有輸入鏈表包含的元素總數。
(\hint 用最小堆來完成 \m{k} 路歸並)
\stopEXERCISE

\startANSWER
從各鏈表中取一個元素,放入最小堆中。
對於每個元素,我們需要跟蹤是從哪個鏈表中取出來的。
歸並時,每從堆中取出最小的元素,則從此元素所屬鏈表中取一個元素放入堆中(除非這個鏈表爲空)。
一直重復此操作直到堆爲空。

一共需 \m{n} 步,每一步都包含一個堆的插入操作(時間爲 \m{\lg{k}})。
\stopANSWER

\stopsection
