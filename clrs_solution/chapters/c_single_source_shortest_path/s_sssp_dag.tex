\startsection[
  title={Single-source shortest paths in directed acyclic graphs},
]

%e24.2-1
\startEXERCISE
請在圖 24-5 上運行 \ALGO{DAG-SHORTEST-PATHS},使用節點 \m{r} 作爲源節點。
\stopEXERCISE

\startANSWER
\externalfigure[output/e24_2_1-6]
\stopANSWER

%e24.2-2
\startEXERCISE
假定將 \ALGO{DAG-SHORTEST-PATHS} 的第 3 行改爲:
\startCLRS
for the first |V| - 1 vertices, taken in topologically sorted order
\stopCLRS
證明:該算法的正確性保持不變。
\stopEXERCISE

\startANSWER
拓撲排序後,沒有從最後一個頂點出發的邊,因此不需要遍歷最後一個頂點。
\stopANSWER

%e24.2-3
\startEXERCISE
上面描述的 PERT 圖的公式有一點不大自然。
在一個更自然的結構下,
圖中的節點代表要執行的工作,
邊代表工作之間的次序限制,
即邊 \m{(u,v)} 表示工作 \m{u} 必須在工作 \m{v} 之前執行。
在這種結構的圖中,我們將權重賦給節點,而不是邊。
請修改 \ALGO{DAG-SHORTEST-PATHS} 過程,
使其可以在線性時間內找出這種有向無環圖一條最長的路徑。
\stopEXERCISE

\startANSWER
替換 \ALGO{DAG-SHORTEST-PATHS} 中的兩個函數:

\CLRSH{NEW-INITIALIZE-SINGLE-SOURCE(G,s,w)}
\startCLRS
for each vertex v in G.V
	v.d = infty
	v.pi = NIL
s.d = w(s)
\stopCLRS

\CLRSH{NEW-RELAX(u,v,w)}
\startCLRS
if v.d > u.d + w(v)
	v.d = u.d + w(v)
	v.pi = u
\stopCLRS
\stopANSWER

%e24.2-4
\startEXERCISE
給出一個有效算法計算有向無環圖中的路徑總數,
並分析其時間複雜度。
\stopEXERCISE

\startANSWER
\m{s.c = 1}, \m{v.c = \sum_{(u,v)\in E}u.c}。

時間複雜度爲 \m{\Theta(V+E)}。

\CLRSH{DAG-PATHS(G,s)}
\startCLRS
topologically sort the vertices of G

for each vertex v in G.V
	v.c = 0
s.c = 1

for each vertex u, taken in topologically sorted order
	for each vertex v in u.Adj
		v.c = v.c + u.c
\stopCLRS
\stopANSWER

\stopsection
