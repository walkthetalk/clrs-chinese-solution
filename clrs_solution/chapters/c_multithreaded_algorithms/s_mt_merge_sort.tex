\startsection[
  title={Multithreaded merge sort},
]

%e27.3-1
\startEXERCISE
試解釋如何加大 \ALGO{P-MERGE} 基礎情形的規模。
\stopEXERCISE

\startANSWER
將 \m{n_1} 與 0 比較改成與 1 比較。
\stopANSWER

%e27.3-2
\startEXERCISE
與 \ALGO{P-MERGE} 在較大數列中找一個中位數的方法不同,
使用\refexercise{9.3-8} 的結果,
請給出一個找出兩個有序數列中的所有元素的中位數的替代方法。
再給出使用這個中位數查找方法的一個有效的多線程歸併算法的僞碼。
分析該算法。
\stopEXERCISE

\startANSWER
\TODO{略。}
\stopANSWER

%e27.3-3
\startEXERCISE[exercise:27.3-3]
如\refsection{desc_quicksort} 中的 \ALGO{PARTITION},
請給出一個有效的多線程算法,
用劃分元劃分一個數列。
你不必原地劃分數列,但要儘可能地並行。
分析該算法。
(\hint 可能需要一個輔助數列,可能需要對輸入元素處理多趟。)
\stopEXERCISE

\startANSWER
\TODO{略。}
\stopANSWER

%e27.3-4
\startEXERCISE
給出一個多線程版本的 \ALGO{RECURSIVE-FFT} (參見\refsection{dft_fft}),
使其儘可能並行。並分析該算法。
\stopEXERCISE

\startANSWER
\TODO{略。}
\stopANSWER

%e27.3-5
\startEXERCISE\DIFFICULT
給出一個多線程版本的 \ALGO{RANDOMIZED-SELECT} (參見
\refsection{selection_in_expected_linear_time}),
使其儘可能並行。並分析該算法。(\hint 使用\refexercise{27.3-3} 中的劃分算法)
\stopEXERCISE

\startANSWER
\TODO{略。}
\stopANSWER

%e27.3-6
\startEXERCISE\DIFFICULT
如何實現\refsection{worstcase_linear_selection} 的多線程 \ALGO{SELECT} 算法,
使其儘可能並行。並分析該算法。
\stopEXERCISE

\startANSWER
\TODO{略。}
\stopANSWER

\stopsection
