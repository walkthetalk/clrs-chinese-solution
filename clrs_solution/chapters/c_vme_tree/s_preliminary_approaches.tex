\startsection[
  title={Preliminary approaches},
]

%e20.1-1
\startEXERCISE
修改本節中的數據結構,使其支持重複關鍵字。
\stopEXERCISE

\startANSWER
每一個節點不再是一個布爾量,而是一個整數,代表此關鍵字的個數。
\stopANSWER

%e20.1-2
\startEXERCISE
修改本節中的數據結構,使其支持帶有衛星數據的關鍵字。
\stopEXERCISE

\startANSWER
每一個節點改成一個指針,指向衛星數據。
如果指針爲空,則代表沒有此節點。
\stopANSWER

%e20.1-3
\startEXERCISE
使用本節的數據結構會發現,
查找 \m{x} 的後繼和前驅並不依賴於 \m{x} 當時是否包含在集合中。
當 \m{x} 不包含在樹中時,
試說明如何在一棵二叉搜索樹中查找 \m{x} 的後繼。
\stopEXERCISE

\startANSWER
沿 \m{x} 向上搜索,直到找到值爲 1 的節點,並且是通過其左孩子找到的,其右孩子也是 1。
然後沿這個節點的右子樹向下搜索,找最左邊的 1。
\stopANSWER

%e20.1-4
\startEXERCISE
假設不使用一棵疊加的度爲 \m{\sqrt{u}} 的樹,
而是使用一棵疊加的度爲 \m{u^{1/k}} 的樹,
這裏 \m{k} 是大於 1 的常數,
則這樣的一棵樹的高度是多少?
又每個操作將需要多長時間?
\stopEXERCISE

\startANSWER
高度是 \m{k},每個操作需要時間爲 \m{O(k u^{1/k})}。
\stopANSWER

\stopsection
