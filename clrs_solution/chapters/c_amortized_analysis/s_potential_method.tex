\startsection[
  title={The potential method},
]

%e17.3-1
\startEXERCISE
假定有勢函數 \m{\Phi},對所有 \m{i} 滿足 \m{\Phi(D_i)\ge\Phi(D_0)},但 \m{\Phi(D_0)\ne 0}。
證明:存在勢函數 \m{\Phi'},使得 \m{\Phi'(D_0)=0},
對所有 \m{i\ge 1} 滿足 \m{\Phi'(D_i)\ge 0},
且使用 \m{\Phi'} 的攤還代價與使用 \m{\Phi} 的攤還代價相同。
\stopEXERCISE

\startANSWER
令 \m{\Phi'(D_i)=\Phi(D_i)-\Phi(D_0)}。
\stopANSWER

%e17.3-2
\startEXERCISE
使用勢能法重做\refexercise{power_cost}。
\stopEXERCISE

\startANSWER
另勢能函數 \m{\Phi(D_0)=0},對於 \m{i>0}, \m{\Phi(D_i) = 2 i - 2^{1+\lfloor \lg i\rfloor}}。
如果 \m{i=1},則:
\startformula
\hat{c_i} = c_i + \Phi(D_i) - \Phi(D_{i-1}) = 1 + 2 - 2^{1+\lceil \lg 1\rceil} - 0 = 1
\stopformula

如果 \m{i>1},且 \m{i} 不是 2 的冪:
\startformula
\hat{c_i} = c_i + \Phi(D_i) - \Phi(D_{i-1}) = 1 + 2i - 2^{1+\lceil \lg i\rceil} - (2(i-1) - 2^{1+\lceil \lg (i-1)\rceil}) = 3
\stopformula

如果 \m{i>2^j}, \m{j\in \blackboard{N}}:
\startformula
\hat{c_i} = c_i + \Phi(D_i) - \Phi(D_{i-1}) = i + 2i - 2^{1+j} - (2(i-1) - 2^{1+j-1}) = i+2i-2i-2i+2+i = 2
\stopformula

因此,每個操作的攤還代價爲 3。
\stopANSWER

%e17.3-3
\startEXERCISE
考慮一個包含 \m{n} 個元素的普通二叉最小堆數據結構,
他支持 \ALGO{INSERT} 和 \ALGO{EXTRACT-MIN} 操作,
最壞情況時間均爲 \m{O(\lg n)}。
給出一個勢函數 \m{\Phi},使得 \ALGO{INSERT} 操作的攤還代價爲 \m{O(\lg n)},
而 \ALGO{EXTRACT-MIN} 操作的攤還代價爲 \m{O(1)},證明他是正確的。
\stopEXERCISE

\startANSWER
令勢能函數爲 \m{\Phi(D_i)=k\lg n},其中 \m{k} 爲堆中元素個數。則 \ALGO{INSERT} 的攤還代價爲:
\startformula
\hat{c_i} = c_i + \Phi(D_i) - \Phi(D_{i-1}) \le \lg n + (k+1)\lg n - k\lg n = 2\lg n \in O(\lg n)
\stopformula

\ALGO{EXTRACT-MIN} 的攤還代價爲:
\startformula
\hat{c_i} = c_i + \Phi(D_i) - \Phi(D_{i-1}) \le \lg n + (k-1)\lg n - k\lg n = 0 \in O(1)
\stopformula
\stopANSWER

%e17.3-4
\startEXERCISE
執行 \m{n} 個 \ALGO{PUSH}、 \ALGO{POP} 和 \ALGO{MULTIPOP} 操作的總代價是多少?
假定初始時棧中包含 \m{s_0} 個對象,結束後包含 \m{s_n} 個對象。
\stopEXERCISE

\startANSWER
\startformula\startmathalignment
\NC \sum_{i=1}^{n}c_i \NC = \sum_{i=1}^{n}\hat{c_i} + \Phi(D_i) - \Phi(D_{i-1}) \NR
\NC \NC = n - s_n + s_0 \NR
\stopmathalignment\stopformula
\stopANSWER

%e17.3-5
\startEXERCISE
假定計數器初值不是 0,而是包含 \m{b} 個 1 的二進制數。
證明:若 \m{n=\Omega(b)},則執行 \m{n} 個 \ALGO{INCREMENT} 操作的代價爲 \m{O(n)}。
(不要假定 \m{b} 是常量)
\stopEXERCISE

\startANSWER
\startformula\startmathalignment
\NC \sum_{i=1}^{n}c_i \NC = \sum_{i=1}^{n}\hat{c_i} + \Phi(D_i) - \Phi(D_{i-1}) \NR
\NC \NC = n - x + b \NR
\NC \NC \le n - x + n \NR
\NC \NC = O(n) \NR
\stopmathalignment\stopformula
\stopANSWER

%e17.3-6
\startEXERCISE
證明:如何用兩個普通的棧實現一個隊列(\refexercise{two_stack_to_queue}),
使得每個 \ALGO{ENQUEUE} 和 \ALGO{DEQUEUE} 操作的攤還代價爲 \m{O(1)}。
\stopEXERCISE

\startANSWER
令兩個棧分別爲 D 和 E,大小均爲 \m{n}。 E 用於入隊, D 用於出隊。

\CLRSH{ENQUEUE(E,x)}
\startCLRS
PUSH(E,x)
\stopCLRS

\CLRSH{DEQUEUE(E,D)}
\startCLRS
if D is empty
	do
		x = POP(E)
		if E is empty
			return x
		PUSH(D, x)
	while true
else
	return POP(D)
\stopCLRS

\ALGO{ENQUEUE} 的攤還代價爲 4, \ALGO{DEQUEUE} 的攤還代價爲 0。
 \m{k} 個入隊操作剩餘信用爲 \m{3k}。
第一個出隊操作會用掉 \m{2k} 的信用。
後續每個出隊操作使用 1 個信用。
\stopANSWER

%e17.3-7
\startEXERCISE
爲動態整數多重集 \m{S} (允許包含重複值)設計一種數據結構,支持如下兩種操作:
\startigBase[2]
\item \ALGO{INSERT(S,x)} 將 \m{x} 插入 \m{S} 中。
\item \ALGO{DELETE-LARGER-HALF(S)} 將最大的 \m{\lceil |S|/2\rceil} 個元素從 \m{S} 中刪除。
\stopigBase
解釋如何實現這種數據結構,
使得任意 \m{m} 個 \ALGO{INSERT} 和 \ALGO{DELETE-LARGER-HALF} 操作的序列能在 \m{O(m)} 時間內完成。
還要實現一個能在 \m{O(|S|)} 時間內輸出所有元素的操作。
\stopEXERCISE

\startANSWER
用數列來存儲元素。

\ALGO{INSERT}:將元素追加到數列中。

\ALGO{DELETE-LARGER-HALF}:在 \m{O(n)} 時間內找到中值,並刪除較大的 \m{\lceil |S|/2\rceil} 個元素。
\stopANSWER

\stopsection
