\startsection[
  title={Push-relabel algorithms},
]

%e26.4-1
\startEXERCISE
證明:算法 \ALGO{INITIALIZE-PREFLOW(G,s)} 終止後,
有 \m{s.e\le -|f^*|},其中 \m{f^*} 是流網絡 \m{G} 的一個最大流。
\stopEXERCISE

\startANSWER
\startformula
s.e = -\sum_{(s,v)\in E}c(s,v)
\stopformula
\stopANSWER

%e26.4-2
\startEXERCISE
說明如何實現通用的推送-重貼標籤算法,
使得每個重貼標籤的操作成本爲 \m{O(V)},
每個推送操作的成本爲 \m{O(1)},
並且可以在 \m{O(1)} 時間內選擇一個合適的操作,
從而使得整個算法運行時間爲 \m{O(V^2E)}。
\stopEXERCISE

\startANSWER
\m{s} 和 \m{t} 不會溢出,也就不會被重貼標籤。
 \m{t} 可以吸收無限多流,也就不會推送。

有邊 \m{(u,v)\in E_f}, \m{u.h = v.h + 1}。
則可以進行推送,一旦調用 \ALGO{PUSH(u,v)},若 \m{u.e = 0},則爲非飽和推送,
在下一次迭代中我們不會對 \m{u} 進行推送或重貼標籤,因爲 \m{u} 不再溢出。
而如果推送後 \m{u.e > 0},則爲飽和推送,
在下一次迭代中我們會對 \m{u} 進行推送或重貼標籤。
除此之外, \m{v.e} 一定大於零,
因爲 \m{v.e} 一直非負,在執行 \ALGO{PUSH(u,v)} 時還會增大。
至於 \m{v},是推送,還是重貼標籤,就很難判斷了。

令 \m{G} 和 \m{G_f} 分別代表流和殘存網絡。
在 \m{G_f} 中,對於每個節點 \m{u\in V_f},
令 \m{u.r} 表示 \m{G_f} 中 \m{u} 的一個鄰接點,即 \m{(u,u.r)\in E_f},
且 \m{u.h = u.r.h + 1}。
如果不存在這樣的節點,則 \m{u.r = NIL},
 \ALGO{RELABEL(u)} 會正確設置 \m{u.r}。
除此之外,我們用兩個鏈表實現通用的推送重貼標籤算法。

\m{L_1} 是待處理鏈表,對於所有 \m{u\in L_1},都有 \m{u.e\le 0}。
根據之前的討論,在下一次迭代中,我們不會對這些節點進行推送或重貼標籤,
因爲他們沒有溢出。

\m{L_2} 是推送重貼標籤鏈表,對於所有 \m{u\in L_2},都有 \m{u.e > 0}。
即這些節點全部溢出,我們會對其進行推送或重貼標籤。
如果 \m{L_2} 爲空,則算法終止。

初始化的時候進行如下操作。
在執行完 \ALGO{INITIALIZE-PREFLOW} 後,對於所有節點 \m{u\in V},
如果 \m{u.e\le 0},則將其移入 \m{L_1},否則將其移入 \m{L_2}。

從 \m{L_2} 中取出一個節點 \m{u} 後,
根據 \m{u.r} 的值進行不同的處理。
如果 \m{u.r} 是 \m{NIL},則重貼標籤,然後插入到 \m{L_2} 的最前端。
否則令 \m{v = u.r},
先檢查 \m{u.h = v.h + 1} 是否成立,
如果不成立,還是重貼標籤,然後插入到 \m{L_2} 的最前端,
否則調用 \ALGO{PUSH(u,v)}。
調用 \ALGO{PUSH(u,v)} 之後,如果 \m{u.e = 0},
則將 \m{u} 移入 \m{L_1};
如果 \m{u.e > 0},則先將 \m{u.h} 重置爲 \m{NIL},
然後將其插入到 \m{L_2} 末尾。
但 \m{v.e} 還是正值,如果下一次迭代中可以推送 \m{v},就必須推送。
幸運的是,通過跟蹤 \m{v.r},很容易做到這一點。
只需在執行完 \ALGO{PUSH(u,v)} 後,將 \m{v} 移到 \m{L_2} 的最前端,
下一次迭代時,算法會自動判斷。

我們知道每個重貼標籤操作的代價是 \m{O(V)},
每個推送操作的代價是 \m{O(1)},
選擇操作需要 \m{O(1)}。

然而我們還有點不放心:
(1)調用完 \ALGO{PUSH(u,v)} 後,
如果 \m{u.e=0},則會將其移到正確位置。
而如果 \m{u.e>0},我們將其移到了 \m{L_2} 尾端,
如果還可以繼續推送 \m{u} (但沒有立刻推送),會出現什麼情況?
最壞情況下,調用完 \ALGO{PUSH(u,v)} 後(正好是一個飽和推送),
我們會額外執行一個重貼標籤操作。
根據引理 26.22,最多會有 \m{2|V||E|} 個飽和推送,
因此總代價最多爲 \m{2|V||E| \times O(V) = O(V^2 E)}。
(2)從 \m{L_2} 中取出一個節點 \m{u},
如果 \m{u.r = v},且 \m{u.h \ne v.h+1},
則從 \m{v} 到 \m{u} 迴流,
即 \m{u} 仍然溢出,不能再從 \m{u} 推送到 \m{v}。
對於每條邊 \m{(u,v)\in E_f},
這種情況最多有 \m{2|V|} 次。
 \m{u.r=v} 意味着之前進行過一次非飽和推送 \m{(u,v)},
即 \m{u.h = v.h+1}。
但現在 \m{u.h\ne v.h + 1},即 \m{u.h\le v.h}, \m{v.h} 增大了。
下一次執行 \ALGO{PUSH(u,v)} 時會發現仍然有 \m{u.h=v.h + 1},
即 \m{u.h} 又增大了。
因此對於每條邊 \m{(u,v)\in E_f},
最多有 \m{2|V|} 次重貼標籤操作。
總代價爲 \m{|E|\times (2|V|) \times O(V)=O(V^2 E)}。
\stopANSWER

%e26.4-3
\startEXERCISE
證明:在通用推送重貼標籤算法中,
所有 \m{O(V^2)} 個重貼標籤操作,總共只用了 \m{O(VE)} 的時間。
\stopEXERCISE

\startANSWER
每次調用 \ALGO{RELABEL(u)},我們會檢查所有邊 \m{(u,v)\in E_f}。
對於每個節點而言,最多重貼標籤 \m{2|V|-1} 次,
而對於每條邊 \m{(u,v)},在重貼標籤過程中,
最多會檢查 \m{4|V|-2} 次,
其中一半用於 \m{u},另一半用於 \m{v}。
殘存網絡中最多有 \m{2|E|=O(E)} 條邊,
因此重貼標籤總時間爲 \m{O(VE)}。
\stopANSWER

%e26.4-4
\startEXERCISE
假定使用推送重貼標籤算法找到了流網絡 \m{G=(V,E)} 的一個最大流,
如何快速找到 \m{G} 的一個最小切割。
\stopEXERCISE

\startANSWER
可以在 \m{O(V)} 時間內做到。

首先,找到 \m{\hat{h}},滿足 \m{0<\hat{h}<|V|},
且算法終止時沒有節點高度爲 \m{\hat{h}}。

由於 \m{s.h = |V|},且 \m{t.h = 0},
我們僅需考慮 \m{|V|-2} 個節點。
也就是說 \m{\hat{h}} 的值有 \m{|V|-1} 種可能,
我們知道在 \m{1,2,\ldots,|V|-1} 中,至少有一個值不是任何節點的高。
因此如何選取 \m{\hat{h}},就很明確了,
很容易通過一個布爾數列,在 \m{O(V)} 時間內就可以找到 \m{\hat{h}}。

令 \m{S=\{u\in V: u.h > \hat{h}\}}, \m{T=\{v\in V: v.h<\hat{h}\}}。
由於 \m{s.h=|V|>\hat{h}},所以 \m{s\in S},
又由於 \m{t.h=0<\hat{h}},所以 \m{t\in T},
這也符合切割的要求。

我們需要使得對於所有 \m{u\in S} 和 \m{v\in T}, \m{f(u,v)=c(u,v)}。
一旦如此,則 \m{f(S,T)=c(S,T)},
根據推論 26.6, \m{(S,T)} 就是最小切割。

爲了反證,設存在節點 \m{u\in S} 和 \m{v\in T}, \m{(u,v)\in E_f}。
因爲 \m{h} 始終是高度函數(引理 26.17),有 \m{u.h\le v.h + 1}。
但同時 \m{v.h < \hat{h} < u.h},
又因爲所有高度都是整數, \m{v.h\le u.h - 2}。
因此, \m{u.h\le v.h + 1\le u.h - 2 + 1 = u.h - 1},
即 \m{0\le -1},顯然不成立。
因此 \m{(S,T)} 就是最小切割。

\stopANSWER

%e26.4-5
\startEXERCISE
給出一個有效的推送重貼標籤算法,
使其可以在一個二分圖中找到一個最大匹配,
並分析其效率。
\stopEXERCISE

\startANSWER
\TODO{略。}
\stopANSWER

%e26.4-6
\startEXERCISE
假定在流網絡 \m{G=(V,E)} 中所有邊的容量都在集合 \m{\{1,2,\ldots,k\}} 中。
分析通用推送重貼標籤算法的運行時間,
請以 \m{|V|}、 \m{|E|} 和 \m{k} 來予以表示。
(\hint 每條邊在變爲飽和之前可以支持多少次非飽和推送操作)
\stopEXERCISE

\startANSWER
\TODO{略。}
\stopANSWER

%e26.4-7
\startEXERCISE
證明:我們可以將 \ALGO{INITIALIZE-PREFLOW} 的第 6 行改爲:
\startformula
s.h = |G.V| - 2
\stopformula
而不會影響通用推送重貼標籤算法的正確性和漸進性能。
\stopEXERCISE

\startANSWER
如果設置 \m{s.h = |V|-2},則需要修改高度函數的定義。

要證明正確性,只需要更新引理 26.18 的證明。
原始證明推導出的是 \m{s.h\le k < |V|} 與 \m{s.h=|V|} 矛盾。
改成 \m{s.h=|V|-2} 之後,這個矛盾不存在了。

在原始證明中,假定有一簡單增廣路徑 \m{\langle v_0,v_1,\ldots,v_k\rangle},
其中 \m{v_0=s}, \m{v_k = t},因此 \m{k<|V|}。
 \m{(s,v_1)} 怎樣才能成爲殘存邊?
此邊在 \ALGO{INITIALIZE-PREFLOW} 中已經飽和了,
這意味着曾經有從 \m{v_1} 推送到 \m{s} 的操作。
爲了做到這一點,必須滿足 \m{v_1.h = s.h + 1}。
如果我們設置了 \m{s.h=|V|-2},
也就意味着 \m{v_1.h} 是 \m{|V|-1}。
在那之後, \m{v_1.h} 不再減小,因此 \m{v_1.h\ge |V|-1}。
回溯增廣路徑,有 \m{v_{k-i}.h\le t.h + i},
對於 \m{i=0,1,\ldots,k} 均成立。
回顧假設,由於增廣路徑是簡單路徑, \m{k<|V|}。
令 \m{i=k-1},有 \m{v_1.h\le t.h+k-1 < 0+|V|-1}。
現在就有矛盾了, \m{v_1.h \ge |V|-1} 和 \m{v_1.h < |V|-1}。
也就是說,引理 26.18 依然成立。

漸進分析沒什麼變化。
\stopANSWER

%e26.4-8
\startEXERCISE
設 \m{\delta_{f}(u,v)} 爲殘存網絡 \m{G_f} 中從節點 \m{u} 到節點 \m{v} 的距離(邊的條數)。
證明: \ALGO{GENERIC-PUSH-RELABEL} 維持性質 \m{u.h < |V|} 意味着 \m{u.h\le \delta_{f}(u,t)},
維持性質 \m{u.h \ge |V|} 意味着 \m{u.h-|V|\le \delta_{f}(u,s)}。
\stopEXERCISE

\startANSWER
\TODO{略。}
\stopANSWER

%e26.4-9
\startEXERCISE\DIFFICULT
如前一個練習,設 \m{\delta_{f}(u,v)} 爲殘存網絡 \m{G_f} 中從節點 \m{u} 到節點 \m{v} 的距離。
請說明如何修改通用推送重貼標籤算法,
使其維持性質 \m{u.h<|V|} 意味着 \m{u.h=\delta_{f}(u,t)},
維持性質 \m{u.h\ge |V|} 意味着 \m{u.h-|V|=\delta_{f}(u,s)}。
你所設計的算法中,維持該性質所用總時間應爲 \m{O(VE)}。
\stopEXERCISE

\startANSWER
\TODO{略。}
\stopANSWER

%e26.4-10
\startEXERCISE\DIFFICULT
證明:在流網絡 \m{G=(V,E)} 上運行 \ALGO{GENERIC-PUSH-RELABEL},
非飽和推送的次數爲 \m{4|V|^{2}|E|},假定 \m{|V|\ge 4}。
\stopEXERCISE

\startANSWER
\TODO{略。}
\stopANSWER

\stopsection
