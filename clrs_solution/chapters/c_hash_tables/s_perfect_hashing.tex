\startsection[
  title={Perfect hashing},
  reference=section:perfect_hashing,
]

%e11.5-1
\startEXERCISE
假設用開放尋址法和均勻散列技術將 n 個關鍵字插入到一個大小爲 m 的散列表中。
設 \m{p(n,m)} 爲沒有衝突發生的概率。
試證明: \m{p(n,m)\le e^{-n(n-1)/2m}}。
(\hint 式(3.12))
論證當 n 超過 \m{\sqrt{m}} 時,不發生衝突的概率快速趨於 0。

附式 3.12: \m{e^x \ge 1 + x}。
\stopEXERCISE

\startANSWER
發生衝突的概率期望值爲 \m{\binom{n}{2}\frac{1}{m}}。
所以沒有衝突發生的概率爲:
\startformula
p(n,m)
  = 1 - \binom{n}{2}\frac{1}{m}
  = 1 - \frac{n(n-1)}{2m}
  = 1 + (- \frac{n(n-1)}{2m})
  \le e^{-n(n-1)/2m}
\stopformula

\TODO{如何證明?}
\stopANSWER

\stopsection
