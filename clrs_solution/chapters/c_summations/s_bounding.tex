\startsection[
  title={Bounding summations},
]

%eA.2-1
\startEXERCISE
證明: \m{\sum_{k=1}^{n}1/k^2} 有常數上界。
\stopEXERCISE

\startANSWER
\startformula\startmathalignment
\NC \sum_{k=1}^{n}\frac{1}{k^2} \NC = 1 + \sum_{k=2}^{n}\frac{1}{k^2} \NR
\NC \NC \le 1 + \int_{1}^{n}\frac{dx}{x^2} \NR
\NC \NC = 1 + \left[-\frac{1}{x}\right]_{1}^{n} \NR
\NC \NC = 1 + \left[-\frac{1}{n} - \left(-\frac{1}{1}\right)\right] \NR
\NC \NC = 2 - \frac{1}{n} \NR
\NC \NC \le 2 \NR
\stopmathalignment\stopformula
將第一項分離出來,是爲了防止積分時除零。
\stopANSWER

%eA.2-2
\startEXERCISE
求下面和式的漸進上界:
\startformula
\sum_{k=0}^{\left\lfloor\lg n\right\rfloor}\left\lceil \frac{n}{2^k}\right\rceil
\stopformula
\stopEXERCISE

\startANSWER
\startformula\startmathalignment
\NC \sum_{k=0}^{\left\lfloor\lg n\right\rfloor}\left\lceil \frac{n}{2^k}\right\rceil
    \NC \le \sum_{k=0}^{\left\lfloor\lg n\right\rfloor} \left(\frac{n}{2^k} + 1\right)\NR
\NC \NC \le \lg n + 1 + \sum_{k=0}^{\infty} \frac{n}{2^k} \NR
\NC \NC = \lg n + 1 + n\cdot \frac{1}{1-\frac{1}{2}} \NR
\NC \NC = \lg n + 1 + 2n \NR
\NC \NC = O(n) \NR
\stopmathalignment\stopformula
\stopANSWER

%eA.2-3
\startEXERCISE
通過分割求和的方式證明第 \m{n} 個調諧數是 \m{\Omega(\lg n)}。
\stopEXERCISE

\startANSWER
\TODO{略。}
\stopANSWER

%eA.2-4
\startEXERCISE
用積分法求 \m{\sum_{k=1}^{n}k^3} 的近似值。
\stopEXERCISE

\startANSWER
\m{k^3} 單調遞增:
\startformula\startmathalignment[n=3]
\NC \int_{0}^{n}x^3 \NC \le \sum_{k=1}^{n} k^3 \NC \le \int_{1}^{n+1} x^3 \NR
\NC \left[\frac{x^4}{4}\right]_0^n \NC \le \sum_{k=1}^{n} k^3
     \NC \le \left[\frac{x^4}{4}\right]_1^{n+1} \NR
\NC \frac{n^4}{4} - 0 \NC \le \sum_{k=1}^{n}k^3 \NC \le \frac{(n+1)^4}{4} - \frac{1}{4} \NR
\NC \frac{n^4}{4} \NC \le \sum_{k=1}^{n}k^3 \NC \le \frac{(n+1)^4 - 1}{4} \NR
\stopmathalignment\stopformula
\stopANSWER

%eA.2-5
\startEXERCISE
爲什麼我們不在 \m{\sum_{k=1}^{n}1/k} 上使用積分近似(公式 A.12)來
獲得第 \m{n} 個調諧數的上界?
\stopEXERCISE

\startANSWER
防止除零。
\stopANSWER

\stopsection
