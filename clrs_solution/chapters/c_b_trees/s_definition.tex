\startsection[
  title={Definition of B-trees},
]
%e18.1-1
\startEXERCISE
爲什麼不允許最小度數 \m{t=1}?
\stopEXERCISE

\startANSWER
根據定義,最小度數 \m{t} 意味着所有節點(根節點除外)都至少含有 \m{t-1} 個關鍵字。
從而使得根節點以外的內部節點都至少有 \m{t} 個孩子節點。
所以,如果 \m{t=1},則意味着根節點之外的節點最少可以只有 \m{t-1=0} 個關鍵字,
根節點之外的內部節點最少可以只有 \m{t=1} 個孩子節點。

因此,最小度數不能是 1。
\stopANSWER

%e18.1-2
\startEXERCISE
當 \m{t} 取何值時,圖 18-1 所示的樹是一棵合法的 B 樹?
\stopEXERCISE

\startANSWER
根節點之外的節點至少要有 \m{t-1} 個關鍵字,最多有 \m{2t-1} 個關鍵字。
根據圖 18-1,有 \m{t-1\le 2} 且 \m{3\le 2t-1},求解得 \m{2\le \le 3},
因此 \m{t} 可以是 2 或 3。
\stopANSWER

%e18.1-3
\startEXERCISE
請給出表示 \m{\{1,2,3,4,5\}} 的最小度數爲 2 的所有合法 B 樹。
\stopEXERCISE

\startANSWER
略。
\stopANSWER

%e18.1-4
\startEXERCISE
一棵高度爲 \m{h} 的 B 樹中,可以存儲最多多少個關鍵字?
用最小度數 \m{t} 的函數表示。
\stopEXERCISE

\startANSWER
每個節點最多包含 \m{2t-1} 個關鍵字。每個節點最多有 \m{2t} 個孩子。
\startformula\startmathalignment
\NC \NC (2t-1)[(2t)^0 + (2t)^1 + \ldots + (2t)^h] \NR
\NC = \NC (2t-1)\sum_{i=0}^{h}(2t)^i \NR
\NC = \NC (2t-1)\frac{(2t)^{h+1} - 1}{2t-1} \NR
\NC = \NC (2t)^{h+1} - 1 \NR
\stopmathalignment\stopformula
\stopANSWER

%18.1-5
\startEXERCISE
如果紅黑樹中每個黑節點吸收他的紅色孩子,
並把他們的孩子併入作爲自己的孩子,
描述這個結果的數據結構。
\stopEXERCISE

\startANSWER
對於黑色節點而言,他可能有一個或兩個紅孩子,也可能沒有,合併後,此節點將有 1、 2 或 3 個關鍵字。
此後所有葉子節點將具有同樣的高度(根據紅黑樹性質 5:從一個節點到所有其後代葉子節點路徑上的黑色節點數目相同)。
因此,這棵紅黑樹變成了一棵 \m{t=2} 的 B 樹,即 2-3-4 樹。
\stopANSWER

\stopsection
