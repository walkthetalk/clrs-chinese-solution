\startsubject[
  title={Problems},
]

%pD-1
\startPROBLEM
(Vandermonde matrix)
給定數值 \m{x_0,x_1,\ldots,x_{n-1}},
證明 Vandermonde 矩陣的行列式
\startformula
V(x_0,x_1,\ldots,x_n)=
\left[\startmatrix
\NC 1 \NC x_0 \NC x_0^2 \NC \cdots \NC x_0^{n-1} \NR
\NC 1 \NC x_1 \NC x_1^2 \NC \cdots \NC x_1^{n-1} \NR
\NC \vdots \NC \vdots \NC \vdots \NC \ddots \NC \vdots \NR
\NC 1 \NC x_{n-1} \NC x_{n-1}^2 \NC \cdots \NC x_{n-1}^{n-1} \NR
\stopmatrix\right]
\stopformula
是
\startformula
\det(V(x_0,x_1,\ldots,x_{n-1}) = \prod_{0\le j < k\le n-1} (x_k - x_j)
\stopformula
(\hint 對於 \m{i=n-1,n-2,\ldots,1},
將第 \m{i} 列乘以 \m{-x_0} 以後加到第 \m{i+1} 列上,
然後使用歸納法。)
\stopPROBLEM

\startANSWER
\TODO{略。}
\stopANSWER

%pD-2
\startPROBLEM
(Permutations defined by matrix-vector multiplication over GF(2))
利用 \m{GF(2)} 上的矩陣乘法可以定義一類集合 \m{S_n=\{0,1,2,\ldots,2^n-1]\}} 中整數的排列。
對於 \m{S_n} 中每個整數,可以將他的二進制表示形式看作一個 \m{n} 位向量
\startformula
\left[\startmatrix
\NC x_0 \NR
\NC x_1 \NR
\NC x_2 \NR
\NC \vdots \NR
\NC x_{n-1} \NR
\stopmatrix\right]
\stopformula
其中 \m{\sum_{i=0}^{n-1} x_i 2^i}。
如果 \m{A} 是一個元素均爲 0 或 1 的 \m{n\times n} 矩陣,
則我們可以定義一個排列。
該排列將 \m{S_n} 中的每一個值 \m{x} 映射到一個數上,
該數的二進制表示形式爲矩陣——向量積 \m{Ax}。
這裏,我們按照 \m{GF(2)} 執行所有算數運算:
所有的值爲 0 或 1,
並且除特例 \m{1+1=0} 外,
其他常規加法、乘法規則均適用。
讀者可以認爲 \m{GF(2)} 算數運算除了只使用最低有效位,
其他均與常規整數算術運算一致。

例如,對於 \m{S_2=\{0,1,2,3\}},矩陣
\startformula
A=\left[\startmatrix
\NC 1 \NC 0 \NR
\NC 1 \NC 1 \NR
\stopmatrix\right]
\stopformula
定義了如下排列 \m{\pi_A: \pi_A(0)=0,\pi_A(1)=3,\pi_A(2)=2,\pi_A(3)=1}。
下面解釋 \m{\pi_A(3)=1} 的理由,在 \m{GF(2)} 中
\startformula
\pi_A(3)=
\left[\startmatrix
\NC 1 \NC 0 \NR
\NC 1 \NC 1 \NR
\stopmatrix\right]
\left[\startmatrix
\NC 1 \NC 1 \NR
\stopmatrix\right]
\left[\startmatrix
\NC 1\cdot 1 + 0\cdot 1 \NR
\NC 1\cdot 1 + 1\cdot 1 \NR
\stopmatrix\right]
\left[\startmatrix
\NC 1 \NR
\NC 0 \NR
\stopmatrix\right]
\stopformula
就是 1 的二進制表示。

我們繼續在 \m{GF(2)} 上討論本問題,
並且所有矩陣和向量的元素均爲 0 或 1。
定義 0-1 矩陣(勻速均爲 0 或 1 的矩陣)在 \m{GF(2)} 上的秩與普通矩陣一致,
但是所有的決定線性相關的算術運算都按 \m{GF(2)} 進行。
定義 \m{n\times n} 0-1 矩陣 \m{A} 的取值範圍爲
\startformula
R(A) = \{y: y=Ax,x\in S_n\}
\stopformula
這樣, \m{R(A)} 是 \m{S_n} 中一類數的集合,
這類數可通過爲 \m{S_n} 中每個值都 \m{x} 乘以 \m{A} 得到。
\startigBase[a]\startitem
如果 \m{r} 是矩陣 \m{A} 的秩,證明 \m{|R(A)|=2^4}。
證明:當且僅當 \m{A} 是滿秩的, \m{A} 才可定義 \m{S_n} 上的一個排列。
\stopitem\stopigBase

\startANSWER
\TODO{略。}
\stopANSWER

對於一個給定的 \m{\times n} 矩陣 \m{A} 和一個給定的值 \m{y\in R(A)},
定義 \m{y} 的{\EMP preimage}爲
\startformula
P(A,y) = \{x:Ax=y\}
\stopformula
從而, \m{P(A,y)} 即爲 \m{S_n} 中的集合,這些集合乘以 \m{A} 後會映射到 \m{y}。

\startigBase[continue]\startitem
如果 \m{r} 是 \m{n\times n} 矩陣 \m{A} 的秩且 \m{y\in R(A)},
證明 \m{|P(A,y)| = 2^{n-r}}。
\stopitem\stopigBase

\startANSWER
\TODO{略。}
\stopANSWER

令 \m{0\le m\le n},假定將集合 \m{S_n} 劃分成相鄰數字的塊,
其中第 \m{i} 個塊包含 \m{2^m} 個數 \m{i2^m,\allowbreak
i2^m+1,\allowbreak
i2^m+2,\allowbreak
\ldots,\allowbreak
(i+1)2^m-1}。
對於任意子集 \m{S\subseteq S_n},
定義 \m{B(S,m)} 爲 \m{S_n} 中的塊的集合,
這些塊的大小爲 \m{2^m},且包含 \m{S} 中的元素。
例如,當 \m{n=3,m=1},且 \m{S=\{1,4,5\}} 時,
 \m{B(S,m)} 包含塊 0 (因爲在第 0 塊中)
和塊 2 (因爲 4 和 5 均在塊 2 中)。

\startigBase[continue]\startitem
令 \m{A} 的左下部 \m{(n-m)\times m} 子矩陣的秩爲 \m{r},
即通過取矩陣 \m{A} 底部 \m{n-m} 行和最左端 \m{m} 列的交獲得的矩陣。
令 \m{S} 是 \m{S_n} 中任意大小爲 \m{2^m} 的塊,
且令 \m{S'=\{y:y=Ax,\exists x\in S\}}。
證明: \m{|B(S',m)|=2^r} 且對於 \m{B(S',m)} 中的每一個塊,
有且僅有 \m{2^{m-r}} 個 \m{S} 中的數映射到該塊上。
\stopitem\stopigBase

\startANSWER
\TODO{略。}
\stopANSWER

將零向量乘以任意矩陣得到的都是零向量,
所以通過 \m{GF(2)} 上乘以滿秩 \m{n\times n} 0-1 矩陣
所定義的 \m{S_n} 的排列集合不能囊括 \m{S_n} 的所有排列。
這裏,對由矩陣——向量乘法定義的那類排列進行擴展,以包含一個附加項,
從而將 \m{x\in S_n} 映射到 \m{Ax+c} 上,
其中 \m{c} 是 \m{n} 位向量,加法按 \m{GF(2)} 執行。
例如,當
\startformulas
\startformula
A=\left[\startmatrix
\NC 1 \NC 0 \NR
\NC 1 \NC 1 \NR
\stopmatrix\right]
\stopformula
\startformula
c=\left[\startmatrix
\NC 0 \NR
\NC 1 \NR
\stopmatrix\right]
\stopformula
\stopformulas
我們可以獲得如下排列 \m{\pi_{A,c}}:
\startformulas
\startformula\startmathalignment[n=1]
\NC \pi_{A,c}(0) = 2 \NR
\NC \pi_{A,c}(1) = 1 \NR
\stopmathalignment\stopformula
\startformula\startmathalignment[n=1]
\NC \pi_{A,c}(2) = 0 \NR
\NC \pi_{A,c}(3) = 3 \NR
\stopmathalignment\stopformula
\stopformulas
對於某個 \m{n\times n} 0-1 滿秩矩陣 \m{A} 和 某個 \m{n} 位向量 \m{c},
如果一個排列可以將 \m{x\in S_n} 映射到 \m{Ax+c},則稱其爲 {\EMP linear permutation}。

\startigBase[continue]\startitem
用計數觀點證明:
 \m{S_n} 線性排列的數目遠小於 \m{S_n} 排列的數目。
\stopitem\stopigBase

\startANSWER
\TODO{略。}
\stopANSWER

\startigBase[continue]\startitem
請給出一個 \m{S_n} 的排列的例子及 \m{n} 的值,
其中該排列不能通過任何線性排列獲得。
(\hint 對於一個給定的排列,考慮矩陣在乘以單位向量後,與矩陣原有列的關係。)
\stopitem\stopigBase

\startANSWER
\TODO{略。}
\stopANSWER
\stopPROBLEM

\stopsubject%Problems
