\startsection[
  title={Indicator random variables},
]

\startEXERCISE
在 \ALGO{HIRE-ASSISTANT} 中,假設應聘者以隨機順序出現,
你只僱傭一次的概率是多少?正好僱傭 \m{n} 次的概率是多少?
\stopEXERCISE
\startANSWER
當最好的應聘者出現在第一個時,我們只需僱傭一次,其概率爲 \m{1/n} (一共有 \m{n!} 種全排列,而最好的出現在第一個的全排列共 \m{(n-1)!} 種)。
而正好僱傭 \m{n} 次,則需應聘者必須以增序出現,概率爲 \m{1/n!}。
\stopANSWER

\startEXERCISE
在 \ALGO{HIRE-ASSISTANT} 中,假設應聘者以隨機順序出現,
你正好僱傭兩次的概率是多少?
\stopEXERCISE
\startANSWER
如果正好僱傭了兩次,那麼假設第一個僱傭的人級別爲 \m{i},
第二次僱傭的人級別肯定爲 \m{n},且 \m{i < n}。

比第一個僱傭的人級別高的應聘者有 \m{n-i} 個,
這些人中級別爲 \m{n} 的那個排在最前面,其概率爲 \m{1/(n-i)}
(我們可以忽略其他應聘者,他們不影響概率)。
因此在第一個應聘者級別爲 \m{i},則正好僱傭兩次的概率爲:
\startformula
\Pr{T_i} = \frac{1}{n}\frac{1}{n-i}
\stopformula
其中第一部分反映的是級別爲 \m{i} 的應聘者排在第一個的概率。

正好僱傭兩次的概率爲:
\startformula\startmathalignment
\NC \Pr{T} \NC= \sum_{i=1}^{n-1}\Pr{T_i} \NR
\NC        \NC= \sum_{i=1}^{n-1}\frac{1}{n}\frac{1}{n-i} \NR
\NC        \NC= \frac{1}{n} \sum_{i=1}^{n-1}\frac{1}{i} \NR
\NC        \NC= \frac{1}{n} (\lg(n-1) + O(1)) \NR
\stopmathalignment\stopformula
\stopANSWER

\startEXERCISE
利用指示器隨機變量計算擲 \m{n} 個骰子之和的期望值。
\stopEXERCISE
\startANSWER
如果是單個骰子,期望值 \m{X_i} 爲:
\startformula\startmathalignment
\NC E[X_k] \NC= \sum_{i=0}^6 i \Pr{X_k = i} \NR
\NC        \NC= \frac{1 + 2 + 3 + 4 + 5 + 6}{6} \NR
\NC        \NC= \frac{21}{6} \NR
\NC        \NC= 3.5 \NR
\stopmathalignment\stopformula

對於多個骰子,則:
\startformula\startmathalignment
\NC E[X] \NC= E[\sum_{i=1}^nX_i] \NR
\NC      \NC= \sum_{i=1}^n E[X_i] \NR
\NC      \NC= \sum_{i=1}^n 3.5 \NR
\NC      \NC= 3.5 \cdot n \NR
\stopmathalignment\stopformula
\stopANSWER

\startEXERCISE
利用指示器隨機變量來求解如下{\EMP 帽子核對問題}(hat-check problem):
\m{n} 位顧客,每個人都將自己的帽子交給服務生。
服務生以隨機次序將帽子歸還給顧客。
請問拿到自己帽子的顧客數目的期望值是多少?
\stopEXERCISE
\startANSWER
每位顧客拿回自己帽子的概率都是 \m{1/n}。
記 \m{X_i} 爲第 \m{i} 個顧客拿回自己帽子的事件。
因此:
\startformula
E[X] = E[X_1 + X_2 + \ldots + X_n]
         = \sum_{i=1}^n E[X_i]
         = \sum_{i=1}^n \frac{1}{n}
         = 1
\stopformula
\stopANSWER

\startEXERCISE
設 \m{A[1..n]} 是由 \m{n} 個不同的數構成的數列。
如果 \m{i < j} 且 \m{A[i] > A[j]},則稱 \m{(i,j)} 爲 \m{A} 的一個{\EMP 逆序對}(inversion)。
(參見\refproblem{inversion} 中更多關於逆序對的例子)
假設 \m{A} 的元素構成 \m{\langle1,2,\ldots,n\rangle} 上的一個均勻隨機排列,
請用指示器隨機變量來計算其中逆序對的數目期望值。
\stopEXERCISE
\startANSWER
記 \m{X_{ij}} 爲 \m{i} 和 \m{j} 爲逆序對的情況。
每個逆序對發生的概率爲 \m{1/2},
因爲有 \m{\binom{n}{2}} 種可能,而大的那個在前面的概率爲 \m{\frac{1}{n(n-1)}}。
兩式相乘得 \m{1/2}。由此可得 \m{E[X_{ij}] = 1/2}。
\startformula\startmathalignment
\NC E[X] \NC= \sum_{i=1}^{n-1} \sum_{j=i+1}^n E[X_{ij}] \NR
\NC      \NC= \sum_{i=1}^{n-1} \sum_{j=i+1}^n \frac{1}{2} \NR
\NC      \NC= \frac{1}{2} \sum_{i=1}^{n-1} \sum_{j=i+1}^n 1 \NR
\NC      \NC= \frac{1}{2} \sum_{i=1}^{n-1} (n-i) \NR
\NC      \NC= \frac{1}{2} \sum_{i=1}^{n-1} i \NR
\NC      \NC= \frac{n(n-1)}{4} \NR
\NC      \NC= \binom{n}{2}/2 \NR
\stopmathalignment\stopformula
\stopANSWER

\stopsection
