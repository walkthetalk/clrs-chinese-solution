\startsubject[
  title={Problems},
]

%p33-1
\startPROBLEM
(Convex layers)
已知平面上點集 \m{Q},
我們用歸納法定義 \m{Q} 的{\EMP 凸層}(convex layer)。
 \m{Q} 的第一凸層是由 \m{Q} 中 \ALGO{CH(Q)} 的頂點組成。
對 \m{i>1},定義 \m{Q_i} 由把 \m{Q} 中所有在凸層 \m{1,2,\ldots,i-1} 中的點
都去除後所剩餘的點構成。
如果 \m{Q_i\ne \empty},
那麼 \m{Q} 的第 \m{i} 凸層爲 \ALGO{CH(Q_i)};
否則,第 \m{i} 凸層無定義。

\startigBase[a]\startitem
寫出一個運行時間爲 \m{O(n^2)} 的算法,
以找出 \m{n} 個點所組成的集合的各凸層。
\stopitem\stopigBase

\startANSWER
\TODO{略。}
\stopANSWER

\startigBase[continue]\startitem
證明:在對 \m{n} 個實數進行排序所需時間爲 \m{\Omega(n\lg n)} 的任何計算模型上,
計算 \m{n} 個點的凸層需要 \m{\Omega(n\lg n)} 時間。
\stopitem\stopigBase

\startANSWER
\TODO{略。}
\stopANSWER
\stopPROBLEM

%p33-2
\startPROBLEM
(Maximal layers)
設 \m{Q} 是平面上 \m{n} 個點所組成的集合。
如果有 \m{x\ge x'} 且 \m{y\ge y'},
則稱點 \m{(x,y)} {\EMP 支配} 點 \m{(x',y')}。
 \m{Q} 中不被其中任何其他點支配的點稱爲{\EMP 最大點}。
注意, \m{Q} 可以包含許多最大點,
可以把這些最大點組織成如下的最大層。
第一最大層 \m{L_1} 是 \m{Q} 中最大點構成的集合。
對於 \m{i>1},第 \m{i} 最大層 \m{L_i} 是 \m{Q-\cup_{j=1}^{i-1}L_j} 中
最大點構成的集合。

假設 \m{Q} 包含 \m{k} 個非空的最大層,
並設 \m{y_i} 是 \m{L_i} 中最左邊點的 \m{y} 座標 \m{(i=1,2,\ldots,k)}。
假定 \m{Q} 中沒有兩個點有相同的 \m{x} 座標或 \m{y} 座標。

\startigBase[a]\startitem
證明 \m{y_1>y_2>\ldots >y_k}。
\stopitem\stopigBase

\startANSWER
\TODO{略。}
\stopANSWER

考慮一個點 \m{(x,y)},
他在 \m{Q} 中任意點的左邊,
並且其 \m{y} 座標與 \m{Q} 中任何點的 \m{y} 座標都不相同。
設 \m{Q'=Q\cup\{(x,y)\}}。

\startigBase[continue]\startitem
設 \m{j} 是滿足 \m{y_j < y} 的最小下標,除非 \m{y<y_k},
在這種情況下,令 \m{j=k+1}。
證明 \m{Q'} 的最大層如下:
\startigBase
\item 如果 \m{j\le k},則 \m{Q'} 的最大層與 \m{Q} 的最大層相同,
只是 \m{L_j} 也包含 \m{(x,y)} 作爲其新的最左點。
\item 如果 \m{j=k+1},則 \m{Q'} 的前 \m{k} 個最大層與 \m{Q} 的相同,
但此外, \m{Q'} 有一個非空的第 \m{k+1} 最大層 \m{L_{k+1}=\{(x,y)\}}。
\stopigBase
\stopitem\stopigBase

\startANSWER
\TODO{略。}
\stopANSWER

\startigBase[continue]\startitem
描述一種時間爲 \m{O(n\lg n)} 的算法,
用於計算出包含 \m{n} 個點的集合 \m{Q} 的各最大層。
(\hint 把一條掃除線從右向左移動。)
\stopitem\stopigBase

\startANSWER
\TODO{略。}
\stopANSWER

\startigBase[continue]\startitem
如果允許輸入點有相同的 \m{x} 座標或 \m{y} 座標,
會不會出現問題?
如果會,提出一種方法來解決這一問題。
\stopitem\stopigBase

\startANSWER
\TODO{略。}
\stopANSWER

\stopPROBLEM

%p33-3
\startPROBLEM
(Ghostbusters and ghosts)
有 \m{n} 個巨人正與 \m{n} 個鬼戰鬥。
每個巨人的武器是一個質子包,
他可以用一串質子流射中鬼從而將鬼消滅。
質子流沿直線行進,在擊中鬼時終止。
巨人聚定採取下列策略。
他們各自尋找一個鬼形成 \m{n} 對巨人-鬼,
然後每個巨人同時向各自對應的鬼射出一串質子流。
我們直到,質子流互相交叉是很危險的,
因此,巨人選擇的配對方式應該使質子流不會交叉。

假定每個巨人和每個鬼的位置都是平面上一個固定的點,
並且沒有三個位置共線。
\startigBase[a]\startitem
論證存在一條通過一個巨人和一個鬼的直線,
使得直線一遍的巨人數與同一邊的鬼數相等。
試說明如何在 \m{O(n\lg n)} 時間內找出這樣一條直線。
\stopitem\stopigBase

\startANSWER
\TODO{略。}
\stopANSWER

\startigBase[continue]\startitem
寫出一個運行時間爲 \m{O(n^2\lg n)} 的算法,
使其以不會有質子流交叉爲條件把巨人與鬼配對。
\stopitem\stopigBase

\startANSWER
\TODO{略。}
\stopANSWER
\stopPROBLEM

%33-4
\startPROBLEM
(Picking up sticks)
 Charon 教授有 \m{n} 根小棍子,
他們以某種方式互相疊放在一起。
每根棍子都用其端點來指定,
每個端點都是一個有序的三元組,
其座標 \m{(x,y,z)} 已知。
所有棍子都不是垂直的。
他希望拾取所有的棍子,
但要滿足如下條件:
一次一根地挑起棍子,
當一根棍子上面沒有壓着其他棍子時,
才能將其挑起來。
\startigBase[a]\startitem
給出一個過程,
取兩根棍子 \m{a} 和 \m{b} 作爲參數,
返回 \m{a} 是在 \m{b} 的上面、下面還是與 \m{b} 無關。
\stopitem\stopigBase

\startANSWER
\TODO{略。}
\stopANSWER

\startigBase[continue]\startitem
給出一個有效的算法,用於確定是否有可能拾取所有的棍子。
如果能,提供一個拾取所有棍子的合法順序。
\stopitem\stopigBase

\startANSWER
\TODO{略。}
\stopANSWER
\stopPROBLEM

%p33-5
\startPROBLEM
(Sparse-hulled distributions)
考慮計算平面上點集的圖包問題,
但這些點是根據某已知的隨機分佈取得的。
有時,從這樣一種分佈中取得的 \m{n} 個點凸包的期望規模爲 \m{O(n^{1-\varepsilon})},
其中 \m{\varepsilon} 爲某個大於 0 的常數。
稱這樣的分佈爲{\EMP 稀疏包分佈}。
稀疏包分佈包括以下幾種:
\startigBase[2]
\item 點是均勻地從一個單位半徑的圓面中取得的,凸包的期望規模爲 \m{\Theta(n^{1/3})}。
\item 點是均勻地從一個具有 \m{k} 條邊的凸多邊形內部取得的(\m{k} 爲任意常數)。
凸包的期望規模爲 \m{\Theta(\lg n)}。
\item 點是根據二維正態分佈取得的。凸包的期望規模爲 \m{\Theta(\sqrt{\lg n})}。
\stopigBase

\startigBase[a]\startitem
已知兩個分別有 \m{n_1} 和 \m{n_2} 個頂點的凸多邊形,
說明如何在 \m{O(n_1+n_2)} 時間內計算出全部 \m{n_1+n_2} 個點的凸包(多邊形可以重疊)。
\stopitem\stopigBase

\startANSWER
\TODO{略。}
\stopANSWER

\startigBase[continue]\startitem
證明:對於根據稀疏包分佈獨立取得的一組 \m{n} 個點,
其凸包可以在 \m{O(n)} 的期望時間內計算出來。
(\hint 採用遞迴方法分別求出前 \m{n/2} 個點和後 \m{n/2} 個點的凸包,
然後再對結果進行合併。)
\stopitem\stopigBase

\startANSWER
\TODO{略。}
\stopANSWER
\stopPROBLEM

\stopsubject%Problems
