\startsection[
  title={Symmetric positive-definite matrices and least-squares approximation},
]

%e28.3-1
\startEXERCISE
證明:一個對稱正定矩陣對角線上的每一個元素都是正的。
\stopEXERCISE

\startANSWER
\TODO{略。}
\stopANSWER

%e28.3-2
\startEXERCISE
設如下矩陣 \m{A} 是一個 \m{2\times 2} 的對稱正定矩陣。
運用類似引理 28.5 證明過程中用過的“完全平方”來證明該行列式的值 \m{ac-b^2} 是正的。
\startformula
A=\left[\startmatrix
\NC a \NC b \NR
\NC b \NC c \NR
\stopmatrix\right]
\stopformula
\stopEXERCISE

\startANSWER
\TODO{略。}
\stopANSWER

%e28.3-3
\startEXERCISE
證明:一個對稱正定矩陣中值最大的元素在對角線上。
\stopEXERCISE

\startANSWER
\TODO{略。}
\stopANSWER

%e28.3-4
\startEXERCISE
證明:一個對稱正定矩陣的每一個主子矩陣的行列式值都是正的。
\stopEXERCISE

\startANSWER
\TODO{略。}
\stopANSWER

%e28.3-5
\startEXERCISE
證明:設 \m{A_k} 表示對稱正定矩陣 \m{A} 的第 \m{k} 個主子矩陣。
證明:在 LU 分解過程中, \m{det(A_k)/det(A_{k-1})} 是第 \m{k} 個主元,
其中爲方便起見, \m{det(A_0)=1}。
\stopEXERCISE

\startANSWER
\TODO{略。}
\stopANSWER

%e28.3-6
\startEXERCISE
證明:找出具有形式 \m{F(x) = c_1 + c_2 x\lg x + c_3 e^x} 的函數,
使其爲下面數據點的最優最小二乘擬合:
 \m{(1,1)}, \m{(2,1)}, \m{(3,3)}, \m{(4,8)}。
\stopEXERCISE

\startANSWER
\TODO{略。}
\stopANSWER

%e28.3-7
\startEXERCISE
證明:請說明僞逆矩陣 \m{A^+} 滿足下面 4 個等式:
\startformula\startmathalignment
\NC AA^+ A \NC = A \NR
\NC A^+ A A^+ \NC = A^+ \NR
\NC (AA^+)^T \NC = A A^+ \NR
\NC (A^+ A)^T \NC = A^+ A \NR
\stopmathalignment\stopformula
\stopEXERCISE

\startANSWER
\TODO{略。}
\stopANSWER

\stopsection
