\startsection[
  title={Disjoint-set forests},
  reference=section:disjoint_set_forests,
]

%e21.3-1
\startEXERCISE
用按秩合併與路徑壓縮啓發式策略的不相交集合森林重做\refexercise{21.2-2}。
\stopEXERCISE

\startANSWER
\TODO{略。}
\stopANSWER

%e21.3-2
\startEXERCISE
寫出使用路徑壓縮的 \ALGO{FIND-SET} 過程的非遞迴版本。
\stopEXERCISE

\startANSWER
\CLRSH{FIND-SET(x)}
\startCLRS
y = x
z = x.p
while y != z
	tmp = z.p
	z.p = y
	y = z
	z = tmp
ret = y
y = z.p
while z != x
	tmp = z.p
	z.p = ret
	z = tmp
x.p = ret
return ret
\stopCLRS
\stopANSWER

%e21.3-3
\startEXERCISE
給出一個包含 \m{m} 個 \ALGO{MAKE-SET}、 \ALGO{UNION} 和 \ALGO{FIND-SET} 操作的序列
(其中有 \m{n} 個是 \ALGO{MAKE-SET} 操作),
當僅使用按秩合併時,需要 \m{\Omega(m\lg n)} 的時間。
\stopEXERCISE

\startANSWER
提示: \ALGO{UNION} 操作的結果是一棵高度爲 \m{\lg n} 的樹。
\stopANSWER

%e21.3-4
\startEXERCISE
假設想要增加一個 \ALGO{PRINT-SET(x)} 操作,
他是對於給定的節點 \m{x} 打印出 \m{x} 所在集合的所有成員,順序可以任意。
如何對一棵不相交集合森林的每個節點僅增加一個屬性,
使得 \ALGO{PRINT-SET(x)} 所花費的時間同 \m{x} 所在集合元素的個數呈線性關係,
並且其他操作的漸進運行時間不改變。
這裏假設我們可在 \m{O(1)} 時間內打印除集合的每個成員。
\stopEXERCISE

\startANSWER
\TODO{略。}
\stopANSWER

%e21.3-5
\startEXERCISE\DIFFICULT
證明:任何具有 \m{m} 個 \ALGO{MAKE-SET}、 \ALGO{FIND-SET} 和 \ALGO{LINK} 操作的序列,
這裏所有的 \ALGO{LINK} 操作都出現在 \ALGO{FIND-SET} 操作之前,
如果同時使用路徑壓縮和按秩合併啓發式策略,
則這些操作只需 \m{O(m)} 的時間。
在同樣情況下,如果只使用路徑壓縮啓發式策略,又會如何?
\stopEXERCISE

\startANSWER
\m{n-1} 個 \ALGO{UNION} 需要 \m{O(n)} 時間,會創建 \m{n-1} 條邊。
 \ALGO{FIND-SET} 會壓縮到根節點路徑上的所有邊。
所以 \m{m} 個 \ALGO{FIND-SET} 需要的時間不會多於 \m{O(m+n)}。
只使用路徑壓縮,結果也一樣。
\stopANSWER

\stopsection
