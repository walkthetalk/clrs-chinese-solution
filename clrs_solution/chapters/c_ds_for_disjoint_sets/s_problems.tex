\startsubject[
  title={Problems},
]

%p21-1
\startPROBLEM
(Off-line minimum)
{\EMP 脫機最小值問題}(off-line minimum problem)是使用 \ALGO{INSERT} 和 \ALGO{EXTRACT-MIN} 操作
維護一個元素取自域 \m{\{1,2,\ldots,n\}} 的動態集合 \m{T}。
給定一組包含 \m{n} 個 \ALGO{INSERT} 和 \m{m} 個 \ALGO{EXTRACT-MIN} 的調用序列 \m{S},
其中屬於 \m{\{1,2,\ldots,n\}} 中的每個關鍵字只被插入一次。
我們希望確定每個 \ALGO{EXTRACT-MIN} 調用返回的是哪個關鍵字。
特別地,希望對一個 \m{extracted[1..m](i=1,2,\ldots,m)} 數列進行填充,
其中 \m{extracted[i]} 是由第 \m{i} 次 \ALGO{EXTRACT-MIN} 調用所返回的關鍵字。
該問題是“脫機的”,其含義就是在確定任何返回的關鍵字之前處理真個序列 \m{S}。

\startigBase[a]\startitem
在下面脫機最小值問題的實例中,每個操作 \ALGO{INSERT(i)} 用一個 \m{i} 值來表示,
並且每個 \ALGO{EXTRACT-MIN} 用字母 \m{E} 來表示:
\startformula
4,8,E,3,E,9,2,6,E,E,E,1,7,E,5
\stopformula
將正確的值填入 \m{extracted} 數列。
\stopitem\stopigBase

\startANSWER
\TODO{略。}
\stopANSWER

爲了設計出解決此問題的算法,我們把序列 \m{S} 劃分成若干個同構的子序列,
即如下表示 \m{S}:
\startformula
I_1,E,I_2,E,I_3,\ldots,I_m,E,I_{m+1}
\stopformula
這裏每個 \m{E} 代表單次 \ALGO{EXTRACT-MIN} 調用,
並且每個 \m{I_j} 代表一個(可能爲空的) \ALGO{INSERT} 調用序列。
對於每個子序列 \m{I_j},
開始時把由這些操作插入的關鍵字插入一個集合 \m{K_j},
如果 \m{I_j} 爲空,那麼他也爲空。然後執行下面的程序:

\CLRSH{OFF-LINE-MINIMUM(m, n)}
\startCLRS
for i = 1 to n
	determine j such that i in K[j]
	if j != m + 1
		extracted[j] = i
		let l be the smallest value greater than j
			for which set K[l] exists
		K[l] = K[j] merge K[l], destroying K[j]
return extracted
\stopCLRS

\startigBase[continue]\startitem
證明:由 \ALGO{OFF-LINE-MINIMUM} 返回的數列 \m{extracted} 是正確的。
\stopitem\stopigBase

\startANSWER
\TODO{略。}
\stopANSWER

\startigBase[continue]\startitem
描述如何用不相交集合數據結構來實現高效的 \ALGO{OFF-LINE-MINIMUM}。
給出其最壞情況下運行時間的緊確界。
\stopitem\stopigBase

\startANSWER
\TODO{略。}
\stopANSWER

\stopPROBLEM

%p21-2
\startPROBLEM
(Depth determination)
在{\EMP 深度確定}(Depth determination)問題中,
我們通過以下三個操作來維護一個有根樹的森林 \m{\cal{F} = \{T_i\}}:

\ALGO{MAKE-TREE(v)}:創建一棵只包含唯一節點 \m{v} 的樹。

\ALGO{FIND-DEPTH(v)}:返回節點 \m{v} 在樹中的深度。

\ALGO{GRAFT(r, v)}:使得節點 \m{r} (假定他爲一棵樹的樹根)成爲節點 \m{v} 的孩子
(假定他在另一棵樹中,但是他本身可能是、也可能不是一棵樹的根)

\startigBase[a]\startitem
假設採用類似於不相交集合森林的樹表示:
 \m{v.p} 是節點 \m{v} 的父節點,
除了 \m{v} 是根時 \m{v.p = v} 的這種情況。
進一步假設,我們可以通過置 \m{r.p=v} 來實現 \ALGO{GRAFT(r,v)},
並且可以通過沿着查找路徑上升至根,
返回一個除 \m{v} 以外的節點樹來實現 \ALGO{FIND-PATH(v)}。
證明:一組 \m{m} 個 \ALGO{MAKE-TREE}、 \ALGO{FIND-DEPTH} 和 \ALGO{GRAFT} 操作
序列的最壞情況運行時間是 \m{\Theta(m^2)}。
\stopitem\stopigBase

\startANSWER
\TODO{略。}
\stopANSWER

通過使用按秩合併與路徑壓縮啓發式策略,
能減少最壞情況運行時間。
我們使用不相交集合森林 \m{{\cal{S}} = \{S_i\}},
其中每個集合 \m{S_i} (他本身是一棵樹)對應於一棵森林 \m{\cal{F}} 中的樹 \m{T_i}。
然而,集合 \m{S_i} 中的樹結構沒有必要對應於 \m{T_i} 的樹結構。
實際上, \m{S_i} 的實現並沒有記錄準確的父子關係,
但他允許我們確定 \m{T_i} 中任意節點的深度。

關鍵的思想是維護每個節點 \m{v} 的一個“僞距離” \m{v.d},
他倍定義爲使得沿着從 \m{v} 到他的集合 \m{S_i} 的根的簡單路徑上的僞距離之和等於 \m{T_i} 中節點 \m{v} 的深度。
也就是說,如果從 \m{v} 到他在 \m{S_i} 的根的簡單路徑爲 \m{v_0,v_1,\ldots,v_k}(這裏 \m{v_0=v} 並且 \m{v_k} 是 \m{S_i} 的根),
那麼節點 \m{v} 在樹 \m{T_i} 上的深度爲 \m{\sum_{j=0}^{k}v_j\cdot d}。

\startigBase[continue]\startitem
給出 \ALGO{MAKE-TREE} 的一種實現。
\stopitem\stopigBase

\startANSWER
\TODO{略。}
\stopANSWER

\startigBase[continue]\startitem
說明應如何修改 \ALGO{FIND-SET} 來實現 \ALGO{FIND-DEPTH}。
你的實現要採用路徑壓縮,
並且他的運行時間應與查找路徑的長度呈線性關係。
試確保你的實現能正確地更新僞距離。
\stopitem\stopigBase

\startANSWER
\TODO{略。}
\stopANSWER

\startigBase[continue]\startitem
說明如何實現 \ALGO{GRAFT(r,v)},
他通過修改 \ALGO{UNION} 和 \ALGO{LINK} 來合併包含 \m{r} 和 \m{v} 的集合。
試確保你的實現能正確地更新僞距離。
並注意到,
集合 \m{S_i} 的根沒有必要是對應樹 \m{T_i} 的根。
\stopitem\stopigBase

\startANSWER
\TODO{略。}
\stopANSWER

\startigBase[continue]\startitem
試給出一組 \m{m} 個 \ALGO{MAKE-TREE}、 \ALGO{FIND-DEPTH} 和 \ALGO{GRAFT} 操作的序列
(其中 \m{n} 個是 \ALGO{MAKE-TREE} 操作)最壞情況運行時間的一個緊確界。
\stopitem\stopigBase

\startANSWER
\TODO{略。}
\stopANSWER

\stopPROBLEM

%p21-3
\startPROBLEM
(Tarjan’s off-line least-common-ancestors algorithm)
在一棵有根樹 \m{T} 中,
兩個節點 \m{u} 和 \m{v} 的{\EMP 最小公共祖先}
(leat common ancestor) \m{\omega} 是節點 \m{u} 和 \m{v} 的一個共同祖先,
且他有最大的深度。
在{\EMP 脫機最小公共祖先問題}(off-line least-common-ancestors problem)中,
給定一棵有根樹 \m{T} 和一個在 \m{T} 中的無序節點對的任意集合 \m{P=\{\{u,v\}\}} ,
我們希望確定 \m{P} 中每對的最小公共祖先。

爲了解決脫機最小公共祖先問題,
下面的過程通過對 \ALGO{LCA(T.root)} 的初始調用,
來執行對 \m{T} 的樹遍歷。
假設在執行遍歷之前,每個節點被着色爲白色。

\CLRSH{LCA(u)}
\startCLRS
MAKE-SET(u)
FIND-SET(u).ancestor = u
for each child v of u in T
	LCA(v)
	UNION(u,v)
	FIND-SET(u).ancestor = u
u.color = BLACK
for each node v such that {u,v} in P
	if v.color = BLACK
		print "The least common ancestor of "
			u " and " v " is " FIND-SET(v).ancestor
\stopCLRS

\startigBase[a]\startitem
證明:對每對 \m{\{u,v\}\in P},第 10 行恰好只執行一次。
\stopitem\stopigBase

\startANSWER
\TODO{略。}
\stopANSWER

\startigBase[continue]\startitem
證明:在調用 \ALGO{LCA(u)} 時,不相交集合數據結構的集合數等於 \m{T} 中 \m{u} 的深度。
\stopitem\stopigBase

\startANSWER
\TODO{略。}
\stopANSWER

\startigBase[continue]\startitem
證明:對於每對 \m{\{u,v\}\in P}, \ALGO{LCA} 能正確地輸出 \m{u} 和 \m{v} 的最小公共祖先。
\stopitem\stopigBase

\startANSWER
\TODO{略。}
\stopANSWER

\startigBase[continue]\startitem
假設我們使用\refsection{disjoint_set_forests} 中的不相交集合數據結構實現,試分析 \ALGO{LCA} 的運行時間。
\stopitem\stopigBase

\startANSWER
\TODO{略。}
\stopANSWER

\stopPROBLEM

\stopsubject%Problems
