\startsection[
  title={Optimal binary search trees},
]

%e15.5-1
\startEXERCISE
設計僞碼 \ALGO{CONSTRUCT-OPTIMAL-BST(root)},
輸入爲表 root,
輸出是最優二叉搜索樹的結構。
例如,對圖 15-10 中的 root 表,應輸出:

\startigBase
\item \m{k_2} 爲根
\item \m{k_1} 爲 \m{k_2} 的左孩子
\item \m{d_0} 爲 \m{k_1} 的左孩子
\item \m{d_1} 爲 \m{k_1} 的右孩子
\item \m{k_5} 爲 \m{k_2} 的右孩子
\item \m{k_4} 爲 \m{k_5} 的左孩子
\item \m{k_3} 爲 \m{k_4} 的左孩子
\item \m{d_2} 爲 \m{k_3} 的左孩子
\item \m{d_3} 爲 \m{k_3} 的右孩子
\item \m{d_4} 爲 \m{k_4} 的右孩子
\item \m{d_5} 爲 \m{k_5} 的右孩子
\stopigBase

與圖 15-9(b) 中的最優二叉搜索樹對應。
\stopEXERCISE

\startANSWER
\CLRSH{CONSTRUCT-OPTIMAL-BST(root)}
\startCLRS
print "k" root[1, n] "是根"
p<-0
PRINT-OPTIMAL-BST(root, 1, n)
\stopCLRS

\CLRSH{PRINT-OPTIMAL-BST(root, i, j)}
\startCLRS
if root[i, root[i, j]-1]
	print "k" root[i, root[i, j]-1] "是k" root[i, j] "的左孩子"
	PRINT-OPTIMAL-BST(root, i, root[i, j]-1)
else
	print "d" p "是k" root[i, j] "的左孩子"
	p<-p+1
if root[root[i, j]+1, j]
	print "k" root[root[i, j]+1, j] "是k" root[i, j] "的右孩子"
	PRINT-OPTIMAL-BST(root, root[i, j]+1, j)
else
	print "d" p "是k" root[i, j] "的右孩子"
	p<-p+1
\stopCLRS
\stopANSWER

%e15.5-2
\startEXERCISE
若 7 個關鍵字的概率如下所示,求其最優二叉搜索樹的結構和代價。
\startxtable[
    option=max,
    align={middle,lohi},
    split=yes,
    header=repeat,
    footer=repeat,
    offset=.25em,
]

% head
\startxtablehead[frame=off,topframe=on,bottomframe=on]
\startxrow[foregroundstyle=bold,]
  \xcell[rightframe=on]{\m{i}}\processcommalist[0,1,2,3,4,5,6,7]{\xcell[align={middle}]}
\stopxrow
\stopxtablehead

% body
\startxtablebody[frame=off,topframe=on,bottomframe=on]
\startxrow \xcell[rightframe=on]{\m{p_i}}
	\processcommalist[,0.04,0.06,0.08,0.02,0.10,0.12,0.14]\xcell \stopxrow
\startxrow \xcell[rightframe=on]{\m{q_i}}
	\processcommalist[0.06,0.06,0.06,0.06,0.05,0.05,0.05,0.05]\xcell \stopxrow
\stopxtablebody

\stopxtable


\stopEXERCISE

\startANSWER
\externalfigure[output/e15_5_2-1]
\externalfigure[output/e15_5_2-2]
\externalfigure[output/e15_5_2-3]
\externalfigure[output/e15_5_2-4]
\stopANSWER

%e15.5-3
\startEXERCISE
假設 \ALGO{OPTIMAL-BST} 不維護表 \m{\omega[i,j]},
而是在第 9 行利用公式(15.12)直接計算 \m{\omega(i,j)},
然後在第 11 行使用此值。
如此改動會對漸進時間複雜度有何影響?
附公式(5-5):
\startformula
\omega(i,j) = \sum_{l=i}^{j}p_l + \sum_{l=i-1}^{j}q_l
\stopformula
\stopEXERCISE

\startANSWER
此改動不會影響算法的漸進時間複雜度。
就改動本身而言,時間由 \m{\Theta(1)} 變爲 \m{\Theta(j-i)}。
但是,後面的循環本來就是 \m{\Theta(j-i)},所以再加一個 \m{\Theta(j-i)},
結果還是 \m{\Theta(j-i)},從而對整個算法而言,時間複雜度沒什麼變化,還是 \m{\Theta(n^3)}。
\stopANSWER

%e15.5-4
\startEXERCISE
Knuth 已經證明,對所有 \m{1\le i < j \le n},
存在最優二叉搜索樹,其根滿足 \m{root[i,j-1]\le root[i,j]\le root[i+1,j]}。
利用這一特性修改算法 \ALGO{OPTIMAL-BST},
使得運行時間減少爲 \m{\Theta(n^2)}。
\stopEXERCISE

\startANSWER
\CLRSH{OPTIMAL-BST‘(p, q, n)}
\startCLRS
for i = 1 to n + 1
	e[i, i-1] = q[i-1]
	w[i, i-1] = q[i-1]
for l = 1 to n
	for i = 1 to n - l + 1
		j = i + l - 1
		e[i, j] = ∞
		w[i, j] = w[i, j-1] + p[j] + q[j]
		for r = root[i, j-1] to root[i+1, j]
			t = e[i, r-1] + e[r+1, j] + w[i, j]
			if t < e[i, j]
				e[i, j] = t
				root[i, j] = r
return e and root
\stopCLRS

計算 \m{root[i,j]} 的時候 r 循環次數爲 \m{root[i+1,j] - root[i, j-1] + 1}。
則 root 中第 k 層所有元素共需要 r 循環 \m{root[k,1] - root[1,k] + n - k} 次。
又因爲 \m{1\le root[1,k] \le root[k,1] \le n},所以每一層的循環次數爲 \m{O(n)}。
而 \m{0\le k\le n-1},所以總循環次數爲 \m{O(n^2)}。
\stopANSWER

\stopsection
