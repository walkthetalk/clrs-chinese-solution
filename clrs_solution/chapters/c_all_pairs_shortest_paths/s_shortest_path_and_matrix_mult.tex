\startsection[
  title={Shortest paths and matrix multiplication},
  reference=section:25.1,
]

%e25.1-1
\startEXERCISE
圖 25-2 爲一帶權重有向圖,
在其上運行算法 \ALGO{SLOW-ALL-PAIRS-SHORTEST-PATHS}。
給出循環的每次迭代所計算出的矩陣。
然後用算法 \ALGO{FASTER-ALL-PAIRS-SHORTEST-PATHS} 重新做一遍。
附圖 25-2:

\externalfigure[output/e25_1_1-1]
\stopEXERCISE

\startANSWER
慢速版本:

\startcombination[2*3]
{\startformula

L^{(1)} = \left(\startmatrix
\NC 0\NC \infty\NC \infty\NC \infty\NC -1\NC \infty \NR
\NC 1\NC 0\NC \infty\NC 2\NC \infty\NC \infty \NR
\NC \infty\NC 2\NC 0\NC \infty\NC \infty\NC -8 \NR
\NC -4\NC \infty\NC \infty\NC 0\NC 3\NC \infty \NR
\NC \infty\NC 7\NC \infty\NC \infty\NC 0\NC \infty \NR
\NC \infty\NC 5\NC 10\NC \infty\NC \infty\NC 0 \NR
\stopmatrix\right)

\stopformula}{}{\startformula

L^{(2)} = \left(\startmatrix
\NC 0\NC 6\NC \infty\NC \infty\NC -1\NC \infty \NR
\NC -2\NC 0\NC \infty\NC 2\NC 0\NC \infty \NR
\NC 3\NC -3\NC 0\NC 4\NC \infty\NC -8 \NR
\NC -4\NC 10\NC \infty\NC 0\NC -5\NC \infty \NR
\NC 8\NC 7\NC \infty\NC 9\NC 0\NC \infty \NR
\NC 6\NC 5\NC 10\NC 7\NC \infty\NC 0 \NR
\stopmatrix\right)

\stopformula}{}{\startformula

L^{(3)} = \left(\startmatrix
\NC 0\NC 6\NC \infty\NC 8\NC -1\NC \infty \NR
\NC -2\NC 0\NC \infty\NC 2\NC -3\NC \infty \NR
\NC -2\NC -3\NC 0\NC -1\NC 2\NC -8 \NR
\NC -4\NC 2\NC \infty\NC 0\NC -5\NC \infty \NR
\NC 5\NC 7\NC \infty\NC 9\NC 0\NC \infty \NR
\NC 3\NC 5\NC 10\NC 7\NC 5\NC 0 \NR
\stopmatrix\right)

\stopformula}{}{\startformula

L^{(4)} = \left(\startmatrix
\NC 0\NC 6\NC \infty\NC 8\NC -1\NC \infty \NR
\NC -2\NC 0\NC \infty\NC 2\NC -3\NC \infty \NR
\NC -5\NC -3\NC 0\NC -1\NC -3\NC -8 \NR
\NC -4\NC 2\NC \infty\NC 0\NC -5\NC \infty \NR
\NC 5\NC 7\NC \infty\NC 9\NC 0\NC \infty \NR
\NC 3\NC 5\NC 10\NC 7\NC 2\NC 0 \NR
\stopmatrix\right)

\stopformula}{}{\startformula

L^{(5)} = \left(\startmatrix
\NC 0\NC 6\NC \infty\NC 8\NC -1\NC \infty \NR
\NC -2\NC 0\NC \infty\NC 2\NC -3\NC \infty \NR
\NC -5\NC -3\NC 0\NC -1\NC -6\NC -8 \NR
\NC -4\NC 2\NC \infty\NC 0\NC -5\NC \infty \NR
\NC 5\NC 7\NC \infty\NC 9\NC 0\NC \infty \NR
\NC 3\NC 5\NC 10\NC 7\NC 2\NC 0 \NR
\stopmatrix\right)

\stopformula}{}
{}{}
\stopcombination

快速版本:

\startcombination[2*2]
{\startformula

L^{(1)} = \left(\startmatrix
\NC 0\NC \infty\NC \infty\NC \infty\NC -1\NC \infty \NR
\NC 1\NC 0\NC \infty\NC 2\NC \infty\NC \infty \NR
\NC \infty\NC 2\NC 0\NC \infty\NC \infty\NC -8 \NR
\NC -4\NC \infty\NC \infty\NC 0\NC 3\NC \infty \NR
\NC \infty\NC 7\NC \infty\NC \infty\NC 0\NC \infty \NR
\NC \infty\NC 5\NC 10\NC \infty\NC \infty\NC 0 \NR
\stopmatrix\right)

\stopformula}{}{\startformula

L^{(2)} = \left(\startmatrix
\NC 0\NC 6\NC \infty\NC \infty\NC -1\NC \infty \NR
\NC -2\NC 0\NC \infty\NC 2\NC 0\NC \infty \NR
\NC 3\NC -3\NC 0\NC 4\NC \infty\NC -8 \NR
\NC -4\NC 10\NC \infty\NC 0\NC -5\NC \infty \NR
\NC 8\NC 7\NC \infty\NC 9\NC 0\NC \infty \NR
\NC 6\NC 5\NC 10\NC 7\NC \infty\NC 0 \NR
\stopmatrix\right)

\stopformula}{}{\startformula

L^{(4)} = \left(\startmatrix
\NC 0\NC 6\NC \infty\NC 8\NC -1\NC \infty \NR
\NC -2\NC 0\NC \infty\NC 2\NC -3\NC \infty \NR
\NC -5\NC -3\NC 0\NC -1\NC -3\NC -8 \NR
\NC -4\NC 2\NC \infty\NC 0\NC -5\NC \infty \NR
\NC 5\NC 7\NC \infty\NC 9\NC 0\NC \infty \NR
\NC 3\NC 5\NC 10\NC 7\NC 2\NC 0 \NR
\stopmatrix\right)

\stopformula}{}{\startformula

L^{(8)} = \left(\startmatrix
\NC 0\NC 6\NC \infty\NC 8\NC -1\NC \infty \NR
\NC -2\NC 0\NC \infty\NC 2\NC -3\NC \infty \NR
\NC -5\NC -3\NC 0\NC -1\NC -6\NC -8 \NR
\NC -4\NC 2\NC \infty\NC 0\NC -5\NC \infty \NR
\NC 5\NC 7\NC \infty\NC 9\NC 0\NC \infty \NR
\NC 3\NC 5\NC 10\NC 7\NC 2\NC 0 \NR
\stopmatrix\right)

\stopformula}{}
\stopcombination
\stopANSWER

%e25.1-2
\startEXERCISE
爲什麼要求對於所有 \m{1\le i\le n},必須滿足 \m{\omega_{ii} = 0}?
\stopEXERCISE

\startANSWER
這樣才能保證遞迴式 25.2 的正確性。附遞迴式 25.2:
\startformula
l_{ij}^{(m)} = \min(l_{ij}^{(m-1)}, \min_{1\le k\le n}\{l_{ik}^{(m-1)} + \omega_{kj}\})
=\min_{1\le k\le n}\{l_{ik}^{(m-1)} + \omega_{kj}\}
\stopformula
\stopANSWER

%e25.1-3
\startEXERCISE
在最短路徑算法中使用的矩陣 \m{L^{(0)}} 對應傳統矩陣乘法裏的什麼?
\startformula
L^{(0)} = \left(\startmatrix
\NC 0 \NC \infty \NC \infty \NC \ldots \NC \infty \NR
\NC \infty \NC 0 \NC \infty \NC \ldots \NC \infty \NR
\NC \infty \NC \infty \NC 0 \NC \ldots \NC \infty \NR
\NC \vdots \NC \vdots \NC \vdots \NC \ddots \NC \vdots \NR
\NC \infty \NC \infty \NC \infty \NC \ldots \NC 0 \NR
\stopmatrix\right)
\stopformula
\stopEXERCISE

\startANSWER
單位矩陣。將單位矩陣中加法所用的 \m{0} 換成 \m{\min} 所用的 \m{\infty},
將乘法所用的 \m{1} 換成加法所用的 \m{0}。
\startformula
I = \left(\startmatrix
\NC 1 \NC 0 \NC 0 \NC \ldots \NC 0 \NR
\NC 0 \NC 1 \NC 0 \NC \ldots \NC 0 \NR
\NC 0 \NC 0 \NC 1 \NC \ldots \NC 0 \NR
\NC \vdots \NC \vdots \NC \vdots \NC \ddots \NC \vdots \NR
\NC 0 \NC 0 \NC 0 \NC \ldots \NC 1 \NR
\stopmatrix\right)
\stopformula
\stopANSWER

%e25.1-4
\startEXERCISE
證明:由 \ALGO{EXTEND-SHORTEST-PATHS} 所定義的矩陣乘法滿足結合律。
\stopEXERCISE

\startANSWER
\m{\min} 和 \m{+} 都滿足結合律。
\stopANSWER

%e25.1-5
\startEXERCISE
說明如何將單源最短路徑問題表示爲矩陣和向量的乘積,
並解釋該乘積的計算過程如何對應 \ALGO{BELLMAN-FORD} 算法?
(請參閱\refsection{24.1})
\stopEXERCISE

\startANSWER
\startformula
L^{(n-1)} = W^{n-1} = L^{(0)}\cdot W^{n-1}
\stopformula
其中 \m{L^{(0)}} 是單位矩陣, \m{l_{ij}^{(n-1)} = \delta(i,j)}。
即矩陣中第 \m{i} 行第 \m{j} 列的元素表示
從第 \m{i} 個節點到第 \m{j} 個節點的最短路徑權重。
第 \m{i} 行元素表示從第 \m{i} 個節點出發到達其他各個節點的最短路徑權重。

矩陣“相乘” \m{C=A\cdot B},
 \m{C} 中第 \m{i} 行元素是由 \m{A} 中第 \m{i} 行元素“乘以”矩陣 \m{B} 得到。
由於我們只需要 \m{C} 中第 \m{i} 行元素,因此也只需要 \m{A} 中第 \m{i} 行元素。

所以源節點爲 \m{i} 的單源最短路徑就是 \m{L_{i}^{(0)} \cdot W^{n-1}},
其中 \m{L_{i}^{(0)}} 就是 \m{L^{(0)}} 的第 \m{i} 行元素,是一個向量,
此向量中第 \m{i} 個元素是 \m{0},其他元素都是 \m{\infty}。

自左至右進行“乘法”操作與 \ALGO{BELLMAN-FORD} 的運行過程一樣。
向量對應的就是 \ALGO{BELLMAN-FORD} 算法中的 \m{d}。

\startigBase[2]
\item 一開始向量中只有源節點所處位置元素爲 \m{0},其他均爲 \m{\infty},
與 \ALGO{INITIALIZE-SINGLE-SOURCE} 相同。
\item 每次向量“乘以” \m{W} 就會對所有邊進行一遍鬆弛操作。
\item 一共執行 \m{n-1} 次乘法,即 \m{n-1} 邊鬆弛操作。
\stopigBase
\stopANSWER

%e25.1-6
\startEXERCISE
假定我們還希望在本節討論的算法裏計算出最短路徑上的節點。
說明如何在 \m{O(n^3)} 時間內從已經計算出的
最短路徑權重矩陣 \m{L} 計算出前驅矩陣 \m{\prod}。
\stopEXERCISE

\startANSWER
\startCLRS
for each edge (u,v) in G.E
	for s in G.V
		if L[s][u] + w(u,v) == L[s][v]
			PI[s][v] = u
\stopCLRS
\stopANSWER

%e25.1-7
\startEXERCISE
我們可以用計算最短路徑權重的辦法來計算最短路徑上的節點。
定義 \m{\pi_{ij}^{(m)}} 爲從 \m{i} 到 \m{j} 的至多
包含 \m{m} 條邊的任意最小權重路徑上節點 \m{j} 的前驅。
請修改 \ALGO{EXTEND-SHORTEST-PATHS} 和 \ALGO{SLOW-ALL-PAIRS-SHORTEST-PATHS},
使其在計算出矩陣 \m{L^{(1)},L^{(2)},\ldots,L^{(n-1)}} 的同時,
計算出矩陣 \m{\prod^{(1)},\prod^{(2)},\ldots,\prod^{(n-1)}}。
\stopEXERCISE

\startANSWER
更新 \m{L} 的同時更新對應的 \m{\prod} 即可。
\stopANSWER

%e25.1-8
\startEXERCISE
本節討論的 \ALGO{FASTER-ALL-PAIRS-SHORTEST-PATHS} 要求
我們保存 \m{\lceil \lg (n-1)\rceil} 個矩陣,
由於每個矩陣有 \m{n^2} 個元素,總存儲空間需求爲 \m{\Theta(n^2\lg n)}。
請修改算法,使其僅僅使用兩個 \m{n\times n} 矩陣,
從而將存儲空間降至 \m{\Theta(n^2)}。
\stopEXERCISE

\startANSWER
\CLRSH{FASTER-ALL-PAIRS-SHORTEST-PATHS(W)}
\startCLRS
n = W.rows
L[1] = W
m = 1
i = 1
while m < n-1
	L[(i+1) mod 2] = EXTEND-SHORTEST-PATHS(L[i mod 2],L[i mod 2])
	m = 2m
	i = i + 1
return L[i mod 2]
\stopCLRS
\stopANSWER

%e25.1-9
\startEXERCISE
修改 \ALGO{FASTER-ALL-PAIRS-SHORTEST-PATHS},
使其可以判斷一個圖是否包含負權重環路。
\stopEXERCISE

\startANSWER
爲算法添加以下內容,若返回 FALSE 則代表有負權重環路,否則沒有:
\startCLRS
for i = 1 upto n
	for each edge(u,v) in G.E
		if L[i][v] > L[i][u] + w(u,v)
			return FALSE
return TRUE
\stopCLRS
或者也可以通過判斷對角線上的元素是否有負值,
不過需要計算 \m{W^n},而不是 \m{W^{n-1}}。
\stopANSWER

%e25.1-10
\startEXERCISE
給出一個有效算法,可以找出圖中所有負權重環路的最短長度(邊的條數)。
\stopEXERCISE

\startANSWER
在計算 \m{W^m} 時,一旦對角線出現負值, \m{m} 即爲所求。
\stopANSWER

\stopsection
