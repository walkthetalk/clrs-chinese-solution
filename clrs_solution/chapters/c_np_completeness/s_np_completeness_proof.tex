\startsection[
  title={NP-completeness proofs},
]

%e34.4-1
\startEXERCISE
考慮一下在定理 34.9 的證明過程中運用直接(非多項式時間)歸約。
描述一個規模爲 \m{n} 的電路,
運用這種歸約思想將其轉換爲一個公式時,
能產生一個規模爲 \m{n} 的指數的公式。
\stopEXERCISE

\startANSWER
\TODO{略。}
\stopANSWER

%e34.4-2
\startEXERCISE
寫出將定理 34.10 中的方法應用於公式(34.3)時所得到的 \ALGO{3-CNF} 公式。
\stopEXERCISE

\startANSWER
\TODO{略。}
\stopANSWER

%e34.4-3
\startEXERCISE
Jagger 教授提出,在定理 34.10 的證明中,
可以僅利用真值表技術而無需其他步驟,
就能證明 \m{\text{\ALGO{SAT}} \le_P \text{\ALGO{3-CNF-SAT}}}。
也就是說,這位教授試圖取布爾公式 \m{\phi},
形成有關其變量的真值表,
根據其真值表導出一個 \ALGO{3-CNF} 形式的、
等價於 \m{\neg \phi} 的公式,
再對公式取反,並運用 DeMorgan 定律,
從而可以得到一個等價於 \m{\phi} 的 \ALGO{3-CNF} 公式。
證明:這一策略不能產生多項式時間的歸約。
\stopEXERCISE

\startANSWER
\TODO{略。}
\stopANSWER

%e34.4-4
\startEXERCISE
證明:確定某一布爾公式是否爲重言式這一問題對 \ALGO{co-NP} 來說是完全的。
(\hint 見\refexercise{34.3-7}。)
\stopEXERCISE

\startANSWER
\TODO{略。}
\stopANSWER

%e34.4-5
\startEXERCISE
證明:可以在多項式時間內求解析取範式形式的布爾公式的可滿足性。
\stopEXERCISE

\startANSWER
\TODO{略。}
\stopANSWER

%e34.4-6
\startEXERCISE
假設已知某一多項式時間算法可以判定公式的可滿足性。
請說明如何利用這一算法在多項式時間內找出可滿足性賦值。
\stopEXERCISE

\startANSWER
\TODO{略。}
\stopANSWER

%e34.4-7
\startEXERCISE
設 \ALGO{2-CNF-SAT} 是 \ALGO{CNF} 範式的、
每個子句中恰有兩個文字的可滿足公式的集合,
證明: \m{\text{\ALGO{2-CNF-SAT}}\in P}。
儘可能優化你的算法效率。
(\hint 注意 \m{x\vee y} 與 \m{\neg x\rightarrow y} 是等價的。
將 \ALGO{2-CNF-SAT} 歸約爲一個在有向圖上高效可解的問題。)
\stopEXERCISE

\startANSWER
\TODO{略。}
\stopANSWER

\stopsection
