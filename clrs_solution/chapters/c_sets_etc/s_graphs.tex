\startsection[
  title={Graphs},
  reference=section:appendix_graphs,
]

%eB.4-1
\startEXERCISE
在一個教職工聚會中,與會者互相握手問候彼此,
每位教授會記住他握手的次數。
在聚會的最後,系主任將所有教授握手的次數相加。
通過證明下面的{\EMP 握手定理}來說明系主任得到的結果是偶數:
如果 \m{G=(V,E)} 是無向圖,則有:
\startformula
\sum_{v\in V} \mfunction{degree}(v) = 2|E|
\stopformula
\stopEXERCISE

\startANSWER
\TODO{略。}
\stopANSWER

%eB.4-2
\startEXERCISE
證明:如果無向圖或有向圖在兩個頂點 \m{u} 和 \m{v} 之間包含一條路徑,
則該圖一定包含一條 \m{u} 與 \m{v} 之間的簡單路徑。
證明:如果一個有向圖包含環,則他一定包含一個簡單環。
\stopEXERCISE

\startANSWER
\TODO{略。}
\stopANSWER

%eB.4-3
\startEXERCISE
證明:任意連通無向圖 \m{G=(V,E)} 滿足 \m{|E|\ge |V|-1}。
\stopEXERCISE

\startANSWER
\TODO{略。}
\stopANSWER

%eB.4-4
\startEXERCISE
驗證在一個無向圖中,“從……可達”關係是圖中頂點的等價關係。
等價關係的三個特性中,
哪些對有向圖點集上的“從……可達”關係成立?
\stopEXERCISE

\startANSWER
\TODO{略。}
\stopANSWER

%eB.4-5
\startEXERCISE
圖 B-2 (a)中有向圖的無向版本是什麼?
圖 B-2 (b)中無向圖的有向版本是什麼?
\stopEXERCISE

\startANSWER
\TODO{略。}
\stopANSWER

%eB.4-6
\startEXERCISE\DIFFICULT
證明:若令超圖中的邊與點的關聯關係對應二分圖中的鄰接關係,
則超圖可表示爲二分圖。
(\hint 令二分圖中的一個頂點集對應超圖的頂點集,
並令二分圖的另一頂點集對應超邊集。)
\stopEXERCISE

\startANSWER
\TODO{略。}
\stopANSWER

\stopsection
