\startcomponent c_front

\title{前言}

讀研時曾選修此課,個別章節亦曾略讀,
工作後深悔蹉跎歲月,復有重學之念。
幸喜網絡日興,查閱資料日益便捷,方能維持斷續之進展。
經數載之功,終得彙編成冊,以期能對各位有所助益。
然才疏學淺,心力有窮,部分問題終不得解,以待高賢。

時至今日,最大成果當數此冊,其間亦不斷學習 {\ConTeXt} 排版,
 metapost 繪圖,甚而據 metapost 寫就一擴展,專用於此冊插圖。
然煩亂之心漸重,一個簡單問題,落實於代碼之上,卻要七拐八繞方能實現。
不同編程語言對正則算式之表述亦五花八門,
宏語言中字符轉義、展開等問題也頗費心力。

於工具而言,強大固然不可或缺,然使用之便利更不可不察。
“所想即所得”怕本是一個謊言,
“所見即所得”纔是應有之面貌。
若當真“所想即所得”,則無需查看輸出,一氣呵成,
單就公式斷行而言,此論即爲大繆。
世界本就複雜,妄圖以簡單規則描述,註定碰壁。
內容與排版絕無法完全隔離,或 texworks 纔是未來。

\title{目錄}
\placelist[part,chapter,section]
%\completecontent
\title{圖}
\placelist[figure]
\title{表}
\placelist[table]

\stopcomponent
