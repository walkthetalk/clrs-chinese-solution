\startsection[
  title={The tails of the binomial distribution},
]

%eC.5-1
\startEXERCISE\DIFFICULT
有一枚均勻硬幣,一種是拋擲 \m{n} 次都是反面朝上,
另一種是拋擲 \m{4n} 次正面朝上的次數小於 \m{n},
哪個的概率小?
\stopEXERCISE

\startANSWER
第一種的概率爲 \m{\left(\frac{1}{2}\right)^{n}}。

下面看第二種,
\startformula\startmathalignment
\NC \Pr\{X<n\}
    \NC = \sum_{i=0}^{n-1}b(i;4n,1/2) \NR
\NC \NC = \left(\frac{1}{2}\right)^{4n}\sum_{i=0}^{n-1}\binom{4n}{i} \NR
\stopmathalignment\stopformula
等價於比較 \m{8^n} 和 \m{\sum_{i=0}^{n-1}\binom{4n}{i}}。

列出兩個值的前幾項,可以看出兩者的比值一直在增長。
前者是等比數列,
而後者相鄰兩個數值的比例也一直在增長(除去 \m{n=1,2} 的情況),
 \m{n\ge 18} 時,後者超過了前者,
也就是說 \m{n<18} 時,前者概率較大,推測 \m{n\ge 18} 時,後者概率較大。
\TODO{還有待證明}。


\startxtable[
    option=max,
    align={left,lohi},
    split=yes,
    header=repeat,
    footer=repeat,
    offset=.25em,
]

% head
\startxtablehead[frame=off,bottomframe=on]
\startxrow[foregroundstyle=bold,align={middle}]
  \xmcell[rightframe=on]{n}
  \processcommalist[
  A_n = 8^n,
  B_n = \sum_{i=0}^{n-1}\binom{4n}{n},
  \frac{B_n}{B_{n-1}},
  \frac{B_n}{A_n}
  ]\xmcell
\stopxrow
\stopxtablehead

% body
\startxtablebody[frame=off]
\startxrow \xmcell[rightframe=on]{ 1}\processcommalist[                 8,                  1, 1.000, 0.125]\xmcell \stopxrow
\startxrow \xmcell[rightframe=on]{ 2}\processcommalist[                64,                  9, 9.000, 0.141]\xmcell \stopxrow
\startxrow \xmcell[rightframe=on]{ 3}\processcommalist[               512,                 79, 8.778, 0.154]\xmcell \stopxrow
\startxrow \xmcell[rightframe=on]{ 4}\processcommalist[              4096,                697, 8.823, 0.170]\xmcell \stopxrow
\startxrow \xmcell[rightframe=on]{ 5}\processcommalist[             32768,               6196, 8.890, 0.189]\xmcell \stopxrow
\startxrow \xmcell[rightframe=on]{ 6}\processcommalist[            262144,              55455, 8.950, 0.212]\xmcell \stopxrow
\startxrow \xmcell[rightframe=on]{ 7}\processcommalist[           2097152,             499178, 9.001, 0.238]\xmcell \stopxrow
\startxrow \xmcell[rightframe=on]{ 8}\processcommalist[          16777216,            4514873, 9.045, 0.269]\xmcell \stopxrow
\startxrow \xmcell[rightframe=on]{ 9}\processcommalist[         134217728,           40999516, 9.081, 0.305]\xmcell \stopxrow
\startxrow \xmcell[rightframe=on]{10}\processcommalist[        1073741824,          373585604, 9.112, 0.348]\xmcell \stopxrow
\startxrow \xmcell[rightframe=on]{11}\processcommalist[        8589934592,         3414035527, 9.139, 0.397]\xmcell \stopxrow
\startxrow \xmcell[rightframe=on]{12}\processcommalist[       68719476736,        31278197839, 9.162, 0.455]\xmcell \stopxrow
\startxrow \xmcell[rightframe=on]{13}\processcommalist[      549755813888,       287191809724, 9.182, 0.522]\xmcell \stopxrow
\startxrow \xmcell[rightframe=on]{14}\processcommalist[     4398046511104,      2642070371194, 9.200, 0.601]\xmcell \stopxrow
\startxrow \xmcell[rightframe=on]{15}\processcommalist[    35184372088832,     24347999094724, 9.215, 0.692]\xmcell \stopxrow
\startxrow \xmcell[rightframe=on]{16}\processcommalist[   281474976710656,    224723513577529, 9.230, 0.798]\xmcell \stopxrow
\startxrow \xmcell[rightframe=on]{17}\processcommalist[  2251799813685248,   2076978797223820, 9.242, 0.922]\xmcell \stopxrow
\startxrow \xmcell[rightframe=on]{18}\processcommalist[ 18014398509481984,  19220104372823340, 9.254, 1.067]\xmcell \stopxrow
\startxrow \xmcell[rightframe=on]{19}\processcommalist[144115188075855872, 178061257422521452, 9.264, 1.236]\xmcell \stopxrow
\stopxtablebody

\stopxtable

\stopANSWER

%eC.5-2
\startEXERCISE\DIFFICULT
證明推論 C.6 和推論 C.7。

附推論 C.6:考慮 \m{n} 次 Bernoulli 試驗,成功概率爲 \m{p}。
令 \m{X} 爲表示成功次數的隨機變量,
則對於 \m{np<k<n},
獲得多於 \m{k} 次成功的概率爲:
\startformula
\Pr\{X>k\} = \sum_{i=k+1}^{n} b(i;n,p) < \frac{(n-k)p}{k-np} b(k;n,p)
\stopformula

附推論 C.7:考慮 \m{n} 次 Bernoulli 試驗,成功概率爲 \m{p},
失敗概率爲 \m{q=1-p},令 \m{X} 爲表示成功次數的隨機變量,
則對於 \m{(np+n)/2<k<n}, \m{\Pr\{X>k\} < \frac{1}{2}\Pr\{X>k-1\}}。
\stopEXERCISE

\startANSWER
先來證明 C.6。先來看下面公式:
\startformula\startmathalignment
\NC \frac{b(i+1;n,p)}{b(i;n,p)}
    \NC = \frac{\binom{n}{i+1}p^{i+1}(1-p)^{n-i-1}}{\binom{n}{i}p^i(1-p)^{n-i}} \NR
\NC \NC = \frac{n-i}{i+1} \frac{p}{1-p} \NR
\startintertext
由於 \m{k\le i<n} (可加強爲 \m{k<i<n}):
\stopintertext
\NC \NC \le \frac{n-k}{k+1} \frac{p}{1-p} \qquad \qquad \text{令此式爲 \m{x}。}\NR
\startintertext
由於 \m{np<k<n}:
\stopintertext
\NC \NC < \frac{n-np}{np+1}\frac{p}{1-p} \NR
\NC \NC = \frac{np}{np+1} \NR
\NC \NC < 1 \NR
\stopmathalignment\stopformula

根據上面推導可知 \m{b(i+1;n,p)<x b(i;n,p)},同理:
\startformula
b(i+2;n,p)<x b(i+1;n,p) < x^2 b(i;n,p)
\stopformula
以此類推,可知:
\startformula\startmathalignment
\NC \Pr\{X>k\} = \sum_{i=k+1}^{n} b(i;n,p)
    \NC < x b(k;n,p) + x^2 b(k;n,p) + \ldots + x^{n-k} b(k;n,p) \NR
\NC \NC < b(k;n,p) \sum_{i=1}^{n-k}x^i \NR
\NC \NC < b(k;n,p) \sum_{i=1}^{\infty} x^i \NR
\NC \NC = b(k;n,p) \frac{x}{1-x} \NR
\startintertext
將 \m{x} 代入得:
\stopintertext
\NC \NC = b(k;n,p) \frac{(n-k)p}{1-p+k-np} \NR
\NC \NC < \frac{(n-k)p}{k-np} b(k;n,p) \NR
\stopmathalignment\stopformula

然後證明 C.7。由於 \m{\frac{np+n}{2}<k<n},
\startformula
\left(\frac{n-k}{k-np}\right) p
  < \left(\frac{n-\frac{np+n}{2}}{\frac{np+n}{2}-np}\right)p
  = p < 1
\stopformula
根據 C.6 可知:
\startformula
\Pr\{X>k\}
  = \sum_{i=k+1}^{n} b(i;n,p)
  < \frac{(n-k)p}{k-np} b(k;n,p)
  < b(k;n,p)
\stopformula
現在就可以證明 C.7 了:
\startformula\startmathalignment
\NC \Pr\{X>k\}
    \NC = \frac{1}{2} \left(\Pr\{X>k\} + \Pr\{X>k\}\right) \NR
\NC \NC < \frac{1}{2} \left(\Pr\{X>k\} + b(k;n,p)\right) \NR
\NC \NC = \frac{1}{2} \Pr\{X>k-1\} \NR
\stopmathalignment\stopformula
\stopANSWER

%eC.5-3
\startEXERCISE\DIFFICULT
證明:對於所有滿足 \m{0<k<na/(a+1)} 的 \m{a>0} 和 \m{k},有
\startformula
\sum_{i=0}^{k-1}\binom{n}{i}a^i
  < (a+1)^n\frac{k}{na-k(a+1)} b(k;n,a/(a+1))
\stopformula
\stopEXERCISE

\startANSWER
因爲 \m{a>0},令 \m{p = \frac{a}{a+1} < 1},則:
\startformula
q = 1-p = 1 - \frac{a}{a+1} = \frac{1}{a+1}
\stopformula
現在考慮二項式 \m{b(i;n,p)} 中的係數 \m{p^i q^{n-i}}:
\startformula
p^i q^{n-i} = \left(\frac{a}{a+1}\right)^i
              \left(\frac{1}{a+1}\right)^{n-i}
            = \frac{a^i}{(a+1)^n}
\stopformula
從中可以看出 \m{a^i = (a+1)^n p^i q^{n-i}},因此:
\startformula\startmathalignment
\NC \sum_{i=0}^{k-1}\binom{n}{i}a^i
    \NC = (a+1)^n \sum_{i=0}^{k-1}\binom{n}{i}p^i q^{n-i} \NR
\NC \NC = (a+1)^n \Pr\{X<k\} \NR
\startintertext
根據定理 C.4 (已滿足 \m{0<k<np}):
\stopintertext
\NC \NC < (a+1)^n \frac{kq}{np-k} b(k;n,p) \NR
\startintertext
將 \m{p,q} 代入可得:
\stopintertext
\NC \NC = (a+1)^n \frac{k}{na-k(a+1)} b(k;n,\frac{a}{a+1})
\stopmathalignment\stopformula
\stopANSWER

%eC.5-4
\startEXERCISE\DIFFICULT
證明:若 \m{0<k<np},其中 \m{0<p<1} 且 \m{q=1-p},則
\startformula
\sum_{i=0}^{k-1}p^i q^{n-i}
  < \frac{kq}{np-k}
    \left(\frac{np}{k}\right)^k
    \left(\frac{nq}{n-k}\right)^{n-k}
\stopformula
\stopEXERCISE

\startANSWER
\startformula\startmathalignment
\NC \sum_{i=0}^{k-1}p^iq^{n-i}
    \NC < \sum_{i=0}^{k-1}\binom{n}{i}p^i q^{n-i} \NR
\NC \NC = \Pr\{X < k\} \qquad \text{(C.4)} \NR
\NC \NC < \frac{kq}{np - k} b(k;n,p) \qquad \text{(C.1)} \NR
\NC \NC < \frac{kq}{np - k}
          \left(\frac{np}{k}\right)^k
          \left(\frac{nq}{n-k}\right)^{n-k}
\stopmathalignment\stopformula
\stopANSWER

%eC.5-5
\startEXERCISE\DIFFICULT
利用定理 C.8 證明:對於 \m{r>n-\mu},
\startformula
\Pr\{\mu - X \ge r\} \le \left(\frac{(n-\mu)e}{r}\right)^r
\stopformula
附定理 C.8:考慮 \m{n} 次 Bernoulli 試驗,其中第 \m{i} 次成功概率爲 \m{p_i},
失敗概率爲 \m{q_i = 1 - p_i}, \m{i=1,2,\ldots,n}。
令隨機變量 \m{X} 表示成功的總次數,令 \m{\mu = \E[X]}。
那麼,對於 \m{r>\mu},
\startformula
\Pr\{X-\mu\ge r\} \le \left(\frac{\mu e}{r}\right)^r
\stopformula

類似地,利用推論 C.9 證明:對於 \m{r>n-np},
\startformula
\Pr\{np - X\ge r\}\le \left(\frac{nqe}{r}\right)^r
\stopformula
附定理 C.9:考慮 \m{n} 次 Bernoulli 試驗,
每次試驗成功概率爲 \m{p},失敗概率爲 \m{q = 1 - p}。
令隨機變量 \m{X} 表示成功的總次數。
那麼,對於 \m{r>np},
\startformula
\Pr\{X-np\ge r\}
  = \sum_{k=\left\lceil np+r\right\rceil}^{n} b(k;n,p)
  \le \left(\frac{npe}{r}\right)^r
\stopformula
\stopEXERCISE

\startANSWER
令 \m{Y=n-X},即失敗次數,則 \m{\nu = \E[Y] = \E[n - X] = n - \E[x] = n - \mu}。
根據定理 C.8:
\startformula\startmathalignment[n=1]
\NC \Pr\{Y - \nu > r\} \le \left(\frac{\nu e}{r}\right)^r \NR
\NC \Downarrow \NR
\NC \Pr\{\mu - X \ge r\} \le \left(\frac{(n - \mu)e}{r}\right)^r \NR
\stopmathalignment\stopformula

另一個證明類似。
\stopANSWER

%eC.5-6
\startEXERCISE
考慮 \m{n} 次 Bernoulli 試驗,其中第 \m{i} 次成功概率爲 \m{p_i},
失敗概率爲 \m{q_i = 1 - p_i}, \m{i=1,2,\ldots,n}。
令隨機變量 \m{X} 表示成功的總次數,令 \m{\mu = \E[X]}。
證明:對於 \m{r\ge 0},有
\startformula
\Pr\{X-\mu\ge r\} \le e^{-r^2/2n}
\stopformula
(\hint 證明 \m{p_i e^{\alpha q_i} + q_i e^{-\alpha p_i} \le e^{\alpha^2/2}}。
然後根據定理 C.8 的證明思路,利用該不等式代替 C.45。)
\stopEXERCISE

\startANSWER
先證明提示所給不等式 \m{p_i e^{\alpha q_i} + q_i e^{-\alpha p_i} \le e^{\alpha^2/2}}。
令:
\startformula
f(x) = e^{x^2/2} - (pe^{qx} + qe^{-px})
\stopformula
我們需要證明 \m{f(x)} 在 \m{x\ge 0} 時單調遞增,
即要證明 \m{f'(x) > 0}:
\startformula
f'(x) = xe^{x^2/2} - pq(e^{qx} - e^{-px})
\stopformula
要證明 \m{f'(x)>0},我們先證明 \m{f''(x) > 0}:
\startformula\startmathalignment[align={middle, right}]
\NC f''(x) = e^{x^2/2} + x^2 e^{x^2/2} - pq(qe^{qx} + pe^{-px}) > 0 \NC \NR
\NC \Uparrow \NC \NR
\NC e^{x^2/2} > pq(qe^{qx} + pe^{-px}) \NC x^2 e^{x^2/2} \ge 0 \NR
\NC \Uparrow \NC \NR
\NC e^{x^2/2} > \frac{1}{4}(qe^{qx} + pe^{-px}) \NC pq = p(1-p) \le 1/4\NR
\NC \Uparrow \NC \NR
\NC 4e^{x^2/2} > e^{qx} + e^{-px} \NC p<1,q<1\NR
\NC \Uparrow \NC \NR
\NC 4e^{x^2/2} > e^x + 1 \NC e^{qx} + e^{-px} = e^{-px} (e^x + 1) < e^x + 1 \NR
\NC \Uparrow \NC \NR
\NC 3e^{x^2/2} > e^x \NC e^{x^2/2} > 1 \NR
\NC \Uparrow \NR
\NC 3e^{x^2/2 - x} > 1 \NC \NR
\NC \Uparrow \NR
\NC 3e^{\frac{(x-1)^2 - 1}{2}} > 1 \NC \NR
\NC \Uparrow \NC \NR
\NC 3 > \sqrt{e} \NC\NR
\stopmathalignment\stopformula

對於 \m{x\ge 0},由於 \m{f''(0)\ge 0},因此 \m{f'(x)} 單調遞增。
由於 \m{f'(x)\ge 0},所以 \m{f(x)} 單調遞增。
因此 \m{f(x)\ge 0},即提示中所給不等式成立。
據此可知:
\startformula\startmathalignment
\NC \E[e^{\alpha (X_i - p_i)}]
    \NC = e^{\alpha(1-p_i)}p_i + e^{\alpha(0-p_i)}q_i \NR
\NC \NC = p_i e^{\alpha q_i} + q_i e^{- \alpha p_i} \NR
\NC \NC \le e^{\alpha^2/2} \NR
\stopmathalignment\stopformula

將其代入下式:
\startformula\startmathalignment
\NC \E[e^{\alpha(X - \mu)}]
    \NC = \prod_{i=1}^n \E[e^{\alpha(X_i - p_i)}] \NR
\NC \NC \le \prod_{i=1}^n \exp(\alpha^2 / 2) \NR
\NC \NC = \exp(n \alpha^2/2) \NR
\stopmathalignment\stopformula

根據 C.43 和 C.44:
\startformula\startmathalignment
\NC \Pr\{X-\mu\ge r\}
    \NC = \Pr\{e^{\alpha(X-\mu)} \ge e^{\alpha r}\} \NR
\NC \NC \le \E[e^{\alpha(X-\mu)}] e^{-\alpha r} \NR
\startintertext
代入上面的不等式:
\stopintertext
\NC \NC \le \exp({n \alpha^2/2}) \exp({-\alpha r}) \NR
\startintertext
要是其值最小,即使 \m{n\alpha^2/2 - \alpha r} 最小,
即其導數 \m{n\alpha - r = },求解得 \m{\alpha = r/n}:
\stopintertext
\NC \NC = \exp({\frac{nr^2}{2n^2} - \frac{r^2}{n}}) \NR
\NC \NC = \exp(-\frac{r^2}{2n}) \NR
\stopmathalignment\stopformula
\stopANSWER

%eC.5-7
\startEXERCISE\DIFFICULT
證明:選擇 \m{\alpha = \ln(r/\mu)} 可使不等式 C.47 右側取最小值。
\stopEXERCISE

\startANSWER
\startformula
\Pr\{X-\mu\ge r\} \le \exp(\mu e^{\alpha} - \alpha r)
\stopformula
要想使右式最小,可對 \m{\alpha} 求其一階導數,使其爲 0 即可。
\startformula
\frac{d}{d\alpha}\exp(\mu e^{\alpha} - \alpha r)
  = (\mu e^{\alpha} - r) \exp(\mu e^{\alpha} - \alpha r)
  = 0
\stopformula
解得 \m{\alpha = \ln (r/\mu)}。

可通過二階導數來證明其正確性。也可以按如下方式進行說明:
當 \m{\alpha > \ln (r/\mu)} 時,一階導數大於 0,原函數遞增,
當 \m{\alpha < \ln (r/\mu)} 時,一階導數大於 0,原函數遞減,
因此原函數在 \m{\alpha = \ln (r/\mu)} 處取得最小值。
\stopANSWER

\stopsection
