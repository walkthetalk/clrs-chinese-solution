\startsubject[
  title={Problems},
]

%p3-1
\startPROBLEM
(多項式的漸近行爲)
設\m{p(n) = \sum_{i = 0}^{d} {a_i n^i}}是關於\m{n}的\m{d}次多項式,
其中\m{a_d > 0},\m{k}是一個常量。用漸進記號的定義證明下列性質。
\startigBase[a]
\item 如果\m{k \geq d},那麼\m{p(n) = O(n^k)}。
\item 如果\m{k \leq d},那麼\m{p(n) = \Omega(n^k)}。
\item 如果\m{k = d},那麼\m{p(n) = \Theta(n^k)}。
\item 如果\m{k > d},那麼\m{p(n) = o(n^k)}。
\item 如果\m{k < d},那麼\m{p(n) = \omega(n^k)}。
\stopigBase
\stopPROBLEM

\startANSWER
取 \m{c = a_d + b},滿足下列不等式:
\startformula
p(n) = \sum_{i = 0}^{d}a_i n^i = a_d n^d + a_{d-1}n^{d-1} + \ldots + a_1 n + a_0 \leq cn^d
\stopformula
兩邊同除以 \m{n^d},有:
\startformula
c = a_d + b \geq a_d + \frac{a_{d-1}}n + \frac{a_{d-2}}{n^2} + \ldots + \frac{a_0}{n^d}
\stopformula
即
\startformula
b \geq \frac{a_{d-1}}n + \frac{a_{d-2}}{n^2} + \ldots + \frac{a_0}{n^d}
\stopformula
如果使 \m{b = 1},則選取 \m{n_0},使得:
\startformula
n_0 = \max(da_{d-1}, d\sqrt{a_{d-2}}, \ldots, d\sqrt[d]{a_0})
\stopformula
則有:
\startformula
p(n) \leq cn^d \quad \text{對於 } n \geq n_0
\stopformula
即 \m{O(n^d)} 的定義。如果選 \m{b = -1},則可得 \m{\Omega(n^d)},綜合可得 \m{\Theta(n^d)}。
另外兩個的證明類似。
\stopANSWER

\startPROBLEM
(相對漸進增長)
對於下表中每對算式 \m{(A, B)},指明 \m{A} 是否是 \m{B} 的 \m{O}、\m{o}、\m{\Omega}、\m{\omega}、\m{\Theta}。
假定 \m{k \geq 1},\m{\epsilon > 0} 且 \m{c > 1},均爲常量。

\bTABLE[align=center]
\bTABLEhead
\bTR
	\bTH \m{A} \eTH
	\bTH \m{B} \eTH
	\bTH \m{O} \eTH
	\bTH \m{o} \eTH
	\bTH \m{\Omega} \eTH
	\bTH \m{\omega} \eTH
	\bTH \m{\Theta} \eTH
\eTR
\eTABLEhead
\bTABLEbody
\bTR
	\bTD \m{\lg^kn} \eTD
	\bTD \m{n^\epsilon} \eTD
	\bTD\startANSWER yes \stopANSWER\eTD
	\bTD\startANSWER yes \stopANSWER\eTD
	\bTD\startANSWER no \stopANSWER\eTD
	\bTD\startANSWER no \stopANSWER\eTD
	\bTD\startANSWER no \stopANSWER\eTD
\eTR
\bTR
	\bTD \m{n^k} \eTD
	\bTD \m{c^n} \eTD
	\bTD\startANSWER yes \stopANSWER\eTD
	\bTD\startANSWER yes \stopANSWER\eTD
	\bTD\startANSWER no \stopANSWER\eTD
	\bTD\startANSWER no \stopANSWER\eTD
	\bTD\startANSWER no \stopANSWER\eTD
\eTR
\bTR
	\bTD \m{\sqrt{n}} \eTD
	\bTD \m{n^{\sin n}} \eTD
	\bTD\startANSWER no \stopANSWER\eTD
	\bTD\startANSWER no \stopANSWER\eTD
	\bTD\startANSWER no \stopANSWER\eTD
	\bTD\startANSWER no \stopANSWER\eTD
	\bTD\startANSWER no \stopANSWER\eTD
\eTR
\bTR
	\bTD \m{2^n} \eTD
	\bTD \m{2^{n/2}} \eTD
	\bTD\startANSWER no \stopANSWER\eTD
	\bTD\startANSWER no \stopANSWER\eTD
	\bTD\startANSWER yes \stopANSWER\eTD
	\bTD\startANSWER yes \stopANSWER\eTD
	\bTD\startANSWER no \stopANSWER\eTD
\eTR
\bTR
	\bTD \m{n^{\lg c}} \eTD
	\bTD \m{c^{\lg n}} \eTD
	\bTD\startANSWER yes \stopANSWER\eTD
	\bTD\startANSWER no \stopANSWER\eTD
	\bTD\startANSWER yes \stopANSWER\eTD
	\bTD\startANSWER no \stopANSWER\eTD
	\bTD\startANSWER yes \stopANSWER\eTD
\eTR
\bTR
	\bTD \m{\lg(n!)} \eTD
	\bTD \m{\lg(n^n)} \eTD
	\bTD\startANSWER yes \stopANSWER\eTD
	\bTD\startANSWER no \stopANSWER\eTD
	\bTD\startANSWER yes \stopANSWER\eTD
	\bTD\startANSWER no \stopANSWER\eTD
	\bTD\startANSWER yes \stopANSWER\eTD
\eTR
\eTABLEbody
\eTABLE
\stopPROBLEM

\startPROBLEM
(根據漸進增長率排序)
\startigBase[a]
\startitem
根據增長的階爲下列函數排序,即求出滿足
\m{g_1 = \Omega(g_2)}、\m{g_2 = \Omega(g_3)}、\m{\ldots}、\m{g_{29} = \Omega(g_{30})}
的函數的一種排列 \m{g_1, g_2, \ldots, g_{30}}。
並將這些函數劃分成等價類,當且僅當 \m{f(n) = \Theta(g(n))}時,\m{f(n)}和\m{g(n)}才再同一類中。

\bTABLE[align=center]
\bTR \bTD \m{\lg(\lg^{\ast}n)} \eTD \bTD \m{2^{\lg^{\ast}n}} \eTD \bTD \m{(\sqrt{2})^{\lg{n}}} \eTD \bTD \m{n^2} \eTD \bTD \m{n!} \eTD \bTD \m{(\lg{n})!} \eTD \eTR
\bTR \bTD \m{(\frac{3}{2})^n} \eTD \bTD \m{n^3} \eTD \bTD \m{\lg^2{n}} \eTD \bTD \m{\lg(n!)} \eTD \bTD \m{2^{2^n}} \eTD \bTD \m{n^{1/\lg{n}}} \eTD \eTR
\bTR \bTD \m{\ln{\ln{n}}} \eTD \bTD \m{\lg^{\ast}n} \eTD \bTD \m{n \cdot 2^n} \eTD \bTD \m{n^{\lg\lg{n}}} \eTD \bTD \m{\ln{n}} \eTD \bTD \m{1} \eTD \eTR
\bTR \bTD \m{2^{\lg{n}}} \eTD \bTD \m{(\lg{n})^{\lg{n}}} \eTD \bTD \m{e^n} \eTD \bTD \m{4^{\lg{n}}} \eTD \bTD \m{(n + 1)!} \eTD \bTD \m{\sqrt{\lg{n}}} \eTD \eTR
\bTR \bTD \m{\lg^{\ast}(\lg{n})} \eTD \bTD \m{2^{\sqrt{2\lg{n}}}} \eTD \bTD \m{n} \eTD \bTD \m{2^n} \eTD \bTD \m{n\lg{n}} \eTD \bTD \m{2^{2^{n + 1}}} \eTD \eTR
\eTABLE
\stopitem

\startANSWER
\startformula\startalign
\NC (\sqrt{2})^{\lg{n}} \NC = \sqrt{n} \NR
\NC \sqrt{2}^{\lg{n}} \NC = 2^{1/2\lg{n}} = 2^{\lg{\sqrt{n}}} = \sqrt{n} \NR
\NC n! < n^n \NC = 2^{\lg{n^n}} = 2^{n\lg{n}} \NR
\NC n^{1/\lg{n}} \NC = n^{\log_n{2}} = 2 \NR
\NC n^{\lg{\lg{n}}} \NC = (2^{\lg{n}})^{\lg\lg{n}} = (2^{\lg\lg{n}})^{\lg{n}} = (\lg{n})^{\lg{n}} \NR
\NC \lg^2{n} \NC = 2^{\lg{\lg^2{n}}} = o(2^{\sqrt{2\lg{n}}}) \NR
\stopalign\stopformula
\startcolumns[n=3,blank=small,distance=2em,balance=yes]
\startigBase[n]
\item \m{1 = n^{1/\lg n}}
\item \m{\lg(\lg^{\ast}n)}
\item \m{\lg^{\ast}n\simeq \lg^{\ast}{\lg{n}}}
\item \m{2^{\lg^{\ast}n}}
\item \m{\ln{\ln{n}}}
\item \m{\sqrt{\lg{n}}}
\item \m{\ln{n}}
\item \m{\lg^2{n}}
\item \m{2^{\sqrt{2\lg{n}}}}
\item \m{(\sqrt{2})^{\lg{n}}}
\item \m{n = 2^{\lg{n}}}
\item \m{n\lg{n} \simeq \lg(n!)}
\item \m{n^2 = 4^{\lg{n}}}
\item \m{n^3}
\item \m{n^{\lg\lg{n}} = (\lg{n})^{\lg{n}}}
\item \m{(\frac{3}{2})^n}
\item \m{2^n}
\item \m{n \cdot 2^n}
\item \m{e^n}
\item \m{n!}
\item \m{(n + 1)!}
\item \m{2^{2^n}}
\item \m{2^{2^{n+1}}}
\stopigBase
\stopcolumns
\stopANSWER

\startitem
給出一個非負函數 \m{f(n)},使得對於所有 \m{g_i(n)},\m{f(n)}既不是 \m{O(g_i(n))},也不是 \m{\Omega(g_i(n))}。
\stopitem

\startANSWER
\m{2^{2^{(n + 1)\sin{x}}}}
\stopANSWER
\stopigBase
\stopPROBLEM

\startPROBLEM
(漸進記號的性質)
假設 \m{f(n)} 和 \m{g(n)} 是漸進正函數,證明或反駁下面的每個猜測。
\startigBase[a]
\item \m{f(n) = O(g(n))} 蘊含 \m{g(n) = O(f(n))}。

\startANSWER
錯誤。 \m{n = O(n^2)},但是 \m{n^2 \neq O(n)}。
\stopANSWER

\item \m{f(n) + g(n) = \Theta(\min(f(n), g(n)))}。

\startANSWER
錯誤。 \m{n^2 + n \neq \Theta(min(n^2, n)) = \Theta(n)}。
\stopANSWER

\item 如果對於足夠的 \m{n},有 \m{\lg(g(n))\geq 1} 且 \m{f(n)\geq 1},那麼 \m{f(n) = O(g(n))} 蘊含 \m{\lg(f(n)) = O(lg(g(n)))}。

\startANSWER
正確。 因爲對於給定 \m{n \geq n_0}, \m{f(n) \geq 1}:
\startformula
\exists c, n_0 : \forall n \geq n_0 : 0 \leq f(n) \leq cg(n)
\stopformula
\startformula
   \Downarrow
\stopformula
\startformula
   0 \leq \lg{f(n)} \leq \lg(cg(n)) = \lg{c} + \lg{g(n)}
\stopformula
需要證明:
\startformula
\lg{f(n)} \leq d\lg{g(n)}
\stopformula
很容易找到 \m{d}:
\startformula
d = \frac{\lg{c} + \lg{g(n)}}{\lg{g(n)}} = \frac{\lg{c}}{\lg{g}} + 1 \leq \lg{c} + 1
\stopformula
最後一步顯然成立,因爲 \m{\lg{g(n)} \geq 1}。
\stopANSWER

\item \m{f(n) = O(g(n))} 蘊含 \m{2^{f(n)} = O(2^{g(n)})}。

\startANSWER
錯誤。 \m{2n = O(n)},但是 \m{2^{2n} = 4^n \neq O(2^n)}。
\stopANSWER

\item \m{f(n) = O((f(n))^2)}。

\startANSWER
正確。只要 \m{f(n) \geq 1}, \m{0 \leq f(n) \leq cf^2(n)} 是很自然的。
當然如果對於所有 \m{n}, \m{f(n) < 1},則錯誤,但我們通常不考慮這種函數。
\stopANSWER

\item \m{f(n) = O(g(n))} 蘊含 \m{g(n) = \Omega(f(n))}。

\startANSWER
正確。如果 \m{f(n) = O(g(n))},則 \m{0 \leq f(n) \leq cg(n)},我們只需證明:
\startformula
0 \leq df(n) \leq g(n)
\stopformula
對於 \m{d = 1/c},上式顯然成立。
\stopANSWER

\item \m{f(n) = \Theta(f(n/2))}。

\startANSWER
錯誤。取 \m{f(n) = 2^n},我們需要證明:
\startformula
\exists c_1, c_2, n: \forall n \geq n_0 : 0 \leq c_1 \cdot 2^{n/2} \leq 2^n
   \leq c_2 \cdot 2^{n/2}
\stopformula
顯然不成立。
\stopANSWER

\item \m{f(n) + o(f(n)) = \Theta(f(n))}。

\startANSWER
正確。設 \m{g(n) = o(f(n))},我們需要證明 \m{c_1f(n) \leq f(n) + g(n) \leq c_2f(n)},我們知道:
\startformula
\forall c \exists n_0 \forall n \geq n_0 : cg(n) < f(n)
\stopformula
因此,只需 \m{c_1 = 1}, \m{c_2 = 2} 即可。
\stopANSWER

\stopigBase
\stopPROBLEM

\startPROBLEM
(\m{O}與\m{\Omega}的一些變形)
某些作者用一種與我們稍微不同的方式來定義 \m{\Omega};
假設我們使用 \m{\mathop{\Omega}\limits^{\infty}} 來表示這種定義。
若存在正常量 \m{c},使得對無窮多個整數 \m{n},有 \m{f(n)\geq cg(n)\geq 0},
則稱 \m{f(n) = \mathop{\Omega}\limits^{\infty}(g(n))}。
\startigBase[a]
\item 證明:對漸進非負的任意兩個函數 \m{f(n)} 和 \m{g(n)},
或者 \m{f(n) = O(g(n))} 或者 \m{f(n) = \mathop{\Omega}\limits^{\infty}(g(n))} 或者二者均成立,
然而,如果用 \m{\Omega} 代替 \m{\mathop{\Omega}\limits^{\infty}},那麼該命題卻不成立。

\startANSWER
我們需要比較 \m{cg(n) \leq f(n)};
如果對於正無窮整數均成立,則有 \m{\mathop{\Omega}\limits^{\infty}};
而如果之對有限的整數成立,則設最大值爲 \m{n_0},有:
\startformula
\forall n > n_0: f(n) < cg(n)
\stopformula
足以說明 \m{f(n) = O(g(n))}。

如果 \m{f(n) = g(n)},顯然兩式均成立。

但是對於 \m{\Omega},卻不一定,
如 \m{n = \mathop{\Omega}\limits^{\infty}(n^{\sin{n}})},
但 \m{n \neq \Omega(n^{\sin{n}})}。
\stopANSWER

\item 描述用 \m{\mathop{\Omega}\limits^{\infty}} 代替 \m{\Omega} 來刻畫程序運行時間的潛在優點和缺點。

\startANSWER
TODO
\stopANSWER

\stopigBase

某些作者也用一種稍微不同的方式定義 \m{O};
假設用 \m{O'} 來表示這種可選的定義。
我們稱 \m{f(n) = O'(g(n))},當且儘當 \m{|f(n)| = O(g(n))}。

\startigBase[a,continue]
\item 如果用 \m{O'} 代替 \m{O},但仍使用 \m{\Omega},定理 3.1 中的“當且儘當”的每個方向將出現什麼情況?

\startANSWER
定理 3.1 中的“當且儘當”要改爲“蘊含”,即 \m{\Theta \Rightarrow O'},反向則不成立。
比如函數 \m{f(n) = n \cdot \sin{n}},即爲 \m{O'(n)},但卻不是 \m{O(n)} 或 \m{\Theta(n)}。
\stopANSWER
\stopigBase

有些作者定義 \m{\tilde{O}} 來意指忽略對數因子的 \m{O}:
\startformula
\tilde{O} = \lbrace f(n) : \exists c > 0, k > 0, n_0 > 0 \forall n \geq n_0: 0 \leq f(n) \leq cg(n)\lg^k(n) \rbrace
\stopformula
\startigBase[a,continue]
\item 用類似方式定義 \m{\tilde{\Omega}} 和 \m{\tilde{\Theta}}。
證明定理 3.1 相對應的類似結論。

\startANSWER
\startformula
\tilde{\Omega} = \lbrace f(n) : \exists c, k, n_0 \forall n > n_0 : 0 \leq cg(n) \lg^{-k}(n) \leq f(n) \rbrace
\stopformula
\startformula
\tilde{\Theta} = \lbrace f(n) : \exists c_1, c_2, k_1, k_2, n_0 \forall n > n_0 : 0 \leq c_1g(n) \lg^{-k_1}(n) \leq f(n) \leq c_2g(n) \lg^{k_2}(n)\rbrace
\stopformula
\stopANSWER
\stopigBase

\stopPROBLEM

\startPROBLEM
(多重函數)我們可以把用於函數 \m{\lg^{\ast}} 中的重復運算符 \m{\ast} 應用於實數集上的任意單調遞增函數 \m{f(n)}。
對給定的常量 \m{c \in R},我們定義多重函數 \m{f_c^{\ast}} 爲:
\startformula
f_c^{\ast}(n) = \min \lbrace i \geq 0 : f^{(i)}(n) \leq c \rbrace
\stopformula
該函數不必在所有情況下都爲良定義的。
換句話說, \m{f_c^{\ast}(n)} 的值就是爲將其參數縮小至 \m{c} 或更小所需要將函數 \m{f} 重復應用的數目。
對如下每個函數 \m{f(n)} 和常量 \m{c},給出 \m{f_c^{\ast}(n)} 的一個盡量緊確的界。

\bTABLE[align=center]
\bTABLEhead
\bTR
	\bTH \m{f(n)} \eTH
	\bTH \m{c} \eTH
	\bTH \m{f_c^{\ast}(n)} \eTH
\eTR
\eTABLEhead
\bTABLEbody
\bTR
	\bTD \m{n - 1} \eTD
	\bTD \m{0} \eTD
	\bTD\startANSWER \m{\Theta(n)} \stopANSWER\eTD
\eTR
\bTR
	\bTD \m{\lg{n}} \eTD
	\bTD \m{1} \eTD
	\bTD\startANSWER \m{\Theta(\lg^{\ast}n)} \stopANSWER\eTD
\eTR
\bTR
	\bTD \m{n/2} \eTD
	\bTD \m{1} \eTD
	\bTD\startANSWER \m{\Theta(\lg{n})} \stopANSWER\eTD
\eTR
\bTR
	\bTD \m{n/2} \eTD
	\bTD \m{2} \eTD
	\bTD\startANSWER \m{\Theta(\lg{n})} \stopANSWER\eTD
\eTR
\bTR
	\bTD \m{\sqrt{n}} \eTD
	\bTD \m{2} \eTD
	\bTD\startANSWER \m{\Theta(\lg\lg{n})} \stopANSWER\eTD
\eTR
\bTR
	\bTD \m{\sqrt{n}} \eTD
	\bTD \m{1} \eTD
	\bTD\startANSWER 無法收斂 \stopANSWER\eTD
\eTR
\bTR
	\bTD \m{n^{1/3}} \eTD
	\bTD \m{2} \eTD
	\bTD\startANSWER \m{\Theta(\log_3\lg{n})} \stopANSWER\eTD
\eTR
\bTR
	\bTD \m{n/\lg{n}} \eTD
	\bTD \m{2} \eTD
	\bTD\startANSWER \m{\omega(\lg\lg{n}), o(\lg{n})} \stopANSWER\eTD
\eTR
\eTABLEbody
\eTABLE

\stopPROBLEM

\stopsubject
