\startsection[
  title={Asymptotic notation},
]

\startEXERCISE
證明 \m{\max(f(n),g(n)) = \Theta(f(n)+g(n))}。
\stopEXERCISE
\startANSWER
由於單調遞增,則:
\startformula\startalign[n=3]
\NC \exists n_1, n_2: \NC f(n) \geq 0 \NC \quad\text{若 \m{n > n_1};} \NR
\NC                   \NC g(n) \geq 0 \NC \quad\text{若 \m{n > n_2}。} \NR
\stopalign\stopformula

設 \m{n_0 = \max(n_1,n_2)},對於 \m{n > n_0}:
\startformula\startalign
\NC f(n) \NC \leq \max(f(n), g(n)) \NR
\NC g(n) \NC \leq \max(f(n), g(n)) \NR
\NC (f(n) + g(n))/2 \NC \leq \max(f(n),g(n)) \NR
\NC \max(f(n), g(n)) \NC \leq f(n) + g(n) \NR
\stopalign\stopformula

對於最後兩個不等式,有:
\startformula
0 \leq \frac{1}{2}(f(n)+g(n)) \leq \max(f(n),g(n)) \leq f(n) + g(n)\text{,若 \m{n > n_0}。}
\stopformula

這與 \m{\Theta(f(n)+g(n))} 的定義一致,其中 \m{c_1 = 1/2}, \m{c_2=1}。

\stopANSWER

\startEXERCISE
證明:對於任意實數常量 \m{a} 和 \m{b},其中 \m{b>0},有:
\startformula
(n+a)^b = \Theta(n^b)
\stopformula
\stopEXERCISE
\startANSWER
\startformula
(n + a)^b = \binom{n}{0}n^b + \binom{n}{1}n^{b-1}b + \cdots + \binom{n}{0}a^b
\stopformula
\stopANSWER

\startEXERCISE
解釋一下爲什麼說“算法 \m{A} 的運行時間至少是 \m{O(n^2)}”沒有任何意義。
\stopEXERCISE
\startANSWER
\m{O} 是指上界,“至少”是指下界。
\stopANSWER

\startEXERCISE
\m{2^{n+1} = O(2^n)}? \m{2^{2n} = O(2^n)}?
\stopEXERCISE
\startANSWER
\m{2^{n+1} = O(2^n)}。

\m{2^{2n} \neq O(2^n)},因爲\m{\nexists c: 2^n \cdot 2^n \leq c 2^n}。
\stopANSWER

\startEXERCISE
證明定理 3.1。

當且僅當 \m{f(n) = O(g(n))} 並且 \m{f(n) = \Omega(g(n))} 時,才有 \m{f(n) = \Theta(g(n))}。
\stopEXERCISE
\startANSWER
如果 \m{f(n) = \Theta(g(n))},則:
\startformula
0 \leq c_1g(n) \leq f(n) \leq c_2g(n) \quad \text{若} n > n_0
\stopformula
將兩個常數 \m{c_1} 和 \m{c_2} 代入 \m{O} 和 \m{\Omega} 的定義中即得:
\startformula\startalign
\NC f(n) \NC = O(g(n)) \NR
\NC f(n) \NC = \Omega(g(n)) \NR
\stopalign\stopformula

如果 \m{f(n) = O(g(n))} 並且 \m{f(n) = \Omega(g(n))},則:
\startformula\startalign
0 \leq c_3g(n) \leq f(n) \NC \quad \text{若} n \geq n_1 \NR
0 \leq f(n) \leq c_4g(n) \NC \quad \text{若} n \geq n_2 \NR
\stopalign\stopformula
設 \m{n_3=\max(n_1,n_2)},合並兩個不等式,得:
\startformula
0 \leq c_3g(n) \leq f(n) \leq c_4g(n) \quad \text{若} n > n_3
\stopformula
\stopANSWER

\startEXERCISE
證明當且僅當算法的最壞情況運行時間爲 \m{O(g(n))},且最好情況運行時間爲 \m{\Omega(g(n))} 時,
其運行時間才是 \m{\Theta(g(n))}。
\stopEXERCISE
\startANSWER
設最壞情況運行時間爲 \m{T_w},最好情況運行時間爲 \m{T_b},則:
\startformula\startalign
\NC 0 \leq c_1g(n) \leq T_b(n) \NC \quad \text{若} n > n_b \NR
\NC 0 \leq T_w(n) \leq c_2g(n) \NC \quad \text{若} n > n_w \NR
\stopalign\stopformula

結合兩式,有:
\startformula
0 \leq c_1g(n) \leq T_b(n) \leq T_w(n) \leq c_2g(n)
   \quad \text{若} n > \max(n_b, n_w)
\stopformula
\stopANSWER

\startEXERCISE
證明集合 \m{o(g(n)) \cap \omega(g(n))} 爲空。
\stopEXERCISE
\startANSWER
對於常量 \m{c > 0}:
\startformula\startalign
\NC \exists n_1 > 0 : \NC 0 \leq f(n) < cg(n) \NR
\NC \exists n_2 > 0 : \NC 0 \leq cg(n) < f(n) \NR
\stopalign\stopformula

如果 \m{n_0 = \max(n_1,n_2)},則:
\startformula
f(n) < cg(n) < f(n)
\stopformula
顯然此不等式不成立,不存在這樣的函式 \m{f(n)}。
\stopANSWER

\startEXERCISE
將參數 \m{n} 推廣爲兩個參數 \m{m} 和 \m{n},
存在正常數 \m{c}、 \m{n_0} 和 \m{m_0},
使得對於所有的 \m{n\geq n_0} 或 \m{m\geq m_0},
有 \m{0 \leq f(n,m) \leq cg(n,m)},即 \m{O(g(n,m)) = f(n,m)}。
給出 \m{\Omega(g(n,m))} 和 \m{\Theta(g(n,m))} 的定義。
\stopEXERCISE
\startANSWER
存在正常數 \m{c}、 \m{n_0} 和 \m{m_0},
使得對於所有的 \m{n\geq n_0} 或 \m{m\geq m_0},
有 \m{0 \leq cg(n,m) \leq f(n,m)},即 \m{\Omega(g(n,m)) = f(n,m)}。

存在正常數 \m{c_1}、 \m{c_2}、 \m{n_0} 和 \m{m_0},
使得對於所有的 \m{n\geq n_0} 或 \m{m\geq m_0},
有 \m{0 \leq c_1g(n,m) \leq f(n,m) \leq c_2g(n,m)},即 \m{\Theta(g(n,m)) = f(n,m)}。
\stopANSWER

\stopsection
