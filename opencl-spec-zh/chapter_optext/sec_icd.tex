% OpenCL Installable Client Driver (ICD)
\section{OpenCL 可安裝的客戶端驅動(ICD)}

\subsection{簡介}

這是一個\cnglo{platform}擴展,其中定義了一種比較簡單的機制,
使得 Khronos OpenCL ICD 裝載器可以將多個獨立的 Vendor ICD 暴露給 OpenCL。
針對 ICD 裝載器編寫的\cnglo{app}可以存取所有供應商實作暴露出來的所有 \cldt{cl_platform_ids},
其中 ICD 裝載器扮演譯碼器的角色。
如果實作支持本擴展,
則 \cenum{CL_PLATFORM_EXTENSIONS} 所對應的字串中應當包含 \clext{cl_khr_icd},
參見\reftab{cldevquery}。

% Inferring Vendors from Function Calls from Arguments
\subsection{如何調用供應商提供的函式}

對於任一 OpenCL 函式調用,
 ICD 裝載器都是用其引數調用供應商所實現的函式。
當一個對象是如下結構體時,就說他是與 ICD 相兼容的:
\startclc[indentnext=no]
struct _cl_<object>
{
	struct _cl_icd_dispatch *dispatch;
	// ... remainder of internal data
};
\stopclc
其中 \ccmm{<object>} 是下列字串之一:
\startigBase
\item \ccmm{platform_id}、
\item \ccmm{device_id}、
\item \ccmm{context}、
\item \ccmm{command_queue}、
\item \ccmm{mem}、
\item \ccmm{program}、
\item \ccmm{kernel}、
\item \ccmm{event}、或
\item \ccmm{sampler}。
\stopigBase

結構體 \cldt{_cl_icd_dispatch} 是一個函式指位器分派表,
用來直接調用特定供應商的實作。
所有由 ICD 兼容對象所創建的對象都必須是 ICD 兼容的。

\cldt{_cl_icd_dispatch} 中函式的次序由 ICD 裝載器的源碼確定。
此結構體只能在後面追加新內容。

不帶引數的函式會將引數部分忽略,但 \clapi{clGetExtensionFunctionAddress} 例外,後面有詳細介紹。

% ICD Data
\subsection{ICD 數據}

Vendor ICD 由兩部分數據定義:
\startigBase
\item Vendor ICD 程式庫,包含 OpenCL 實作所實現的所有 OpenCL 入口點。
此程式庫的檔案名應當包含供應商的名字或特定供應商的實作 ID。

\item Vendor ICD 擴展後綴,一個短字串,
此供應商所實現的所有擴展缺省都帶有此後綴。
這個供應商後綴是可選的。
\stopigBase

% ICD Loader Vendor Enumeration on Windows
\subsection{Windows 上 ICD 裝載器所用的供應商列表}

在 Windows 上,為了列舉 Vendor ICD, ICD 裝載器會
掃描註冊表中的鍵值 \ccmm{HKEY_LOCAL_MACHINE\SOFTWARE\Khronos\OpenCL\Vendors}。
對於此鍵值中的任一 DWORD 數據,
只要其值為 0, ICD 裝載器就會用 \clapi{LoadLibraryA} 打開其名字對應的動態鏈接庫。

例如,如果註冊表包含下列值:
\startclc[indentnext=no]
[HKEY_LOCAL_MACHINE\SOFTWARE\Khronos\OpenCL\Vendors]
"c:\\vendor a\\vndra_ocl.dll"=dword:00000000
\stopclc
那麼 ICD 就會打開程式庫“\ccmm{c:\vendor a\vndra_ocl.dll}”。


% ICD Loader Vendor Enumeration on Linux
\subsection{Linux 上 ICD 裝載器所用的供應商列表}

在 Linux 上,為了列舉 Vendor ICD, ICD 裝載器會
掃描路徑 \ccmm{/etc/OpenCL/vendors} 中的所有檔案。
對於此路徑中的任一檔案, ICD 裝載器都會以純文本模式打開此檔案。
此檔案的格式應該是一行文本,指定了 Vendor ICD 的程式庫。
 ICD 裝載器會試圖用 \capi{dlopen()} 打開此程式庫。
指定程式庫時可能是絕對路徑,或者僅僅是檔案名。

例如,存在檔案 \ccmm{/etc/OpenCL/vendors/VendorA.icd},
其中包含文本 \ccmm{libVendorAOpenCL.so},
則 ICD 裝載器會裝載程式庫“\ccmm{libVendorAOpenCL.so}”。

% Adding a Vendor Library
\subsection{添加供應商程式庫}

在成功裝載 Vendor ICD 的程式庫後,
 ICD 裝載器會查詢如下函式:
\startigBase[indentnext=no]
\item \cenum{clIcdGetPlatformIDsKHR},
\item \cenum{clGetPlatformInfo},和
\item \cenum{clGetExtensionFunctionAddress}。
\stopigBase
如果其中任一函式不存在,則 ICD 裝載器會關閉此程式庫並將其忽略。

接下來, ICD 裝載器會用 \clapi{clIcdGetPlatformIDsKHR} 來
查詢程式庫中使能了 ICD 的可用\cnglo{platform}。
對於每個這樣的\cnglo{platform},
 ICD 裝載器都會查詢其擴展字串來驗證他確實支持 \clext{cl_khr_icd},
然後以 \cenum{CL_PLATFORM_ICD_SUFFIX_KHR} 調用 \clapi{clGetPlatformInfo} 查詢
其 Vendor ICD 擴展後綴,

如果上面任一步驟失敗, ICD 裝載器就會忽略這個 Vendor ICD 並繼續裝載下一個。

% New Procedures and Functions
\subsection{新例程和新函式}

\startCLFUNC
cl_int clIcdGetPlatformIDsKHR (
			cl_uint num_entries,
			cl_platform_id *platforms,
			cl_uint *num_platforms);
\stopCLFUNC

% New Tokens
\subsection{新符記}

函式 \clapi{clGetPlatformInfo} 的參數 \carg{param_name} 接受以下值:
\startclc
CL_PLATFORM_ICD_SUFFIX_KHR	0x0920
\stopclc

如果沒有找到可用\cnglo{platform},則 \clapi{clGetPlatformIDs} 會返回:
\startclc
CL_PLATFORM_NOT_FOUND_KHR	-1001
\stopclc

% Additions to Chapter 4 of the OpenCL 1.2 Specification
\subsection{對第四章的補充}

以下列內容取代\refsection{queryPlf}中對 \clapi{clGetPlatformIDs} 的返回值的描述:
\startreplacepar
如果執行成功,並且找到了至少一個可用\cnglo{platform},
則 \clapi{clGetPlatformIDs} 會返回 \cenum{CL_SUCCESS}。
如果沒有找到可用 \cnglo{platform},則返回 \cenum{CL_PLATFORM_NOT_FOUND_KHR}。
如果 \carg{num_entries} 是零,但 \carg{platforms} 不是 \cmacro{NULL};
或者 \carg{num_platforms} 和 \carg{platforms} 都是 \cmacro{NULL},
則返回 \cenum{CL_INVALID_VALUE}。
\stopreplacepar

\refsection{queryPlf}中,在 \clapi{clGetPlatformIDs} 的描述後添加以下內容:
\startreplacepar
\topclfunc{clIcdGetPlatformIDsKHR}

\startCLFUNC
cl_int clIcdGetPlatformIDsKHR (
			cl_uint num_entries,
			cl_platform_id *platforms,
			cl_uint *num_platforms);
\stopCLFUNC

此函式用來獲取可以通過 Khronos ICD 裝載器來訪問的\cnglo{platform}列表。

\carg{num_entries} 即 \carg{platforms} 中可以存放的 \cldt{cl_platform_id} 的數目。
如果 \carg{platforms} 不是 \cmacro{NULL},則 \carg{num_entries} 必須大於零。

\carg{platforms} 用來返回可以通過 Khronos ICD 裝載器來訪問的\cnglo{platform}列表。
其中的 \cldt{cl_platform_id} 是 ICD 兼容的,
可用來識別特定 OpenCL \cnglo{platform}。
如果引數 \carg{platforms} 是 \cmacro{NULL},則將其忽略。
所返回\cnglo{platform}的數目是 \carg{num_entries} 和
實際可用\cnglo{platform}數目中較小的那個。

\carg{num_platforms} 返回實際可用\cnglo{platform}的數目。
如果 \carg{num_platforms} 是 \cmacro{NULL},則將其忽略。

如果執行成功,並且至少有一個可用\cnglo{platform},
則 \clapi{clIcdGetPlatformIDsKHR} 會返回 \cenum{CL_SUCCESS}。
如果沒有找到可用 \cnglo{platform},則返回 \cenum{CL_PLATFORM_NOT_FOUND_KHR}。
如果 \carg{num_entries} 是零,但 \carg{platforms} 不是 \cmacro{NULL};
或者 \carg{num_platforms} 和 \carg{platforms} 都是 \cmacro{NULL},
則返回 \cenum{CL_INVALID_VALUE}。
\stopreplacepar

將\reftab{plf_query_icd}中的內容添加到\reftab{plfquery}中。

\placetable[here][tab:plf_query_icd]
{OpenCL 平台查询}
{\startETD[cl_platform_info][返回型別]

\clETD{CL_PLATFORM_ICD_SUFFIX_KHR}{char[]}{
函式名後綴,由 ICD 裝載器用來識別此\cnglo{platform}中的擴展函式。
}

\stopETD

}

% Additions to Chapter 9 of the OpenCL 1.2 Extension Specification
\subsection{對第九章的補充}

在\refsection{getFuncPtr}的最後添加以下內容:
\startreplacepar
對於 ICD 裝載器所支持的那些函式,
 \clapi{clGetExtensionFunctionAddress} 會返回 ICD 裝載器實作的函式指位器。
而對於 ICD 裝載所不知道的那些函式,
 \clapi{clGetExtensionFunctionAddress} 會根據傳入的字串來確定供應商實作。
 \clapi{clGetExtensionFunctionAddress} 會在 ICD 裝載器所列舉的 Vendor ICD 中進行查詢,
並且查詢時會在所查詢函式名後加上 ICD 後綴。
如果不存在這樣的供應商,或者函式後綴是 \ccmm{KHR} 或 \ccmm{EXT},
則 \clapi{clGetExtensionFunctionAddress} 會返回 \cmacro{NULL}。
\stopreplacepar

% Issues
\subsection{問題}

\startQUESTION
一些 OpenCL 函式沒有對象引數,無法據此識別其供應商程式庫(例如, \clapi{clUnloadCompiler}),
這種情況如何處理?
\stopQUESTION
\startANSWER
已解決:通過 ICD 調用這種函式時,全部為空操作。
\stopANSWER

\startQUESTION
如何處理 OpenCL 擴展?
\stopQUESTION
\startANSWER
已解決:只要有一個供應商實現了某個擴展函式,就應當立刻將其加入 ICD。
對於那些還未加入 ICD 的供應商擴展,使用後綴機制來訪問。
\stopANSWER

\startQUESTION
ICD 如何處理為 \cmacro{NULL} 的 \cldt{cl_platform_id}?
\stopQUESTION
\startANSWER
已解決: ICD 不支持 \cmacro{NULL} \cnglo{platform}。
\stopANSWER

\startQUESTION
無法卸載 ICD,是否應當提供一種機制來做此事?
\stopQUESTION
\startANSWER
已解決:由於沒有一個標準機制來卸載供應商實作,所以也不為 ICD 添加這種機制。
\stopANSWER

