\subsection{由 GL 材質創建 CL 圖像對象}

\topclfunc{clCreateFromGLTexture}

\startCLFUNC
cl_mem clCreateFromGLTexture (
			cl_context context,
			cl_mem_flags flags,
			GLenum texture_target,
			GLint miplevel,
			GLuint texture,
			cl_int *errcode_ret)
\stopCLFUNC

此函式可以:
\startigBase
\item 由 OpenGL 2D 材質對象、或 OpenGL cubemap 材質對象的某一面創建 OpenCL 2D \cnglo{imgobj}。

\item 由 OpenGL 2D 材質陣列對象創建 OpenCL 2D 圖像陣列對象。

\item 由 OpenGL 1D 材質對象創建 OpenCL 1D 圖像對象。

\item 由 OpenGL 材質緩衝對象創建 OpenCL 1D 圖像緩衝對象。

\item 由 OpenGL 1D 材質陣列對象創建 OpenCL 1D 圖像陣列對象。

\item 由 OpenGL 3D 材質對象創建 OpenCL 3D 圖像對象。
\stopigBase

\carg{context} 是一個由 OpenGL \cnglo{context}創建的 OpenCL \cnglo{context}。

\carg{flags} 是位欄,用於指定用法資訊。參見\reftab{clmemflags}。
此處只能使用 \cenum{CL_MEM_READ_ONLY}、 \cenum{CL_MEM_WRITE_ONLY} 和 \cenum{CL_MEM_READ_WRITE}。

\carg{texture_target} 必須是下列值之一:
\startigBase[indentnext=no]
\item \cenum{GL_TEXTURE_1D}
\item \cenum{GL_TEXTURE_1D_ARRAY}
\item \cenum{GL_TEXTURE_BUFFER}
\item \cenum{GL_TEXTURE_2D}
\item \cenum{GL_TEXTURE_2D_ARRAY}
\item \cenum{GL_TEXTURE_3D}
\item \cenum{GL_TEXTURE_CUBE_MAP_POSITIVE_X}
\item \cenum{GL_TEXTURE_CUBE_MAP_POSITIVE_Y}
\item \cenum{GL_TEXTURE_CUBE_MAP_POSITIVE_Z}
\item \cenum{GL_TEXTURE_CUBE_MAP_NEGATIVE_X}
\item \cenum{GL_TEXTURE_CUBE_MAP_NEGATIVE_Y}
\item \cenum{GL_TEXTURE_CUBE_MAP_NEGATIVE_Z}
\item \cenum{GL_TEXTURE_RECTANGLE}\footnote{%
要求 OpenGL 3.1。
如果支持 OpenGL 擴展 \clext{GL_ARG_texture_rectangle},
也可以是 \cenum{GL_TEXTURE_RECTANGLE_ARG}。}
\stopigBase
\carg{texture_target} 僅用來定義 \carg{texture} 的圖像類型。
此參數不會引用所綁定的 GL 材質對象。

\carg{miplevel} 即所用的 mipmap 級別\footnote{%
如果 \carg{miplevel} 的值 > 0,則實作可能返回 \cenum{CL_INVALID_OPERATION}。}。
如果 \carg{texture_target} 是 \cenum{GL_TEXTURE_BUFFER},
則 \carg{miplevel} 必須是 0。

\carg{texture} 是一個 GL 1D、 2D、 3D、 1D 陣列、 2D 陣列、 cubemap、 矩形或緩衝材質對象。
這個材質對象(按照材質上 OpenGL 規則的完整性)必須是完整的材質。
OpenGL 為 \carg{miplevel} 的 \carg{texture} 所定義的格式和維度
會被用來創建 OpenCL \cnglo{imgobj}。
僅當 GL 材質對象的內部格式可以映射到
\reftab{imgChannelOrder}和\reftab{imgChannelDataType}中所列通道次序和數據型別時,
才可以用來創建 OpenCL \cnglo{imgobj}。

\carg{errcode_ret} 會返回相應的錯誤碼,見下文。
如果他是 \cmacro{NULL},則不會返回錯誤碼。

如果成功創建了\cnglo{imgobj},則 \clapi{clCreateFromGLTexture} 會將其返回,
並將 \carg{errcode_ret} 置為 \cenum{CL_SUCCESS}。
否則,返回 \cmacro{NULL},並將 \carg{errcode_ret} 置為下列錯誤碼之一:
\startigBase
\item \cenum{CL_INVALID_CONTEXT},
如果 \carg{context} 無效,或者不是由 GL \cnglo{context}創建的。

\item \cenum{CL_INVALID_VALUE},如果 \carg{flags} 的值無效,
或者 \carg{texture_target} 的值不再上面所述範圍內。

\item \cenum{CL_INVALID_MIP_LEVEL},
如果 \carg{miplevel} 小於(OpenGL 實作的) level\low{base} 或者(對於 OpenGL ES 實作而言)是零;
或者大於 q (無論是 OpenGL 還是 OpenGL ES)。
 OpenGL 2.1 規範中的{\ftRef{節 3.8.10}} 和 OpenGL ES 2.0 中的{\ftRef{節 3.7.10}} 中
定義了 level\low{base} 和 q 。

\item \cenum{CL_INVALID_MIP_LEVEL},
如果 \carg{miplevel} 大於零,並且 OpenGL 實作不支持由非零 mipmap 級別創建。

\item \cenum{CL_INVALID_GL_OBJECT},
如果 \carg{texture} 不是 \carg{texture_target} 類型的 GL 材質對象;
或者沒有定義 \carg{miplevel} 的 \carg{texture};
或者指定的 \carg{miplevel} 的寬或高是零。

\item \cenum{CL_INVALID_IMAGE_FORMAT_DESCRIPTOR},
如果 OpenGL 材質內部格式不能映射到所支持的 OpenCL 圖像格式上。

\item \cenum{CL_INVALID_OPERATION},
如果 \carg{texture} 是邊界寬度值大於零的 GL 材質對象。

\item \cenum{CL_OUT_OF_RESOURCES},如果\scdevfailres。

\item \cenum{CL_OUT_OF_HOST_MEMORY},如果\schostfailres。
\stopigBase

如果存在對應的 CL \cnglo{imgobj},
但是通過 GL API (如 \capi{glTexImage2D}、 \capi{glTexImage3D} 或者
修改了材質參數 \cenum{GL_TEXTURE_BASE_LEVEL} 或 \cenum{GL_TEXTURE_MAX_LEVEL})
修改了 GL 材質對象的狀態,
則後續使用 CL \cnglo{imgobj}時會導致\cnglo{undef}的行為。

可使用函式 \clapi{clRetainMemObject} 和 \clapi{clReleaseMemObject} 來
\cnglo{retain}和\cnglo{release}此\cnglo{imgobj}。

\reftab{ImgFmtMap}列出了 GL 材質內部格式與 CL 圖像格式間的對應關係。
只要 GL 材質對象是\reftab{ImgFmtMap}中所列格式,就保證能將其映射到表中對應的 CL 圖像格式上。
如果材質對象是其他 OpenGL 內部格式,
則可能(但不保證)也能映射到 CL 圖像格式上;
如果存在這樣的映射,則保證保留所有顏色組件、數據型別,並且每個組件的位數至少與分配 OpenGL 對象時相同。

\placetable[here][tab:ImgFmtMap]
{GL 內部格式到 CL 圖像格式的映射}
{\startCLOO[GL 內部格式][CL 圖像格式\par(通道次序、通道數據型別)]

\clOO{\cenum{GL_RGBA8}}{%
\cenum{CL_RGBA}、\cenum{CL_UNORM_INT8} 或\par
\cenum{CL_BGRA}、\cenum{CL_UNORM_INT8}}

\clOO{\cenum{GL_RGBA}\par
\cenum{GL_UNSIGNED_INT_8_8_8_8_REV}}{%
\cenum{CL_RGBA}、\cenum{CL_UNORM_INT8}}

\clOO{\cenum{GL_BGRA}\par
\cenum{GL_UNSIGNED_INT_8_8_8_8_REV}}{%
\cenum{CL_BGRA}、\cenum{CL_UNORM_INT8}}

\clOO{\cenum{GL_RGBA16}}{%
\cenum{CL_RGBA}、\cenum{CL_UNORM_INT16}}

\clOO{\cenum{GL_RGBA8I}、\cenum{GL_RGBA8I_EXT}}{%
\cenum{CL_RGBA}、\cenum{CL_SIGNED_INT8}}

\clOO{\cenum{GL_RGBA16I}、\cenum{GL_RGBA16I_EXT}}{%
\cenum{CL_RGBA}、\cenum{CL_SIGNED_INT16}}

\clOO{\cenum{GL_RGBA32I}、\cenum{GL_RGBA32I_EXT}}{%
\cenum{CL_RGBA}、\cenum{CL_SIGNED_INT32}}

\clOO{\cenum{GL_RGBA8UI}、\cenum{GL_RGBA8UI_EXT}}{%
\cenum{CL_RGBA}、\cenum{CL_UNSIGNED_INT8}}

\clOO{\cenum{GL_RGBA16UI}、\cenum{GL_RGBA16UI_EXT}}{%
\cenum{CL_RGBA}、\cenum{CL_UNSIGNED_INT16}}

\clOO{\cenum{GL_RGBA32UI}、\cenum{GL_RGBA32UI_EXT}}{%
\cenum{CL_RGBA}、\cenum{CL_UNSIGNED_INT32}}

\clOO{\cenum{GL_RGBA16F}、\cenum{GL_RGBA16F_ARB}}{%
\cenum{CL_RGBA}、\cenum{CL_HALF_FLOAT}}

\clOO{\cenum{GL_RGBA32F}、\cenum{GL_RGBA32F_ARB}}{%
\cenum{CL_RGBA}、\cenum{CL_FLOAT}}

\stopCLOO

}

