\section{內建函式}

OpenCL C 編程語言提供了一套豐富的內建函式用於標量和矢量運算。
這些函式中有很多跟通用 C 庫中提供的函式名字類似,
不同的是他們所支持的參數型別即可是標量,也可是矢量。
\cnglo{app}應儘可能使用這些內建函式,而不是去實現自己的版本。

用戶自定義的 OpenCL C 函式,按照函式的 C 標準規則運行(C99-TC2-{\ftRef{節 6.9.1}})。
在函式入口處,對於所有會動態變化的參數(variably modified parameter),都會求出其大小,
並且按照\refsection{usualArithConv}中描述的常見算術轉換規則
將所有引數算式的值都轉換成對應參數的型別。
本節所描述的內建函式行為類似,
不過由於同一個內建函式可能有多種形式,為避免歧義,不會有隱式的標量拓寬。
然而需要注意的是,可能某些內建函式的某些形式是針對標量以及矢量型別的混合運算。

% work-item functions
\subsection{\cnglo{workitem}函式}

\reftab{workItemFunction}列出了內建的\cnglo{workitem}函式,
這些函式可用來查詢指定給 \capi{clEnqueueNDRangeKernel} 的維數、全局和局部的索引空間大小、
以及在\cnglo{device}上執行此\cnglo{kernel}時
每個\cnglo{workitem}的\cnglo{glbid}和\cnglo{locid}。
如果使用的是 \capi{clEnqueueTask},則維數、全局和局部索引空間的大小都是一。

\placetable[here,split][tab:workItemFunction]
{\cnglo{workitem}函式表}
{% get_work_dim
\startbuffer[funcproto:get_work_dim]
uint get_work_dim()
\stopbuffer

\startbuffer[funcdesc:get_work_dim]
返回所用的維度數目。即 \capi{clEnqueueNDRangeKernel} 的引數 \carg{work_dim} 的值。

如果用的是 \capi{clEnqueueTask},則返回 1。
\stopbuffer

% get_global_size
\startbuffer[funcproto:get_global_size]
size_t get_global_size(uint dimindx)
\stopbuffer

\startbuffer[funcdesc:get_global_size]
返回在 \carg{dimindx} 所標識的維度上指定的全局\cnglo{workitem}的數目。
即 \capi{clEnqueueNDRangeKernel} 的引數 \carg{global_work_size}。
\carg{dimindx} 的有效範圍為 0 到 \math{\mapiemp{get_work_dim}()-1}。
如果 \carg{dimindx} 是其他值,則 \capi{get_global_size} 返回 1。

如果用的是 \capi{clEnqueueTask},則返回 1。
\stopbuffer

% get_global_id
\startbuffer[funcproto:get_global_id]
size_t get_global_id(uint dimindx)
\stopbuffer

\startbuffer[funcdesc:get_global_id]
返回\cnglo{workitem}在 \carg{dimindx} 所標識的維度上的唯一\cnglo{glbid}。
此 ID 基於用來執行此\cnglo{kernel}的全局\cnglo{workitem}的數目。
\carg{dimindx} 的有效範圍為 0 到 \math{\mapiemp{get_work_dim}() - 1}。
如果 \carg{dimindx} 是其他值,則 \capi{get_global_id} 返回 0。

如果用的是 \capi{clEnqueueTask},則返回 0。
\stopbuffer

% get_local_size
\startbuffer[funcproto:get_local_size]
size_t get_local_size(uint dimindx)
\stopbuffer

\startbuffer[funcdesc:get_local_size]
返回在 \carg{dimindx} 所標識的維度上指定的局部\cnglo{workitem}的數目。
如果 \capi{clEnqueueNDRangeKernel} 的引數 \carg{local_work_size} 不是 \cmacro{NULL},
則返回的就是此引數的值;
否則 OpenCL 實作會選擇一個恰當的 \carg{local_work_size} 並將其返回。
\carg{dimindx} 的有效範圍為 0 到 \math{\mapiemp{get_work_dim}()-1}。
如果 \carg{dimindx} 是其他值,則 \capi{get_local_size} 返回 1。

如果用的是 \capi{clEnqueueTask},則返回 1。
\stopbuffer

% get_local_id
\startbuffer[funcproto:get_local_id]
size_t get_local_id(uint dimindx)
\stopbuffer

\startbuffer[funcdesc:get_local_id]
返回\cnglo{workitem}在\cnglo{workgrp}內 \carg{dimindx} 所標識的維度上的唯一\cnglo{locid}。
\carg{dimindx} 的有效範圍為 0 到 \math{\mapiemp{get_work_dim}() - 1}。
如果 \carg{dimindx} 是其他值,則 \capi{get_local_id} 返回 0。

如果用的是 \capi{clEnqueueTask},則返回 0。
\stopbuffer

% get_num_groups
\startbuffer[funcproto:get_num_groups]
size_t get_num_groups(uint dimindx)
\stopbuffer

\startbuffer[funcdesc:get_num_groups]
返回執行此\cnglo{kernel}的\cnglo{workgrp}在 \carg{dimindx} 所標識的維度上的數目。
\carg{dimindx} 的有效範圍為 0 到 \math{\mapiemp{get_work_dim}() - 1}。
如果 \carg{dimindx} 是其他值,則 \capi{get_num_groups} 返回 1。

如果用的是 \capi{clEnqueueTask},則返回 1。
\stopbuffer

% get_group_id
\startbuffer[funcproto:get_group_id]
size_t get_group_id(uint dimindx)
\stopbuffer

\startbuffer[funcdesc:get_group_id]
所返回的\cnglo{workgrp} ID 取值範圍為
 \math{0 ... \mapiemp{get_work_dim}(\marg{dimindx}) - 1}。
\carg{dimindx} 的有效範圍為 0 到 \math{\mapiemp{get_work_dim}() - 1}。
如果 \carg{dimindx} 是其他值,則 \capi{get_group_id} 返回 0。

如果用的是 \capi{clEnqueueTask},則返回 0。
\stopbuffer

% get_global_offset
\startbuffer[funcproto:get_global_offset]
size_t get_global_offset (uint dimindx)
\stopbuffer

\startbuffer[funcdesc:get_global_offset]
返回的是
為 \capi{clEnqueueNDRangeKernel} 的引數 \carg{global_work_offset} 所指定的偏移值。
\carg{dimindx} 的有效範圍為 0 到 \math{\mapiemp{get_work_dim}() - 1}。
如果 \carg{dimindx} 是其他值,則 \capi{get_group_id} 返回 0。

如果用的是 \capi{clEnqueueTask},則返回 0。
\stopbuffer


% begin TABLE
\startCLFD

\clFD{get_work_dim}
\clFD{get_global_size}
\clFD{get_global_id}
\clFD{get_local_size}
\clFD{get_local_id}
\clFD{get_num_groups}
\clFD{get_group_id}
\clFD{get_global_offset}

\stopCLFD

}

% Math Functions
\subsection[section:mathFunc]{數學函式}

\reftab{svMathFunc}中列出了內建的數學函式。
內建的數學函式分為兩種:
\startigBase
\item 第一種函式有兩個版本,一個版本的引數是標量,另一個版本的引數是矢量;

\item 第二種函式只有一個版本,引數為標量浮點數。
\stopigBase

矢量版本的數學函式按組件逐一進行運算。
描述也是針對單個組件的。

無論調用環境中使用哪種捨入模式,
內建數學函式始終捨入為最近偶數,返回的結果也始終如一。

\reftab{svMathFunc}中所列函式即可接受標量引數,也可接受矢量引數。
泛型 \ctype{gentype} 表示函式引數的型別可以是
 \ctype{float}、 \ctype{float2}、 \ctype{float3}、 \ctype{float4}、
 \ctype{float8}、 \ctype{float16}、 \ctype{double}、 \ctype{double2}、
 \ctype{double3}、 \ctype{double4}、 \ctype{double8} 或 \ctype{double16}。
泛型 \ctype{gentypef} 表示函式引數的型別可以是
 \ctype{float}、 \ctype{float2}、 \ctype{float3}、 \ctype{float4}、
 \ctype{float8} 或 \ctype{float16}。
泛型 \ctype{gentyped} 表示函式引數的型別可以是
 \ctype{double}、 \ctype{double2}、 \ctype{double3}、 \ctype{double4}、
 \ctype{double8} 或 \ctype{double16}。
如果沒有特殊說明,函式的返回值與引數的型別都相同。

\placetable[here,split][tab:svMathFunc]
{引數既可為標量,也可為矢量的內建數學函式表}
{\startCLFD

\clFD{acos}
\clFD{acosh}
\clFD{acospi}
\clFD{asin}
\clFD{asinh}
\clFD{asinpi}
\clFD{atan}
\clFD{atan2}
\clFD{atanh}
\clFD{atanpi}
\clFD{atan2pi}
\clFD{cbrt}
\clFD{ceil}
\clFD{copysign}
\clFD{cos}
\clFD{cosh}
\clFD{cospi}
\clFD{erfc}
\clFD{erf}
\clFD{exp}
\clFD{exp2}
\clFD{exp10}
\clFD{expm1}
\clFD{fabs}
\clFD{fdim}
\clFD{floor}
\clFD{fma}
\clFD{fmax}
\clFD{fmin}
\clFD{fmod}
\clFD{fract}
\clFD{frexpf}
\clFD{frexpd}
\clFD{hypot}
\clFD{ilogb}
\clFD{ldexp}
\clFD{lgamma}
\clFD{log}
\clFD{log2}
\clFD{log10}
\clFD{log1p}
\clFD{logb}
\clFD{mad}
\clFD{maxmag}
\clFD{minmag}
\clFD{modf}
\clFD{nan}
\clFD{nextafter}
\clFD{pow}
\clFD{pown}
\clFD{powr}
\clFD{remainder}
\clFD{remquof}
\clFD{remquod}
\clFD{rint}
\clFD{rootn}
\clFD{rsqrt}
\clFD{sin}
\clFD{sincos}
\clFD{sinh}
\clFD{sinpi}
\clFD{sqrt}
\clFD{tan}
\clFD{tanh}
\clFD{tanpi}
\clFD{tgamma}
\clFD{trunc}

\stopCLFD
}

% half_ & native_ math function
\reftab{hnMathFunc}中列出了下列函式:
\startigBase
\item \reftab{svMathFunc}中的部分函式,但定義時帶有前綴 \ccmm{half_}。
實現這些函式時,精度至少要有 10 位,即所有 ULP 值都要小於等於 8192 ulp
(ULP:units in the last place,最後一位的進退位)。

\item \reftab{svMathFunc}中的部分函式,但定義時帶有前綴 \ccmm{native_}。
這些函式可能會映射到一條或多條原生的\cnglo{device}指令上,
性能通常比對應的不帶前綴 \ccmm{native_} 的函式更好。
這些函式的精度(以及某些情況下的輸入範圍)\cnglo{impdef}。

\item 用於除法和倒數運算的 \ccmm{half_} 和 \ccmm{native_} 函式。
\stopigBase

\startnotepar
對於帶前綴 \ccmm{native_} 的函式,
參數的取值範圍以及取最大值時會產生什麼錯誤都\cnglo{impdef}。
\stopnotepar

在\reftab{hnMathFunc}中,泛型 \ctype{gentype} 表示函式引數的型別可以是
 \ctype{float}、 \ctype{float2}、 \ctype{float3}、 \ctype{float4}、
 \ctype{float8} 或 \ctype{float16}。

\placetable[here][tab:hnMathFunc]
{內建的 \ccmm{half_} 和 \ccmm{native_} 數學函式}
{% half_cos
\startbuffer[funcproto:half_cos]
gentype half_cos (gentype x) 
\stopbuffer
\startbuffer[funcdesc:half_cos]
計算餘弦。 \carg{x} 的取值範圍為 \math{-2^{16} \cdots +2^{16}}。
\stopbuffer

% half_divide
\startbuffer[funcproto:half_divide]
gentype half_divide (gentype x, 
		gentype y) 
\stopbuffer
\startbuffer[funcdesc:half_divide]
計算 \math{x/y}。
\stopbuffer

% half_exp
\startbuffer[funcproto:half_exp]
gentype half_exp (gentype x) 
\stopbuffer
\startbuffer[funcdesc:half_exp]
計算 \math{e^x}。
\stopbuffer

% half_exp2
\startbuffer[funcproto:half_exp2]
gentype half_exp2 (gentype x) 
\stopbuffer
\startbuffer[funcdesc:half_exp2]
計算 \math{2^x}。
\stopbuffer

% half_exp10
\startbuffer[funcproto:half_exp10]
gentype half_exp10 (gentype x) 
\stopbuffer
\startbuffer[funcdesc:half_exp10]
計算 \math{10^x}。
\stopbuffer

% half_log
\startbuffer[funcproto:half_log]
gentype half_log (gentype x) 
\stopbuffer
\startbuffer[funcdesc:half_log]
計算自然對數。
\stopbuffer

% half_log2
\startbuffer[funcproto:half_log2]
gentype half_log2 (gentype x) 
\stopbuffer
\startbuffer[funcdesc:half_log2]
計算底為 2 的對數。
\stopbuffer

% half_log10
\startbuffer[funcproto:half_log10]
gentype half_log10 (gentype x) 
\stopbuffer
\startbuffer[funcdesc:half_log10]
計算底為 10 的對數。
\stopbuffer

% half_powr
\startbuffer[funcproto:half_powr]
gentype half_powr (gentype x,
		gentype y)
\stopbuffer
\startbuffer[funcdesc:half_powr]
計算 \math{x^y},其中 \math{x\geq 0}。
\stopbuffer

% half_recip
\startbuffer[funcproto:half_recip]
gentype half_recip (gentype x)
\stopbuffer
\startbuffer[funcdesc:half_recip]
計算倒數。
\stopbuffer

% half_rsqrt
\startbuffer[funcproto:half_rsqrt]
gentype half_rsqrt (gentype x)
\stopbuffer
\startbuffer[funcdesc:half_rsqrt]
計算 \math{ 1 / \sqrt{x}}。
\stopbuffer

% half_sin
\startbuffer[funcproto:half_sin]
gentype half_sin (gentype x)
\stopbuffer
\startbuffer[funcdesc:half_sin]
計算正弦。 \carg{x} 的取值範圍為 \math{-2^{16} \cdots +2^{16}}。
\stopbuffer

% half_sqrt
\startbuffer[funcproto:half_sqrt]
gentype half_sqrt (gentype x)
\stopbuffer
\startbuffer[funcdesc:half_sqrt]
計算 \math{\sqrt{x}}。
\stopbuffer

% half_tan
\startbuffer[funcproto:half_tan]
gentype half_tan (gentype x)
\stopbuffer
\startbuffer[funcdesc:half_tan]
計算正切。 \carg{x} 的取值範圍為 \math{-2^{16} \cdots +2^{16}}。
\stopbuffer

% native_cos
\startbuffer[funcproto:native_cos]
gentype native_cos (gentype x) 
\stopbuffer
\startbuffer[funcdesc:native_cos]
計算餘弦。
\stopbuffer

% native_divide
\startbuffer[funcproto:native_divide]
gentype native_divide (gentype x, 
		gentype y) 
\stopbuffer
\startbuffer[funcdesc:native_divide]
計算 \math{x/y}。
\stopbuffer

% native_exp
\startbuffer[funcproto:native_exp]
gentype native_exp (gentype x) 
\stopbuffer
\startbuffer[funcdesc:native_exp]
計算 \math{e^x}。
\stopbuffer

% native_exp2
\startbuffer[funcproto:native_exp2]
gentype native_exp2 (gentype x) 
\stopbuffer
\startbuffer[funcdesc:native_exp2]
計算 \math{2^x}。
\stopbuffer

% native_exp10
\startbuffer[funcproto:native_exp10]
gentype native_exp10 (gentype x) 
\stopbuffer
\startbuffer[funcdesc:native_exp10]
計算 \math{10^x}。
\stopbuffer

% native_log
\startbuffer[funcproto:native_log]
gentype native_log (gentype x) 
\stopbuffer
\startbuffer[funcdesc:native_log]
計算自然對數。
\stopbuffer

% native_log2
\startbuffer[funcproto:native_log2]
gentype native_log2 (gentype x) 
\stopbuffer
\startbuffer[funcdesc:native_log2]
計算底為 2 的對數。
\stopbuffer

% native_log10
\startbuffer[funcproto:native_log10]
gentype native_log10 (gentype x) 
\stopbuffer
\startbuffer[funcdesc:native_log10]
計算底為 10 的對數。
\stopbuffer

% native_powr
\startbuffer[funcproto:native_powr]
gentype native_powr (gentype x,
		gentype y)
\stopbuffer
\startbuffer[funcdesc:native_powr]
計算 \math{x^y},其中 \math{x\geq 0}。
\stopbuffer

% native_recip
\startbuffer[funcproto:native_recip]
gentype native_recip (gentype x)
\stopbuffer
\startbuffer[funcdesc:native_recip]
計算倒數。
\stopbuffer

% native_rsqrt
\startbuffer[funcproto:native_rsqrt]
gentype native_rsqrt (gentype x)
\stopbuffer
\startbuffer[funcdesc:native_rsqrt]
計算 \math{ 1 / \sqrt{x}}。
\stopbuffer

% native_sin
\startbuffer[funcproto:native_sin]
gentype native_sin (gentype x)
\stopbuffer
\startbuffer[funcdesc:native_sin]
計算正弦。
\stopbuffer

% native_sqrt
\startbuffer[funcproto:native_sqrt]
gentype native_sqrt (gentype x)
\stopbuffer
\startbuffer[funcdesc:native_sqrt]
計算 \math{\sqrt{x}}。
\stopbuffer

% native_tan
\startbuffer[funcproto:native_tan]
gentype native_tan (gentype x)
\stopbuffer
\startbuffer[funcdesc:native_tan]
計算正切。
\stopbuffer


% begin table
\startCLFD
\clFD{half_cos}
\clFD{half_divide}
\clFD{half_exp}
\clFD{half_exp2}
\clFD{half_exp10}
\clFD{half_log}
\clFD{half_log2}
\clFD{half_log10}
\clFD{half_powr}
\clFD{half_recip}
\clFD{half_rsqrt}
\clFD{half_sin}
\clFD{half_sqrt}
\clFD{half_tan}

\clFD{native_cos}
\clFD{native_divide}
\clFD{native_exp}
\clFD{native_exp2}
\clFD{native_exp10}
\clFD{native_log}
\clFD{native_log2}
\clFD{native_log10}
\clFD{native_powr}
\clFD{native_recip}
\clFD{native_rsqrt}
\clFD{native_sin}
\clFD{native_sqrt}
\clFD{native_tan}
\stopCLFD
}

實作可以自行決定 \capi{half_} 函式是否支持去規格化值。
如果引數是去規格化數, \capi{half_} 函式可以返回任何值,只要\refsection{ftzmECB}允許就行,
無論 \ccmm{-cl-denorms-are-zero} (參見\refsection{MathIntrinsicsOption})是否有效。

有下列符號常量可用。這些值的型別都是 \ctype{float},在單精度浮點數的精度內是準確的。
\startCLOD[常量名][描述]

\clOD{\cmacro{MAXFLOAT}}{
最大的有限單精度浮點數。
}

\clOD{\cmacro{HUGE_VALF}}{
正浮點常量算式。其求值結果為 \math{+\infty},由內建數學函式用作返回值表明錯誤。
}

\clOD{\cmacro{INFINITY}}{
常量算式,型別為 \ctype{float},表示正的或無符號的無窮。
}

\clOD{\cmacro{NAN}}{
常量算式,型別為 \ctype{float},表示 quiet NaN。
}
\stopCLOD


如果\cnglo{device}支持雙精度浮點數,還有下列符號常量可用:
\startCLOD[常量名][描述]

\clOD{\cmacro{HUGE_VAL}}{
正浮點常量算式。型別為 \ctype{double}。
其求值結果為 \math{+\infty},由內建數學函式用作返回值表明錯誤。
}

\stopCLOD


% Floating-point macros and pragmas
\subsubsection{浮點巨集和雜注}

雜注(pragma) \cpragmaemp{FP_CONTRACT} 可用來允許(如果狀態是 \ccmm{on})
或禁止(如果狀態是 \ccmm{off})實作化簡算式。
他可位於外部聲明的外面,也可位於複合語句中的顯式聲明或語句的前面。
當在外部聲明外面時,在遇到下一個 \cpragmaemp{FP_CONTRACT} 或者翻譯單元結束時就無效了。
當在複合語句中時,在遇到下一個 \cpragmaemp{FP_CONTRACT}
(包括嵌套的複合語句中的 \cpragmaemp{FP_CONTRACT})或者複合語句結束時就無效了;
在複合語句末尾處,會恢復成此語句之前的狀態。
在其他任何上下文中使用此雜注,其行為都是未定義的。

這樣設置 \cpragmaemp{FP_CONTRACT}:
\startclc
#pragma OPENCL FP_CONTRACT on-off-switch

on-off-switch is one of:
	/BTEX\ftEmp{ON}/ETEX, /BTEX\ftEmp{OFF}/ETEX or /BTEX\ftEmp{DEFAULT}/ETEX.
	The /BTEX\ftEmp{DEFAULT}/ETEX value is /BTEX\ftEmp{ON}/ETEX.
\stopclc

% float - single precision
巨集 \cmacroemp{FP_FAST_FMAF} 用來指明對於單精度浮點數,
函式 \capi{fma} 是否比直接編碼更快。
如果定義了此巨集,則表明對算元為 \ctype{float} 的乘、加運算,
函式 \capi{fma} 一般跟直接編碼一樣快,或者更快。

OpenCL C 編程語言定義了如下巨集,他們必須使用指定的值。
可以在預處理指示 \ccmm{#if} 中使用這些常量算式。
\startclc
#define FLT_DIG		6
#define FLT_MANT_DIG	24
#define FLT_MAX_10_EXP	+38
#define FLT_MAX_EXP	+128
#define FLT_MIN_10_EXP	-37
#define FLT_MIN_EXP	-125
#define FLT_RADIX	2
#define FLT_MAX		0x1.fffffep127f
#define FLT_MIN		0x1.0p-126f
#define FLT_EPSILON	0x1.0p-23f
\stopclc

\reftab{tblFltMacroAndApp}中給出了上面所列巨集與\cnglo{app}所用的巨集名字之間的對應關係。

\placetable[here][tab:tblFltMacroAndApp]
{單精度浮點巨集與應用程式所用巨集的對應關係}
{\startCLOO[OpenCL 語言中的巨集][\cnglo{app}所用的巨集]

\clMMF{DIG}
\clMMF{MANT_DIG}
\clMMF{MAX_10_EXP}
\clMMF{MAX_EXP}
\clMMF{MIN_10_EXP}
\clMMF{MIN_EXP}
\clMMF{RADIX}
\clMMF{MAX}
\clMMF{MIN}
\clMMF{EPSILSON}

\stopCLOO
}

下列兩個巨集將會展開成整數常量算式。
如果 \carg{x} 是 0 或 NaN,則 \math{\mapiemp{ilogb}(x)} 會分別返回這兩個值。
\startigBase
\item \cmacroemp{FP_ILOGB0} 為 \ccmm{{INT_MIN}} 或 \ccmm{-{INT_MAX}}。
\item \cmacroemp{FP_ILOGBNAN} 為 \ccmm{{INT_MAX}} 或 \ccmm{{INT_MIN}}。
\stopigBase

除此之外,還有一些常量可用,如\reftab{tblFltMacroConst}所示。
他們的型別都是 \ctype{float},在 \ctype{float} 型別的精度內是準確的。

\placetable[here][tab:tblFltMacroConst]
{單精度浮點常量}
{\startCLOO[常量][描述]

\clCM{M_E_F}{e}
\clCM{M_LOG2E_F}{log_{2}e}
\clCM{M_LOG10E_F}{log_{10}e}
\clCM{M_LN2_F}{log_{e}2}
\clCM{M_LN10_F}{log_{e}10}
\clCM{M_PI_F}{\pi}
\clCM{M_PI_2_F}{\pi/2}
\clCM{M_PI_4_F}{\pi/4}
\clCM{M_1_PI_F}{1/\pi}
\clCM{M_2_PI_F}{2/\pi}
\clCM{M_2_SQRTPI_F}{2/\sqrt{\pi}}
\clCM{M_SQRT2_F}{\sqrt{2}}
\clCM{M_SQRT1_2_F}{1/\sqrt{2}}

\stopCLOO
}

% double - double precision
如果\cnglo{device}支持雙精度浮點數,還有下列巨集和常量可用:
\startigBase
\item 巨集 \cmacroemp{FP_FAST_FMA} 指明處理雙精度浮點數時,
 \capi{fma} 系列函式是否比直接編碼更快。
如果定義了此巨集,則表明對 \ctype{double} 算元的乘、加運算,
函式 \capi{fma} 一般跟直接編碼一樣快,或者更快。
\stopigBase

OpenCL C 編程語言定義了如下巨集,他們必須使用指定的值。
可以在預處理指示 \ccmm{#if} 中使用這些常量算式。
\startclc
#define DBL_DIG		15
#define DBL_MANT_DIG	53
#define DBL_MAX_10_EXP	+308
#define DBL_MAX_EXP	+1024
#define DBL_MIN_10_EXP	-307
#define DBL_MIN_EXP	-1021
#define DBL_MAX		0x1.fffffffffffffp1023
#define DBL_MIN		0x1.0p-1022
#define DBL_EPSILON	0x1.0p-52
\stopclc

\reftab{tblDblMacroAndApp}中給出了上面所列巨集與\cnglo{app}所用的巨集名字之間的對應關係。

\placetable[here][tab:tblDblMacroAndApp]
{雙精度浮點巨集與應用程式所用巨集的對應關係}
{\startCLOO[OpenCL 語言中的巨集][\cnglo{app}所用的巨集]

\clMMD{DIG}
\clMMD{MANT_DIG}
\clMMD{MAX_10_EXP}
\clMMD{MAX_EXP}
\clMMD{MIN_10_EXP}
\clMMD{MIN_EXP}
\clMMD{MAX}
\clMMD{MIN}
\clMMD{EPSILSON}

\stopCLOO
}

除此之外,還有一些常量可用,如\reftab{tblDblMacroConst}所示。
他們的型別都是 \ctype{double},在 \ctype{double} 型別的精度內是準確的。

\placetable[here][tab:tblDblMacroConst]
{雙精度浮點常量}
{\startCLOO[常量][描述]

\clCM{M_E}{e}
\clCM{M_LOG2E}{log_{2}e}
\clCM{M_LOG10E}{log_{10}e}
\clCM{M_LN2}{log_{e}2}
\clCM{M_LN10}{log_{e}10}
\clCM{M_PI}{\pi}
\clCM{M_PI_2}{\pi/2}
\clCM{M_PI_4}{\pi/4}
\clCM{M_1_PI}{1/\pi}
\clCM{M_2_PI}{2/\pi}
\clCM{M_2_SQRTPI}{2/\sqrt{\pi}}
\clCM{M_SQRT2}{\sqrt{2}}
\clCM{M_SQRT1_2}{1/\sqrt{2}}

\stopCLOO
}


%Integer Functions
\subsection[section:integerFunc]{整數函式}

\reftab{svIntegerFunc}中列出了內建的整數函式,其引數即可為標量,亦可為矢量。
矢量版本的整數函式按組件逐一運算。其中的描述針對單個組件的。

泛型 \ctype{gentype} 表示函式引數的型別可以是:
\startigBase[indentnext=no]
\item \ctype{char}、 \ctype{char{2|3|4|8|16}}、
\item \ctype{uchar}、 \ctype{uchar{2|3|4|8|16}}、
\item \ctype{short}、 \ctype{short{2|3|4|8|16}}、
\item \ctype{ushort}、 \ctype{ushort{2|3|4|8|16}}、
\item \ctype{int}、 \ctype{int{2|3|4|8|16}}、
\item \ctype{uint}、 \ctype{uint{2|3|4|8|16}}、
\item \ctype{long}、 \ctype{long{2|3|4|8|16}}、
\item \ctype{ulong} 或 \ctype{ulong{2|3|4|8|16}}。
\stopigBase
而泛型 \ctype{ugentype} 則指代無符號版本的 \ctype{gentype}。
例如,如果 \ctype{gentype} 為 \ctype{char4},則 \ctype{ugentype} 為 \ctype{uchar4}。
同時,泛型 \ctype{sgentype} 指明函式的引數可以是標量(即
 \ctype{char}、 \ctype{uchar}、 \ctype{short}、 \ctype{ushort}、
 \ctype{int}、 \ctype{uint}、 \ctype{long} 或 \ctype{ulong})。
對於既有 \ctype{gentype} 引數,又有 \ctype{sgentype} 引數的內建整數函式,
 \ctype{gentype} 必須是標量或矢量版本的 \ctype{sgentype}。
例如,如果 \ctype{sgentype} 是 \ctype{uchar},
則 \ctype{gentype} 必須是 \ctype{uchar} 或 \ctype{uchar{2|3|4|8|16}}。
對於矢量版本, \ctype{sgentype} 只是簡單的拓寬成 \ctype{gentype},
參見\refsection{operator}中的\refitem{arithoperator}。

對於任一函式的任一具體應用,所有引數以及返回值的型別均相同,除非明確指定了其型別。

\placetable[here][tab:svIntegerFunc]
{引數既可為標量整數,也可為矢量整數的內建函式}
{% abs
\startbuffer[funcproto:abs]
ugentype abs(gentype x)
\stopbuffer
\startbuffer[funcdesc:abs]
返回 \math{|\marg{x}|}。
\stopbuffer

% abs_diff
\startbuffer[funcproto:abs_diff]
ugentype abs_diff(gentype x,
		gentype y)
\stopbuffer
\startbuffer[funcdesc:abs_diff]
返回 \math{|\marg{x} - \marg{y}|},不會有模上溢。
\stopbuffer

% add_sat
\startbuffer[funcproto:add_sat]
gentype add_sat(gentype x,
		gentype y)
\stopbuffer
\startbuffer[funcdesc:add_sat]
返回 \math{\marg{x} + \marg{y}},並使結果飽和。
\stopbuffer

% hadd
\startbuffer[funcproto:hadd]
gentype hadd (gentype x,
		gentype y)
\stopbuffer
\startbuffer[funcdesc:hadd]
返回 \math{(\marg{x} + \marg{y})>>1},中間的加法不會有模上溢。
\stopbuffer

% rhadd
\startbuffer[funcproto:rhadd]
gentype rhadd (gentype x,
		gentype y)
\stopbuffer
\startbuffer[funcdesc:rhadd]
返回 \math{(\marg{x} + \marg{y} + 1)>>1},中間的加法不會有模上溢\footnote{
矢量運算經常需要 \math{n+1} 個臨時位才能算出結果。
而 \capi{rhadd} 指令具有這個額外的位,從而無需上升取樣(upsample)或下降取樣(downsample)。
其性能優勢非常大。}。
\stopbuffer

% clamp
\startbuffer[funcproto:clamp]
gentype clamp (gentype x,
		gentype minval,
		gentype maxval)
gentype clamp (gentype x,
		sgentype minval,
		sgentype maxval)
\stopbuffer
\startbuffer[funcdesc:clamp]
返回 \math{\mapiemp{min}(\mapiemp{max}(\marg{x}, \marg{minval}), \marg{maxval})}。
如果 \math{\marg{minval} > \marg{maxval}},則結果未定義。
\stopbuffer

% clz
\startbuffer[funcproto:clz]
gentype clz (gentype x)
\stopbuffer
\startbuffer[funcdesc:clz]
返回 \carg{x} 中前導 0 的位數,從最高有效位開始。
\stopbuffer

% mad_hi
\startbuffer[funcproto:mad_hi]
gentype mad_hi (gentype a,
		gentype b,
		gentype c)
\stopbuffer
\startbuffer[funcdesc:mad_hi]
返回 \math{\mapiemp{mul_hi}(\marg{a}, \marg{b}) + \marg{c}}。
\stopbuffer

% mad_sat
\startbuffer[funcproto:mad_sat]
gentype mad_sat (gentype a,
		gentype b,
		gentype c)
\stopbuffer
\startbuffer[funcdesc:mad_sat]
返回 \math{\marg{a} \times \marg{b} + \marg{c}},並使結果飽和。
\stopbuffer

% max
\startbuffer[funcproto:max]
gentype max (gentype x, gentype y)
gentype max (gentype x, sgentype y)
\stopbuffer
\startbuffer[funcdesc:max]
如果 \math{\marg{x} < \marg{y}},則返回 \carg{y},否則返回 \carg{x}。
\stopbuffer

% min
\startbuffer[funcproto:min]
gentype min (gentype x, gentype y)
gentype min (gentype x, sgentype y)
\stopbuffer
\startbuffer[funcdesc:min]
如果 \math{\marg{y} < \marg{x}},則返回 \carg{y},否則返回 \carg{x}。
\stopbuffer

% mul_hi
\startbuffer[funcproto:mul_hi]
gentype mul_hi (gentype x,
		gentype y)
\stopbuffer
\startbuffer[funcdesc:mul_hi]
計算 \math{\marg{x} \times \marg{y}},並返回乘積的高位半部(high half)。
\stopbuffer

% rotate
\startbuffer[funcproto:rotate]
gentype rotate (gentype v, gentype i)
\stopbuffer
\startbuffer[funcdesc:rotate]
將 \carg{v} 中的每個元素都左移,其位數就是 \carg{i} 中對應元素的值
(遵守\refsection{operator}中的移位取模規則)。
由左側移出的位再從右側移入。
\stopbuffer

% sub_sat
\startbuffer[funcproto:sub_sat]
gentype sub_sat (gentype x, gentype y)
\stopbuffer
\startbuffer[funcdesc:sub_sat]
返回 \math{\marg{x} - \marg{y}},並使結果飽和。
\stopbuffer

% upsample
\startbuffer[funcproto:upsampleshort]
short upsample (char hi, uchar lo)
ushort upsample (uchar hi, uchar lo)
shortn upsample (charn hi, ucharn lo)
ushortn upsample (ucharn hi, ucharn lo)
\stopbuffer
\startbuffer[funcdesc:upsampleshort]
\math{result[i] = ((short)hi[i] << 8) | lo[i]}\par
\math{result[i] = ((ushort)hi[i] << 8) | lo[i]}
\stopbuffer
\startbuffer[funcproto:upsampleint]
int upsample (short hi, ushort lo)
uint upsample (ushort hi, ushort lo)
intn upsample (shortn hi, ushortn lo)
uintn upsample (ushortn hi, ushortn lo)
\stopbuffer
\startbuffer[funcdesc:upsampleint]
\math{result[i] = ((int)hi[i] << 16) | lo[i]}\par
\math{result[i] = ((uint)hi[i] << 16) | lo[i]}
\stopbuffer
\startbuffer[funcproto:upsamplelong]
long upsample (int hi, uint lo)
ulong upsample (uint hi, uint lo)
longn upsample (intn hi, uintn lo)
ulongn upsample (uintn hi, uintn lo)
\stopbuffer
\startbuffer[funcdesc:upsamplelong]
\math{result[i] = ((long)hi[i] << 32) | lo[i]}\par
\math{result[i] = ((ulong)hi[i] << 32) | lo[i]}
\stopbuffer

% popcount
\startbuffer[funcproto:popcount]
gentype popcount (gentype x)
\stopbuffer
\startbuffer[funcdesc:popcount]
返回 \carg{x} 中非零位的數目。
\stopbuffer

% begin TABLE
\startCLFD

\clFD{abs}
\clFD{abs_diff}
\clFD{add_sat}
\clFD{hadd}
\clFD{rhadd}
\clFD{clamp}
\clFD{clz}
\clFD{mad_hi}
\clFD{mad_sat}
\clFD{max}
\clFD{min}
\clFD{mul_hi}
\clFD{rotate}
\clFD{sub_sat}
\clFD{upsampleshort}
\clFD{upsampleint}
\clFD{upsamplelong}
\clFD{popcount}

\stopCLFD

}

\reftab{fastIntegerFunc}中列出了優化\cnglo{kernel}性能時可用的快速整數函式。
泛型 \ctype{gentype} 表明函式引數的型別可以是
 \ctype{int}、 \ctype{int2}、 \ctype{int3}、
 \ctype{int4}、 \ctype{int8}、 \ctype{int16}、
 \ctype{uint}、 \ctype{uint2}、 \ctype{uint3}、
 \ctype{uint4}、 \ctype{uint8} 或 \ctype{uint16}。

\placetable[here][tab:fastIntegerFunc]
{內建的快速整數函式}
{% mad24
\startbuffer[funcproto:mad24]
gentype mad24 (gentype x, 
		gentype y,
		gentype z)
\stopbuffer
\startbuffer[funcdesc:mad24]
計算 \math{x \times y +z},其中 \math{x} 和 \math{y} 均為 24 位整數,
\math{z} 和返回值均為 32 位整數。
至於如何實施 24 位整數乘法,請參考 \capi{mul24} 的定義。
\stopbuffer

% mul24
\startbuffer[funcproto:mul24]
gentype mul24 (gentype x,
		gentype y)
\stopbuffer
\startbuffer[funcdesc:mul24]
計算 \math{x} 和 \math{y} 的乘積。
其中 \math{x} 和 \math{y} 均為 32 位整數,但僅使用低 24 位。
\math{z} 和返回值均為 32 位整數。
至於如何實施 24 位整數乘法,請參考 \capi{mul24} 的定義。
要使用此函式,必須滿足以下條件:
如果 \math{x} 和 \math{y} 是帶符號整數,則其值必須在區間 \math{[-2^{23}, 2^{23}-1]} 內;
如果 \math{x} 和 \math{y} 是無符號整數,則其值必須在區間 \math{[0, 2^{24}-1]} 內。
如果不滿足上述條件,則其結果\cnglo{impdef}。
\stopbuffer

% begin TABLE
\startCLFD

\clFD{mad24}
\clFD{mul24}

\stopCLFD

}

OpenCL C 編程語言定義了如下巨集,他們必須使用指定的值。
可以在預處理指示 \ccmm{#if} 中使用這些常量算式。

\startclc
#define CHAR_BIT	8
#define CHAR_MAX	SCHAR_MAX
#define CHAR_MIN	SCHAR_MIN
#define INT_MAX		2147483647
#define INT_MIN		(-2147483647 - 1)
#define LONG_MAX	0x7fffffffffffffffL
#define LONG_MIN	(-0x7fffffffffffffffL - 1)
#define SCHAR_MAX	127
#define SCHAR_MIN	(-127 - 1)
#define SHRT_MAX	32767
#define SHRT_MIN	(-32767 - 1)
#define UCHAR_MAX	255
#define USHRT_MAX	65535
#define UINT_MAX	0xffffffff
#define ULONG_MAX	0xffffffffffffffffUL
\stopclc

\reftab{integerMacroAndApp}中給出了上面所列巨集與\cnglo{app}所用的巨集名字之間的對應關係。

\placetable[here][tab:integerMacroAndApp]
{整形巨集與應用程式所用巨集之間的對應關係}
{\startCLOO[OpenCL 語言中的巨集][\cnglo{app}所用的巨集]

\clMM{CHAR_BIT}
\clMM{CHAR_MAX}
\clMM{CHAR_MIN}
\clMM{INT_MAX}
\clMM{INT_MIN}
\clMM{LONG_MAX}
\clMM{LONG_MIN}
\clMM{SCHAR_MAX}
\clMM{SCHAR_MIN}
\clMM{SHRT_MAX}
\clMM{SHRT_MIN}
\clMM{UCHAR_MAX}
\clMM{USHRT_MAX}
\clMM{UINT_MAX}
\clMM{ULONG_MAX}

\stopCLOO
}


\subsection[section:commonFunc]{公共函式}

\reftab{svCommonFunc}中列出了內建的公共函式。
這些函式都是按組件逐一運算的,其中的描述也是針對單個組件的。
泛型 \ctype{gentype} 表明函式的引數可以是:
\startigBase[indentnext=no]
\item \ctype{float}、
\item \ctype{float{2|3|4|8|16}}、
\item \ctype{double}、
\item \ctype{double{2|3|4|8|16}}。
\stopigBase
泛型 \ctype{gentypef} 表明函式的引數可以是:
\startigBase[indentnext=no]
\item \ctype{float}、
\item \ctype{float{2|3|4|8|16}}。
\stopigBase
泛型 \ctype{gentyped} 表明函式的引數可以是:
\startigBase[indentnext=no]
\item \ctype{double}、
\item \ctype{double{2|3|4|8|16}}。
\stopigBase

內建的公共函式實現時用的捨入模式是捨入為最近偶數。

\startnotepar
可以使用化簡(如 \capi{mad} 或 \capi{fma})來實現 \capi{mix} 和 \capi{smoothstep}。
\stopnotepar

\placetable[here][tab:svCommonFunc]
{引數既可為標量整數,也可為矢量整數的內建公共函式}
{\startCLFD

\clFD{clamp}
\clFD{degrees}
\clFD{max}
\clFD{min}
\clFD{mix}
\clFD{radians}
\clFD{step}
\clFD{smoothstep}
\clFD{sign}

\stopCLFD
}

\subsection[section:geomtricFunc]{幾何函式}

\reftab{svGeometricFunc}中列出了內建的幾何函式。
這些函式都是按組件逐一運算的,其中的描述也是針對單個組件的。
\cldt[n]{float} 表示
 \ctype{float}、 \ctype{float2}、 \ctype{float3} 或 \ctype{float4};
而 \cldt[n]{double} 則表示
 \ctype{double}、 \ctype{double2}、 \ctype{double3} 或 \ctype{double4}。

內建的幾何函式實現時用的捨入模式是捨入為最近偶數。

\startnotepar
可以使用化簡(如 \capi{mad} 或 \capi{fma})來實現幾何函式。
\stopnotepar

\placetable[here][tab:svGeometricFunc]
{引數既可為標量,也可為矢量的內建幾何函式}
{\startCLFD

\clFD{cross}
\clFD{dot}
\clFD{distance}
\clFD{length}
\clFD{normalize}
\clFD{fast_distance}
\clFD{fast_length}
\clFD{fast_normalize}

\stopCLFD
}

\subsection[section:relationFunc]{關係函式}

可以使用關係算子和相等算子(<、 <=、 >、 >=、 !=、 ==)對內建標量和矢量型別進行關係運算,
所產生的結果分別為標量或矢量帶符號整形,參見\refsection{operator}。

\reftab{svRelationalFunc}中所列函式\footnote{
如果實作對規範進行了擴充,從而支持 IEEE-754 标志和異常,
則當有一個或多個算數是 NaN 時,
\reftab{svRelationalFunc}中所定義的內建函式不會引發{\ftRef{無效(invalid)}}浮點異常。}
可以內建標量或矢量型別為引數,返回的結果為標量或矢量整形。
泛型 \ctype{gentype} 指代下列內建型別:
\startigBase[indentnext=no]
\item \cldt{char}、 \cldt[n]{char}、 \cldt{uchar}、 \cldt[n]{uchar}、
\item \cldt{short}、 \cldt[n]{short}、 \cldt{ushort}、 \cldt[n]{ushort}、
\item \cldt{int}、 \cldt[n]{int}、 \cldt{uint}、 \cldt[n]{uint}、
\item \cldt{long}、 \cldt[n]{long}、 \cldt{ulong}、 \cldt[n]{ulong}、
\item \cldt{float}、 \cldt[n]{float}、 \cldt{double} 和 \cldt[n]{double}。
\stopigBase
泛型 \ctype{igentype} 指代內建帶符號整形,即:
\startigBase[indentnext=no]
\item \cldt{char}、 \cldt[n]{char}、
\item \cldt{short}、 \cldt[n]{short}、
\item \cldt{int}、 \cldt[n]{int}、
\item \cldt{long} 和 \cldt[n]{long}。
\stopigBase
泛型 \ctype{ugentype} 指代內建無符號整形,即:
\startigBase[indentnext=no]
\item \cldt{uchar}、 \cldt[n]{uchar}、
\item \cldt{ushort}、 \cldt[n]{ushort}、
\item \cldt{uint}、 \cldt[n]{uint}、
\item \cldt{ulong} 和 \cldt[n]{ulong}。
\stopigBase
其中 \ccmmsuffix{n} 為 2、 3、 4、 8 或 16。

對於標量型別的引數,如果所指定的關係為 {\ftRef{false}},則下列函式(參見\reftab{svRelationalFunc})會返回 0,否則返回 1:
\startigBase[indentnext=no]
\item \capi{isequal}、 \capi{isnotequal}、
\item \capi{isgreater}、 \capi{isgreaterequal}、
\item \capi{isless}、 \capi{islessequal}、
\item \capi{islessgreater}、
\item \capi{isfinite}、 \capi{isinf}、
\item \capi{isnan}、 \capi{isnormal}、
\item \capi{isordered}、 \capi{isunordered} 和
\item \capi{signbit}。
\stopigBase
而對於矢量型別的引數,如果所指定的關係為 {\ftRef{false}},則返回 0,
否則返回 -1 (即所有位都是 1)。

如果任一引數為 NaN,則下列關係函式返回 0:
\startigBase[indentnext=no]
\item \capi{isequal}、
\item \capi{isgreater}、 \capi{isgreaterequal}、
\item \capi{isless}、 \capi{islessequal} 和
\item \capi{islessgreater}。
\stopigBase
如果引數為標量,則當任一引數為 NaN 時, \capi{isnotequal} 返回 1;
而如果引數為矢量,則當任一引數為 NaN 時, \capi{isnotequal} 返回 -1。

\placetable[here,split][tab:svRelationalFunc]
{標量和矢量關係函式}
{\startCLFD

\clFD{isequal}
\clFD{isnotequal}
\clFD{isgreater}
\clFD{isgreaterequal}
\clFD{isless}
\clFD{islessequal}
\clFD{islessgreater}
\clFD{isfinite}
\clFD{isinf}
\clFD{isnan}
\clFD{isnormal}
\clFD{isordered}
\clFD{isunordered}
\clFD{signbit}
\clFD{any}
\clFD{all}
\clFD{bitselect}
\clFD{select}

\stopCLFD
}


\subsection[section:vectorLsFunc]{矢量數據裝載和存儲函式}

\startbuffer
僅在擴展 \clext{cl_khr_fp16} 中才定義有 \cldt[n]{half}%
(參見《OpenCL 1.2 擴展規範》的{\ftRef{節 9.5}})。
\stopbuffer
\reftab{vectorLsFunc}中列出了用來讀寫矢量型別數據的內建函式。
泛型 \cldt{gentype} 表示內建數據型別
 \cldt{char}、 \cldt{uchar}、 \cldt{short}、 \cldt{ushort}、
 \cldt{int}、 \cldt{uint}、 \cldt{long}、 \cldt{ulong}、
 \cldt{float} 或 \cldt{double}。
泛型 \cldt[n]{gentype} 表示具有 n 個 \cldt{gentype} 元素的矢量。
我們用 \cldt[n]{half} 表示具有 n 個 \cldt{half} 元素的矢量\footnote{\getbuffer}。
函式名中也有後綴 \ccmmsuffix{n} (即 \clapi[n]{vload}、 \clapi[n]{vstore} 等),
其中 \ccmmsuffix{n} 為 2、 3、 4、 8 或 16。

\startnotepar
\capi{vload3}、 \capi{vload_half3}、 \capi{vstore3} 和 \capi{vstore_half3}
所用位址為 \math{(\marg{p} + (\marg{offset}\times 3))};
而 \capi{vloada_half3} 和 \capi{vstorea_half3}
所用位址為 \math{(\marg{p} + (\marg{offset}\times 4))}。
\stopnotepar

\placetable[here,split][tab:vectorLsFunc]
{矢量數據裝載、存儲函式表}
{\startCLFD
\clFD{vloadn}
\clFD{vstoren}
\clFD{vload_half}
\clFD{vload_halfn}
\clFD{vstore_half_float}
\clFD{vstore_halfn_float}
\clFD{vstore_half_double}
\clFD{vstore_halfn_double}
\clFD{vloada_halfn}
\clFD{vstorea_halfn_float}
\clFD{vstorea_halfn_double}
\stopCLFD
}

使用這些函式裝載、存儲矢量數據時,
如果所讀寫的位址沒有按\reftab{vectorLsFunc}中所描述的方式對齊,
則結果未定義。
\reftab{vectorLsFunc}中存儲函式的指位器引數 \carg{p} 可以指向
 \cqlf{__global}、 \cqlf{__local} 或 \cqlf{__private} 內存。
\reftab{vectorLsFunc}中裝載函式的指位器引數 \carg{p} 可以指向
 \cqlf{__global}、 \cqlf{__local}、 \cqlf{__constant} 或 \cqlf{__private} 內存。

\subsection{同步函式}

OpenCL C 編程語言實現了\reftab{syncFunc}中所列的同步函式。

\placetable[here][tab:syncFunc]
{內建同步函式}
{% barrier
\startbuffer[funcproto:barrier]
void barrier (cl_mem_fence_flags flags) 
\stopbuffer
\startbuffer[funcdesc:barrier]
同一\cnglo{workgrp}中的\cnglo{workitem}在處理器上執行此\cnglo{kernel}時,
其中任一\cnglo{workitem}要想越過 \capi{barrier} 繼續執行,
所有\cnglo{workitem}都得先執行此函式。
所有\cnglo{workitem}必須都能執行到此函式。

如果 \capi{barrier} 在條件語句內,
只要有一個\cnglo{workitem}會進入此條件語句具有並執行 \capi{barrier},
那麼所有\cnglo{workitem}都必須進入此條件語句。

如果 \capi{barrier} 在迴圈語句內,
那麼在每一次迭代過程中,
任一\cnglo{workitem}要想越過 \capi{barrier} 繼續執行,
所有\cnglo{workitem}都得先執行此函式。

\capi{barrier} 函式還會用內存隔柵(讀、寫都包括)來確保局部、全局內存操作的正確順序。

引數 \carg{flags} 指定內存位址空間,可以是下列常值的組合:
\startigBase
\item \cenum{CLK_LOCAL_MEM_FENCE},
函式 \capi{barrier} 會通過刷新存儲在局部內存中的所有變量,
或者用內存隔柵確保局部內存操作的正確順序。

\item \cenum{CLK_GLOBAL_MEM_FENCE},
函式 \capi{barrier} 會用內存隔柵確保全局內存操作的正確順序。
例如,\cnglo{workitem}寫入\cnglo{bufobj}或\cnglo{imgobj}後又想讀取更新過的數據,
這時此功能就派上用場了。
\stopigBase
\stopbuffer

% begin table
\startCLFD

\clFD{barrier}

\stopCLFD

}

\subsection{顯式內存隔柵函式}

OpenCL C 編程語言實現了\reftab{memfenceFunc}中所列的顯式內存隔柵函式,
可對\cnglo{workitem}中的內存操作進行定序。

\placetable[here][tab:memfenceFunc]
{內建顯式內存隔柵函式}
{% mem_fence
\startbuffer[funcproto:mem_fence]
void mem_fence (cl_mem_fence_flags flags) 
\stopbuffer
\startbuffer[funcdesc:mem_fence]
\cnglo{workitem}執行\cnglo{kernel}時,為其中的裝載和存儲進行定序。
這意味着在執行 \capi{mem_fence} 之後的裝載和存儲之前,
會先將 \capi{mem_fence} 之前的裝載和存儲提交給內存。

引數 \carg{flags} 指定內存位址空間,可以是下列常值的組合:
\startigBase
\item \cenum{CLK_LOCAL_MEM_FENCE}
\item \cenum{CLK_GLOBAL_MEM_FENCE}
\stopigBase
\stopbuffer

% read_mem_fence
\startbuffer[funcproto:read_mem_fence]
void read_mem_fence (cl_mem_fence_flags flags)
\stopbuffer
\startbuffer[funcdesc:read_mem_fence]
讀內存屏障,僅對裝載定序。

引數 \carg{flags} 指定內存位址空間,可以是下列常值的組合:
\startigBase
\item \cenum{CLK_LOCAL_MEM_FENCE}
\item \cenum{CLK_GLOBAL_MEM_FENCE}
\stopigBase
\stopbuffer

% write_mem_fence
\startbuffer[funcproto:write_mem_fence]
void write_mem_fence (cl_mem_fence_flags flags)
\stopbuffer
\startbuffer[funcdesc:write_mem_fence]
寫內存屏障,僅對存儲定序。

引數 \carg{flags} 指定內存位址空間,可以是下列常值的組合:
\startigBase
\item \cenum{CLK_LOCAL_MEM_FENCE}
\item \cenum{CLK_GLOBAL_MEM_FENCE}
\stopigBase
\stopbuffer


% begin table
\startCLFD
\clFD{mem_fence}
\clFD{read_mem_fence}
\clFD{write_mem_fence}
\stopCLFD
}


\subsection[section:asyncCopyPrefetch]{在全局內存和局部內存間的異步拷貝以及預取}

OpenCL C 編程語言實現了\reftab{asyncCopyPrefetch}中所列函式,
可在\cnglo{glbmem}和\cnglo{locmem}間進行異步拷貝,
以及從\cnglo{glbmem}中預取(prefetch)。

如無特殊說明,泛型 \ctype{gentype} 表示函式引數可以是下列內建數據型別:
\startigBase[indentnext=no]
\item \cldt{char}、 \cldt[n]{char}、 \cldt{uchar}、 \cldt[n]{uchar}、
\item \cldt{short}、 \cldt[n]{short}、 \cldt{ushort}、 \cldt[n]{ushort}、
\item \cldt{int}、 \cldt[n]{int}、 \cldt{uint}、 \cldt[n]{uint}、
\item \cldt{long}、 \cldt[n]{long}、 \cldt{ulong}、 \cldt[n]{ulong}、
\item \cldt{float}、 \cldt[n]{float} 或 \cldt{double}、 \cldt[n]{double},
\stopigBase
其中 \ccmmsuffix{n} 可以是 2、 3\footnote{
對 \capi{async_work_group_copy} 和 \capi{async_work_group_strided_copy} 而言,
矢量型別的組件數目是 3 還是 4 沒有什麼區別。}、 4、 8、 16。

\placetable[here][tab:asyncCopyPrefetch]
{內建異步拷貝和預取函式}
{\startCLFD
\clFD{async_work_group_copy}
\clFD{async_work_group_strided_copy}
\clFD{wait_group_events}
\clFD{prefetch}
\stopCLFD
}

\startnotepar
\cnglo{kernel}必須用內建函式 \capi{wait_group_events} 等待所有異步拷貝全部完成後再退出,
否則其行為未定義。
\stopnotepar


\subsection[section:atomicFunc]{原子函式}

OpenCL C 編程語言實現了\reftab{atomicFunc}中所列函式,
可用來對位於 \cqlf{__global} 或 \cqlf{__local} 內存中的 32 位帶符號、
無符號整數以及單精度浮點數\footnote{
只有 \capi{atomic_xchg} 才支持單精度浮點數據型別。}進行原子操作。

\startnotepar
OpenCL 1.0 規範的節 9.5 和節 9.6 中列有如下擴展:
\startigBase
\item \clext{cl_khr_global_int32_base_atomics}
\item \clext{cl_khr_global_int32_extended_atomics}
\item \clext{cl_khr_local_int32_base_atomics}
\item \clext{cl_khr_local_int32_extended_atomics}
\stopigBase
其中所定義的帶有前綴 \capi{atom_} 的內建原子函式也在支持之列。
\stopnotepar

\placetable[here,split][tab:atomicFunc]
{內建原子函式}
{% atomic_add
\startbuffer[funcproto:atomic_add]
int atomic_add (
	volatile __global int *p,
	int val)
unsigned int atomic_add (
	volatile __global unsigned int *p,
	unsigned int val)

int atomic_add (
	volatile __local int *p,
	int val)
unsigned int atomic_add (
	volatile __local unsigned int *p,
	unsigned int val)
\stopbuffer
\startbuffer[funcdesc:atomic_add]
讀取 \carg{p} 所指向的 32 位值(記為 \math{old})。
計算 \math{(old + \marg{val})} 並將結果存儲到 \carg{p} 所指位置中。
此函式返回 \math{old}。
\stopbuffer

% atomic_sub
\startbuffer[funcproto:atomic_sub]
int atomic_sub (
	volatile __global int *p,
	int val)
unsigned int atomic_sub (
	volatile __global unsigned int *p,
	unsigned int val)

int atomic_sub (
	volatile __local int *p,
	int val)
unsigned int atomic_sub (
	volatile __local unsigned int *p,
	unsigned int val)
\stopbuffer
\startbuffer[funcdesc:atomic_sub]
讀取 \carg{p} 所指向的 32 位值(記為 \math{old})。
計算 \math{(old - \marg{val})} 並將結果存儲到 \carg{p} 所指位置中。
此函式返回 \math{old}。
\stopbuffer

% atomic_xchg
\startbuffer[funcproto:atomic_xchg]
int atomic_xchg (
	volatile __global int *p,
	int val)
unsigned int atomic_xchg (
	volatile __global unsigned int *p,
	unsigned int val)
float atomic_xchg (
	volatile __global float *p,
	float val)

int atomic_xchg (
	volatile __local int *p,
	int val)
unsigned int atomic_xchg (
	volatile __local unsigned int *p,
	unsigned int val)
float atomic_xchg (
	volatile __local float *p,
	float val)
\stopbuffer
\startbuffer[funcdesc:atomic_xchg]
將位置 \carg{p} 中所存儲的值 \math{old} 和 \carg{val} 中的新值相互交換。
返回 \math{old}。
\stopbuffer

% atomic_inc
\startbuffer[funcproto:atomic_inc]
int atomic_inc (volatile __global int *p)
unsigned int atomic_inc (
	volatile __global unsigned int *p)

int atomic_inc (volatile __local int *p)
unsigned int atomic_inc (
	volatile __local unsigned int *p)
\stopbuffer
\startbuffer[funcdesc:atomic_inc]
讀取 \carg{p} 所指向的 32 位值(記為 \math{old})。
計算 \math{(old+1)} 並將結果存儲到 \carg{p} 所指位置中。
此函式返回 \math{old}。
\stopbuffer

% atomic_dec
\startbuffer[funcproto:atomic_dec]
int atomic_dec (volatile __global int *p)
unsigned int atomic_dec (
	volatile __global unsigned int *p)

int atomic_dec (volatile __local int *p)
unsigned int atomic_dec (
	volatile __local unsigned int *p)
\stopbuffer
\startbuffer[funcdesc:atomic_dec]
讀取 \carg{p} 所指向的 32 位值(記為 \math{old})。
計算 \math{(old-1)} 並將結果存儲到 \carg{p} 所指位置中。
此函式返回 \math{old}。
\stopbuffer

% atomic_cmpchg
\startbuffer[funcproto:atomic_cmpxchg]
int atomic_cmpxchg (
	volatile __global int *p,
	int cmp, int val)
unsigned int atomic_cmpxchg (
	volatile __global unsigned int *p,
	unsigned int cmp,
	unsigned int val)

int atomic_cmpxchg (
	volatile __local int *p,
	int cmp,
	int val)
unsigned int atomic_cmpxchg (
	volatile __local unsigned int *p,
	unsigned int cmp,
	unsigned int val)
\stopbuffer
\startbuffer[funcdesc:atomic_cmpxchg]
讀取 \carg{p} 所指向的 32 位值(記為 \math{old})。
計算 \math{(old == cmp) ? val : old} 並將結果存儲到 \carg{p} 所指位置中。
此函式返回 \math{old}。
\stopbuffer

% atomic_min
\startbuffer[funcproto:atomic_min]
int atomic_min (
	volatile __global int *p,
	int val)
unsigned int atomic_min (
	volatile __global unsigned int *p,
	unsigned int val)

int atomic_min (
	volatile __local int *p,
	int val)
unsigned int atomic_min (
	volatile __local unsigned int *p,
	unsigned int val)
\stopbuffer
\startbuffer[funcdesc:atomic_min]
讀取 \carg{p} 所指向的 32 位值(記為 \math{old})。
計算 \math{\mapiemp{min}(old, \marg{val})} 並將結果存儲到 \carg{p} 所指位置中。
此函式返回 \math{old}。
\stopbuffer

% atomic_max
\startbuffer[funcproto:atomic_max]
int atomic_max (
	volatile __global int *p,
	int val)
unsigned int atomic_max (
	volatile __global unsigned int *p,
	unsigned int val)

int atomic_max (
	volatile __local int *p,
	int val)
unsigned int atomic_max (
	volatile __local unsigned int *p,
	unsigned int val)
\stopbuffer
\startbuffer[funcdesc:atomic_max]
讀取 \carg{p} 所指向的 32 位值(記為 \math{old})。
計算 \math{\mapiemp{max}(old, \marg{val})} 並將結果存儲到 \carg{p} 所指位置中。
此函式返回 \math{old}。
\stopbuffer

% atomic_and
\startbuffer[funcproto:atomic_and]
int atomic_and (
	volatile __global int *p,
	int val)
unsigned int atomic_and (
	volatile __global unsigned int *p,
	unsigned int val)

int atomic_and (
	volatile __local int *p,
	int val)
unsigned int atomic_and (
	volatile __local unsigned int *p,
	unsigned int val)
\stopbuffer
\startbuffer[funcdesc:atomic_and]
讀取 \carg{p} 所指向的 32 位值(記為 \math{old})。
計算 \math{(old \mcmm{&} \marg{val})} 並將結果存儲到 \carg{p} 所指位置中。
此函式返回 \math{old}。
\stopbuffer

% atomic_or
\startbuffer[funcproto:atomic_or]
int atomic_or (
	volatile __global int *p,
	int val)
unsigned int atomic_or (
	volatile __global unsigned int *p,
	unsigned int val)

int atomic_or (
	volatile __local int *p,
	int val)
unsigned int atomic_or (
	volatile __local unsigned int *p,
	unsigned int val)
\stopbuffer
\startbuffer[funcdesc:atomic_or]
讀取 \carg{p} 所指向的 32 位值(記為 \math{old})。
計算 \math{(old \mcmm{|} \marg{val})} 並將結果存儲到 \carg{p} 所指位置中。
此函式返回 \math{old}。
\stopbuffer

% atomic_xor
\startbuffer[funcproto:atomic_xor]
int atomic_xor (
	volatile __global int *p,
	int val)
unsigned int atomic_xor (
	volatile __global unsigned int *p,
	unsigned int val)

int atomic_xor (
	volatile __local int *p,
	int val)
unsigned int atomic_xor (
	volatile __local unsigned int *p,
	unsigned int val)
\stopbuffer
\startbuffer[funcdesc:atomic_xor]
讀取 \carg{p} 所指向的 32 位值(記為 \math{old})。
計算 \math{(old \mcmm{^} \marg{val})} 並將結果存儲到 \carg{p} 所指位置中。
此函式返回 \math{old}。
\stopbuffer


% begin table
\startCLFD
\clFD{atomic_add}
\clFD{atomic_sub}
\clFD{atomic_xchg}
\clFD{atomic_inc}
\clFD{atomic_dec}
\clFD{atomic_cmpxchg}
\clFD{atomic_min}
\clFD{atomic_max}
\clFD{atomic_and}
\clFD{atomic_or}
\clFD{atomic_xor}
\stopCLFD

}

\subsection{雜類矢量函式}

另外,OpenCL C 編程語言還實現了\reftab{miscVectorFunc}中所列內建矢量函式。
如無特殊說明,泛型 \cldt[n]{gentype} (或 \cldt[m]{gentype})表示函式引數的型別可以為:
\startigBase[indentnext=no]
\item \ctype{char{2|4|8|16}}、 \ctype{uchar{2|4|8|16}}、
\item \ctype{short{2|4|8|16}}、 \ctype{ushort{2|4|8|16}}、
\item \ctype{half{2|4|8|16}}\footnote{%
僅當支持擴展 \clext{cl_khr_fp16} 時才有效。}、
\item \ctype{int{2|4|8|16}}、 \ctype{uint{2|4|8|16}}、
\item \ctype{long{2|4|8|16}}、 \ctype{ulong{2|4|8|16}}、
\item \ctype{float{2|4|8|16}} 或 \ctype{double{2|4|8|16}}\footnote{%
僅當支持雙精度浮點數時才有效}。
\stopigBase
而泛型 \cldt[n]{ugentype} 則為內建無符號整形。

\placetable[here][tab:miscVectorFunc]
{內建雜類矢量函式}
{% vec_step
\startbuffer[funcproto:vec_step]
int vec_step (gentypen a) 

int vec_step (char3 a)
int vec_step (uchar3 a)
int vec_step (short3 a)
int vec_step (ushort3 a)
int vec_step (half3 a)
int vec_step (int3 a)
int vec_step (uint3 a)
int vec_step (long3 a)
int vec_step (ulong3 a)
int vec_step (float3 a)
int vec_step(double3 a)

int vec_step(type)
\stopbuffer
\startbuffer[funcdesc:vec_step]
此函式的引數為標量或矢量,返回值為引數中的元素數目。

對於所有標量型別,返回值為 1。

對於 3 組件矢量,返回值為 4。

引數也可以是純型別,如 \math{\mapiemp{vec_step}(\mtype{float2})}。
\stopbuffer

% shuffle
\startbuffer[funcproto:shuffle]
gentypen shuffle (
	gentypem x,
	ugentypen mask)
gentypen shuffle2 (
	gentypem x,
	gentypem y,
	ugentypen mask)
\stopbuffer
\startbuffer[funcdesc:shuffle]
輸入為一個或兩個型別相同的矢量,由其中的元素構造一個排列並返回,
所返回的矢量元素型別與輸入相同,矢量長度與掩碼相同。
掩碼中每個元素的大小必須與結果中元素大小相同。

對於 \capi{shuffle},
只考慮 \carg{mask} 中每個元素的 \math{\mapiemp{ilogb}(2m-1)} 個最低有效位。
而對於 \capi{shuffle2},
則會考慮 \carg{mask} 中每個元素的 \math{\mapiemp{ilogb}(2m-1)+1} 個最低有效位。
忽略 \carg{mask} 中的其他位。

因為要對輸入矢量中的元素從左到右進行編號(第二個輸入矢量的編號續接第一個的),
因此需要用 \math{\mapiemp{vec_step}(\text{\cldt[m]{gentype}})} 得到矢量元素數目。
引數 \carg{mask} 用來確定結果中的相應元素選自輸入矢量中的哪個元素。

例:
\startcintbl
uint4 mask = (uint4)(3, 2, 1, 0);
float4 a;
float4 r = shuffle(a, mask);
// r.s0123 = a.wzyx

uint8 mask = (uint8)(0, 1, 2, 3, 4, 5, 6, 7);
float4 a, b;
float8 r = shuffle2(a, b, mask);
// r.s0123 = a.xyzw
// r.s4567 = b.xyzw

uint4 mask;
float8 a;
float4 b;
b = shuffle(a, mask);
\stopcintbl

而無效的例子有:
\startcintbl
uint8 mask;
short16 a;
short8 b;
b = shuffle(a, mask); <- not valid
\stopcintbl
\stopbuffer

% begin table
\startCLFD
\clFD{vec_step}
\clFD{shuffle}
\stopCLFD
}

\subsection{printf}

OpenCL C 編程語言還實現有\reftab{printfFunc}中所列的 \capi{printf} 函式。

\placetable[here][tab:printfFunc]
{內建 printf 函式}
{% printf
\startbuffer[funcproto:printf]
int printf(constant char * restrict format, ...) 
\stopbuffer
\startbuffer[funcdesc:printf]
在 \carg{format} 所指字串(指定後續引數如何轉換)的控制下,
將輸出寫入實作所定義的數據流,如 stdout。
如果引數不足(對格式字串而言),則其行為未定義。
而如果引數有剩餘,則會對剩餘的引數求值(一直這樣)但將其忽略。
此函式解析完格式字串就會返回。

如果執行成功,此函式返回 0,否則返回 -1。
\stopbuffer

\startCLFD
\clFD{printf}
\stopCLFD
}

\subsubsection{printf 輸出同步}

當特定\cnglo{kernel}所關聯的事件完成後,
此\cnglo{kernel}調用 \capi{printf} 得到的輸出會刷入實作所定義的輸出數據流。
在\cnglo{cmdq}上調用 \capi{clFinish} 會將所有擱置的 \capi{printf} 輸出
(由之前入隊並完成的\cnglo{cmd}產生)刷入實作所定義的輸出數據流。
在多個\cnglo{workitem}並發執行 \capi{printf} 的情況下,寫入數據的順序沒有任何保證。
例如,\cnglo{glbid}為 \math{(0,0,1)}的\cnglo{workitem}的輸出
與\cnglo{glbid}為 \math{(0,0,4)}的\cnglo{workitem}的輸出很可能混雜在一起。

\subsubsection{printf 格式字串}

格式是一個字符序列,其開始和結束均位於其初始轉義狀態(shift state)下。
此格式可能包含零個或多個指示:普通字符(非 \cemp 引入,下面按順序列出了跟在 \cemp{%} 後面的內容:
\startigBase
\item 零個或多個{\ftRef{標誌}}({\ftRef{flag}})(順序任意),可修正轉換規約的意義。

\item 最小{\ftRef{欄寬}}({\ftRef{field width}}),可選。
如果轉換後,字符數小於欄寬,則在左側以空格填補達到欄寬
(缺省使用空格,如果有後面所描述的左側調整標誌,則會在右側填補)。
欄寬為非負十進制整數\footnote{注意,如果欄寬以 {\ftEmp{0}} 開頭,則將 {\ftEmp{0}} 當作標誌。}。

\item {\ftRef{精度}}({\ftRef{precision}}),可選。
如果出現在 \cemp{d}、 \cemp{i}、 \cemp{o}、 \cemp{u}、
 \cemp{x} 和 \cemp{X} 轉換中,則作為數字的最小位數;
如果出現在 \cemp{a}、 \cemp{A}、 \cemp{e}、 \cemp{E}、
 \cemp{f} 和 \cemp{F} 轉換中,則作為小數點後面數字的位數;
如果出現在 \cemp{g} 和 \cemp{G} 轉換中,則作為尾數數字的最大位數;
如果出現在 \cemp{s} 轉換中,則為要寫入的最大字節數。
精度的形式為點(.)後跟一個可選的十進制整數;
如果只有點,則用零作精度。
如果精度與其他任一轉換規約一同出現,則其行為未定義。

\item {\ftRef{矢量說明符}}({\ftRef{vector specifier}}),可選。

\item {\ftRef{長度修飾符}}({\ftRef{length modifier}}),指定引數的大小。
長度修飾符與矢量說明符一起指定矢量型別。
不允許在矢量型別間進行隱式轉換(遵守\refsection{implicityConversion})。
如果沒有指定矢量說明符,則長度修飾符是可選的。

\item {\ftRef{轉換說明符}}({\ftRef{conversion specifier}}),
用來指定要應用哪種轉換。
\stopigBase

標誌字符及其意義如下:

\startclSpecifier{\cemp{-}}
轉換結果在欄位內左對齊(如果沒有此標誌則為右對齊)。
\stopclSpecifier

\startclSpecifier{\cemp{+}}
帶符號的轉換,結果總會帶有正負號(如果沒有此標誌,則只有負數才帶有符號\footnote{
轉換浮點數時,負零以及捨入為零的負值都會使結果帶有負號。})。
\stopclSpecifier

\startclSpecifier{\cemp{space}}
如果帶符號轉換的第一個字符不是正負號,或者帶符號轉換的結果為空,
則在結果前添一個空格。
如果同時指定了 \ccmm{space} 和 \ccmm{+},則忽略 \ccmm{space}。
\stopclSpecifier

\startbuffer
\cemp{#}
\stopbuffer
\startclSpecifier{\getbuffer}
結果將是“另一種形式”。
對於 \cemp{o} 轉換,他會增加精度強制結果的第一個數字為零
(僅在有必要時才這樣做,如果值本身就是 0,則只打印一個 0)。
對於 \cemp{x} (或 \cemp{X})轉換,
非零結果將帶有前綴 \cemp{0x} (或 \cemp{0X})。
對於 \cemp{a}、 \cemp{A}、 \cemp{e}、 \cemp{E}、 \cemp{f}、
 \cemp{F}、 \cemp{g} 和 \cemp{G} 轉換,浮點數的轉換結果始終帶有小數點,
即使後面沒有數字(通常,後面有數字時才有小數點)。
對於 \cemp{g} 和 \cemp{G} 轉換,{\ftRef{不會}}移除結果中尾隨的零。
對於其他轉換,其行為未定義。
\stopclSpecifier

\startclSpecifier{\cemp{0}}
對於 \cemp{d}、 \cemp{i}、 \cemp{o}、 \cemp{u}、 \cemp{x}、
 \cemp{X}、 \cemp{a}、 \cemp{A}、 \cemp{e}、 \cemp{E}、
 \cemp{f}、 \cemp{F}、 \cemp{g} 和 \cemp{G} 轉換,
用前導零(跟在正負號或尾數後面)而不是空格將結果填補到欄寬,轉換無窮或 NaN 時除外。
如果 \cemp{0} 和 \cemp{-} 同時存在,則忽略 \cemp{0}。
對於 \cemp{d}、 \cemp{i}、 \cemp{o}、 \cemp{u}、 \cemp{x} 和
 \cemp{X} 轉換,如果指定了精度,則忽略 \cemp{0};
對於其他轉換,其行為未定義。
\stopclSpecifier

矢量說明符及其意義如下:

\startclSpecifier{\clemp[n]{v}}
表明後面的 \cemp{a}、 \cemp{A}、 \cemp{e}、 \cemp{E}、
 \cemp{f}、 \cemp{F}、 \cemp{g}、 \cemp{G}、
 \cemp{d}、 \cemp{i}、 \cemp{o}、 \cemp{u}、
 \cemp{x} 或 \cemp{X} 轉換說明符作用到矢量引數上,
其中 \cempsuffix{n} 是矢量的大小,必須是 2、 3、 4、 8 或 16。

矢量值按如下形式顯示:
\startclc[indentnext=no]
value1 C value2 C ... C valuen
\stopclc
其中 \ccmm{C} 是分隔符,此分隔符為逗號(\ccmm{,})。
\stopclSpecifier

% no vector specifier
如果沒有使用矢量說明符,長度修飾符及其意義如下:
\startclSpecifier{\cemp{hh}}
表明後面的 \cemp{d}、 \cemp{i}、 \cemp{o}、 \cemp{u}、
 \cemp{x} 或 \cemp{X} 轉換說明符作用到 \ctypeemp{char} 或 \ctypeemp{uchar} 引數上
(引數還是先按整數相應規則進行型別晉陞,
不過在打印前會先轉換成 \ctypeemp{char} 或 \ctypeemp{uchar})。
\stopclSpecifier

\startclSpecifier{\cemp{h}}
表明後面的 \cemp{d}、 \cemp{i}、 \cemp{o}、 \cemp{u}、
 \cemp{x} 或 \cemp{X} 轉換說明符作用到 \ctypeemp{short} 或 \ctypeemp{ushort} 引數上
(引數還是先按整數相應規則進行型別晉陞,
不過在打印前會先轉換成 \ctypeemp{short} 或 \ctypeemp{ushort})。
\stopclSpecifier

\startclSpecifier{\cemp{l}(\ccmm{ell})}
表明後面的 \cemp{d}、 \cemp{i}、 \cemp{o}、 \cemp{u}、
 \cemp{x} 或 \cemp{X} 轉換說明符作用到 \ctypeemp{long} 或 \ctypeemp{ulong} 引數上。
完整規格的 OpenCL 是支持修飾符 \cemp{l} 的。
而對於嵌入式規格的 OpenCL,僅當\cnglo{device}支持 64 位整數時才支持修飾符 \cemp{l}。
\stopclSpecifier

% has vector specifier
如果使用了矢量說明符,長度修飾符及其意義如下:
\startclSpecifier{\cemp{hh}}
表明後面的 \cemp{d}、 \cemp{i}、 \cemp{o}、 \cemp{u}、 \cemp{x} 或 \cemp{X}
 轉換說明符作用到 \cldtemp[n]{char} 或 \cldtemp[n]{uchar} 引數上
(不會對引數進行型別晉陞)。
\stopclSpecifier

\startclSpecifier{\cemp{h}}
表明後面的 \cemp{d}、 \cemp{i}、 \cemp{o}、 \cemp{u}、 \cemp{x} 或 \cemp{X}
 轉換說明符作用到 \cldtemp[n]{short} 或 \cldtemp[n]{ushort} 引數上
(不會對引數進行型別晉陞);
而後面的 \cemp{a}、 \cemp{A}、 \cemp{e}、 \cemp{E}、 \cemp{f}、 \cemp{F}、
 \cemp{g} 或 \cemp{G} 轉換說明符作用到 \cldtemp[n]{half} 引數上。
\stopclSpecifier

\startclSpecifier{\cemp{hl}}
此修飾符可以與矢量說明符一起使用。
表明後面的 \cemp{d}、 \cemp{i}、 \cemp{o}、 \cemp{u}、 \cemp{x} 或 \cemp{X}
 轉換說明符作用到 \cldtemp[n]{int} 或 \cldtemp[n]{uint} 引數上;
而後面的 \cemp{a}、 \cemp{A}、 \cemp{e}、 \cemp{E}、 \cemp{f}、 \cemp{F}、
 \cemp{g} 或 \cemp{G} 轉換說明符作用到 \cldtemp[n]{float} 引數上。
\stopclSpecifier

\startclSpecifier{\cemp{l}(\ccmm{ell})}
表明後面的 \cemp{d}、 \cemp{i}、 \cemp{o}、 \cemp{u}、 \cemp{x} 或 \cemp{X}
 轉換說明符作用到 \cldtemp[n]{long} 或 \cldtemp[n]{ulong} 引數上;
而後面的 \cemp{a}、 \cemp{A}、 \cemp{e}、 \cemp{E}、 \cemp{f}、 \cemp{F}、
 \cemp{g} 或 \cemp{G} 轉換說明符作用到 \cldtemp[n]{double} 引數上。
完整規格的 OpenCL 是支持修飾符 \cemp{l} 的。
而對於嵌入式規格的 OpenCL,
僅當\cnglo{device}支持 64 位整數或雙精度浮點數時才支持修飾符 \cemp{l}。
\stopclSpecifier

\startnotepar
如果沒有長度說明符,只有矢量說明符,則其行為未定義。
矢量說明符以及長度修飾符所描述的矢量數據型別必須與引數的型別一致;否則其行為未定義。

如果長度修飾符不是與上述轉換說明符一起出現的,則其行為未定義。
\stopnotepar

% conversion specifier
轉換說明符及其意義如下:

\startclSpecifier{\cemp{d},\cemp{i}}
將 \cldtemp{int}、 \cldtemp[n]{char}、 \cldtemp[n]{short}、
 \cldtemp[n]{int} 或 \cldtemp[n]{long} 型別的引數轉換成帶符號十進制數,
樣式為 \cfmt{[-]dddd}。
精度指定所顯式數字的最少位數;
如果所轉換的值可以用更少的數字表示,則添加前導零。
缺省精度為 1。
用精度零轉換零值,其結果為空,沒有任何字符。
\stopclSpecifier

\startclSpecifier{\cemp{o},\cemp{u},\cemp{x},\cemp{X}}
將 \cldtemp{unsigned int}、 \cldtemp[n]{uchar}、 \cldtemp[n]{ushort}、
 \cldtemp[n]{uint} 或 \cldtemp[n]{ulong} 型別的引數轉換成無符號八進制數(\cemp{o})、
無符號十進制數(\cemp{u})或無符號十六進制數(\cemp{x} 或 \cemp{X}),樣式為 \cfmt{dddd};
\cemp{x} 使用字母 \cemp{abcdef},而 \cemp{X} 使用字母 \cemp{ABCDEF}。
精度指定所顯式數字的最少位數;
如果所轉換的值可以用更少的數字表示,則添加前導零。
缺省精度為 1。
用精度零轉換零值,其結果為空,沒有任何字符。
\stopclSpecifier

\startclSpecifier{\cemp{f},\cemp{F}}
將 \cldtemp{double}、 \cldtemp[n]{half}、 \cldtemp[n]{float}
 或 \cldtemp[n]{double} 型別的浮點引數轉換成十進制符號,樣式為 \cfmt{[-]ddd.dddd},
其中小數點後面數字的位數等於精度。
如果沒有指定精度,則使用缺省值 6;
如果精度為零,並且沒有指定 \cemp{#} 標誌,則不會有小數部分(包括小數點)。
如果有小數點,則小數點前至少有一位數字。
會對引數進行捨入。
如果引數為無窮,
則轉換後樣式為 \cfmt{[-]inf} 或 \cfmt{[-]infinity}——至於是哪一種\cnglo{impdef}。
如果引數為 NaN,
則轉換後樣式為 \cfmt{[-]nan} 或 \cfmt{[-]nan(n-char-sequence)}
——至於是哪一種,以及\cfmt{n-char-sequence}的意義都\cnglo{impdef}。
\startbuffer
當作用於無窮和 NaN 值時, \cemp{-}、 \cemp{+} 和 \cemp{space} 標誌的意義不變;
而 \cemp{#} 和 \cemp{0} 標誌沒有效果。
\stopbuffer
轉換說明符 \cemp{F} 則會產生 \cemp{INF}、 \cemp{INFINITY} 或 \cemp{NAN},
分別代替 \cemp{inf}、 \cemp{infinity} 或 \cemp{nan}\footnote{\getbuffer}。
\stopclSpecifier

\startclSpecifier{\cemp{e}、 \cemp{E}}
將 \cldtemp{double}、 \cldtemp[n]{half}、 \cldtemp[n]{float}
 或 \cldtemp[n]{double} 型別的浮點引數轉換成十進制符號,
樣式為 \cfmt{[-]d.dddd e±dd},
小數點前面有一位數字(如果引數非零,則此數字也非零),
小數點後面數字的位數等於精度;
如果沒有指定精度,則使用缺省值 6;
如果精度為零,並且沒有指定 \cemp{#} 標誌,則不會有小數部分(包括小數點)。
會對引數進行捨入。
轉換說明符 \cemp{E} 所產生的結果會將指數前面的 \cemp{e} 換成 \cemp{E}。
指數至少包含兩位數字,如果多於兩位,則只包含表示指數所必要的數字。
如果引數的值為零,則指數也是零。
如果引數為無窮或 NaN,則轉換結果與轉換說明符 \cemp{f} 或 \cemp{F} 一樣。
\stopclSpecifier

\startclSpecifier{\cemp{g}、 \cemp{G}}
將 \cldtemp{double}、 \cldtemp[n]{half}、 \cldtemp[n]{float}
 或 \cldtemp[n]{double} 型別的浮點引數按 \cemp{f} 或 \cemp{e} 的樣式進行轉換
(如果是 \cemp{G},則按 \cemp{F} 或 \cemp{E} 的樣式進行轉換),
具體按哪種轉換取決於所要轉換的值以及精度。
如果精度非零,將其記為 \math{P};如果省略了精度,則讓 \math{P} 為 6;
如果精度為零,則讓 \math{P} 為 1。
然後,將按 \cemp{E} 轉換時的指數記為 \math{X},如果 \math{P > X \geq -4},
則按 \cemp{f}(或 \cemp{F})以及精度 \math{P-(X+1)} 進行轉換。
否則,按 \cemp{e}(或 \cemp{E})以及精度 \math{P-1} 進行轉換。
最後,除非使用了標誌 \cemp{#},否則移除小數部分中所有尾隨的零,
如果這樣小數部分為空,就連小數點一併移除。
如果引數為無窮或 NaN,則轉換結果與轉換說明符 \cemp{f} 或 \cemp{F} 一樣。
\stopclSpecifier

\startclSpecifier{\cemp{a}、 \cemp{A}}
\cldtemp{double}、 \cldtemp[n]{half}、 \cldtemp[n]{float}
 或 \cldtemp[n]{double} 型別的浮點引數轉換後的樣式為:
\starttyping
		/BTEX\ftRef{[-]}/ETEX/BTEX\ftEmp{0xh}/ETEX/BTEX\ftRef{.hhhh}/ETEX /BTEX\ftEmp{p}/ETEX±d
\stoptyping
其中小數點前有一位十六進制數字(如果引數為規格化浮點數,則此數字非零,否則未規定\footnote{
實作可以自行選擇小數點左邊的數字,以使後續數字按半字節(4 位)邊界對齊。}),
小數點後十六進制數字的位數與精度相同;
如果沒有指定精度,則選用剛好能確切表示其值的精度;
如果精度為零,而且沒有指定標誌 \cemp{#},則不會有小數部分(包括小數點)。
 \cemp{a} 使用字母 \cemp{abcdef}, \cemp{A} 使用字母 \cemp{ABCDEF}。
 \cemp{A} 會用 \cemp{X} 和 \cemp{P} 分別取代樣式中的 \cemp{x} 和 \cemp{p}。
指數至少包含一位數字,否則僅為表示十進制指數所必要的數字。
如果引數為零,則指數也是零。
如果引數為無窮或 NaN,則轉換結果與轉換說明符 \cemp{f} 或 \cemp{F} 一樣。
\stopclSpecifier

\startnotepar
僅當支持數據型別 \ctype{double} 時,
轉換說明符 \cemp{e}、 \cemp{E}、 \cemp{g}、 \cemp{G}、 \cemp{a}、 \cemp{A} 才會
將 \ctype{float} 或 \ctype{half} 型別的引數轉換成 \ctype{double}。
如果不支持 \ctype{double},則用 \ctypeemp{float} 取代 \ctypeemp{double} 作為引數,
同時會將 \ctype{half} 轉換為 \ctype{float}。
\stopnotepar

\startclSpecifier{\cemp{c}}
會將 \ctypeemp{int} 引數轉換為 \ctypeemp{unsigned char},並輸出字符。
\stopclSpecifier

\startclSpecifier{\cemp{s}}
引數必須是常值字串\footnote{
沒有專門針對多字節字符的條款。
對於 \capiemp{printf} 的轉換說明符 \cemp{s} 而言,
如果引數不是常值字串,則其行為未定義。}。
常值字串陣列中的字符都會被寫入輸出數據流,直到遇到 null 終止符,終止符不會被寫入。
如果指定了精度,所寫入字符的數目不會超過精度值。
如果沒有指定精度或者精度值大於陣列的大小,陣列將包含 null 字符\problem{}。
\stopclSpecifier

\startclSpecifier{\cemp{p}}
引數必須是指向 \ctypeemp{void} 的指位器。
此指位器可以指代 \cqlf{global}、 \cqlf{constant}、 \cqlf{local} 或 \cqlf{private}
 位址空間中的某個\cnglo{memregion}。
指位器的值轉換成一個可打印字符的序列,其內容\cnglo{impdef}。
\stopclSpecifier

\startbuffer
\cemp,不會轉換任何引數。
完整的轉換規約為 \cemp。
\stopclSpecifier

如果轉換規約無效,其行為未定義。
如果任一轉換規約的對應引數型別不正確,其行為也是未定義的。

任何情況下,欄寬較小和欄位不存在都不會引起欄位的截斷;
如果轉換結果比欄寬寬,則會對此欄位進行擴充,以包含整個轉換結果。

對於 \cemp{a} 和 \cemp{A} 轉換,會將引數正確捨入成具有給定精度的十六進制浮點數。

下面給出了 \capi{printf} 的一些例子。
\startclc
float4 f = (float4)(1.0f, 2.0f, 3.0f, 4.0f);
uchar4 uc = (uchar4)(0xFA, 0xFB, 0xFC, 0xFD);

printf("f4 = %2.2v4hlf\n", f);
printf("uc = %#v4hhx\n", uc);
\stopclc

上面兩次 \capi{printf} 調用打印結果如下:
\startclc
f4 = 1.00,2.00,3.00,4.00
uc = 0xfa,0xfb,0xfc,0xfd
\stopclc

下面是關於 \capi{printf} 的轉換說明符 \cemp{s} 的合法用例。
引數必須是常值字串:
\startclc
kernel void my_kernel( ... )
{
	printf(“%s\n”, “this is a test string\n”);
}
\stopclc

下面是關於 \capi{printf} 的轉換說明符 \cemp{s} 的無效用例。
\startclc
kernel void my_kernel(global char *s, ... )
{
	printf(“%s\n”, s);
}

constant char *p = “this is a test string\n”;
printf(“%s\n”, p);
printf(“%s\n”, &p[3]);
\stopclc

下面也是一個 \capi{printf} 的無效用例,
矢量說明符和長度修飾符所確定的數據型別與引數的型別不一致:
\startclc
kernel void my_kernel(global char *s, ... )
{
	uint2 ui = (uint2)(0x12345678, 0x87654321);
	printf("unsigned short value = (%#v2hx)\n", ui)
	printf("unsigned char value = (%#v2hhx)\n", ui)
}
\stopclc

% Differences between OpenCL C and C99 printf
\subsubsection{printf 在 OpenCL 和 C99 中的差異}

\startigBase
\item OpenCL C 中,不支持在轉換說明符 \cemp{c} 或 \cemp{s} 前使用修飾符 \cemp{l}。

\item OpenCL C 不支持長度說明符 \cemp{ll}、 \cemp{j}、 \cemp{z}、 \cemp{t} 以及
 \cemp{L},不過暫時保留以備將來可能使用。

\item OpenCL C 添加了可選的矢量說明符 \clemp[n]{v} 用以支持矢量型別的打印。

\item 僅當支持數據型別 \ctype{double} 時,
轉換說明符 \cemp{f}、 \cemp{F}、 \cemp{e}、 \cemp{E}、 \cemp{g}、 \cemp{G}、
 \cemp{a}、 \cemp{A} 才會將 \ctype{float} 引數轉換成 \ctype{double}。
參見 \reftab{cldevquery} 中的 \cenum{CL_DEVICE_DOUBLE_FP_CONFIG}。
如果不支持 \ctype{double},則引數為 \ctype{float} 而不是 \ctype{double}。

\item 對於嵌入式規格,僅當支持 64 位整數時才支持長度說明符 \cemp{l}。

\item 在 OpenCL C 中,如果執行成功, \capi{printf} 會返回 0,否則返回 -1。
而在 C99 中, \capi{printf} 會返回所打印字符的數目,
如果發生了輸出錯誤或編碼錯誤則返回一個負值。

\item 在 OpenCL C 中,轉換說明符 \cemp{s} 只能用於常值字串。
\stopigBase


% Image Read and Write Functions
\subsection[section:imgRwFunc]{圖像讀寫函式}

本節中所定義的內建函式僅能與\cnglo{imgobj}一起使用。

聲明可被\cnglo{kernel}讀取的\cnglo{imgobj}時應帶有限定符 \cqlf{__read_only}。
對帶有 \cqlf{__read_only} 的\cnglo{imgobj}調用 \capi{write_image} 會造成編譯錯誤。
聲明可被\cnglo{kernel}寫入的\cnglo{imgobj}時應帶有限定符 \cqlf{__write_only}。
對帶有 \cqlf{__write_only} 的\cnglo{imgobj}調用 \capi{read_image} 會造成編譯錯誤。
不支持在同一\cnglo{kernel}中對同一\cnglo{imgobj}
同時調用 \capi{read_image} 和 \capi{write_image}。

\capi{read_image} 返回的是一個四組件矢量浮點數、整數或無符號整數顏色值。
此值用 \ccmm{x}、 \ccmm{y}、 \ccmm{z}、 \ccmm{w} 來標識,
其中 \ccmm{x} 指代紅色分量, \ccmm{y} 指代綠色分量, \ccmm{z} 指代藍色分量,
 \ccmm{w} 指代 alpha 分量,每個分量都是一個矢量組件。

% Samplers
\subsubsection[section:samplerInImgRw]{採樣器}

圖像讀取函式的引數中有一個就是\cnglo{sampler}。
\cnglo{sampler}可作為引數由 \capi{clSetKernelArg} 傳給\cnglo{kernel},
也可以在 \cqlf{kernel} 函式的最外層聲明\cnglo{sampler},
或者是\cnglo{program}源碼中聲明的型別為 \ctype{sampler_t} 的常量。

\cnglo{program}中所聲明\cnglo{sampler}變量的型別為 \ctype{sampler_t}。
這種變量必須用 32 位無符號整形常數進行初始化,
按位欄解釋此常量,其位欄指定了下列屬性:
\startigBase[indentnext=no]
\item 尋址模式
\item 濾波模式
\item 歸一化坐標
\stopigBase
這些屬性控制着 \capi{read_image{f|i|ui}} 如何讀取圖像中的元素。

也可在\cnglo{program}源碼中將\cnglo{sampler}聲明為全局常量,下列三種語法均有效:
\startclc
const sampler_t		<sampler name> = <value>
constant sampler_t	<sampler name> = <value>
__constant sampler_t	<sampler_name> = <value>
\stopclc

\startnotepar
在計算每個\cnglo{device}中指向常數位址空間的引數數目或常數位址空間的大小時,
不考慮帶有限定符 \cqlf{constant} 的\cnglo{sampler}(參見\reftab{cldevquery}中的
 \cenum{CL_DEVICE_MAX_CONSTANT_ARGS} 和
 \cenum{CL_DEVICE_MAX_CONSTANT_BUFFER_SIZE})。
\stopnotepar

\placetable[here][tab:samplerDesc]
{採樣器描述符}
{\startCLOD[\cnglo{sampler}屬性][描述]

\clOD{\ccmm{<normalized coords>}}{
指定所傳入的坐標 \math{x}、 \math{y} 和 \math{z} 是否已歸一化。
他必須是常值,可以是下列預定義枚舉中的一個:
\startigBase
\item \cenum{CLK_NORMALIZED_COORDS_TRUE}
\item \cenum{CLK_NORMALIZED_COORDS_FALSE}
\stopigBase

在單個\cnglo{kernel}中針對同一圖像多次調用 \capi{read_image{f|i|ui}} 時,
所用\cnglo{sampler}中 \ccmm{<normalized coords>} 的值必須相同。
}

\clOD{\ccmm{<addressing mode>}}{
指定圖像的尋址模式,即圖像坐標溢出時如何處置。
他必須是常值,可以是下列預定義枚舉中的一個:
\startigBase
\item \cenum{CLK_ADDRESS_MIRRORED_REPEAT}——在整數接點處翻轉圖像坐標。
這種尋址模式只能用於歸一化坐標。
如果使用的不是歸一化坐標,則此模式生成的圖像坐標未定義。

\item \cenum{CLK_ADDRESS_REPEAT}——溢出的坐標會繞回到有效區間內。
這種尋址模式只能用於歸一化坐標。
如果使用的不是歸一化坐標,則此模式生成的圖像坐標未定義。

\item \cenum{CLK_ADDRESS_CLAMP_TO_EDGE}——溢出的坐標會被壓入有效範圍內。

\item \cenum{CLK_ADDRESS_CLAMP}\footnote{
與尋址模式\cenum{CLK_ADDRESS_CLAMP_TO_EDGE}類似。}——溢出的坐標會返回顏色極值。

\item \cenum{CLK_ADDRESS_NONE}——此模式下,由程式員保證坐標不會溢出,否則結果未定義。
\stopigBase

對於 1D 和 2D 圖像陣列,尋址模式僅對坐標 \math{x} 和 \math{(x,y)} 有效。
坐標中的陣列索引所用尋址模式始終是 \cenum{CLK_ADDRESS_CLAMP_TO_EDGE}。
}

\clOD{\ccmm{filter mode}}{
指定所用的濾波模式。
他必須是常值,可以是下列預定義枚舉中的一個:
\startigBase
\item \cenum{CLK_FILTER_NEAREST}
\item \cenum{CLK_FILTER_LINEAR}
\stopigBase

對於這些濾波模式的描述,請參見\refsection{addressFilterMode}。
}

\stopCLOD

}

例:
\startclc[indentnext=no]
const sampler_t /BTEX\ftEmp{samplerA}/ETEX = CLK_NORMALIZED_COORDS_TRUE
			| CLK_ADDRESS_REPEAT
			| CLK_FILTER_NEAREST;
\stopclc
\cnglo{sampler} {\ftEmp{samplerA}} 用的是規格化坐標、重複尋址模式和最近濾波。

對於一個\cnglo{kernel}中所能聲明\cnglo{sampler}的最大數目,
可以用 \capi{clGetDeviceInfo} 以 \cenum{CL_DEVICE_MAX_SAMPLERS} 進行查詢。

% Determining the border color
\subsubsubsection{確定顏色極值}

如果\cnglo{sampler}中的 \ccmm{<addressing mode>} 是 \cenum{CLK_ADDRESS_CLAMP},
則溢出的圖像坐標會返回顏色極值。
所選顏色極值取決於圖像通道順序,可以是下列中的一個:
\startigBase
\item 如果圖像通道順序為 \cenum{CL_A}、 \cenum{CL_INTENSITY}、 \cenum{CL_Rx}、
\cenum{CL_RA}、 \cenum{CL_RGx}、 \cenum{CL_RGBx, CL_ARGB}、 \cenum{CL_BGRA}
 或 \cenum{CL_RGBA},則所選顏色極值為 \ccmm{(0.0f, 0.0f, 0.0f, 0.0f)}。

\item 如果圖像通道順序為 \cenum{CL_R}、 \cenum{CL_RG}、 \cenum{CL_RGB}
 或 \cenum{CL_LUMINANCE},則所選顏色極值為 \ccmm{(0.0f, 0.0f, 0.0f, 1.0f)}。
\stopigBase

% Built-in Image Read Functions
\subsubsection[section:builtInImgReadFunc]{內建圖像讀取函式}

下列帶有\cnglo{sampler}的內建函式可用於讀取圖像\footnote{%
这些函式不支持型别为 \ctype{mage1d_buffer_t} 的图像。%
}。

\placetable[here,split][tab:imgReadFunc]
{內建圖像讀取函式}
{\startCLFD
\clFD{read_imagef_2d}
\clFD{read_imagei_2d}
\clFD{read_imagef_3d}
\clFD{read_imagei_3d}
\clFD{read_imagef_2da}
\clFD{read_imagei_2da}
\clFD{read_imagef_1d}
\clFD{read_imagei_1d}
\clFD{read_imagef_1da}
\clFD{read_imagei_1da}
\stopCLFD

}

% Built-in Image Sampler-less Read Functions
\subsubsection{內建無採樣器圖像讀取函式}

下列無\cnglo{sampler}的內建函式也可用於讀取圖像。
其行為與\refsection{builtInImgReadFunc}中坐標為整數、並帶有\cnglo{sampler}的對應函式一樣,
相當於\cnglo{sampler}的濾波模式為 \cenum{CLK_FILTER_NEAREST}、
歸一化坐標為 \cenum{CLK_NORMALIZED_COORDS_FALSE}、
尋址模式為 \cenum{CLK_ADDRESS_NONE}。

\placetable[here,split][tab:imgReadWithoutSamplerFunc]
{內建無採樣器圖像讀取函式}
{\startCLFD
\clFD{read_imagef_2d_s}
\clFD{read_imagei_2d_s}
\clFD{read_imagef_3d_s}
\clFD{read_imagei_3d_s}
\clFD{read_imagef_2da_s}
\clFD{read_imagei_2da_s}
\clFD{read_imagef_1d_s}
\clFD{read_imagei_1d_s}
\clFD{read_imagef_1da_s}
\clFD{read_imagei_1da_s}
\stopCLFD

}

% Built-in Image Write Functions
\subsubsection{內建圖像寫入函式}

下列內建函式可用於寫入圖像。

\placetable[here,split][tab:imgWriteFunc]
{內建圖像寫入函式}
{\startCLFD
\clFD{write_image_2d}
\clFD{write_image_2da}
\clFD{write_image_1d}
\clFD{write_image_1da}
\stopCLFD

}

% Built-in Image Query Functions
\subsubsection{內建圖像查詢函式}

下列內建函式可用於查詢圖像資訊。

\placetable[here,split][tab:imgQueryFunc]
{內建圖像查詢函式}
{% get_image_width
\startbuffer[funcproto:get_image_width]
int get_image_width (image1d_t image)
int get_image_width (
	image1d_buffer_t image)
int get_image_width (image2d_t image)
int get_image_width (image3d_t image)
int get_image_width (
	image1d_array_t image)
int get_image_width (
	image2d_array_t image)
\stopbuffer
\startbuffer[funcdesc:get_image_width]
返回圖像寬度,單位像素。
\stopbuffer

% get_image_height
\startbuffer[funcproto:get_image_height]
int get_image_height (image2d_t image)
int get_image_height (image3d_t image)
int get_image_height (
	image2d_array_t image)
\stopbuffer
\startbuffer[funcdesc:get_image_height]
返回圖像高度,單位像素。
\stopbuffer

% get_image_depth
\startbuffer[funcproto:get_image_depth]
int get_image_depth (image3d_t image)
\stopbuffer
\startbuffer[funcdesc:get_image_depth]
返回圖像深度,單位像素。
\stopbuffer

% get_image_channel_data_type
\startbuffer[funcproto:get_image_channel_data_type]
int get_image_channel_data_type (
	image1d_t image)
int get_image_channel_data_type (
	image1d_buffer_t image)
int get_image_channel_data_type (
	image2d_t image)
int get_image_channel_data_type (
	image3d_t image)
int get_image_channel_data_type (
	image1d_array_t image)
int get_image_channel_data_type (
	image2d_array_t image)
\stopbuffer
\startbuffer[funcdesc:get_image_channel_data_type]
返回通道數據類型,有效值有:
\startigBase
\item \cenum{CLK_SNORM_INT8}
\item \cenum{CLK_SNORM_INT16}
\item \cenum{CLK_UNORM_INT8}
\item \cenum{CLK_UNORM_INT16}
\item \cenum{CLK_UNORM_SHORT_565}
\item \cenum{CLK_UNORM_SHORT_555}
\item \cenum{CLK_UNORM_SHORT_101010}
\item \cenum{CLK_SIGNED_INT8}
\item \cenum{CLK_SIGNED_INT16}
\item \cenum{CLK_SIGNED_INT32}
\item \cenum{CLK_UNSIGNED_INT8}
\item \cenum{CLK_UNSIGNED_INT16}
\item \cenum{CLK_UNSIGNED_INT32}
\item \cenum{CLK_HALF_FLOAT}
\item \cenum{CLK_FLOAT}
\stopigBase
\stopbuffer

% get_image_channel_order
\startbuffer[funcproto:get_image_channel_order]
int get_image_channel_order (
	image1d_t image)
int get_image_channel_order (
	image1d_buffer_t image)
int get_image_channel_order (
	image2d_t image)
int get_image_channel_order (
	image3d_t image)
int get_image_channel_order (
	image1d_array_t image)
int get_image_channel_order (
	image2d_array_t image)
\stopbuffer
\startbuffer[funcdesc:get_image_channel_order]
返回通道順序,有效值有:
\startigBase
\item \cenum{CLK_A}
\item \cenum{CLK_R}
\item \cenum{CLK_Rx}
\item \cenum{CLK_RG}
\item \cenum{CLK_RGx}
\item \cenum{CLK_RA}
\item \cenum{CLK_RGB}
\item \cenum{CLK_RGBx}
\item \cenum{CLK_RGBA}
\item \cenum{CLK_ARGB}
\item \cenum{CLK_BGRA}
\item \cenum{CLK_INTENSITY}
\item \cenum{CLK_LUMINANCE}
\stopigBase
\stopbuffer

% get_image_dim_2d
\startbuffer[funcproto:get_image_dim_2d]
int2 get_image_dim (image2d_t image)
int2 get_image_dim (
	image2d_array_t image)
\stopbuffer
\startbuffer[funcdesc:get_image_dim_2d]
將 2D 圖像的寬度和高度存入 \ctype{int2} 返回。
其中組件 \carg{x} 是寬度,組件 \carg{y} 是高度。
\stopbuffer

% get_image_dim_3d
\startbuffer[funcproto:get_image_dim_3d]
int4 get_image_dim (image3d_t image)
\stopbuffer
\startbuffer[funcdesc:get_image_dim_3d]
將 3D 圖像的寬度、高度和深度存入 \ctype{int4} 返回。
其中組件 \carg{x} 是寬度,組件 \carg{y} 是高度,組件 \carg{z} 是深度,組件 \carg{w} 是 0。
\stopbuffer

% get_image_array_size_2d
\startbuffer[funcproto:get_image_array_size_2d]
size_t get_image_array_size(
	image2d_array_t image)
\stopbuffer
\startbuffer[funcdesc:get_image_array_size_2d]
返回 2D 圖像陣列中圖像的個數。
\stopbuffer

% get_image_array_size_1d
\startbuffer[funcproto:get_image_array_size_1d]
size_t get_image_array_size(
	image1d_array_t image)
\stopbuffer
\startbuffer[funcdesc:get_image_array_size_1d]
返回 1D 圖像陣列中圖像的個數。
\stopbuffer

% begin table
\startCLFD
\clFD{get_image_width}
\clFD{get_image_height}
\clFD{get_image_depth}
\clFD{get_image_channel_data_type}
\clFD{get_image_channel_order}
\clFD{get_image_dim_2d}
\clFD{get_image_dim_3d}
\clFD{get_image_array_size_2d}
\clFD{get_image_array_size_1d}
\stopCLFD
}

\reftab{imgQueryFunc}中,
\capi{get_image_channel_data_type} 和 \capi{get_image_channel_order} 所返回的帶有前綴 \cenum{CLK_} 的值
分別對應於\reftab{imgChannelDataType}和\reftab{imgChannelOrder}中帶有前綴 \cenum{CL_} 的值。
例如, \cenum{CL_UNORM_INT8} 和 \cenum{CLK_UNORM_INT8} 都是指通道數據類型為非歸一化的 8 位整數。

\subsubsection{将图像通道映射到 read_image 所返回的颜色值上、%
将传递给 write_image 的颜色值映射到图像通道上}

\reftab{ecComponentAndImgChannel}中列出了圖像元素的各通道的顏色值
與 \ctype{float4}、 \ctype{int4} 或 \ctype{uint4} 中組件的映射關係,
這些矢量由 \capi{read_image{f|i|ui}} 返回或作為 \capi{write_image{f|i|ui}} 的參數 \carg{color}。
對於未映射的組件,如果是紅、綠、藍幾個通道,則將其值置為 \ccmm{0.0},
而如果是 alpha 通道,則將其值置為 \math{1.0}。

\placetable[here][tab:ecComponentAndImgChannel]
{矢量組件與圖像通道的對應關係}{
\startCLOO[通道順序][矢量組件中的通道數據]
\clOO{\cenum{CL_R}、 \cenum{CL_Rx}}{\ccmm{(r, 0.0, 0.0, 1.0)}}
\clOO{\cenum{CL_A}}{\ccmm{(0.0, 0.0, 0.0, a)}}

\clOO{\cenum{CL_RGB}、 \cenum{CL_RGBx}}{\ccmm{(r, g, 0.0, 1.0)}}
\clOO{\cenum{CL_RA}}{\ccmm{(r, 0.0, 0.0, a)}}

\clOO{\cenum{CL_RG}、 \cenum{CL_RGx}}{\ccmm{(r, g, b, 1.0)}}
\clOO{\cenum{CL_RGBA}、 \cenum{CL_BGRA}、 \cenum{CL_ARGB}}{\ccmm{(r, g, b, a)}}

\clOO{\cenum{CL_INTENSITY}}{\ccmm{(I, I, I, I)}}
\clOO{\cenum{CL_LUMINANCE}}{\ccmm{(L, L, L, 1.0)}}
\stopCLOO
}

如果\cnglo{kernel}對多個圖像使用同一個尋址模式為 \cenum{CL_ADDRESS_CLAMP} 的\cnglo{sampler},
則可能導致實作內部使用額外的\cnglo{sampler}。
如果通過 \capi{read_image{f | i | ui}} 對多個圖像使用同一\cnglo{sampler},
則實作可能需要分配額外的\cnglo{sampler}來處理不同的顏色極值(這取決於所用的圖像格式)。
在計算\cnglo{device}所支持\cnglo{sampler}的最大數目時(\cenum{CL_DEVICE_MAX_SAMPLERS}),
會將這些實作自行分配的\cnglo{sampler}考慮在內。
如果所入隊的\cnglo{kernel}需要的\cnglo{sampler}超過了這個最大值,
則會導致返回 \cenum{CL_OUT_OF_RESOURCES}。


