% get_work_dim
\startbuffer[funcproto:get_work_dim]
uint get_work_dim()
\stopbuffer

\startbuffer[funcdesc:get_work_dim]
返回所用的維度數目。即 \capi{clEnqueueNDRangeKernel} 的引數 \carg{work_dim} 的值。

如果用的是 \capi{clEnqueueTask},則返回 1。
\stopbuffer

% get_global_size
\startbuffer[funcproto:get_global_size]
size_t get_global_size(uint dimindx)
\stopbuffer

\startbuffer[funcdesc:get_global_size]
返回在 \carg{dimindx} 所標識的維度上指定的全局\cnglo{workitem}的數目。
即 \capi{clEnqueueNDRangeKernel} 的引數 \carg{global_work_size}。
\carg{dimindx} 的有效範圍為 0 到 \math{\mapiemp{get_work_dim}()-1}。
如果 \carg{dimindx} 是其他值,則 \capi{get_global_size} 返回 1。

如果用的是 \capi{clEnqueueTask},則返回 1。
\stopbuffer

% get_global_id
\startbuffer[funcproto:get_global_id]
size_t get_global_id(uint dimindx)
\stopbuffer

\startbuffer[funcdesc:get_global_id]
返回\cnglo{workitem}在 \carg{dimindx} 所標識的維度上的唯一\cnglo{glbid}。
此 ID 基於用來執行此\cnglo{kernel}的全局\cnglo{workitem}的數目。
\carg{dimindx} 的有效範圍為 0 到 \math{\mapiemp{get_work_dim}() - 1}。
如果 \carg{dimindx} 是其他值,則 \capi{get_global_id} 返回 0。

如果用的是 \capi{clEnqueueTask},則返回 0。
\stopbuffer

% get_local_size
\startbuffer[funcproto:get_local_size]
size_t get_local_size(uint dimindx)
\stopbuffer

\startbuffer[funcdesc:get_local_size]
返回在 \carg{dimindx} 所標識的維度上指定的局部\cnglo{workitem}的數目。
如果 \capi{clEnqueueNDRangeKernel} 的引數 \carg{local_work_size} 不是 \cmacro{NULL},
則返回的就是此引數的值;
否則 OpenCL 實作會選擇一個恰當的 \carg{local_work_size} 並將其返回。
\carg{dimindx} 的有效範圍為 0 到 \math{\mapiemp{get_work_dim}()-1}。
如果 \carg{dimindx} 是其他值,則 \capi{get_local_size} 返回 1。

如果用的是 \capi{clEnqueueTask},則返回 1。
\stopbuffer

% get_local_id
\startbuffer[funcproto:get_local_id]
size_t get_local_id(uint dimindx)
\stopbuffer

\startbuffer[funcdesc:get_local_id]
返回\cnglo{workitem}在\cnglo{workgrp}內 \carg{dimindx} 所標識的維度上的唯一\cnglo{locid}。
\carg{dimindx} 的有效範圍為 0 到 \math{\mapiemp{get_work_dim}() - 1}。
如果 \carg{dimindx} 是其他值,則 \capi{get_local_id} 返回 0。

如果用的是 \capi{clEnqueueTask},則返回 0。
\stopbuffer

% get_num_groups
\startbuffer[funcproto:get_num_groups]
size_t get_num_groups(uint dimindx)
\stopbuffer

\startbuffer[funcdesc:get_num_groups]
返回執行此\cnglo{kernel}的\cnglo{workgrp}在 \carg{dimindx} 所標識的維度上的數目。
\carg{dimindx} 的有效範圍為 0 到 \math{\mapiemp{get_work_dim}() - 1}。
如果 \carg{dimindx} 是其他值,則 \capi{get_num_groups} 返回 1。

如果用的是 \capi{clEnqueueTask},則返回 1。
\stopbuffer

% get_group_id
\startbuffer[funcproto:get_group_id]
size_t get_group_id(uint dimindx)
\stopbuffer

\startbuffer[funcdesc:get_group_id]
所返回的\cnglo{workgrp} ID 取值範圍為
 \math{0 ... \mapiemp{get_work_dim}(\marg{dimindx}) - 1}。
\carg{dimindx} 的有效範圍為 0 到 \math{\mapiemp{get_work_dim}() - 1}。
如果 \carg{dimindx} 是其他值,則 \capi{get_group_id} 返回 0。

如果用的是 \capi{clEnqueueTask},則返回 0。
\stopbuffer

% get_global_offset
\startbuffer[funcproto:get_global_offset]
size_t get_global_offset (uint dimindx)
\stopbuffer

\startbuffer[funcdesc:get_global_offset]
返回的是
為 \capi{clEnqueueNDRangeKernel} 的引數 \carg{global_work_offset} 所指定的偏移值。
\carg{dimindx} 的有效範圍為 0 到 \math{\mapiemp{get_work_dim}() - 1}。
如果 \carg{dimindx} 是其他值,則 \capi{get_group_id} 返回 0。

如果用的是 \capi{clEnqueueTask},則返回 0。
\stopbuffer


% begin TABLE
\startCLFD

\clFD{get_work_dim}
\clFD{get_global_size}
\clFD{get_global_id}
\clFD{get_local_size}
\clFD{get_local_id}
\clFD{get_num_groups}
\clFD{get_group_id}
\clFD{get_global_offset}

\stopCLFD

