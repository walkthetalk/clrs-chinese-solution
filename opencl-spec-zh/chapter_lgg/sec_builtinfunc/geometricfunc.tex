\subsection[section:geomtricFunc]{幾何函式}

\reftab{svGeometricFunc}中列出了內建的幾何函式。
這些函式都是按組件逐一運算的,其中的描述也是針對單個組件的。
\cldt[n]{float} 表示
 \ctype{float}、 \ctype{float2}、 \ctype{float3} 或 \ctype{float4};
而 \cldt[n]{double} 則表示
 \ctype{double}、 \ctype{double2}、 \ctype{double3} 或 \ctype{double4}。

內建的幾何函式實現時用的捨入模式是捨入為最近偶數。

\startnotepar
可以使用化簡(如 \capi{mad} 或 \capi{fma})來實現幾何函式。
\stopnotepar

\placetable[here][tab:svGeometricFunc]
{引數既可為標量,也可為矢量的內建幾何函式}
{\startCLFD

\clFD{cross}
\clFD{dot}
\clFD{distance}
\clFD{length}
\clFD{normalize}
\clFD{fast_distance}
\clFD{fast_length}
\clFD{fast_normalize}

\stopCLFD
}
