\subsection{雜類矢量函式}

另外,OpenCL C 編程語言還實現了\reftab{miscVectorFunc}中所列內建矢量函式。
如無特殊說明,泛型 \cldt[n]{gentype} (或 \cldt[m]{gentype})表示函式引數的型別可以為:
\startigBase[indentnext=no]
\item \ctype{char{2|4|8|16}}、 \ctype{uchar{2|4|8|16}}、
\item \ctype{short{2|4|8|16}}、 \ctype{ushort{2|4|8|16}}、
\item \ctype{half{2|4|8|16}}\footnote{%
僅當支持擴展 \clext{cl_khr_fp16} 時才有效。}、
\item \ctype{int{2|4|8|16}}、 \ctype{uint{2|4|8|16}}、
\item \ctype{long{2|4|8|16}}、 \ctype{ulong{2|4|8|16}}、
\item \ctype{float{2|4|8|16}} 或 \ctype{double{2|4|8|16}}\footnote{%
僅當支持雙精度浮點數時才有效}。
\stopigBase
而泛型 \cldt[n]{ugentype} 則為內建無符號整形。

\placetable[here][tab:miscVectorFunc]
{內建雜類矢量函式}
{% vec_step
\startbuffer[funcproto:vec_step]
int vec_step (gentypen a) 

int vec_step (char3 a)
int vec_step (uchar3 a)
int vec_step (short3 a)
int vec_step (ushort3 a)
int vec_step (half3 a)
int vec_step (int3 a)
int vec_step (uint3 a)
int vec_step (long3 a)
int vec_step (ulong3 a)
int vec_step (float3 a)
int vec_step(double3 a)

int vec_step(type)
\stopbuffer
\startbuffer[funcdesc:vec_step]
此函式的引數為標量或矢量,返回值為引數中的元素數目。

對於所有標量型別,返回值為 1。

對於 3 組件矢量,返回值為 4。

引數也可以是純型別,如 \math{\mapiemp{vec_step}(\mtype{float2})}。
\stopbuffer

% shuffle
\startbuffer[funcproto:shuffle]
gentypen shuffle (
	gentypem x,
	ugentypen mask)
gentypen shuffle2 (
	gentypem x,
	gentypem y,
	ugentypen mask)
\stopbuffer
\startbuffer[funcdesc:shuffle]
輸入為一個或兩個型別相同的矢量,由其中的元素構造一個排列並返回,
所返回的矢量元素型別與輸入相同,矢量長度與掩碼相同。
掩碼中每個元素的大小必須與結果中元素大小相同。

對於 \capi{shuffle},
只考慮 \carg{mask} 中每個元素的 \math{\mapiemp{ilogb}(2m-1)} 個最低有效位。
而對於 \capi{shuffle2},
則會考慮 \carg{mask} 中每個元素的 \math{\mapiemp{ilogb}(2m-1)+1} 個最低有效位。
忽略 \carg{mask} 中的其他位。

因為要對輸入矢量中的元素從左到右進行編號(第二個輸入矢量的編號續接第一個的),
因此需要用 \math{\mapiemp{vec_step}(\text{\cldt[m]{gentype}})} 得到矢量元素數目。
引數 \carg{mask} 用來確定結果中的相應元素選自輸入矢量中的哪個元素。

例:
\startcintbl
uint4 mask = (uint4)(3, 2, 1, 0);
float4 a;
float4 r = shuffle(a, mask);
// r.s0123 = a.wzyx

uint8 mask = (uint8)(0, 1, 2, 3, 4, 5, 6, 7);
float4 a, b;
float8 r = shuffle2(a, b, mask);
// r.s0123 = a.xyzw
// r.s4567 = b.xyzw

uint4 mask;
float8 a;
float4 b;
b = shuffle(a, mask);
\stopcintbl

而無效的例子有:
\startcintbl
uint8 mask;
short16 a;
short8 b;
b = shuffle(a, mask); <- not valid
\stopcintbl
\stopbuffer

% begin table
\startCLFD
\clFD{vec_step}
\clFD{shuffle}
\stopCLFD
}
