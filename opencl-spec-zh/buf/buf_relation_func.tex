% isequal
\startbuffer[funcproto:isequal]
int isequal (float x, float y)
intn isequal (floatn x, floatn y)
int isequal (double x, double y)
longn isequal (doublen x, doublen y)
\stopbuffer
\startbuffer[funcdesc:isequal]
按組件逐一比較 \ccmm{x == y}。
\stopbuffer

% isequal_half
\startbuffer[funcproto:isequal_half]
int isequal (half x, half y)
shortn isequal (halfn x, halfn y)
\stopbuffer
\startbuffer[funcdesc:isequal_half]
按組件逐一比較 \ccmm{x == y}。
\stopbuffer

% isnotequal
\startbuffer[funcproto:isnotequal]
int isnotequal (float x, float y)
intn isnotequal (floatn x, floatn y)
int isnotequal (double x, double y)
longn isnotequal (doublen x, doublen y)
\stopbuffer
\startbuffer[funcdesc:isnotequal]
按組件逐一比較 \ccmm{x != y}。
\stopbuffer

% isnotequal_half
\startbuffer[funcproto:isnotequal_half]
int isnotequal (half x, half y)
shortn isnotequal (halfn x, halfn y)
\stopbuffer
\startbuffer[funcdesc:isnotequal_half]
按組件逐一比較 \ccmm{x != y}。
\stopbuffer

% isgreater
\startbuffer[funcproto:isgreater]
int isgreater (float x, float y)
intn isgreater (floatn x, floatn y)
int isgreater (double x, double y)
longn isgreater (doublen x, doublen y)
\stopbuffer
\startbuffer[funcdesc:isgreater]
按組件逐一比較 \ccmm{x > y}。
\stopbuffer

% isgreater_half
\startbuffer[funcproto:isgreater_half]
int isgreater (half x, half y)
shortn isgreater (halfn x, halfn y)
\stopbuffer
\startbuffer[funcdesc:isgreater_half]
按組件逐一比較 \ccmm{x > y}。
\stopbuffer

% isgreaterequal
\startbuffer[funcproto:isgreaterequal]
int isgreaterequal (float x, float y)
intn isgreaterequal (floatn x, floatn y)
int isgreaterequal (double x,
		double y)
longn isgreaterequal (doublen x,
		doublen y)
\stopbuffer
\startbuffer[funcdesc:isgreaterequal]
按組件逐一比較 \ccmm{x >= y}。
\stopbuffer

% isgreaterequal_half
\startbuffer[funcproto:isgreaterequal_half]
int isgreaterequal (half x, half y)
shortn isgreaterequal (halfn x, halfn y)
\stopbuffer
\startbuffer[funcdesc:isgreaterequal_half]
按組件逐一比較 \ccmm{x >= y}。
\stopbuffer

% isless
\startbuffer[funcproto:isless]
int isless (float x, float y)
intn isless (floatn x, floatn y)
int isless (double x, double y)
longn isless (doublen x, doublen y)
\stopbuffer
\startbuffer[funcdesc:isless]
按組件逐一比較 \ccmm{x < y}。
\stopbuffer

% isless_half
\startbuffer[funcproto:isless_half]
int isless (half x, half y)
shortn isless (halfn x, halfn y)
\stopbuffer
\startbuffer[funcdesc:isless_half]
按組件逐一比較 \ccmm{x < y}。
\stopbuffer

% islessequal
\startbuffer[funcproto:islessequal]
int islessequal (float x, float y)
intn islessequal (floatn x, floatn y)
int islessequal (double x, double y)
longn islessequal (doublen x, doublen y)
\stopbuffer
\startbuffer[funcdesc:islessequal]
按組件逐一比較 \ccmm{x <= y}。
\stopbuffer

% islessequal_half
\startbuffer[funcproto:islessequal_half]
int islessequal (half x, half y)
shortn islessequal (halfn x, halfn y)
\stopbuffer
\startbuffer[funcdesc:islessequal_half]
按組件逐一比較 \ccmm{x <= y}。
\stopbuffer

% islessgreater
\startbuffer[funcproto:islessgreater]
int islessgreater (float x, float y)
intn islessgreater (floatn x, floatn y)
int islessgreater (double x, double y)
longn islessgreater (doublen x, doublen y)
\stopbuffer
\startbuffer[funcdesc:islessgreater]
按組件逐一比較 \ccmm{(x < y) || (x > y)}。
\stopbuffer

% islessgreater_half
\startbuffer[funcproto:islessgreater_half]
int islessgreater (half x, half y)
shortn islessgreater (halfn x, halfn y)
\stopbuffer
\startbuffer[funcdesc:islessgreater_half]
按組件逐一比較 \ccmm{(x < y) || (x > y)}。
\stopbuffer

% isfinite
\startbuffer[funcproto:isfinite]
int isfinite (float)
intn isfinite (floatn)
int isfinite (double)
longn isfinite (doublen)
\stopbuffer
\startbuffer[funcdesc:isfinite]
測試引數是否為有限值。
\stopbuffer

% isfinite_half
\startbuffer[funcproto:isfinite_half]
int isfinite (half)
shortn isfinite (halfn)
\stopbuffer
\startbuffer[funcdesc:isfinite_half]
測試引數是否為有限值。
\stopbuffer

% isinf
\startbuffer[funcproto:isinf]
int isinf (float)
intn isinf (floatn)
int isinf (double)
longn isinf (doublen)
\stopbuffer
\startbuffer[funcdesc:isinf]
測試引數是否為無窮值(正數或負數)。
\stopbuffer

% isinf_half
\startbuffer[funcproto:isinf_half]
int isinf (half)
shortn isinf (halfn)
\stopbuffer
\startbuffer[funcdesc:isinf_half]
測試引數是否為無窮值(正數或負數)。
\stopbuffer

% isnan
\startbuffer[funcproto:isnan]
int isnan (float)
intn isnan (floatn)
int isnan (double)
longn isnan (doublen)
\stopbuffer
\startbuffer[funcdesc:isnan]
測試引數是否為 NaN。
\stopbuffer

% isnan_half
\startbuffer[funcproto:isnan_half]
int isnan (half)
shortn isnan (halfn)
\stopbuffer
\startbuffer[funcdesc:isnan_half]
測試引數是否為 NaN。
\stopbuffer

% isnormal
\startbuffer[funcproto:isnormal]
int isnormal (float)
intn isnormal (floatn)
int isnormal (double)
longn isnormal (doublen)
\stopbuffer
\startbuffer[funcdesc:isnormal]
測試引數是否為規格化值。
\stopbuffer

% isnormal_half
\startbuffer[funcproto:isnormal_half]
int isnormal (half)
shortn isnormal (halfn)
\stopbuffer
\startbuffer[funcdesc:isnormal_half]
測試引數是否為規格化值。
\stopbuffer

% isordered
\startbuffer[funcproto:isordered]
int isordered (float x, float y)
intn isordered (floatn x, floatn y)
int isordered (double x, double y)
longn isordered (doublen x, doublen y)
\stopbuffer
\startbuffer[funcdesc:isordered]
測試引數是否規則。
相當於 \ccmm{isequal(x, x) && isequal(y, y)}。
\stopbuffer

% isordered_half
\startbuffer[funcproto:isordered_half]
int isordered (half x, half y)
shortn isordered (halfn x, halfn y)
\stopbuffer
\startbuffer[funcdesc:isordered_half]
測試引數是否規則。
相當於 \ccmm{isequal(x, x) && isequal(y, y)}。
\stopbuffer

% isunordered
\startbuffer[funcproto:isunordered]
int isunordered (float x, float y)
intn isunordered (floatn x, floatn y)
int isunordered (double x, double y)
longn isunordered (doublen x, doublen y)
\stopbuffer
\startbuffer[funcdesc:isunordered]
測試引數是否不規則。
如果引數 \carg{x} 或 \carg{y} 是 NaN,則返回非零值,否則返回零。
\stopbuffer

% isunordered_half
\startbuffer[funcproto:isunordered_half]
int isunordered (half x, half y)
shortn isunordered (halfn x, halfn y)
\stopbuffer
\startbuffer[funcdesc:isunordered_half]
測試引數是否不規則。
如果引數 \carg{x} 或 \carg{y} 是 NaN,則返回非零值,否則返回零。
\stopbuffer

% signbit
\startbuffer[funcproto:signbit]
int signbit (float)
intn signbit (floatn)
int signbit (double)
longn signbit (doublen)
\stopbuffer
\startbuffer[funcdesc:signbit]
測試符號位。
對於此函式的標量版本,如果設置了符號位,則返回 1,否則返回 0。
而在此函式的標量版本中,對於矢量的每個組件,
如果設置了符號位則返回 -1 (即所有位都是 1),否則返回 0。
\stopbuffer

% signbit_half
\startbuffer[funcproto:signbit_half]
int signbit (half)
shortn signbit (halfn)
\stopbuffer
\startbuffer[funcdesc:signbit_half]
測試符號位。
對於此函式的標量版本,如果設置了符號位,則返回 1,否則返回 0。
而在此函式的標量版本中,對於矢量的每個組件,
如果設置了符號位則返回 -1 (即所有位都是 1),否則返回 0。
\stopbuffer

% any
\startbuffer[funcproto:any]
int any (igentype x)
\stopbuffer
\startbuffer[funcdesc:any]
如果 \carg{x} 中任一組件的最高位是 1,則返回 1;否則返回 0。
\stopbuffer

% all
\startbuffer[funcproto:all]
int all (igentype x)
\stopbuffer
\startbuffer[funcdesc:all]
如果 \carg{x} 中所有組件的最高位都是 1,則返回 1;否則返回 0。
\stopbuffer

% bitselect
\startbuffer[funcproto:bitselect]
gentype bitselect (gentype a,
		gentype b,
		gentype c)
\stopbuffer
\startbuffer[funcdesc:bitselect]
如果 \carg{c} 中的某一位為 0,則選取 \carg{a} 中的對應位作為結果中對應位的值;
否則選取 \carg{b} 中的對應位作為結果中對應位的值。
\stopbuffer

% bitselect_half
\startbuffer[funcproto:bitselect_half]
halfn bitselect (halfn a,
		halfn b,
		halfn c)
\stopbuffer
\startbuffer[funcdesc:bitselect_half]
如果 \carg{c} 中的某一位為 0,則選取 \carg{a} 中的對應位作為結果中對應位的值;
否則選取 \carg{b} 中的對應位作為結果中對應位的值。
\stopbuffer

% select
\startbuffer[funcproto:select]
gentype select (gentype a,
		gentype b,
		igentype c)
gentype select (gentype a,
		gentype b,
		ugentype c)
\stopbuffer
\startbuffer[funcdesc:select]
對於矢量型別中的每個組件,
如果 \ccmm{c[i]} 的最高位為 1,則結果為 \ccmm{b[i]},否則為 \ccmm{a[i]}。

對於標量型別,\ccmm{result = c ? b : a}。

\ctype{igentype} 和 \ctype{ugentype} 的元素數目以及元素的位數
都必須與 \ctype{gentype} 相同。
\stopbuffer

% select_half
\startbuffer[funcproto:select_half]
halfn select (halfn a,
		halfn b,
		shortn c)
halfn select (halfn a,
		halfn b,
		ushortn c)
\stopbuffer
\startbuffer[funcdesc:select_half]
對於矢量型別中的每個組件,\ccmm{result[i] = 如果 c[i] 的最高位為 1 ? b[i] : a[i]}。

\ctype{igentype} 和 \ctype{ugentype} 的元素數目以及元素的位數
都必須與 \ctype{gentype} 相同。
\stopbuffer
