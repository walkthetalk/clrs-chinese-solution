\startETD[cl_context_info][返回型別]

\clETD{CL_CONTEXT_REFERENCE_COUNT}{cl_unit}{
返回 \carg{context} 的\cnglo{refcnt}\footnote{%
在返回的那一刻,此\cnglo{refcnt}就已過時。\cnglo{app}中一般不太適用。提供此特性主要是為了檢測內存泄漏。}。
}

\clETD{CL_CONTEXT_NUM_DEVICES}{cl_unit}{
返回 \carg{context} 中\cnglo{device}的數目。
}

\clETD{CL_CONTEXT_DEVICES}{cl_device_id[]}{
返回 \carg{context} 中\cnglo{device}的清單。
}

\clETD{CL_CONTEXT_PROPERTIES}{cl_context_properties[]}{
返回調用 \capi{clCreateContext} 或 \capi{clCreateContextFromType}
時所指定的引數 \carg{properties}。

對於調用 \capi{clCreateContext} 或 \capi{clCreateContextFromType}
創建 \carg{context} 時所指定的引數 \carg{properties} 而言,
如果此引數不是 \cmacro{NULL},實作必須返回此引數的值。
而如果此引數是 \cmacro{NULL},實作可以選擇將 \carg{param_value_size_ret} 置為 0,
即没有返回属性值,
也可以將 \carg{param_value} 的內容置為 0
(0 用作\cnglo{context}屬性清單的終止標記)。
}

\stopETD

