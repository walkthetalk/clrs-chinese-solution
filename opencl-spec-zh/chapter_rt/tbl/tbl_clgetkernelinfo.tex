\startETD[cl_kernel_info][返回型別]

\clETD{CL_KERNEL_FUNCTION_NAME}{char[]}{
返回\cnglo{kernel}函式的名字。
}

\clETD{CL_KERNEL_NUM_ARGS}{cl_uint}{
返回 \carg{kernel} 的參數個數。
}

\clETD{CL_KERNEL_REFERENCE_COUNT}{cl_uint}{
返回 \carg{kernel} 的\cnglo{refcnt}\footnote{%
在返回的那一刻,此\cnglo{refcnt}就已過時。
\cnglo{app}中一般不太適用。提供此特性主要是為了檢測內存泄漏。}。
}

\clETD{CL_KERNEL_CONTEXT}{cl_context}{
返回 \carg{kernel} 所在的\cnglo{context}。
}

\clETD{CL_KERNEL_PROGRAM}{cl_program}{
返回 \carg{kernel} 所關聯的\cnglo{programobj}。
}

\clETD{CL_KERNEL_ATTRIBUTES}{char[]}{
返回\cnglo{program}源碼中聲明\cnglo{kernel}函式時
所有通過限定符 \cqlf{__attribute__} 指定的特性。
這些特性包括\refsection{funcAttr}中所列特性以及實作所支持的其他特性。

所返回的特性就是聲明時 \ccmm{__attribute__((...))} 中的內容,
但是會移除兩頭的空格以及內嵌的換行。
如果有多個特性,則在所返回的字串中以空格來分隔。
}

\stopETD
