\section[section:profileMoKernel]{對內存對象以及內核的評測}

一些 OpenCL 函式可以作為\cnglo{cmd}插入到\cnglo{cmdq}中,
所入隊的\cnglo{cmd}是通過唯一的\cnglo{evtobj}來標識的。
本節將描述對這些函式的評測(profiling)。
這些函式\footnote{%
《OpenCL 1.2 擴展規範》的{\ftRef{節 9.6.6}} 中定義的
\capi{clEnqueueAcquireGLObjects} 和 \capi{clEnqueueReleaseGLObjects} 也包含在內。}%
是:
\startigBase
\item \capi{clEnqueue{Read|Write|Map}Buffer}
\item \capi{clEnqueue{Read|Write}BufferRect}
\item \capi{clEnqueue{Read|Write|Map}Image}
\item \capi{clEnqueueUnmapMemObject}
\item \capi{clEnqueueCopyBuffer}
\item \capi{clEnqueueCopyBufferRect}
\item \capi{clEnqueueCopyImage}
\item \capi{clEnqueueCopyImageToBuffer}
\item \capi{clEnqueueCopyBufferToImage}
\item \capi{clEnqueueNDRangeKernel}
\item \capi{clEnqueueTask}
\item \capi{clEnqueueNativeKernel}。
\stopigBase

\topclfunc{clGetEventProfilingInfo}

\startCLFUNC
cl_int clGetEventProfilingInfo (cl_event event,
			cl_profiling_info param_name,
			size_t param_value_size,
			void *param_value,
			size_t *param_value_size_ret)
\stopCLFUNC

如果使能了評測,則此函式會返回 \carg{event} 所指代\cnglo{cmd}的評測資訊。

\carg{event} 即\cnglo{evtobj}。

\carg{param_name} 指定所要查詢的評測數據。
\reftab{clGetEventProfilingInfo}中列出了所支持的資訊類型以及 \carg{param_value} 中返回的資訊。

\carg{param_value} 指向的內存用來存儲查詢結果。
如果 \carg{param_value} 是 \cmacro{NULL},則忽略。

\carg{param_value_size} 即 \carg{param_value} 所指內存塊的大小。
其值必須 >= \reftab{clGetEventProfilingInfo}中返回型別的大小。

\carg{param_value_size_ret} 返回查詢結果的實際大小。
如果 \carg{param_value_size_ret} 是 \cmacro{NULL},則忽略。

\placetable[here][tab:clGetEventProfilingInfo]
{\capi{clGetEventProfilingInfo} 所支持的 \carg{param_names}}
{\startETD[cl_profiling_info][返回型別]

\clETD{CL_PROFILING_COMMAND_QUEUED}{cl_ulong}{
64 位值,
描述的是當 \carg{event} 所標識的\cnglo{cmd}入隊時,
\cnglo{device}上時間計數器的值。
單位:納秒。
}

\clETD{CL_PROFILING_COMMAND_SUBMIT}{cl_ulong}{
64 位值,
描述的是當 \carg{event} 所標識的\cnglo{cmd}提交給\cnglo{cmdq}所在\cnglo{device}時,
\cnglo{device}上時間計數器的值。
單位:納秒。
}

\clETD{CL_PROFILING_COMMAND_START}{cl_ulong}{
64 位值,
描述的是當 \carg{event} 所標識的\cnglo{cmd}在\cnglo{device}上開始執行時,
\cnglo{device}上時間計數器的值。
單位:納秒。
}

\clETD{CL_PROFILING_COMMAND_QUEUED}{cl_ulong}{
64 位值,
描述的是當 \carg{event} 所標識的\cnglo{cmd}在\cnglo{device}上執行完畢時,
\cnglo{device}上時間計數器的值。
單位:納秒。
}

\stopETD
}

所返回的無符號 64 位值可用來衡量 OpenCL \cnglo{cmd}所消耗的時間,單位:納秒。

在\cnglo{device}的頻率和電源的狀態發生變化時,
要求 OpenCL \cnglo{device}能夠正確地跟蹤時間。
 \cenum{CL_DEVICE_PROFILING_TIMER_RESOLUTION} 指定了時鐘的精度,
即,定時器增一時所經過的納秒數。

如果執行成功,且記錄了評測資訊,
則 \capi{clGetEventProfilingInfo} 會返回 \cenum{CL_SUCCESS}。
否則,返回下列錯誤碼之一:
\startigBase
\startitem
\cenum{CL_PROFILING_INFO_NOT_AVAILABLE},如果:
\startigBase
\item \cnglo{cmdq}沒有設置 \cenum{CL_QUEUE_PROFILING_ENABLE};
\item 或者 \carg{event} 所標識\cnglo{cmd}的執行狀態不是 \cenum{CL_COMPLETE};
\item 或者 \carg{event} 是用戶\cnglo{evtobj}。
\stopigBase
\stopitem

\item \cenum{CL_INVALID_VALUE},如果 \carg{param_name} 無效;
或者 \carg{param_value_size} 的值 < \reftab{clGetEventProfilingInfo}中返回型別的大小,
且 \carg{param_value} 不是 \cmacro{NULL}。

\item \cenum{CL_INVALID_EVENT},如果 \carg{event} 無效。

\item \cenum{CL_OUT_OF_RESOURCES},如果\scdevfailres。

\item \cenum{CL_OUT_OF_HOST_MEMORY},如果\schostfailres。
\stopigBase
