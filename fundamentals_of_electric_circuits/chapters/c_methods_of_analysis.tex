\startcomponent c_methods_of_analysis
\startchapter[
  title={Methods of Analysis},
]

% 3.1
\startQ[Q:3.1]
Using \reffig{3.50}, design a problem to help other
students better understand nodal analysis.
\placefigq{3.50}{3.1}
\stopQ

\startA
...
\stopA

% 3.2
\startQ[Q:3.2]
For the circuit in \reffig{3.51}, obtain $v_1$ and $v_2$.
\placefigq{3.51}{3.2}
\stopQ

\startA
\startformula\startmathalignment
\NC \frac{v_2-v_1}{2} - \frac{v_1}{10} - \frac{v_1}{5} = 6 \NC \NR
\NC \frac{v_2-v_1}{2} + \frac{v_2}{4} = 6 + 3 \NC \NR
\stopmathalignment\stopformula

$v_1 = \sunit{0 volt}$, $v_2 = \sunit{12 volt}$.
\stopA

% 3.3
\startQ[Q:3.3]
Find the current $I_1$ through $I_4$ and the voltage $v_o$
in the circuit of \reffig{3.52}.
\placefigq{3.52}{3.3}
\stopQ

\startA
$I_1 = \sunit{6 ampere}$\\
$I_2 = \sunit{3 ampere}$\\
$I_3 = \sunit{2 ampere}$\\
$I_4 = \sunit{1 ampere}$\\
\stopA

% 3.4
\startQ[Q:3.4]
Given the circuit in \reffig{3.53},
caculate the currents $i_1$ through $i_4$.
\placefigq{3.53}{3.4}
\stopQ

\startA
$I_1 = \sunit{3 ampere}$\\
$I_2 = \sunit{6 ampere}$\\
$I_3 = \sunit{-0.5 ampere}$\\
$I_4 = \sunit{-0.5 ampere}$\\
\stopA

% 3.5
\startQ[Q:3.5]
Obtain $v_o$ in the circuit of \reffig{3.54}.
\placefigq{3.54}{3.5}
\stopQ

\startA
\startformula
\frac{v_o}{120k} + \frac{v_o + 120}{120k} + \frac{v_o+120-60}{30k} = 0
\stopformula
$v_o = \sunit{-60 volt}$.
\stopA

% 3.6
\startQ[Q:3.6]
Solve for $V_1$ in the circuit of \reffig{3.55}
using nodal analysis.
\placefigq{3.55}{3.6}
\stopQ

\startA
$\frac{V_1}{10} = \frac{20-V_1}{4} + \frac{10-V_1}{10/3}$\\
$V_1 = \frac{160}{13}\approx \sunit{12.308 volt}$.
\stopA

% 3.7
\startQ[Q:3.7]
Apply nodal analysis to solve for $V_x$ in the circuit of \reffig{3.56}.
\placefigq{3.56}{3.7}
\stopQ

\startA
$\frac{V_x}{60} + \frac{V_x}{30} + 0.05 V_x = 2$\\
$V_x = \sunit{20 volt}$.
\stopA

% 3.8
\startQ[Q:3.8]
Using nodal analysis, find $v_o$ in the circuit of \reffig{3.57}.
\placefigq{3.57}{3.8}
\stopQ

\startA
\startformula\startmathalignment
\NC i_o = \frac{v_o}{4} \NC \NR
\NC v_1 = i_o \times (6+4) \NC \NR
\NC \frac{v_1 - 5 v_o}{20} + i_o = \frac{60-v_1}{20} \NC \NR
\stopmathalignment\stopformula
$v_o = \sunit{12 volt}$.
\stopA

% 3.9
\startQ[Q:3.9]
Determine $I_b$ in the circuit in \reffig{3.58} using nodal analysis.
\placefigq{3.58}{3.9}
\stopQ

\startA
\startformula\startmathalignment
\NC V_{50} = 24 - 250 I_b \NC \NR
\NC I_b = \frac{V_{50}}{50} + \frac{V_{50}-60I_b}{150} \NC \NR
\stopmathalignment\stopformula

$I_b = \frac{48}{605}\sunit[l]{ampere}\approx 79.34\sunit[l]{milli ampere}$.
\stopA

% 3.10
\startQ[Q:3.10]
Find $I_o$ in the circuit of \reffig{3.59}.
\placefigq{3.59}{3.10}
\stopQ

\startA
\startformula\startmathalignment
\NC I_4 + 2 I_o = I_o + 4\NC\NR
\NC I_o = I_2 + I_4 \NC\NR
\NC 8I_o + 4I_4 + (4 + I_o) = 0 \NC\NR
\stopmathalignment\stopformula

$I_o = -4\sunit[l]{ampere}$.
\stopA

% 3.11
\startQ[Q:3.11]
Find $V_o$ and the power dissipated in all the resistors
in the circuit of \reffig{3.60}.
\placefigq{3.60}{3.11}
\stopQ

\startA
\startformula
\frac{60-V_o}{12} = \frac{V_o}{12} + \frac{V_o + 24}{6}
\stopformula

$V_o = \sunit{3 volt}$\\
$p_1 = \frac{(60-3)^2}{12} = 270.75\sunit[l]{watt}$\\
$p_2 = \frac{3^2}{12} = 0.75\sunit[l]{watt}$\\
$p_3 = \frac{(3+24)^2}{6} = 121.5\sunit[l]{watt}$.
\stopA

% 3.12
\startQ[Q:3.12]
Using nodal analysis, determine $V_o$ in the circuit in \reffig{3.61}.
\placefigq{3.61}{3.12}
\stopQ

\startA
\startformula\startmathalignment
\NC\NC (\frac{V_o}{10} - 4 I_x + I_x) \times 20 + I_x \times 20 = 40 \NR
\NC\NC (\frac{V_o}{10} - 4 I_x)\times 10 + V_o = I_x \times 20 \NR
\stopmathalignment\stopformula

$V_o = 60\sunit[l]{volt}$, $I_x = 2\sunit[l]{ampere}$.
\stopA

% 3.13
\startQ[Q:3.13]
Calculate $v_1$ and $v_2$ in the circuit of \reffig{3.62} using nodal analysis.
\placefigq{3.62}{3.13}
\stopQ

\startA
\startformula\startmathalignment
\NC\NC \frac{v_2}{30} + \frac{v_1}{50} = 15 \NR
\NC\NC \frac{v_1}{50}\times 10 + v_1 = 600 + v_2 \NR
\stopmathalignment\stopformula

$v_1 = 1750/3 \approx 583.3\sunit[l]{volt}$,
$v_2 = 100\sunit[l]{volt}$.
\stopA

% 3.14
\startQ[Q:3.14]
Using nodal analysis, find $v_o$ in the circuit of \reffig{3.63}.
\placefigq{3.63}{3.14}
\stopQ

\startA
\startformula\startmathalignment
\NC\NC v_x + 100 + (12.5 + \frac{v_x}{1})\times 2 = v_o \NR
\NC\NC \frac{v_x}{1} + \frac{v_o}{4} + \frac{v_o+50}{8} = 0 \NR
\stopmathalignment\stopformula

$v_x = -25\sunit[l]{volt}$,
$v_o = 50\sunit[l]{volt}$.
\stopA

% 3.15
\startQ[Q:3.15]
Apply nodal analysis to find $i_o$ and the power dissipated
in each resistor in the circuit of \reffig{3.64}.
\placefigq{3.64}{3.15}
\stopQ

\startA
\startformula\startmathalignment
\NC\NC \frac{i_o}{6} + \frac{i_o-4}{5} = 10 \NR
\stopmathalignment\stopformula

$i_o = \frac{324}{11}\approx 29.45\sunit[l]{ampere}$\\
$p_6 = \frac{17496}{121} \approx 144.60\sunit[l]{watt}$\\
$p_5 = \frac{15680}{121} \approx 129.59\sunit[l]{watt}$\\
$p_3 = 12\sunit[l]{watt}$
\stopA

% 3.16
\startQ[Q:3.16]
Determine voltages $v_1$ through $v_3$ in the circuit
of \reffig{3.65} using nodal analysis.
\placefigq{3.65}{3.16}
\stopQ

\startA
\startformula\startmathalignment
\NC\NC v_2 = v_o \NR
\NC\NC v_1 = 2 v_o + v_2 \NR
\NC\NC v_1 \times 1 - 2 + v_2\times 4
       + (v_2-v_3)\times 8 + (v_1-v_3)\times 2 = 0 \NR
\stopmathalignment\stopformula

$v_3 = \sunit{13 volt}$\\
$v_2 = \frac{44}{7} \approx \sunit{6.286 volt}$\\
$v_1 = \frac{132}{7} \approx \sunit{18.857 volt}$
\stopA

% 3.17
\startQ[Q:3.17]
Using nodal analysis, find current $i_o$ in the circuit of \reffig{3.66}.
\placefigq{3.66}{3.17}
\stopQ

\startA
\startformula\startmathalignment
\NC\NC i_2 = i_o - \frac{60 - 4i_o}{8} \NR
\NC\NC (i_2 + 3i_o)\times 10 + 4i_o + 2i_2 = 0 \NR
\stopmathalignment\stopformula

$i_o=\frac{45}{26}\approx \sunit{1.73 ampere}$.
\stopA

\stopchapter
\stopcomponent
