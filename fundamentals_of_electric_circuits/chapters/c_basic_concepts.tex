\startcomponent c_basic_concepts
\startchapter[
  title={Basic Concepts},
]

% 1.1
\startQ
How much charge is represented by these number of electrons?
\startIG
\item $\munit{6.482e17}$
\item $\munit{1.24=e18}$
\item $\munit{2.46=e19}$
\item $\munit{1.628e20}$
\stopIG
\stopQ

\startA
\startIG
\item $q = (\munit{1.602e-19 coulomb}) \times \munit{6.482e17} = \munit{_0.10384 coulomb}$
\item $q = (\munit{1.602e-19 coulomb}) \times \munit{1.24=e18} = \munit{_0.19865 coulomb}$
\item $q = (\munit{1.602e-19 coulomb}) \times \munit{2.46=e19} = \munit{_3.941== coulomb}$
\item $q = (\munit{1.602e-19 coulomb}) \times \munit{1.628e20} = \munit{26.08=== coulomb}$
\stopIG
\stopA

% 1.2
\startQ
Determine the current flowing through an element if the charge flow is given by
\startIG
\item $q(t) = (3t+8)               \munit[l]{milli coulomb}$
\item $q(t) = (8t^2 + 4t − 2)      \munit[l]{coulomb}$
\item $q(t) = (3e^{−t} − 5e^{−2t}) \munit[l]{nano coulomb}$
\item $q(t) = 10 \sin 120π t       \munit[l]{pico coulomb}$
\item $q(t) = 20e^{−4t} \cos 50t   \munit[l]{micro coulomb}$
\stopIG
\stopQ

\startA
\startIG
\item $i = \frac{d q}{d t} = 3                      \munit[l]{milli ampere}$
\item $i = \frac{d q}{d t} = (16t + 4)              \munit[l]{ampere}$
\item $i = \frac{d q}{d t} = (-3e^{−t} + 10e^{−2t}) \munit[l]{nano ampere}$
\item $i = \frac{d q}{d t} = 1200π \cos 120π t      \munit[l]{pico ampere}$
\item $i = \frac{d q}{d t} = -(80 \cos 50t + 1000 \sin 50 t) e^{−4t} \munit[l]{micro ampere}$
\stopIG
\stopA

% 1.3
\startQ
Find the charge $q(t)$ flowing through a device if the current is:
\startIG
\item $i(t) = \munit{3 ampere}, q(0) = \munit{1 coulomb}$
\item $i(t) = (2t + 5) \munit[l]{milli ampere}, q(0) = 0$
\item $i(t) = 20 \cos (10t + π / 6) \munit[l]{micro ampere}, q(0) = \munit{2 micro coulomb}$
\item $i(t) = 10e^{−30t} \sin 40t \munit[l]{ampere}, q(0) = 0$
\stopIG
\stopQ

\startA
\startIG
\item $q(t) = \int_0 i(t) + q(0) = (3t+1) \munit[l]{coulomb}$
\item $q(t) = \int_0 i(t) + q(0) = (t^2 + 5t) \munit[l]{milli coulomb}$
\item $q(t) = \int_0 i(t) + q(0) = (2 \sin (10t + π / 6) + 1) \munit[l]{micro coulomb}$
\item $q(t) = \int_0 i(t) + q(0) = (-e^{-30t}(0.16\cos 40t + 0.12\sin 40t) + 0.16) \munit[l]{coulomb}$
\stopIG
\stopA

% 1.4
\startQ
A total charge of $\munit{300 coulomb}$ flows past a given cross section of a conductor in 30 seconds.
What is the value of the current?
\stopQ

\startA
$\munit[r]{300 coulomb} / \munit[l]{30 second} = \munit{10 ampere}$
\stopA

% 1.5
\startQ
Determine the total charge transferred over the time interval
of $0 \le t \le \munit{10 second}$ when $i(t) = \frac{1}{2}t \munit[l]{ampere}$.
\stopQ

\startA
\startformula
q(t) = \int_0^{10}\frac{1}{2}t = \left.\frac{t^2}{4}\right|_0^{10} = 25
\stopformula
\stopA

% 1.6
\startQ[Q:1.6]
The charge entering a certain element is shown in \reffig{1.23}. Find the current at:
\startIG
\item $t = \munit{_1 milli second}$
\item $t = \munit{_6 milli second}$
\item $t = \munit{10 milli second}$
\stopIG
\placefigure[here][fig:1.23]{for \refq{1.6}}{
\startMPcode
	defcoord(10,2,15,15,1.5);
	defdim.x("t(ms)")(2,4,6,8,10,12);
	defdim.y("q(t)(mC)")(30);
	draw coordsys;
	drawemp((0,0)--(2,30)--(8,30)--(12,0));
	drawaux((2,0)--(2,30));
	drawaux((8,0)--(8,30));
\stopMPcode
}
\stopQ

\startA
\startIG
\item $i(\munit{t}) = \frac{dq}{dt} = \frac{d(15t)}{dt}= 15, i(1) = 15 \munit[l]{coulomb}$
\item $i(\munit{t}) = \frac{dq}{dt} = \frac{d(30)}{dt} = 0, i(6) = 0 \munit[l]{coulomb}$
\item $i(\munit{t}) = \frac{dq}{dt} = \frac{d(90-7.5t)}{dt} = -7.5, i(10) = -7.5 \munit[l]{coulomb}$
\stopIG
\stopA

% 1.7
\startQ[Q:1.7]
The charge flowing in a wire is plotted in \reffig{1.24}. Sketch the corresponding current.
\placefigure[here][fig:1.24]{for \refq{1.7}}{
\startMPcode
	defcoord(20,2,15,15,1.5);
	defdim.x("t(s)")(1,2,3,4);
	defdim.y("q(C)")(-10,10);
	draw coordsys;
	drawemp((0,0)--(1,10)--(2,-10)--(3,-10)--(4,0));
\stopMPcode
}
\stopQ

\startA
\startMPcode
	defcoord(20,2,15,15,1.5);
	defdim.x("t(s)")(1,2,3,4);
	setloc.x("lrt")(1,2);
	defdim.y("q(C)")(-20,-10,10);
	draw coordsys;
	drawemp((0,10)--(1,10)--(1,-20)--(2,-20)--(2,0)--(3,0)--(3,10)--(4,10));
\stopMPcode
\stopA

% 1.8
\startQ[Q:1.8]
 The current flowing past a point in a device is shown in \reffig{1.25}.
 Calculate the total charge through the point.
\placefigure[here][fig:1.25]{for \refq{1.8}}{
\startMPcode
	defcoord(30,3,15,15,1.5);
	defdim.x("t(ms)")(1,2);
	defdim.y("i(mA)")(10);
	draw coordsys;
	drawemp((0,0)--(1,10)--(2,10)--(2,0));
\stopMPcode
}
\stopQ

\startA
\startformula
q = \int i dt = \frac{10\times 1}{2} + 10\times 1 = \munit{15 milli coulomb}
\stopformula
\stopA

% 1.9
\startQ[Q:1.9]
The current through an element is shown in \reffig{1.26}.
Determine the total charge that passed through the element at:
\startcolumns[n=3]
\startIG
\item $t=\munit{1 second}$
\item $t=\munit{3 second}$
\item $t=\munit{5 second}$
\stopIG
\stopcolumns
\placefigure[here][fig:1.26]{for \refq{1.9}}{
\startMPcode
	defcoord(15,3,15,15,1.5);
	defdim.x("t(s)")(1,2,3,4,5);
	defdim.y("i(A)")(5,10);
	draw coordsys;
	drawemp((0,10)--(1,10)--(2,5)--(4,5)--(5,0));
\stopMPcode
}
\stopQ

\startA
\startIG
\item $q=\int_0^1 i dt = \int_0^1 10 dt = \munit{10 coulomb}$
\item $q=\int_0^3 i dt
        = \int_0^1 i dt + \int_1^2 i dt         + \int_2^3 i dt
        =\int_0^1 10 dt + \int_1^2 (15 - 5t) dt + \int_2^3 5 dt
        =10             + 7.5                   + 5
        =\munit{22.5 coulomb}$
\item $q=\int_0^5 i dt
        =\int_0^3 i dt + \int_3^4 i dt + \int_4^5 i dt
        = 22.5         + 5             + 2.5
        =\munit{30 coulomb}$
\stopIG
\stopA

% 1.10
\startQ
A lightning bolt with \munit{10 kilo ampere} strikes an object for \munit{15 micro second}.
How much charge is deposited on the object?
\stopQ

\startA
$q = (\munit{10e3 ampere}) \times (\munit{15e-6 second})
   = \munit{150e-3 coulomb} = \munit{0.15 coulomb}$
\stopA

% 1.11
\startQ
A rechargeable flashlight battery is capable of delivering
\munit{90 milli ampere} for about \munit{12 hour}.
How much charge can it release at that rate?
If its terminal voltage is \munit{1.5 volt},
how much energy can the battery deliver?
\stopQ
\startA
$q = (\munit{90e-3 ampere}) \times (12 \times 3600 \munit{hour})
   = \munit{3888 coulomb}$

$E = pt = ivt = qv = \munit{3888 coulomb} \times \munit{1.5 volt} = \munit{5832 joule}$
\stopA

% 1.12
\startQ
If the current flowing through an element is given by
\startformula
i(t)=\startmathcases
\NC 3t \munit[l]{ampere}, \MC 0 \le t < \munit{6 second} \NR
\NC \munit{18 ampere},    \MC 6 \le t < \munit{10 second} \NR
\NC \munit{-12 ampere},   \MC 10\le t < \munit{15 second} \NR
\NC 0, \MC t \ge \munit{15 second} \NR
\stopmathcases
\stopformula
Plot the charge stored in the element over $0 < t < \munit{20 second}$.

\startMPcode
	defcoord(5,2,15,15,1.5);
	defdim.x("(t) s")(6,10,15);
	setloc.x("lrt")(10,15);
	defdim.y("i(t) A")(-12,18);
	draw coordsys;
	drawemp((0,0)--(6,18)--(10,18)--(10,-12)--(15,-12)--(15,0)--(17,0));
\stopMPcode
\stopQ

\startA
\startformula
q(t)=\startmathcases
\NC 1.5t^2 \munit[l]{coulomb},  \MC 0 \le t < \munit{6 second} \NR
\NC 18t-54 \munit[l]{ampere},   \MC 6 \le t < \munit{10 second} \NR
\NC -12t+246 \munit[l]{ampere}, \MC 10\le t < \munit{15 second} \NR
\NC 66 \munit[l]{ampere},       \MC t \ge \munit{15 second} \NR
\stopmathcases
\stopformula

\startMPcode
	defcoord(7,0.7,15,15,1.5);
	defdim.x("(t) s")(6,10,15);
	setloc.x("lrt")(10,15);
	defdim.y("q(t) C")(18,54,66,126);
	draw coordsys;

	%drawref((0,0)--(6,18)--(10,18)--(10,-12)--(15,-12)--(15,0)--(16,0));

	polydef.main(0,6,1,20)(0,0,1.5);
	drawauxp.p((6,54),(10,126),(15,66));
	drawemp(polycurve.main--(10,126)--(15,66)--(16,66));
\stopMPcode
\stopA

% 1.13
\startQ
The charge entering the positive terminal of an element is
$q = 5 \sin 4 π t \sunit[l]{milli coulomb}$,
while the voltage across the element (plus to minus) is
$v = 3 \cos 4 π t \sunit[l]{volt}$,
\startIG
\item Find the power delivered to the element at $t = \sunit{0.3 second}$.
\item Calculate the energy delivered to the element between 0 and \sunit{0.6 second}.
\stopIG
\stopQ

\startA
\startIG
\item $i = \frac{dq}{dt} = 20π\cos 4π t \sunit[l]{milli ampere}$,
$p = vi = 60π\cos^2 4π t \sunit[l]{milli watt}$,
at $t = \sunit{0.3 second}$:
\startformula
p(0.3) = 60π\cos^2 (4π \times 0.3) = 123.37 \sunit[l]{milli watt}
\stopformula

\item $E = \int p d t = \int 60π\cos^2 4π t dt = 60π\int \cos^2 4π t dt
	 = 60π \int (1 + \cos 8π t) dt \sunit[l]{milli joule}$
\startformula
E = 60π (\left.t + \frac{\sin 8π t}{8π}\right|_0^{0.6}) = 117.5 \sunit[l]{milli joule}
\stopformula
\stopIG
\stopA

% 1.14
\startQ
The voltage v(t) across a device and the current i(t) through it are
$v(t) = 20 \sin (4t) \sunit[l]{volt}$ and $i(t) = 10(1 + e^{−2t}) \sunit[l]{milli ampere}$,
Calculate:
\startIG
\item the total charge in the device at $t = \sunit{1 second}$, $q(0) = 0$.
\item the power consumed by the device at $t = \sunit{1 second}$.
\stopIG
\stopQ

\startA
\startIG
\item $q = \int i(t) dt
         = \int 10(1+e^{-2t}) dt
         = \left.(10t - 5 e^{-2t})\right|_0^1
         = 5-5e^{-2}
         = 4.323 \sunit{milli ampere}$
\item $p = v(t)i(t)
         = 20\sin(4t) \times 10 (1 + e^{-2t})
         = 200\sin(4t) \times (1 + e^{-2t})$, $p(1) = -171.845 \sunit[l]{milli watt}$
\stopIG
\stopA

% 1.15
\startQ
The current entering the positive terminal of a device is
$i(t) = 6e^{−2t} \sunit[l]{milli ampere}$
and the voltage across the device is
$v(t) = 10\frac{di}{dt}\sunit[l]{volt}$.
\startIG
\item Find the charge delivered to the device between
$t = 0$ and $t = \sunit{2 second}$.
\item Calculate the power absorbed.
\item Determine the energy absorbed in \sunit{3 second}.
\stopIG
\stopQ

\startA
\startIG
\item $q = \int i(t) dt
         = \left.-3e^{-2t}\right|_0^2
         = 3 - 3 e^{-4}
         = 2.945 \sunit[l]{milli coulomb}$
\item $v(t) = 10\frac{di}{dt} = -120e^{-2t}\sunit[l]{volt}$,
$p = v(t) i(t) = -720e^{-4t} \sunit[l]{milli watt}$
\item $E = \int p dt = -0.99889 \sunit[l]{milli joule}$
\stopIG
\stopA

% 1.16
\startQ[Q:1.16]
\reffig{1.27} shows the current through and the voltage across an element.
\startIG
\item Sketch the power delivered to the element for $t>0$.
\item Fnd the total energy absorbed by the element for the period of $0 < t < \sunit{4 second}$.
\stopIG
\placefigure[here][fig:1.27]{for \refq{1.16}}{
\startcombination[2*1]
{\startMPcode
	defcoord(20,1,15,15,1.5);
	defdim.x("t(s)")(2,4);
	defdim.y("i(mA)")(60);
	draw coordsys;

	drawemp((0,0)--(2,60)--(4,0));
\stopMPcode}{a}{\startMPcode
	defcoord(20,5,15,15,1.5);
	defdim.x("t(s)")(2,4);
	setloc.x("lrt")(2,4);
	defdim.y("v(V)")(-5,5);
	draw coordsys;

	drawemp((0,5)--(2,5)--(2,-5)--(4,-5)--(4,0));
\stopMPcode}{b}
\stopcombination
}
\stopQ

\startA
\startformula\startmathalignment[align={left}]
\NC
  i(t) = \startmathcases
         \NC 30t \sunit[l]{milli ampere} \MC 0 < t < 2 \NR
         \NC 120-30t \sunit[l]{milli ampere} \MC 2 < t < 4 \NR
         \stopmathcases
\NC
  v(t) = \startmathcases
         \NC 5 \sunit[l]{volt} \MC 0 < t < 2 \NR
         \NC -5\sunit[l]{volt} \MC 2 < t < 4 \NR
         \stopmathcases
\NR

\NC
  p(t) = \startmathcases
         \NC 150t    \sunit[l]{milli watt} \MC 0 < t < 2 \NR
         \NC 150t-600\sunit[l]{milli watt} \MC 2 < t < 4 \NR
         \stopmathcases
\NC
  E = \int p(t) dt = \startmathmatrix[left={\left(},right={\right)}]
                     \NC \left.75t^2\right|_0^2 \NR
                     \NC + \NR
                     \NC \left.75t^2-600t\right|_2^4 \NR
                     \stopmathmatrix
                   = \sunit{0 joule}
\NR
\stopmathalignment\stopformula

\startMPcode
	defcoord(20,0.1,15,15,1.5);
	defdim.x("t(s)")(2,4);
	defdim.y("p(mW)")(-300,300);
	draw coordsys;

	drawemp((0,0)--(2,300)--(2,-300)--(4,0));
\stopMPcode
\stopA

% 1.17
\startQ[Q:1.17]
\reffig{1.28} shows a circuit with four elements,
$p_1 = \sunit{60 watt}$ absorbed, $p_3 = \sunit{−145 watt}$ absorbed,
and $p_4 = \sunit{75 watt}$ absorbed.
How many watts does element $2$ absorb?
\placefigure[here][fig:1.28]{for \refq{1.17}}{
\startMPcode
	clearObjsForTwiceRun;
	setX("fig128");

	newR.X(r1) "angle(90)";
	ObjLabel.X(r1)(textext("1")) "labcolor(FireBrick)";

	newR.X(r2) "angle(90)";
	ObjLabel.X(r2)(textext("2")) "labcolor(FireBrick)";

	newR.X(r4) "angle(90)";
	ObjLabel.X(r4)(textext("4")) "labcolor(FireBrick)";

	newR.X(r3) "angle( 0)";
	ObjLabel.X(r3)(textext("3")) "labcolor(FireBrick)";

	X(r1).c = origin;
	X(r2).c = X(r1).c+(30,0);
	X(r4).c = X(r2).c+(60,0);
	X(r3).c = (X(r2).c+X(r4).c)/2 + (0,30);

	drawObj(X(r1),X(r2),X(r3),X(r4));

	pcsimple(X(r3))(2)(X(r4))(2);
	pcsimple(X(r3))(1)(X(r2))(2);
	pcsimple(X(r1))(2)(X(r3))(1);
	pcsimple(X(r2))(1)(X(r4))(1)
		"armA(ypart(X(r3).c-X(r2).n))";
	pcsimple(X(r2))(1)(X(r1))(1)
		"armA(ypart(X(r3).c-X(r2).n))";
\stopMPcode
}
\stopQ

\startA
\stopA

% 1.18
\startQ[Q:1.18]
Find the power absorbed by each of the elements in \reffig{1.29}.

\placefigure[here][fig:1.29]{for \refq{1.18}}{
\startMPcode
	clearObjsForTwiceRun;
	setX("fig129");

	newIVS.X(u1) "angle(90)";
	ObjMLabel.X(u1)("p_1")
		"labpoint(is)", "labdir(rt)";
	ObjMLabel.X(u1)("30V")
		"labpoint(in)", "labdir(lft)";

	newR.X(r2);
	ObjMLabel.X(r2)("p_2")
		"labpoint(is)", "labdir(bot)";
	ObjMLabel.X(r2)("+")
		"labpoint(inw)", "labdir(top)";
	ObjMLabel.X(r2)("-")
		"labpoint(ine)", "labdir(top)";
	ObjMLabel.X(r2)("10V")
		"labpoint(in)", "labdir(top)";

	newR.X(r3) "angle(90)";
	ObjMLabel.X(r3)("p_3")
		"labpoint(is)", "labdir(rt)";
	ObjMLabel.X(r3)("+")
		"labpoint(ine)", "labdir(lft)";
	ObjMLabel.X(r3)("-")
		"labpoint(inw)", "labdir(lft)";
	ObjMLabel.X(r3)("20V")
		"labpoint(in)", "labdir(lft)";

	newR.X(r4);
	ObjMLabel.X(r4)("p_4")
		"labpoint(is)", "labdir(bot)";
	ObjMLabel.X(r4)("+")
		"labpoint(inw)", "labdir(top)";
	ObjMLabel.X(r4)("-")
		"labpoint(ine)", "labdir(top)";
	ObjMLabel.X(r4)("8V")
		"labpoint(in)", "labdir(top)";

	newDCS.X(u2) "angle(90)";
	ObjMLabel.X(u2)("p_5")
		"labpoint(is)", "labdir(urt)";
	ObjMLabel.X(u2)("0.4I")
		"labpoint(is)", "labdir(lrt)";
	ObjMLabel.X(u2)("12V")
		"labpoint(in)", "labdir(lft)";
	ObjMLabel.X(u2)("+")
		"labpoint(ie)", "labdir(ulft)";
	ObjMLabel.X(u2)("-")
		"labpoint(iw)", "labdir(llft)";

	newCDM.X(cdm0) "angle(180)";
	ObjMLabel.X(cdm0)("4A")
		"labpoint(is)", "labdir(top)";
	newCDM.X(cdm1);
	ObjMLabel.X(cdm1)("I=10A")
		"labpoint(in)", "labdir(top)";
	newCDM.X(cdm2) "angle(-90)";
	ObjMLabel.X(cdm2)("14A")
		"labpoint(in)", "labdir(rt)";

	numeric uc;
	uc := 45;
	X(u1).c = origin;
	X(r2).c = X(u1).c+(uc,uc);
	X(r3).c = X(u1).c+(2uc,0);
	X(r4).c = X(r2).c+(2uc,0);
	X(u2).c = X(r3).c + (2uc,0);

	X(cdm0).inw = (xpart(X(u2).c), ypart(X(r4).c));
	X(cdm1).isw = (xpart(X(u1).c), ypart(X(r2).c));
	X(cdm2).ese = X(r3).ie;

	drawObj(X(u1),X(r2),X(r3),X(r4),X(u2));
	drawObj(X(cdm0),X(cdm1),X(cdm2));

	pcsimple(X(u2))(2)(X(r4))(2);
	pcsimple(X(r3))(1)(X(u2))(1) "armA(ypart(X(r4).iw-X(r3).ie))";
	pcsimple(X(r3))(1)(X(u1))(1) "armA(ypart(X(r4).iw-X(r3).ie))";
	pcsimple(X(r3))(2)(X(r4))(1);
	pcsimple(X(r3))(2)(X(r2))(2);
	pcsimple(X(u1))(2)(X(r2))(1);

	ShowIFrame(X(u1));
	ShowFrame(X(u1));
	ShowIFrame(X(cdm2));
	ShowFrame(X(cdm2));
	ShowIFrame(X(cdm0));
	ShowFrame(X(cdm0));
\stopMPcode
}
\stopQ

\startA
\stopA



\stopchapter
\stopcomponent
