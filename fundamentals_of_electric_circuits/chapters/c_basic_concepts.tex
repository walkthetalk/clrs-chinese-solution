\startcomponent c_basic_concepts
\startchapter[
  title={Basic Concepts},
]

% 1.1
\startQ
How much charge is represented by these number of electrons?
\startIG
\item $\sunit{6.482e17}$
\item $\sunit{1.24=e18}$
\item $\sunit{2.46=e19}$
\item $\sunit{1.628e20}$
\stopIG
\stopQ

\startA
\startIG
\item $q = (\sunit{1.602e-19 coulomb}) \times \sunit{6.482e17} = \sunit{_0.10384 coulomb}$
\item $q = (\sunit{1.602e-19 coulomb}) \times \sunit{1.24=e18} = \sunit{_0.19865 coulomb}$
\item $q = (\sunit{1.602e-19 coulomb}) \times \sunit{2.46=e19} = \sunit{_3.941== coulomb}$
\item $q = (\sunit{1.602e-19 coulomb}) \times \sunit{1.628e20} = \sunit{26.08=== coulomb}$
\stopIG
\stopA

% 1.2
\startQ
Determine the current flowing through an element if the charge flow is given by
\startIG
\item $q(t) = (3t+8)               \sunit[l]{milli coulomb}$
\item $q(t) = (8t^2 + 4t − 2)      \sunit[l]{coulomb}$
\item $q(t) = (3e^{−t} − 5e^{−2t}) \sunit[l]{nano coulomb}$
\item $q(t) = 10 \sin 120π t       \sunit[l]{pico coulomb}$
\item $q(t) = 20e^{−4t} \cos 50t   \sunit[l]{micro coulomb}$
\stopIG
\stopQ

\startA
\startIG
\item $i = \frac{d q}{d t} = 3                      \sunit[l]{milli ampere}$
\item $i = \frac{d q}{d t} = (16t + 4)              \sunit[l]{ampere}$
\item $i = \frac{d q}{d t} = (-3e^{−t} + 10e^{−2t}) \sunit[l]{nano ampere}$
\item $i = \frac{d q}{d t} = 1200π \cos 120π t      \sunit[l]{pico ampere}$
\item $i = \frac{d q}{d t} = -(80 \cos 50t + 1000 \sin 50 t) e^{−4t} \sunit[l]{micro ampere}$
\stopIG
\stopA

% 1.3
\startQ
Find the charge $q(t)$ flowing through a device if the current is:
\startIG
\item $i(t) = \sunit{3 ampere}, q(0) = \sunit{1 coulomb}$
\item $i(t) = (2t + 5) \sunit[l]{milli ampere}, q(0) = 0$
\item $i(t) = 20 \cos (10t + π / 6) \sunit[l]{micro ampere}, q(0) = \sunit{2 micro coulomb}$
\item $i(t) = 10e^{−30t} \sin 40t \sunit[l]{ampere}, q(0) = 0$
\stopIG
\stopQ

\startA
\startIG
\item $q(t) = \int_0 i(t) + q(0) = (3t+1) \sunit[l]{coulomb}$
\item $q(t) = \int_0 i(t) + q(0) = (t^2 + 5t) \sunit[l]{milli coulomb}$
\item $q(t) = \int_0 i(t) + q(0) = (2 \sin (10t + π / 6) + 1) \sunit[l]{micro coulomb}$
\item $q(t) = \int_0 i(t) + q(0) = (-e^{-30t}(0.16\cos 40t + 0.12\sin 40t) + 0.16) \sunit[l]{coulomb}$
\stopIG
\stopA

% 1.4
\startQ
A total charge of $\sunit{300 coulomb}$ flows past a given cross section of a conductor in 30 seconds.
What is the value of the current?
\stopQ

\startA
$\sunit[r]{300 coulomb} / \sunit[l]{30 second} = \sunit{10 ampere}$
\stopA

% 1.5
\startQ
Determine the total charge transferred over the time interval
of $0 \le t \le \sunit{10 second}$ when $i(t) = \frac{1}{2}t \sunit[l]{ampere}$.
\stopQ

\startA
\startformula
q(t) = \int_0^{10}\frac{1}{2}t = \left.\frac{t^2}{4}\right|_0^{10} = 25
\stopformula
\stopA

% 1.6
\startQ
The charge entering a certain element is shown in Fig. 1.23. Find the current at:
\startIG
\item $t = \sunit{_1 milli second}$
\item $t = \sunit{_6 milli second}$
\item $t = \sunit{10 milli second}$
\stopIG
\stopQ

\startA
\startIG
\item $i(\sunit{_1}) = \left.15t\right|_{\sunit{1 milli second}}= \sunit{15 milli coulomb}$
\item $i(\sunit{_6}) = \left.30\right|_{\sunit{6 milli second}} = \sunit{30 milli coulomb}$
\item $i(\sunit{10}) = \left.90-7.5t\right|_{\unit{10 milli second}} = \sunit{15 milli coulomb}$
\stopIG
\stopA

\stopchapter
\stopcomponent
