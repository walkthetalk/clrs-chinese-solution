\startcomponent c_basic_concepts
\startchapter[
  title={Basic Concepts},
]

% 2.1
\startQ
Design a problem, complete with a solution,
to help students to better understand Ohm’s law.
Use at least two resistors and one voltage source.
Hint, you could use both resistors at once or one at a time,
it is up to you. Be creative.
\stopQ

\startA
...
\stopA

% 2.2
\startQ
Find the hot resistance of a light bulb rated \sunit{60 watt}, \sunit{120 volt}.
\stopQ

\startA
$\frac{v^2}{R} = p \rightarrow
R = \frac{v^2}{p}
  = \frac{60^2}{120}
  = 30 \sunit{ohm}$
\stopA

% 2.3
\startQ
A bar of silicon is 4 cm long with a circular cross section.
If the resistance of the bar is 240 Ω at room temperature,
what is the cross-sectional radius of the bar?
\stopQ

\startA
For silicon, $\rho = \sunit{6.4e2 ohm meter}$. $A=\pi r^2$.
Hence:
\startformula\startmathalignment
\NC R \NC =\frac{\rho L}{A}=\frac{\rho L}{\pi r^2} \NR
\NC r \NC = \sqrt{\frac{\rho L}{\pi R}} \NR
\NC   \NC = \sqrt{\frac{6.4\times 10^2 \times \frac{4}{100}}
                       {\pi\times 240}} \NR
\NC   \NC = \sqrt{0.033953} \NR
\NC   \NC = 0.1843\sunit[l]{meter} \NR
\stopmathalignment\stopformula
\stopA

% 2.4
\startQ[Q:2.4]
\startIG
\item Calculate current $i$ in \reffig{2.68} when the switch is in position 1.
\item Find the current when the switch is in position 2.
\stopIG
\placeschemq{2.68}{2.4}
\stopQ

\startA
\startIG
\item $40 / 100 = 0.4\sunit[l]{ampere}$
\item $40 / 250 = 0.16\sunit[l]{ampere}$
\stopIG
\stopA

% 2.5
\startQ[Q:2.5]
For the network graph in \reffig{2.69},
find the number of nodes, branches, and loops.
\placefigure{2.69}{2.5}
\stopQ

\startA
$n=9$, $l=7$; $b=n+l-1=15$.
\stopA

% 2.6
\startQ[Q:2.6]
In the network graph shown in \reffig{2.70},
determine the number of branches and nodes.
\placeschemq{2.70}{2.6}
\stopQ

\startA
$n=12$; $l=8$; $b=n+l-1=19$.
\stopA

% 2.7
\startQ[Q:2.7]
Determine the number of branches and nodes in the circuit of \reffig{2.71}.
\placeschemq{2.71}{2.7}
\stopQ

\startA
$n=4$,$l=3$,$b=n+l-1=6$.
\stopA

% 2.8
\startQ[Q:2.8]
Design a problem, complete with a solution,
to help other students better understand Kirchhoff’s Current Law.
Design the problem by specifying values of $i_a$, $i_b$, and $i_c$,
shown in \reffig{2.72},
and asking them to solve for values of $i_1$, $i_2$, and $i_3$.
Be careful to specify realistic currents.
\placeschemq{2.72}{2.8}
\stopQ

\startA
\stopA

% 2.9
\startQ[Q:2.9]
Find $i_1$, $i_2$ and $i_3$ in \reffig{2.73}.
\placeschemq{2.73}{2.9}
\stopQ

\startA
$i_1 + (-2) + (-3) = 0 \rightarrow i_1 = 5\sunit[l]{ampere}$.

$2 = (-2) + i_3 \rightarrow i_3 = 4\sunit[l]{ampere}$.

$i_2 + i_3 = (-4) \rightarrow i_2 = -8\sunit[l]{ampere}$.
\stopA

% 2.10
\startQ[Q:2.10]
Determine $i_1$ and $i_2$ in the circuit of \reffig{2.74}.
\placeschemq{2.74}{2.10}
\stopQ

\startA
$ i_1 = (-8) + (-6) = -14\sunit[l]{ampere}$.

$i_2 = 4 - (-6) = 10\sunit[l]{ampere}$.
\stopA

% 2.11
\startQ[Q:2.11]
In the circuit of \reffig{2.75}, calculate $V_1$ and $V_2$.
\placeschemq{2.75}{2.11}
\stopQ

\startA
$V_1 = 1 + 5 = 6\sunit[l]{volt}$.

$V_2 = 5 - 2 = 3\sunit[l]{volt}$.
\stopA

% 2.12
\startQ[Q:2.12]
In the circuit in \reffig{2.76}, obtain $v_1$, $v_2$, and $v_3$.
\placeschemq{2.76}{2.12}
\stopQ

\startA
$40 = 50 + 20 + v_1 \rightarrow v_1 = -30\sunit[l]{volt}$.

$30 = 20 + v_2 \rightarrow v_2 = 10\sunit[l]{volt}$.

$v_1 = v_2 + v_3 \rightarrow v_3 = v_1 - v_2 = -40\sunit[l]{volt}$.
\stopA

% 2.13
\startQ[Q:2.13]
For the circuit in \reffig{2.77},
use KCL to find the branch currents $I_1$ to $I_4$.
\placeschemq{2.77}{2.13}
\stopQ

\startA
$2 = i_4 + 4 \rightarrow i_4 = -2\sunit[l]{ampere}$.

$i_3 = 7 + i4 = 5\sunit[l]{ampere}$.

$i_2 + 7 + 3 = 0 \rightarrow i_2 = -10\sunit[l]{ampere}$.

$i_1 + i_2 = 2 \rightarrow i_1 = 12\sunit[l]{ampere}$.
\stopA

% 2.14
\startQ[Q:2.14]
Given the circuit in \reffig{2.78},
use KVL to find the branch voltages $V_1$ to $V_4$.
\placeschemq{2.78}{2.14}
\stopQ

\startA
$V_4 = 5 + 2 = 7\sunit[l]{volt}$.

$V_3 = -4 - V_4 = -11\sunit[l]{volt}$.

$V_1 = 3 + V_3 = -8\sunit[l]{volt}$.

$V_2 = -2 - V_1 = 6\sunit[l]{volt}$.
\stopA

% 2.15
\startQ[Q:2.15]
Calculate $v$ and $i_x$ in the circuit of \reffig{2.79}.
\placeschemq{2.79}{2.15}
\stopQ

\startA
$v = 10 - 4 = 6\sunit[l]{volt}$.

$16 + 3i_x = 4 \rightarrow i_x = -4\sunit[l]{ampere}$.
\stopA

% 2.16
\startQ[Q:2.16]
Determine $V_0$ in the circuit in \reffig{2.80}.
\placeschemq{2.80}{2.16}
\stopQ

\startA
\startformula
(25 - 10) \times \frac{16}{16+14} + 10 = 18\sunit[l]{volt}
\stopformula
\stopA

% 2.17
\startQ[Q:2.17]
Obtain $v_1$ through $v_3$ in the circuit of \reffig{2.81}.
\placeschemq{2.81}{2.17}
\stopQ

\startA
$v_3 = 10\sunit[l]{volt}$.

$v_2 = -12-v_3 = -22\sunit[l]{volt}$.

$v_1 = 24+v_2 = 2\sunit[l]{volt}$.
\stopA

% 2.18
\startQ[Q:2.18]
Find $I$ and $V$ in the circuit of \reffig{2.82}.
\placeschemq{2.82}{2.18}
\stopQ

\startA
\startformula\startmathalignment
\NC \NC 3 - \frac{V}{20} - \frac{V}{10} = I \NR
\NC \NC I - 4 + (-2) = \frac{V}{20} \NR
\stopmathalignment\stopformula

$I=5\frac{1}{4}\sunit[l]{ampere}$, $V=-15\sunit[l]{volt}$.
\stopA

% 2.19
\startQ[Q:2.19]
From the circuit in \reffig{2.83}, find $I$,
the power dissipated by the resistor,
and the power supplied by each source.
\placeschemq{2.83}{2.19}
\stopQ

\startA
Applying KVL around the loop, we obtain:
\startformula
-12+10-(-8)+3I=0 \rightarrow I = -2\sunit[l]{ampere}
\stopformula

Power dissipated by the resistor: $p_4=I^2R=12\sunit[l]{watt}$.

Power supplied by the sources:
\startformula\startmathalignment
\NC p_1 \NC = 10\times (-(-2)) = 20\sunit[l]{watt} \NR
\NC p_2 \NC = 12\times (-2) = -24\sunit[l]{watt} \NR
\NC p_3 \NC = (-8)\times (-2) = 16\sunit[l]{watt} \NR
\stopmathalignment\stopformula
\stopA

% 2.20
\startQ[Q:2.20]
Determine $i_0$ in the circuit of \reffig{2.84}.
\placeschemq{2.84}{2.20}
\stopQ

\startA
$i_0\times 22 + 5i_0 = 54$, $i_0 = 2\sunit[l]{ampere}$.
\stopA

% 2.21
\startQ[Q:2.21]
Find $V_x$ in the circuit of \reffig{2.85}.
\placeschemq{2.85}{2.21}
\stopQ

\startA
assume current is $I$, then:
\startformula
-15 + (1+5+2)I + 2V_x = 0
\stopformula

because $V_x=5I$, then $I=\frac{5}{6}\sunit[l]{ampere}$,
$V_x=\frac{25}{6}\approx4.167\sunit[l]{volt}$.
\stopA

% 2.22
\startQ[Q:2.22]
Find $V_0$ in the circuit in \reffig{2.86}
and the power absorbed by the dependent source.
\placeschemq{2.86}{2.22}
\stopQ

\startA
$25 + 2V_0 + \frac{V_0}{10} = 0$,
so $V_0 = -\frac{25}{21} \approx -1.19\sunit[l]{volt}$.

$p=2V_0 \times 2V_0 \approx 5.67\sunit[l]{watt}$.
\stopA

\stopchapter
\stopcomponent
