\startcomponent c_basic_concepts
\startchapter[
  title={Basic Concepts},
]

% 2.1
\startQ
Design a problem, complete with a solution,
to help students to better understand Ohm’s law.
Use at least two resistors and one voltage source.
Hint, you could use both resistors at once or one at a time,
it is up to you. Be creative.
\stopQ

\startA
...
\stopA

% 2.2
\startQ
Find the hot resistance of a light bulb rated \sunit{60 watt}, \sunit{120 volt}.
\stopQ

\startA
$\frac{v^2}{R} = p \rightarrow
R = \frac{v^2}{p}
  = \frac{60^2}{120}
  = 30 \sunit{ohm}$
\stopA

% 2.3
\startQ
A bar of silicon is 4 cm long with a circular cross section.
If the resistance of the bar is 240 Ω at room temperature,
what is the cross-sectional radius of the bar?
\stopQ

\startA
For silicon, $\rho = \sunit{6.4e2 ohm meter}$. $A=\pi r^2$.
Hence:
\startformula\startmathalignment
\NC R \NC =\frac{\rho L}{A}=\frac{\rho L}{\pi r^2} \NR
\NC r \NC = \sqrt{\frac{\rho L}{\pi R}} \NR
\NC   \NC = \sqrt{\frac{6.4\times 10^2 \times \frac{4}{100}}
                       {\pi\times 240}} \NR
\NC   \NC = \sqrt{0.033953} \NR
\NC   \NC = 0.1843\sunit[l]{meter} \NR
\stopmathalignment\stopformula
\stopA

% 2.4
\startQ[Q:2.4]
\startIG
\item Calculate current $i$ in \reffig{2.68} when the switch is in position 1.
\item Find the current when the switch is in position 2.
\stopIG
\placefigure[here][fig:2.68]{for \refq{2.4}}{
\startMPcode
\stopMPcode
}
\stopQ

\stopchapter
\stopcomponent
