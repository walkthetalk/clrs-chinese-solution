\startcomponent c_basic_concepts
\startchapter[
  title={Basic Concepts},
]

% 2.1
\startQ
Design a problem, complete with a solution,
to help students to better understand Ohm’s law.
Use at least two resistors and one voltage source.
Hint, you could use both resistors at once or one at a time,
it is up to you. Be creative.
\stopQ

\startA
...
\stopA

% 2.2
\startQ
Find the hot resistance of a light bulb rated \sunit{60 watt}, \sunit{120 volt}.
\stopQ

\startA
$\frac{v^2}{R} = p \rightarrow
R = \frac{v^2}{p}
  = \frac{60^2}{120}
  = 30 \sunit{ohm}$
\stopA

% 2.3
\startQ
A bar of silicon is 4 cm long with a circular cross section.
If the resistance of the bar is 240 Ω at room temperature,
what is the cross-sectional radius of the bar?
\stopQ

\startA
For silicon, $\rho = \sunit{6.4e2 ohm meter}$. $A=\pi r^2$.
Hence:
\startformula\startmathalignment
\NC R \NC =\frac{\rho L}{A}=\frac{\rho L}{\pi r^2} \NR
\NC r \NC = \sqrt{\frac{\rho L}{\pi R}} \NR
\NC   \NC = \sqrt{\frac{6.4\times 10^2 \times \frac{4}{100}}
                       {\pi\times 240}} \NR
\NC   \NC = \sqrt{0.033953} \NR
\NC   \NC = 0.1843\sunit[l]{meter} \NR
\stopmathalignment\stopformula
\stopA

% 2.4
\startQ[Q:2.4]
\startIG
\item Calculate current $i$ in \reffig{2.68} when the switch is in position 1.
\item Find the current when the switch is in position 2.
\stopIG
\placefigure[here][fig:2.68]{for \refq{2.4}}{
\startMPcode
	clearObjsForTwiceRun;
	setX("fig268");

	newIVS.X(v0)"angle(90)";
	ObjMLabel.X(v0)("40V")
		"labpoint(is)", "labdir(rt)";
	newRa.X(r0) "angle(90)";
	ObjMLabel.X(r0)("100\Omega")
		"labpoint(is)", "labdir(rt)";
	newRa.X(r1) "angle(90)";
	ObjMLabel.X(r1)("250\Omega")
		"labpoint(is)", "labdir(rt)";
	newSwitch.X(s0)(1) "angle(90)";
	ObjMLabel.X(s0)("1")
		"labpoint(pin[1])", "labdir(lft)";
	ObjMLabel.X(s0)("2")
		"labpoint(pin[2])", "labdir(rt)";
	newCDM.X(cdm0) "angle(90)";
	ObjMLabel.X(cdm0)("i")
		"labpoint(is)", "labdir(rt)";

	numeric uc;
	uc := 10;
	X(v0).c = origin;
	X(s0).c = X(v0).c  + (0, 5uc);
	X(r0).c = X(v0).ie + (-6uc, 0);
	X(r1).c = X(v0).ie + ( 6uc, 0);
	X(cdm0).nw = X(v0).ie;

	drawObj(X(v0),X(r0),X(r1),X(s0),X(cdm0));
	pcsimple(X(s0))(1)(X(r0))(2);
	pcsimple(X(s0))(2)(X(r1))(2);
	pcsimple(X(s0))(3)(X(v0))(2);
	pcsimple(X(v0))(1)(X(r0))(1);
	pcsimple(X(v0))(1)(X(r1))(1);
\stopMPcode
}
\stopQ

\startA
\startIG
\item $40 / 100 = 0.4\sunit[l]{ampere}$
\item $40 / 250 = 0.16\sunit[l]{ampere}$
\stopIG
\stopA

% 2.5
\startQ[Q:2.5]
For the network graph in \reffig{2.69},
find the number of nodes, branches, and loops.
\placefigure[here][fig:2.69]{for \refq{2.5}}{
\startMPcode
	clearObjsForTwiceRun;
	setX("fig269");

	newNode.X(n0);
	newNode.X(n1);
	newNode.X(n2);
	newNode.X(n3);
	newNode.X(n4);
	newNode.X(n5);
	newNode.X(n6);
	newNode.X(n7);
	newNode.X(n8);

	numeric uc;
	uc := 20;
	X(n0).c = origin;
	X(n2).c = X(n0).c + (uc, 0);
	X(n4).c = X(n2).c + (uc, 0);
	X(n7).c = X(n4).c + (uc, 0);
	X(n1).c = X(n0).c + (0, uc);
	X(n3).c = X(n1).c + (uc, 0);
	X(n5).c = X(n3).c + (uc, 0);
	X(n8).c = X(n5).c + (uc, 0);
	X(n6).c = X(n5).c + (0, uc);

	drawObj(X(n0),X(n1),X(n2),X(n3),X(n4),X(n5),X(n6),X(n7),X(n8));

	pcline(X(n0))(X(n1));
	pcline(X(n0))(X(n2));
	pcline(X(n1))(X(n2));
	pcline(X(n1))(X(n3));
	pcline(X(n2))(X(n3));
	pcline(X(n2))(X(n4));
	pcline(X(n3))(X(n5));
	pcline(X(n4))(X(n5));
	pcline(X(n4))(X(n7));
	pcline(X(n5))(X(n8));
	pcline(X(n7))(X(n8));
	pcline(X(n5))(X(n7));
	pcline(X(n3))(X(n6));
	pcline(X(n5))(X(n6));
	pcline(X(n8))(X(n6));
\stopMPcode
}
\stopQ

\startA
$n=9$, $l=7$; $b=n+l-1=15$.
\stopA

% 2.6
\startQ[Q:2.6]
In the network graph shown in \reffig{2.70},
determine the number of branches and nodes.
\placefigure[here][fig:2.70]{for \refq{2.6}}{
\startMPcode
	clearObjsForTwiceRun;
	setX("fig270");

	save _n_;
	def _n_(expr i)=
		%X(n)[i]
		X(sc_("n" & decimal(i)))
	enddef;
	for i:=1 upto 12:
		newNode._n_(i);
	endfor;

	numeric uc;
	uc := 20;
	for i:= 1 upto 6:
		_n_(i).c   = (uc,0) rotated ((i-1)*60);
		_n_(i+6).c = _n_(2).c shifted (uc,0) rotated ((i-1)*60);
	endfor;

	for i:=1 upto 12:
		drawObj(_n_(i));
	endfor;

	for i:=1 upto 6:
		save _next;numeric _next; _next := i mod 6 + 1;
		ncline(_n_(i))(_n_(_next));
		ncline(_n_(i))(_n_(i+6));
		ncline(_n_(i+6))(_n_(_next));
	endfor;
	ncbar(_n_(8))(_n_(11)) "angleA(180)", "armA(2uc)";
\stopMPcode
}
\stopQ

\startA
$n=12$; $l=8$; $b=n+l-1=19$.
\stopA

\stopchapter
\stopcomponent
