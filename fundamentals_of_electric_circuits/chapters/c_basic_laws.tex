\startcomponent c_basic_concepts
\startchapter[
  title={Basic Concepts},
]

% 2.1
\startQ
Design a problem, complete with a solution,
to help students to better understand Ohm’s law.
Use at least two resistors and one voltage source.
Hint, you could use both resistors at once or one at a time,
it is up to you. Be creative.
\stopQ

\startA
...
\stopA

% 2.2
\startQ
Find the hot resistance of a light bulb rated \sunit{60 watt}, \sunit{120 volt}.
\stopQ

\startA
$\frac{v^2}{R} = p \rightarrow
R = \frac{v^2}{p}
  = \frac{60^2}{120}
  = 30 \sunit{ohm}$
\stopA

% 2.3
\startQ
A bar of silicon is 4 cm long with a circular cross section.
If the resistance of the bar is 240 Ω at room temperature,
what is the cross-sectional radius of the bar?
\stopQ

\startA
For silicon, $\rho = \sunit{6.4e2 ohm meter}$. $A=\pi r^2$.
Hence:
\startformula\startmathalignment
\NC R \NC =\frac{\rho L}{A}=\frac{\rho L}{\pi r^2} \NR
\NC r \NC = \sqrt{\frac{\rho L}{\pi R}} \NR
\NC   \NC = \sqrt{\frac{6.4\times 10^2 \times \frac{4}{100}}
                       {\pi\times 240}} \NR
\NC   \NC = \sqrt{0.033953} \NR
\NC   \NC = 0.1843\sunit[l]{meter} \NR
\stopmathalignment\stopformula
\stopA

% 2.4
\startQ[Q:2.4]
\startIG
\item Calculate current $i$ in \reffig{2.68} when the switch is in position 1.
\item Find the current when the switch is in position 2.
\stopIG
\placefigq{2.68}{2.4}
\stopQ

\startA
\startIG
\item $40 / 100 = 0.4\sunit[l]{ampere}$
\item $40 / 250 = 0.16\sunit[l]{ampere}$
\stopIG
\stopA

% 2.5
\startQ[Q:2.5]
For the network graph in \reffig{2.69},
find the number of nodes, branches, and loops.
\placefigq{2.69}{2.5}
\stopQ

\startA
$n=9$, $l=7$; $b=n+l-1=15$.
\stopA

% 2.6
\startQ[Q:2.6]
In the network graph shown in \reffig{2.70},
determine the number of branches and nodes.
\placefigq{2.70}{2.6}
\stopQ

\startA
$n=12$; $l=8$; $b=n+l-1=19$.
\stopA

% 2.7
\startQ[Q:2.7]
Determine the number of branches and nodes in the circuit of \reffig{2.71}.
\placefigq{2.71}{2.7}
\stopQ

\startA
$n=4$,$l=3$,$b=n+l-1=6$.
\stopA

% 2.8
\startQ[Q:2.8]
Design a problem, complete with a solution,
to help other students better understand Kirchhoff’s Current Law.
Design the problem by specifying values of $i_a$, $i_b$, and $i_c$,
shown in \reffig{2.72},
and asking them to solve for values of $i_1$, $i_2$, and $i_3$.
Be careful to specify realistic currents.
\placefigq{2.72}{2.8}
\stopQ

\startA
\stopA

% 2.9
\startQ[Q:2.9]
Find $i_1$, $i_2$ and $i_3$ in \reffig{2.73}.
\placefigq{2.73}{2.9}
\stopQ

\startA
$i_1 + (-2) + (-3) = 0 \rightarrow i_1 = 5\sunit[l]{ampere}$.

$2 = (-2) + i_3 \rightarrow i_3 = 4\sunit[l]{ampere}$.

$i_2 + i_3 = (-4) \rightarrow i_2 = -8\sunit[l]{ampere}$.
\stopA

% 2.10
\startQ[Q:2.10]
Determine $i_1$ and $i_2$ in the circuit of \reffig{2.74}.
\placefigq{2.74}{2.10}
\stopQ

\startA
$ i_1 = (-8) + (-6) = -14\sunit[l]{ampere}$.

$i_2 = 4 - (-6) = 10\sunit[l]{ampere}$.
\stopA

% 2.11
\startQ[Q:2.11]
In the circuit of \reffig{2.75}, calculate $V_1$ and $V_2$.
\placefigq{2.75}{2.11}
\stopQ

\startA
$V_1 = 1 + 5 = 6\sunit[l]{volt}$.

$V_2 = 5 - 2 = 3\sunit[l]{volt}$.
\stopA

% 2.12
\startQ[Q:2.12]
In the circuit in \reffig{2.76}, obtain $v_1$, $v_2$, and $v_3$.
\placefigq{2.76}{2.12}
\stopQ

\startA
$40 = 50 + 20 + v_1 \rightarrow v_1 = -30\sunit[l]{volt}$.

$30 = 20 + v_2 \rightarrow v_2 = 10\sunit[l]{volt}$.

$v_1 = v_2 + v_3 \rightarrow v_3 = v_1 - v_2 = -40\sunit[l]{volt}$.
\stopA

% 2.13
\startQ[Q:2.13]
For the circuit in \reffig{2.77},
use KCL to find the branch currents $I_1$ to $I_4$.
\placefigq{2.77}{2.13}
\stopQ

\startA
$2 = i_4 + 4 \rightarrow i_4 = -2\sunit[l]{ampere}$.

$i_3 = 7 + i4 = 5\sunit[l]{ampere}$.

$i_2 + 7 + 3 = 0 \rightarrow i_2 = -10\sunit[l]{ampere}$.

$i_1 + i_2 = 2 \rightarrow i_1 = 12\sunit[l]{ampere}$.
\stopA

% 2.14
\startQ[Q:2.14]
Given the circuit in \reffig{2.78},
use KVL to find the branch voltages $V_1$ to $V_4$.
\placefigq{2.78}{2.14}
\stopQ

\startA
$V_4 = 5 + 2 = 7\sunit[l]{volt}$.

$V_3 = -4 - V_4 = -11\sunit[l]{volt}$.

$V_1 = 3 + V_3 = -8\sunit[l]{volt}$.

$V_2 = -2 - V_1 = 6\sunit[l]{volt}$.
\stopA

% 2.15
\startQ[Q:2.15]
Calculate $v$ and $i_x$ in the circuit of \reffig{2.79}.
\placefigq{2.79}{2.15}
\stopQ

\startA
$v = 10 - 4 = 6\sunit[l]{volt}$.

$16 + 3i_x = 4 \rightarrow i_x = -4\sunit[l]{ampere}$.
\stopA

% 2.16
\startQ[Q:2.16]
Determine $V_0$ in the circuit in \reffig{2.80}.
\placefigq{2.80}{2.16}
\stopQ

\startA
\startformula
(25 - 10) \times \frac{16}{16+14} + 10 = 18\sunit[l]{volt}
\stopformula
\stopA

% 2.17
\startQ[Q:2.17]
Obtain $v_1$ through $v_3$ in the circuit of \reffig{2.81}.
\placefigq{2.81}{2.17}
\stopQ

\startA
$v_3 = 10\sunit[l]{volt}$.

$v_2 = -12-v_3 = -22\sunit[l]{volt}$.

$v_1 = 24+v_2 = 2\sunit[l]{volt}$.
\stopA

% 2.18
\startQ[Q:2.18]
Find $I$ and $V$ in the circuit of \reffig{2.82}.
\placefigq{2.82}{2.18}
\stopQ

\startA
\startformula\startmathalignment
\NC \NC 3 - \frac{V}{20} - \frac{V}{10} = I \NR
\NC \NC I - 4 + (-2) = \frac{V}{20} \NR
\stopmathalignment\stopformula

$I=5\frac{1}{4}\sunit[l]{ampere}$, $V=-15\sunit[l]{volt}$.
\stopA

% 2.19
\startQ[Q:2.19]
From the circuit in \reffig{2.83}, find $I$,
the power dissipated by the resistor,
and the power supplied by each source.
\placefigq{2.83}{2.19}
\stopQ

\startA
Applying KVL around the loop, we obtain:
\startformula
-12+10-(-8)+3I=0 \rightarrow I = -2\sunit[l]{ampere}
\stopformula

Power dissipated by the resistor: $p_4=I^2R=12\sunit[l]{watt}$.

Power supplied by the sources:
\startformula\startmathalignment
\NC p_1 \NC = 10\times (-(-2)) = 20\sunit[l]{watt} \NR
\NC p_2 \NC = 12\times (-2) = -24\sunit[l]{watt} \NR
\NC p_3 \NC = (-8)\times (-2) = 16\sunit[l]{watt} \NR
\stopmathalignment\stopformula
\stopA

% 2.20
\startQ[Q:2.20]
Determine $i_0$ in the circuit of \reffig{2.84}.
\placefigq{2.84}{2.20}
\stopQ

\startA
$i_0\times 22 + 5i_0 = 54$, $i_0 = 2\sunit[l]{ampere}$.
\stopA

% 2.21
\startQ[Q:2.21]
Find $V_x$ in the circuit of \reffig{2.85}.
\placefigq{2.85}{2.21}
\stopQ

\startA
assume current is $I$, then:
\startformula
-15 + (1+5+2)I + 2V_x = 0
\stopformula

because $V_x=5I$, then $I=\frac{5}{6}\sunit[l]{ampere}$,
$V_x=\frac{25}{6}\approx4.167\sunit[l]{volt}$.
\stopA

% 2.22
\startQ[Q:2.22]
Find $V_0$ in the circuit in \reffig{2.86}
and the power absorbed by the dependent source.
\placefigq{2.86}{2.22}
\stopQ

\startA
$25 + 2V_0 + \frac{V_0}{10} = 0$,
so $V_0 = -\frac{25}{21} \approx -1.19\sunit[l]{volt}$.

$p=2V_0 \times 2V_0 \approx 5.67\sunit[l]{watt}$.
\stopA

% 2.23
\startQ[Q:2.23]
In the circuit shown in \reffig{2.87},
determine $V_x$ and the power absorbed by the 60-Ω resistor.
\placefigq{2.87}{2.23}
\stopQ

\startA
$R_{15||30} = 10\sunit[l]{ohm}$, $R_{40||60} = 24\sunit[l]{ohm}$.

$R_{20+10} = 30\sunit[l]{ohm}$, $R_{6+24} = 30\sunit[l]{ohm}$.

$R_{30||30} = 15\sunit[l]{ohm}$.

$R_{5+15} = 20\sunit[l]{ohm}$.

$I_{10} \times 10 = I_{5} \times 20$, $I_{10} + I_{5} = 60$, so $I_{5} = 20\sunit[l]{ampere}$.

$v_x = 5 \times -20 = -100\sunit[l]{volt}$.

$I_{6} = I_{20}$, so $I_{6} = 10\sunit[l]{ampere}$.

$I_{40} \times 40 = I_{60} \times 60$, so $I_{60} = 4\sunit[l]{ampere}$.

$p_{60} = I^2R = 4\times 4\times 60 = 960\sunit[l]{watt}$.
\stopA

% 2.24
\startQ[Q:2.24]
For the circuit in \reffig{2.88},
find $V_o / V_s$ in terms of $\alpha$,​ $R_1$, $R_2$, $R_3$, and $R_4$.
If $R_1 = R_2 = R_3 = R_4$,
what value of $\alpha$ will produce $|V_o / V_s| =​10$?
\placefigq{2.88}{2.24}
\stopQ

\startA
\startformula\startmathalignment
\NC V_s \NC = I_o (R_1 + R_2) \NR
\NC V_o \NC = \alpha I_o \frac{R_3 R_4}{R_3 + R_4} \NR
\NC \frac{V_o}{V_s} \NC = \frac{\alpha R_3 R_4}{(R_1+R_2)(R_3+R_4)} \NR
\stopmathalignment\stopformula

If $R_1 = R_2 = R_3 = R_4$, then $V_o/V_s = \alpha/4$.
$\alpha = 40$ will produce $|V_o / V_s| =​10$.
\stopA

% 2.25
\startQ[Q:2.25]
For the network in \reffig{2.89}, find the current, voltage,
and power associated with the 20-kΩ resistor.
\placefigq{2.89}{2.25}
\stopQ

\startA
\startformula\startmathalignment
\NC V_o    \NC = (5\times 10^{-3})\times(10\times 10^{3}) = 50\sunit[l]{volt} \NR
\NC I_{20} \NC = \frac{5}{5+20} \times 0.01V_o = 0.1\sunit[l]{ampere} \NR
\NC V_{20} \NC = I_{20} \times 20\times 10^3 = 2000\sunit[l]{volt} \NR
\NC P_{20} \NC = V_{20} I_{20} = 200\sunit[l]{watt} \NR
\stopmathalignment\stopformula
\stopA

% 2.26
\startQ[Q:2.26]
For the circuit in \reffig{2.90}, $i_o = 3\sunit[l]{ampere}$.
Calculate $i_x$ and the total power absorbed by the entire circuit.
\placefigq{2.90}{2.26}
\stopQ

\startA
\startformula\startmathalignment
\NC v_o \NC = i_o \times 40 = 120\sunit[l]{volt} \NR
\NC i_x \NC = \frac{v_o}{20} + \frac{v_o}{10}
              + \frac{v_o}{5} + i_o = 45\sunit[l]{ampere} \NR
\NC p \NC = v_o i_x + i_x^2\times 25 = 56025\sunit[l]{watt} \NR
\stopmathalignment\stopformula
\stopA

% 2.27
\startQ[Q:2.27]
Calculate $I_o$ in the circuit of \reffig{2.91}.
\placefigq{2.91}{2.27}
\stopQ

\startA
\startformula\startmathalignment
\NC R \NC = 8 + \frac{3\times 6}{3+6} = 10\sunit[l]{ohm} \NR
\NC I_o \NC = \frac{10}{R} = 1\sunit[l]{ampere} \NR
\stopmathalignment\stopformula
\stopA

% 2.28
\startQ[Q:2.28]
Design a problem, using \reffig{2.92},
to help other students better understand series and parallel circuits.
\placefigq{2.92}{2.28}
\stopQ

\startA
\stopA

% 2.29
\startQ[Q:2.29]
All resistors (R) in \reffig{2.93} are \sunit{10 ohm} each.
Find $R_{eq}$.
\placefigq{2.93}{2.29}
\stopQ

\startA
\startformula\startmathalignment
\NC R_{eq} \NC = (R||R)||((R||R)+(R||(R+R))) \NR
\NC        \NC = \frac{R}{2} || (\frac{R}{2} + \frac{2R}{3}) \NR
\NC        \NC = \frac{R}{2} || \frac{2R}{7} \NR
\NC        \NC = \frac{2}{11}R \NR
\stopmathalignment\stopformula
\stopA

% 2.30
\startQ[Q:2.30]
Find $R_{eq}$ for the circuit in \reffig{2.94}.
\placefigq{2.94}{2.30}.
\stopQ

\startA
$R_{eq}=25+48=73\sunit[l]{ohm}$.
\stopA

% 2.31
\startQ[Q:2.31]
For the circuit in \reffig{2.95},
determine $i_1$ to $i_5$.
\placefigq{2.95}{2.31}
\stopQ

\startA
\startformula\startmathalignment
\NC i_1 \NC = i_2 + i_3 \NR
\NC i_3 \NC = i_4 + i_5 \NR
\NC i_4\times 1 \NC = i_5 \times 2 = i_2\times 4 \NR
\NC i_1\times 3 + i_2\times 4 \NC = 200 \NR
\stopmathalignment\stopformula

$i_1=7i_2 = 56\sunit[l]{ampere}$

$i_2=8\sunit[l]{ampere}$

$i_3=6i_2 = 48\sunit[l]{ampere}$

$i_4=4i_2 = 32\sunit[l]{ampere}$

$i_5=2i_2 = 16\sunit[l]{ampere}$
\stopA

% 2.32
\startQ[Q:2.32]
Find $i_1$ through $i_4$ in the circuit in \reffig{2.96}.
\placefigq{2.96}{2.32}
\stopQ

\startA
$R = (60||40) || (200||50) = 24 || 40 = 15\sunit[l]{ohm}$,
$V = 16R = 240\sunit[l]{volt}$.

$i_1=240/50=4.8\sunit[l]{ampere}$,
$i_2=240/200=1.2\sunit[l]{ampere}$,
$i_3=240/40=6\sunit[l]{ampere}$,
$i_4=240/60=4\sunit[l]{ampere}$.
\stopA

% 2.33
\startQ[Q:2.33]
Obtain $v$ and $i$ in the circuit of \reffig{2.97}.
\placefigq{2.97}{2.33}
\stopQ

\startA
$R=\frac{62}{73}S$,
$i=\frac{99}{73}\approx 1.356\sunit[l]{ampere}$,
$v=\frac{62}{73}S\times 9 \approx 7.64S$.
\stopA

% 2.34
\startQ[Q:2.34]
Using series/parallel resistance combination,
find the equivalent resistance seen by the source in the circuit of \reffig{2.98}.
Find the overall absorbed power by the resistor network.
\placefigq{2.98}{2.34}
\stopQ

\startA
$R=250\sunit[l]{ohm}$, $P=U^2/R=1440\sunit[l]{watt}$.
\stopA

% 2.35
\startQ[Q:2.35]
Calculate $V_o$ 和 $I_o$ in the circuit of \reffig{2.99}.
\placefigq{2.99}{2.35}
\stopQ

\startA
$70||30=21$, $20||5=4$, $V_o=200\times \frac{4}{21+4}=32\sunit[l]{volt}$.

$I_o = \frac{32}{5} - \frac{200-32}{30} = 0.8\sunit[l]{ampere}$.
\stopA

% 2.36
\startQ[Q:2.36]
Find $i$ and $V_o$ in the circuit of \reffig{2.100}.
\placefigq{2.100}{2.36}
\stopQ

\startA
$R=100\sunit[l]{ohm}$, $i=\frac{20}{100}=0.2\sunit[l]{ampere}$.

$V_o = 0.02\times 30 = 0.6\sunit[l]{volt}$.
\stopA

% 2.37
\startQ[Q:2.37]
Given the circuit in \reffig{2.101} and that the resistance, $R_{eq}$,
looking into the circuit from the left is equal to \sunit{100 ohm},
determine the value of $R_1$.
\placefigq{2.101}{2.37}
\stopQ

\startA
$R_{eq}=\frac{5}{3}R_1$, $R_q=\frac{3}{5}R_{eq}=60\sunit[l]{ohm}$.
\stopA

% 2.38
\startQ[Q:2.38]
Find $R_{eq}$ and $i_o$ in the circuit of \reffig{2.102}.
\placefigq{2.102}{2.38}
\stopQ

\startA
$R=\sunit{10 ohm}$, $i_o=3.5\sunit[l]{ampere}$, $R_{eq}=7.5\sunit[l]{ohm}$.
\stopA

% 2.39
\startQ[Q:2.39]
Evaluate $R_{eq}$ looking into each set of terminals for
each of the circuits shown in \reffig{2.103}.
\placefigq[2*1][a,b]{2.103}{2.39}
\stopQ

\startA
$R_a = (6||3 + 6) || 3 = 24/11 \approx 2.182\sunit[l]{ohm}$.

$R_b = (6||6 + 3) || 2 = 1.5\sunit[l]{kilo ohm}$.
\stopA

% 2.40
\startQ[Q:2.40]
For the ladder network in \reffig{2.104}, find $I$ and $R_{eq}$.
\placefigq{2.104}{2.40}
\stopQ

\startA
$R_{eq} = 10\sunit[l]{ohm}$, $I=1.5\sunit[l]{ampere}$.
\stopA

% 2.41
\startQ[Q:2.41]
If $R_{eq}=50\sunit[l]{ohm}$ in the circuit of \reffig{2.105}, find $R$.
\placefigq{2.105}{2.41}
\stopQ

\startA
$(14+R)||60 + 30 = 50$, $R=16\sunit[l]{ohm}$.
\stopA

% 2.42
\startQ[Q:2.42]
Reduce each of the circuits in \reffig{2.106} to
a single resistor at terminals $a$-$b$.
\placefigq[2*1][a,b]{2.106}{2.42}
\stopQ

\startA
$R_{(a)}=4\sunit[l]{ohm}$.\\
$R_{(b)}\approx 5.818\sunit[l]{ohm}$.
\stopA

% 2.43
\startQ[Q:2.43]
Calculate the equivalent resistance $R_{ab}$ at terminals
$a$-$b$ for each of the circuits in \reffig{2.107}.
\placefigq[2*1][a,b]{2.107}{2.43}
\stopQ

\startA
$R=12\sunit[l]{ohm}$, $R=16\sunit[l]{ohm}$.
\stopA

% 2.44
\startQ[Q:2.44]
For the circuit in \reffig{2.108},
obtain the equivalent resistance at terminals $a$-$b$.
\placefigq{2.108}{2.44}
\stopQ

\startA
$R=12\sunit[l]{ohm}$.
\stopA

% 2.45
\startQ[Q:2.45]
Find the equivalent resistance at terminals $a$-$b$ of each circuit in \reffig{2.109}.
\placefigq[2*1][a,b]{2.109}{2.45}
\stopQ

\startA
$R=59.8\sunit[l]{ohm}$. $R=27.5\sunit[l]{ohm}$.
\stopA

% 2.46
\startQ[Q:2.46]
Find $I$ in the circuit of \reffig{2.110}.
\placefigq{2.110}{2.46}
\stopQ

\startA
$R=70\sunit[l]{ohm}$, $I=2\sunit[l]{ampere}$.
\stopA

% 2.47
\startQ[Q:2.47]
Find the equivalent resistance $R_{ab}$ in
the circuit of \reffig{2.111}.
\placefigq{2.111}{2.47}
\stopQ

\startA
$R_{ab}=24\sunit[l]{ohm}$.
\stopA

% 2.48
\startQ[Q:2.48]
Convert the circuits in \reffig{2.112} from $Y$ to $\Delta$.
\placefigq[2*1][a,b]{2.112}{2.48}
\stopQ

\startA
\placefiga[2*1][a,b]{2.48}
\stopA

% 2.49
\startQ[Q:2.49]
Transform the circuits in \reffig{2.113} from $\Delta$ to $Y$.
\placefigq[2*1][a,b]{2.113}{2.49}
\stopQ

\startA
\placefiga[2*1][a,b]{2.49}
\stopA

% 2.50
\startQ[Q:2.50]
Design a problem to help other students better
understand wye-delta transformations using \reffig{2.114}.
\placefigq{2.114}{2.50}
\stopQ

\startA
$R_{eq} = R$.
\stopA

% 2.51
\startQ[Q:2.51]
Obtain the equivalent resistance at the terminals $a$-$b$
for each of the circuits in \reffig{2.115}.
\placefigq[2*1][a,b]{2.115}{2.51}
\stopQ

\startA
$R_{(a)}=\frac{120}{13}\approx 9.23\sunit[l]{ohm}$.\\
$R_{(b)}=36.25\sunit[l]{ohm}$.
\stopA

% 2.52
\startQ[Q:2.52]
For the circuit shown in \reffig{2.116},
find the equivalent resistance.
All resistors are \sunit{3 ohm}.
\placefigq{2.116}{2.52}
\stopQ

\startA
$R_{eq}=\frac{507}{55}\approx 9.218\sunit[l]{ohm}$.
\stopA

% 2.53
\startQ[Q:2.53]
Obtain the equivalent resistance $R_{ab}$ in each of the
circuits of \reffig{2.117}.
In (b), all resistors have a value of \sunit{30 ohm}.
\placefigq[2*1][a,b]{2.117}{2.53}
\stopQ

\startA
$R_{ab} = 142\frac{32}{99} \approx 142.32\sunit[l]{ohm}$.\\
$R_{ab} = 33\frac{1}{3} \approx 33.33\sunit[l]{ohm}$.
\stopA

% 2.54
\startQ[Q:2.54]
Consider the circuit in \reffig{2.118}.
Find the equivalent resistance at terminals: (a) a-b, (b) c-d.
\placefigq{2.118}{2.54}
\stopQ

\startA
$R_{ab}=250\sunit[l]{ohm}$, $R_{cd}=300\sunit[l]{ohm}$.
\stopA

% 2.55
\startQ[Q:2.55]
Calculate $I_o$ in the circuit of \reffig{2.119}.
\placefigq{2.119}{2.55}
\stopQ

\startA
$R = 835\frac{5}{57}\approx 835.088\sunit[l]{ohm}$,
$I_o = \frac{57}{476} \approx 0.120\sunit[l]{ampere}$.
\stopA

% 2.56
\startQ[Q:2.56]
Determine $V$ in the circuit of \reffig{2.120}.
\placefigq{2.120}{2.56}
\stopQ

\startA
$V=\frac{13545}{32113}\approx 42.18\sunit[l]{volt}$.
\stopA

% 2.57
\startQ[Q:2.57]
Find $R_{eq}$ and $I$ in the circuit of \reffig{2.121}.
\placefigq{2.121}{2.57}
\stopQ

\startA
$R_{eq} = \frac{14500}{447} \approx 32.44\sunit[l]{ohm}$,
$I = \frac{447}{290} \approx 1.54\sunit[l]{ampere}$.
\stopA

% 2.58
\startQ[Q:2.58]
The \sunit{150 watt} light bulb in \reffig{2.122} is rated at 110 volts.
Calculate the value of $V_s$ to make the light bulb operate at its rated conditions.
\placefigq{2.122}{2.58}
\stopQ

\startA
$110 + (110/100 + 150/110)\times 50 = 2455/11 \approx 223.18\sunit[l]{ohm}$.
\stopA

% 2.59
\startQ[Q:2.59]
An enterprising young man travels to Europe carrying
three light bulbs he had purchased in North America.
The light bulbs he has are a 100-W light bulb,
a 60-W light bulb, and a 40-W light bulb.
Each light bulb is rated at 110 V. He wishes to connect
these to a 220-V system that is found in Europe.
For reasons we are not sure of, he connects the 40-W
light bulb in series with a parallel combination of the
60-W light bulb and the 100-W light bulb as shown \reffig{2.123}.
How much power is actually being delivered to each light bulb?
What does he see when he first turns on the light bulbs?

Is there a better way to connect these light bulbs
in order to have them work more effectively?
\placefigq{2.123}{2.59}
\stopQ

\startA
The 40-W light bulb will choked.
\placefiga{2.59}
\stopA

% 2.60
\startQ[Q:2.60]
If the three bulbs of \refq{2.59} are connected
in parallel to the 120-V source,
calculate the current through each bulb.
\stopQ

\startA
$I_{40}=\frac{192}{605}\approx 0.317\sunit[l]{ampere}$.\\
$I_{60}=\frac{72}{605}\approx 0.119\sunit[l]{ampere}$.\\
$I_{100}=\frac{24}{121}\approx 0.198\sunit[l]{ampere}$.
\stopA

% 2.61
\startQ[Q:2.61]
As a design engineer,
you are asked to design a lighting system
consisting of a 70-W power supply and
two light bulbs as shown in \reffig{2.124}.
You must select the two bulbs from
the following three available bulbs.

$R_1 = 80 Ω$, cost = \$0.60 (standard size)
$R_2 = 90 Ω$, cost = \$0.90 (standard size)
$R_3 = 100 Ω$, cost = \$0.75 (nonstandard size)

The system should be designed for minimum cost
such that $I$ lies within the range $I = 1.2 A ± 5\%$.
\placefigq{2.124}{2.61}
\stopQ

\startA
Select $R_1$ and $R_3$ ($R_2$ and $R_3$ will cost more).
\stopA

% 2.62
\startQ[Q:2.62]
A three-wire system supplies two loads $A$ and $B$ as
shown in \reffig{2.125}.
Load $A$ consists of a motor drawing a current
of $8\sunit[l]{ampere}$,
while load B is a PC drawing $2\sunit[l]{ampere}$.
Assuming 10 h/day of use for 365 days and 6 cents/kWh,
calculate the annual energy cost of the system.
\placefigq{2.125}{2.62}
\stopQ

\startA
$110\times(8+2)/1000 \times 10 \times 365 \times 0.06 = \$240.9$.
\stopA

% 2.63
\startQ[Q:2.63]
If an ammeter with an internal resistance of \sunit{100 ohm}
and a current capacity of \sunit{2 milli ampere} is to measure \sunit{5 ampere},
determine the value of the resistance needed.
Calculate the power dissipated in the shunt resistor.
\stopQ

\startA
$100\times 0.002 / (5 - 0.002) = \frac{100}{2499} \approx 0.04\sunit[l]{ohm}$.\\
$p=100\times 0.002\times (5-0.002) = 0.9996\sunit[l]{watt}$.
\stopA

% 2.64
\startQ[Q:2.64]
The potentiometer (adjustable resistor) $R_x$ in \reffig{2.126}
is to be designed to adjust current $i_x$ from \sunit{10 milli ampere} to \sunit{1 ampere}.
Calculate the values of $R$ and $R_x$ to achieve this.
\placefigq{2.126}{2.64}
\stopQ

\startA
$R=110/1=110\sunit[l]{ohm}$.\\
$R_x=110/0.010 - 110=10890\sunit[l]{ohm}$.
\stopA

% 2.65
\startQ[Q:2.65]
Design a circuit that uses a d'Arsonval meter
(with an internal resistance of \sunit{2 kilo ohm}
that requires a current of \sunit{5 milli ampere}
to cause the meter to deflect full scale) to build
a voltmeter to read values of voltages up to 100 volts.
\stopQ

\startA
series resistor:\\
$R=100/(5\times 10^-3) - 2\times 10^3
  = 18\times 10^3\sunit[l]{ohm}
  = 18\sunit[l]{kilo ohm}$.
\stopA

% 2.66
\startQ[Q:2.66]
A 20-kΩ/V voltmeter reads 10 V full scale.
\startIG
\item What series resistance is required to make the
meter read 50 V full scale?
\item What power will the series resistor dissipate
when the meter reads full scale?
\stopIG
\stopQ

\startA
\startIG
\item $50/\frac{10}{20k} - 20k = 80\sunit[l]{kilo ohm}$.
\item $\frac{10}{20k}\times (50-10) = 0.02\sunit[l]{watt}$.
\stopIG
\stopA

% 2.67
\startQ[Q:2.67]
\startIG
\item Obtain the voltage $V_o$ in the circuit of \reffig{2.127}(a).
\item Determine the voltage $V_o'$ measured when a voltmeter with
6-kΩ internal resistance is connected as shown in \reffig{2.127}(b).
\startitem The finite resistance of the meter introduces an
error into the measurement.
Calculate the percent error as
\startformula
\left|\frac{V_o - V_o'}{V_o}\right| \times 100\%
\stopformula
\stopitem
\item Find the percent error if the internal resistance were 36 kΩ.
\stopIG
\placefigq[2*1][a,b]{2.127}{2.67}
\stopQ

\startA
\startIG
\startitem
By current division,\\
$i_o=\frac{5}{5+5}\times \sunit{2 milli ampere} = \sunit{1 milli ampere}$;\\
$V_o=i_o\times \sunit{4 kilo ohm} = \sunit{4 volt}$.
\stopitem
\startitem
$4k||6k=2.4\sunit[l]{kilo ohm}$, by current division,\\
$i_o=\frac{5}{1+2.4+5}\times \sunit{2 milli ampere}
    = \frac{25}{21}\sunit[l]{milli ampere}$;\\
$V_o=i_o\times \sunit{2.4 kilo ohm}
    = \frac{20}{7}\sunit[l]{volt}
    \approx 2.857\sunit[l]{volt}$.
\stopitem
\startitem
$\frac{2}{7}\approx 28.57\%$.
\stopitem
\startitem
$\frac{12}{12+5R} = \frac{1}{16} \approx 6.25\%$.
\stopitem
\stopIG
\stopA

% 2.68
\startQ[Q:2.68]
\startIG
\item Find the current $I$ in the circuit of \reffig{2.128}(a).
\item An ammeter with an internal resistance of \sunit{1 ohm}
is inserted in the network to measure $I'$ ​as shown in \reffig{2.128}(b).
What is $I' $?
\startitem Calculate the percent error introduced by the meter as
\startformula
\left|\frac{I-I'}{I}\right|\times 100\%
\stopformula
\stopitem
\stopIG
\placefigq[2*1][a,b]{2.128}{2.68}
\stopQ

\startA
\startIG
\startitem
$40||60=\sunit{24 ohm}$,\\
$I=\frac{4}{16+24}=0.1\sunit{ampere}$.
\stopitem

\startitem
$I'=\frac{4}{16+24+1}=\frac{4}{41}\approx=0.098\sunit[l]{ampere}$.
\stopitem

\startitem
$\left|\frac{I-I'}{I}\right|\times 100\%=\frac{1}{41}\approx 2.44\%$.
\stopitem
\stopIG
\stopA

% 2.69
\startQ[Q:2.69]
A voltmeter is used to measure $V_o$ in the circuit in \reffig{2.129}.
The voltmeter model consists of an ideal voltmeter
in parallel with a 250-kΩ resistor.
Let $V_s = \sunit{95 volt}$, $R_s = \sunit{25 kilo ohm}$,
and $R_1 = \sunit{40 kilo ohm}$.
Calculate $V_o$ with and without the voltmeter when
\startIG
\item $R_2 = \sunit{5 kilo ohm}$
\item $R_2 = \sunit{250 kilo ohm}$
\item $R_2 = \sunit{25 kilo ohm}$
\stopIG
\placefigq{2.129}{2.69}
\stopQ

\startA
Without voltmeter, $V_o$ is
$95/14\approx 6.786$,
$475/18\approx 26.389$,
$4750/63\approx75.397$ \sunit[l]{volt}.
\startformula
V_o=\frac{R_2}{R_s+R_1+R_2}\times V_s
   =\frac{95 R_2}{65+R_2}
\stopformula

With voltmeter, $V_o$ is
$4750/713\approx 6.662$,
$4750/193\approx 24.611$,
$125/2\approx 62.5$ \sunit[l]{volt}.
\startformula
V_o=\frac{R_2'}{R_s+R_1+R_2'}\times V_s
   =\frac{95 R_2'}{65+R_2'}
   =\frac{4750 R_2}{63 R_2 + 3250}
\stopformula
\stopA

% 2.70
\startQ[Q:2.70]
\startIG
\item Consider the Wheatstone bridge shown in \reffig{2.130}.
Calculate $v_a$, $v_b$, and $v_{ab}$.
\item Rework part (a) if the ground is placed at $a$ instead of $o$.
\stopIG
\placefigq{2.130}{2.70}
\stopQ

\startA
\startIG
\item $v_a = \sunit{15 volt}$, $v_b = \sunit{10 volt}$, $v_{ab}=\sunit{5 volt}$.
\item $v_a = \sunit{0 volt}$, $v_b = \sunit{-5 volt}$, $v_{ab}=\sunit{5 volt}$.
\stopIG
\stopA

% 2.71
\startQ[Q:2.71]
\reffig{2.131} represents a model of a solar photovoltaic panel.
Given that $V_s = \sunit{95 volt}$, $R_1 = \sunit{25 ohm}$,
and $i_L = \sunit{2 ampere}$, find $R_L$.
\placefigq{2.131}{2.71}
\stopQ

\startA
$R_L=\frac{V_s}{i_L} - R_1 = \sunit{22.5 ohm}$
\stopA

% 2.72
\startQ[Q:2.72]
Find $V_o$ in the two-way power divider circuit in \reffig{2.132}.
\placefigq{2.132}{2.72}
\stopQ

\startA
$V_o = \sunit{5 volt}$.
\stopA

% 2.73
\startQ[Q:2.73]
An ammeter model consists of an ideal ammeter
in series with a 20-Ω resistor.
It is connected with a current source and an
unknown resistor $R_x$ as shown in \reffig{2.133}.
The ammeter reading is noted.
When a potentiometer $R$ is added and adjusted
until the ammeter reading drops to one half its
previous reading, then $R = \sunit{65 ohm}$.
What is the value of $R_x$ ?
\placefigq{2.133}{2.73}
\stopQ

\startA
$R=R_x + 20$, $R_x=65-20=\sunit{45 ohm}$.
\stopA

% 2.74
\startQ[Q:2.74]
The circuit in \reffig{2.134} is to control the speed of a motor such
that the motor draws currents \sunit{5 ampere}, \sunit{3 ampere}, and
\sunit{1 ampere} when the switch is at high, medium, and low positions, respectively.
The motor can be modeled as a load resistance of \sunit{20 milli ohm}.
Determine the series dropping resistances $R_1$, $R_2$, and $R_3$.
\placefigq{2.134}{2.74}
\stopQ

\startA
$R_3 = \frac{V}{I_h} - R_f - R_m = 6/5 - 0.01 - 0.02 = 1.17\sunit[l]{ohm}$,\\
$R_2 = \frac{V}{I_m} - \frac{V}{I_h} = 0.8\sunit[l]{ohm}$,\\
$R_1 = \frac{V}{I_l} - \frac{V}{I_m} = 4\sunit[l]{ohm}$.
\stopA

\stopchapter
\stopcomponent
