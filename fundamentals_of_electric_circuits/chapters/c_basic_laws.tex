\startcomponent c_basic_concepts
\startchapter[
  title={Basic Concepts},
]

% 2.1
\startQ
Design a problem, complete with a solution,
to help students to better understand Ohm’s law.
Use at least two resistors and one voltage source.
Hint, you could use both resistors at once or one at a time,
it is up to you. Be creative.
\stopQ

\startA
...
\stopA

% 2.2
\startQ
Find the hot resistance of a light bulb rated \sunit{60 watt}, \sunit{120 volt}.
\stopQ

\startA
$\frac{v^2}{R} = p \rightarrow
R = \frac{v^2}{p}
  = \frac{60^2}{120}
  = 30 \sunit{ohm}$
\stopA

% 2.3
\startQ
A bar of silicon is 4 cm long with a circular cross section.
If the resistance of the bar is 240 Ω at room temperature,
what is the cross-sectional radius of the bar?
\stopQ

\startA
For silicon, $\rho = \sunit{6.4e2 ohm meter}$. $A=\pi r^2$.
Hence:
\startformula\startmathalignment
\NC R \NC =\frac{\rho L}{A}=\frac{\rho L}{\pi r^2} \NR
\NC r \NC = \sqrt{\frac{\rho L}{\pi R}} \NR
\NC   \NC = \sqrt{\frac{6.4\times 10^2 \times \frac{4}{100}}
                       {\pi\times 240}} \NR
\NC   \NC = \sqrt{0.033953} \NR
\NC   \NC = 0.1843\sunit[l]{meter} \NR
\stopmathalignment\stopformula
\stopA

% 2.4
\startQ[Q:2.4]
\startIG
\item Calculate current $i$ in \reffig{2.68} when the switch is in position 1.
\item Find the current when the switch is in position 2.
\stopIG
\placeschemq{2.68}{2.4}
\stopQ

\startA
\startIG
\item $40 / 100 = 0.4\sunit[l]{ampere}$
\item $40 / 250 = 0.16\sunit[l]{ampere}$
\stopIG
\stopA

% 2.5
\startQ[Q:2.5]
For the network graph in \reffig{2.69},
find the number of nodes, branches, and loops.
\placefigure{2.69}{2.5}
\stopQ

\startA
$n=9$, $l=7$; $b=n+l-1=15$.
\stopA

% 2.6
\startQ[Q:2.6]
In the network graph shown in \reffig{2.70},
determine the number of branches and nodes.
\placeschemq{2.70}{2.6}
\stopQ

\startA
$n=12$; $l=8$; $b=n+l-1=19$.
\stopA

% 2.7
\startQ[Q:2.7]
Determine the number of branches and nodes in the circuit of \reffig{2.71}.
\placeschemq{2.71}{2.7}
\stopQ

\startA
$n=4$,$l=3$,$b=n+l-1=6$.
\stopA

% 2.8
\startQ[Q:2.8]
Design a problem, complete with a solution,
to help other students better understand Kirchhoff’s Current Law.
Design the problem by specifying values of $i_a$, $i_b$, and $i_c$,
shown in \reffig{2.72},
and asking them to solve for values of $i_1$, $i_2$, and $i_3$.
Be careful to specify realistic currents.
\placeschemq{2.72}{2.8}
\stopQ

\startA
\stopA

% 2.9
\startQ[Q:2.9]
Find $i_1$, $i_2$ and $i_3$ in \reffig{2.73}.
\placeschemq{2.73}{2.9}
\stopQ

\startA
$i_1 + (-2) + (-3) = 0 \rightarrow i_1 = 5\sunit[l]{ampere}$.

$2 = (-2) + i_3 \rightarrow i_3 = 4\sunit[l]{ampere}$.

$i_2 + i_3 = (-4) \rightarrow i_2 = -8\sunit[l]{ampere}$.
\stopA

% 2.10
\startQ[Q:2.10]
Determine $i_1$ and $i_2$ in the circuit of \reffig{2.74}.
\placeschemq{2.74}{2.10}
\stopQ

\startA
$ i_1 = (-8) + (-6) = -14\sunit[l]{ampere}$.

$i_2 = 4 - (-6) = 10\sunit[l]{ampere}$.
\stopA

% 2.11
\startQ[Q:2.11]
In the circuit of \reffig{2.75}, calculate $V_1$ and $V_2$.
\placeschemq{2.75}{2.11}
\stopQ

\startA
$V_1 = 1 + 5 = 6\sunit[l]{volt}$.

$V_2 = 5 - 2 = 3\sunit[l]{volt}$.
\stopA

% 2.12
\startQ[Q:2.12]
In the circuit in \reffig{2.76}, obtain $v_1$, $v_2$, and $v_3$.
\placeschemq{2.76}{2.12}
\stopQ

\startA
\stopA

\stopchapter
\stopcomponent
